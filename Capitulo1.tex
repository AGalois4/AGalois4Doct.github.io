<?xml version="1.0" encoding="UTF-8"?>
<!DOCTYPE html PUBLIC "-//W3C//DTD XHTML 1.0 Strict//EN" "DTD/xhtml1-strict.dtd">
<html xmlns="http://www.w3.org/1999/xhtml">
<head>
<meta http-equiv="Content-Type" content="text/html; charset=UTF-8" />
<meta name="Generator" content="Kate, the KDE Advanced Text Editor" />
<title>Capitulo1.tex</title>
</head>
<!-- Highlighting: "LaTeX" -->
<body>
<pre style='color:#1f1c1b;background-color:#ffffff;'>
<b><span style='color:#644a9b;'>\documentclass</span></b>[letterpaper,11pt,openany]{<b><span style='color:#0095ff;'>book</span></b>}
<b><span style='color:#644a9b;'>\usepackage</span></b>[utf8]{<b><span style='color:#0095ff;'>inputenc</span></b>}
<b><span style='color:#644a9b;'>\usepackage</span></b>[sort&amp;compress]{<b><span style='color:#0095ff;'>natbib</span></b>}
<b><span style='color:#644a9b;'>\usepackage</span></b>[left=3cm, right=2.5cm, top=2.5cm, bottom=2.5cm]{<b><span style='color:#0095ff;'>geometry</span></b>}
<b><span style='color:#644a9b;'>\usepackage</span></b>{<b><span style='color:#0095ff;'>bigints,relsize</span></b>}
<b><span style='color:#644a9b;'>\usepackage</span></b>[spanish,es-lcroman,es-tabla]{<b><span style='color:#0095ff;'>babel</span></b>}<span style='color:#898887;'>%es-lcroman admite números romanos en minúscula (i),(ii)...</span>
<b><span style='color:#644a9b;'>\usepackage</span></b>[nottoc]{<b><span style='color:#0095ff;'>tocbibind</span></b>}<span style='color:#898887;'>%agrega bibliografía y demás. Opciones: [nottoc] no incluir el índice general, [notlot] no incluir el </span>
<span style='color:#898887;'>%índice de tablas, [notlof] no incluir el índice de figuras, [notindex] no incluir el índice alfabético, [notbib] no incluir la bibliografía </span>
<b><span style='color:#644a9b;'>\usepackage</span></b>{<b><span style='color:#0095ff;'>makeidx</span></b>}<span style='color:#898887;'>%permite agregar un índice de palabras</span>
<b><span style='color:#644a9b;'>\usepackage</span></b>[small]{<b><span style='color:#0095ff;'>caption</span></b>}
<b><span style='color:#644a9b;'>\usepackage</span></b>{<b><span style='color:#0095ff;'>subfigure</span></b>}
<span style='color:#898887;'>%\usepackage[sort&amp;compress]{natbib}</span>

<b><span style='color:#644a9b;'>\usepackage</span></b>{<b><span style='color:#0095ff;'>lscape</span></b>}<span style='color:#898887;'>%una sóla página horizontal con landscape</span>

<b><span style='color:#644a9b;'>\usepackage</span></b>[center]{<b><span style='color:#0095ff;'>titlesec</span></b>} 
<b><span style='color:#644a9b;'>\usepackage</span></b>{<b><span style='color:#0095ff;'>longtable</span></b>}
<span style='color:#898887;'>%\titleformat{\chapter}[display]%[frame]%</span>
<span style='color:#898887;'>%{\Huge}</span>
<span style='color:#898887;'>%{\filcenter\LARGE \ - \thechapter \ - }</span>
<span style='color:#898887;'>%{8pt}</span>
<span style='color:#898887;'>%{\bfseries\LARGE\filcenter}</span>
<span style='color:#898887;'>%{} </span>

<span style='color:#644a9b;'>\newpagestyle</span>{estiloA}[<span style='color:#644a9b;'>\normalsize</span>]{<span style='color:#644a9b;'>\headrule</span>
<span style='color:#644a9b;'>\sethead</span>{ <span style='color:#644a9b;'>\ </span>}{ <span style='color:#644a9b;'>\ </span>}{<span style='color:#644a9b;'>\thepage</span>}}
<span style='color:#644a9b;'>\pagestyle</span>{estiloA}
<span style='color:#644a9b;'>\renewcommand</span>{<b><span style='color:#0095ff;'>\makeheadrule</span></b>}{<span style='color:#898887;'>%</span>
<span style='color:#644a9b;'>\makebox</span>[0pt][l]{<span style='color:#644a9b;'>\rule</span>[.9<span style='color:#644a9b;'>\baselineskip</span>]{<span style='color:#644a9b;'>\linewidth</span>}{0pt}}<span style='color:#898887;'>%</span>
<span style='color:#644a9b;'>\rule</span>[-.4<span style='color:#644a9b;'>\baselineskip</span>]{<span style='color:#644a9b;'>\linewidth</span>}{0.0pt}}


<b><span style='color:#644a9b;'>\usepackage</span></b>{<b><span style='color:#0095ff;'>lmodern</span></b>}
<b><span style='color:#644a9b;'>\usepackage</span></b>{<b><span style='color:#0095ff;'>amsmath,amsfonts,amsthm,amssymb,latexsym,stmaryrd,textcomp,mathrsfs,stackrel,mathtools</span></b>}
<b><span style='color:#644a9b;'>\usepackage</span></b>[T1]{<b><span style='color:#0095ff;'>fontenc</span></b>}
<b><span style='color:#644a9b;'>\usepackage</span></b>{<b><span style='color:#0095ff;'>ragged2e,enumerate</span></b>} 
<b><span style='color:#644a9b;'>\usepackage</span></b>{<b><span style='color:#0095ff;'>color</span></b>}
<b><span style='color:#644a9b;'>\usepackage</span></b>{<b><span style='color:#0095ff;'>bigints,relsize</span></b>}

<b><span style='color:#644a9b;'>\usepackage</span></b>{<b><span style='color:#0095ff;'>xcolor</span></b>}
<span style='color:#898887;'>%\usepackage{fancyhdr}%encabezados,al activar no funciona \tableofcontents</span>
<b><span style='color:#644a9b;'>\usepackage</span></b>{<b><span style='color:#0095ff;'>multirow, array</span></b>} <span style='color:#898887;'>% para las tablas</span>

<b><span style='color:#644a9b;'>\usepackage</span></b>{<b><span style='color:#0095ff;'>url</span></b>}
<b><span style='color:#644a9b;'>\usepackage</span></b>[pdftex,colorlinks=true,linkcolor=blue,citecolor=black,urlcolor=blue,breaklinks=true]{<b><span style='color:#0095ff;'>hyperref</span></b>}
<b><span style='color:#644a9b;'>\usepackage</span></b>[pdftex]{<b><span style='color:#0095ff;'>graphicx</span></b>}
<b><span style='color:#644a9b;'>\usepackage</span></b>{<b><span style='color:#0095ff;'>float</span></b>}
<b><span style='color:#644a9b;'>\usepackage</span></b>[final]{<b><span style='color:#0095ff;'>listofsymbols</span></b>} <span style='color:#898887;'>%modifique el archivo listofsymbols.sty cambiando List of symbols por Lista de Símbolos</span>

<span style='color:#644a9b;'>\setlength\parindent</span>{0em} <span style='color:#898887;'>%modifica la sangría en todo el documento; en 0pt no hay sangría</span>
<span style='color:#644a9b;'>\providecommand</span>{<b><span style='color:#0095ff;'>\abs</span></b>}[1]{<span style='color:#644a9b;'>\lvert</span>#1<span style='color:#644a9b;'>\rvert</span>} <span style='color:#898887;'>% agregar valor absoluto</span>
<span style='color:#644a9b;'>\providecommand</span>{<b><span style='color:#0095ff;'>\norm</span></b>}[1]{<span style='color:#644a9b;'>\lVert</span>#1<span style='color:#644a9b;'>\rVert</span>} <span style='color:#898887;'>% agregar norma</span>

<span style='color:#644a9b;'>\def\-</span>{<span style='color:#644a9b;'>\vspace</span>{.5em}}
<span style='color:#644a9b;'>\def\R</span>{<span style='color:#644a9b;'>\mathbb</span>{R}}
<span style='color:#644a9b;'>\def\N</span>{<span style='color:#644a9b;'>\mathbb</span>{N}}
<span style='color:#644a9b;'>\def\D</span>{<span style='color:#644a9b;'>\Delta</span>}
<span style='color:#644a9b;'>\def\Z</span>{<span style='color:#644a9b;'>\mathbb</span>{Z}}
<span style='color:#644a9b;'>\def\C</span>{<span style='color:#644a9b;'>\mathbb</span>{C}}
<span style='color:#644a9b;'>\def\Q</span>{<span style='color:#644a9b;'>\mathbb</span>{Q}}
<span style='color:#644a9b;'>\def\I</span>{<span style='color:#644a9b;'>\mathbb</span>{I}}
<span style='color:#644a9b;'>\def\M</span>{<span style='color:#644a9b;'>\mathscr</span>{M}}
<span style='color:#644a9b;'>\def\ru</span>{<span style='color:#644a9b;'>\rightline</span>{<span style='color:#644a9b;'>\rule</span>{1.5ex}{1.5ex}}}
<span style='color:#644a9b;'>\def\ur</span>{<span style='color:#644a9b;'>\rightline</span>{<span style='color:#ff5500;'>$</span><span style='color:#3daee9;'>\boxtimes</span><span style='color:#ff5500;'>$</span>}}
<span style='color:#644a9b;'>\def\u</span>{<span style='color:#644a9b;'>\textup</span>}
<span style='color:#644a9b;'>\def\ul</span>{<span style='color:#644a9b;'>\underline</span>{}}
<span style='color:#644a9b;'>\def\n</span>{<span style='color:#644a9b;'>\norm</span>}
<span style='color:#644a9b;'>\def\F</span>{<span style='color:#644a9b;'>\mathbb</span>{F}}
<span style='color:#644a9b;'>\def\l</span>{<span style='color:#644a9b;'>\lim</span> <span style='color:#644a9b;'>\limits</span>_}
<span style='color:#644a9b;'>\def\ie</span>{<b>\begin</b>{<b><span style='color:#0095ff;'>eqnarray</span></b>}<span style='color:#ff5500;'>}</span>
<span style='color:#3daee9;'>\def\fe</span><span style='color:#ff5500;'>{</span><b>\end</b>{<b><span style='color:#0095ff;'>eqnarray</span></b>}}
<span style='color:#644a9b;'>\def\ies</span>{<b>\begin</b>{<b><span style='color:#0095ff;'>eqnarray*</span></b>}<span style='color:#ff5500;'>}</span>
<span style='color:#3daee9;'>\def\fes</span><span style='color:#ff5500;'>{</span><b>\end</b>{<b><span style='color:#0095ff;'>eqnarray*</span></b>}}
<span style='color:#644a9b;'>\def\f</span>{<span style='color:#644a9b;'>\forall\ </span>}
<span style='color:#644a9b;'>\def\t</span>{<span style='color:#ff5500;'>$</span><span style='color:#3daee9;'>\therefore\ </span><span style='color:#ff5500;'>$</span>}
<span style='color:#644a9b;'>\def\ex</span>{<span style='color:#644a9b;'>\exists\ </span>}
<span style='color:#644a9b;'>\def\di</span>{<span style='color:#644a9b;'>\displaystyle\int\limits</span>_}
<span style='color:#644a9b;'>\def\e</span>{<span style='color:#644a9b;'>\varepsilon</span>}
<span style='color:#644a9b;'>\def\rr</span>{<span style='color:#644a9b;'>\rightrightarrows</span>}
<span style='color:#644a9b;'>\def\ovr</span>{<span style='color:#644a9b;'>\overrightarrow</span>}
<span style='color:#644a9b;'>\def\O</span>{<span style='color:#644a9b;'>\Omega</span>}
<span style='color:#644a9b;'>\def\o</span>{<span style='color:#644a9b;'>\omega</span>}
<span style='color:#644a9b;'>\def\G</span>{<span style='color:#644a9b;'>\Gamma</span>}
<span style='color:#644a9b;'>\def\g</span>{<span style='color:#644a9b;'>\gamma</span>}
<span style='color:#644a9b;'>\def\vp</span>{<span style='color:#644a9b;'>\bar</span>{<span style='color:#644a9b;'>\varphi</span>}}
<span style='color:#644a9b;'>\def\a</span>{<span style='color:#644a9b;'>\alpha</span>}
<span style='color:#644a9b;'>\def\b</span>{<span style='color:#644a9b;'>\beta</span>}
<span style='color:#644a9b;'>\def\dl</span>{<span style='color:#644a9b;'>\delta</span>}
<span style='color:#644a9b;'>\def\bu</span>{<span style='color:#644a9b;'>\bar</span>{u}}
<span style='color:#644a9b;'>\def\bx</span>{<span style='color:#644a9b;'>\bar</span>{x}}
<span style='color:#644a9b;'>\def\by</span>{<span style='color:#644a9b;'>\bar</span>{y}}
<span style='color:#644a9b;'>\def\wh</span>{<span style='color:#644a9b;'>\widehat</span>}
<span style='color:#644a9b;'>\def\wt</span>{<span style='color:#644a9b;'>\widetilde</span>}
<span style='color:#644a9b;'>\def\S</span>{<span style='color:#644a9b;'>\Sigma</span>}
<span style='color:#644a9b;'>\def\s</span>{<span style='color:#644a9b;'>\sigma</span>}
<span style='color:#644a9b;'>\def\l</span>{<span style='color:#644a9b;'>\lambda</span>}
<span style='color:#644a9b;'>\def\sb</span>{<span style='color:#644a9b;'>\subseteq</span>}
<span style='color:#644a9b;'>\def\mT</span>{<span style='color:#644a9b;'>\mathit</span>{T}}
<span style='color:#644a9b;'>\def\mX</span>{<span style='color:#644a9b;'>\mathit</span>{X}}
<span style='color:#644a9b;'>\def\mY</span>{<span style='color:#644a9b;'>\mathit</span>{Y}}
<span style='color:#644a9b;'>\def\mU</span>{<span style='color:#644a9b;'>\mathit</span>{U}}
<span style='color:#644a9b;'>\def\H</span>{<span style='color:#644a9b;'>\mathscr</span>{H}}
<span style='color:#644a9b;'>\def\hu</span>{<span style='color:#644a9b;'>\hat</span>{u}}
<span style='color:#644a9b;'>\def\hx</span>{<span style='color:#644a9b;'>\hat</span>{x}}
<span style='color:#644a9b;'>\def\hy</span>{<span style='color:#644a9b;'>\hat</span>{y}}
<span style='color:#644a9b;'>\def\mt</span>{<span style='color:#644a9b;'>\mbox</span>{It}_n(<span style='color:#644a9b;'>\R</span>)}
<span style='color:#644a9b;'>\def\M</span>{<span style='color:#644a9b;'>\mathcal</span>{M}}
<span style='color:#644a9b;'>\def\O</span>{<span style='color:#644a9b;'>\Omega</span>}
<span style='color:#644a9b;'>\def\ul</span>{<span style='color:#644a9b;'>\underline</span>}

<span style='color:#644a9b;'>\DeclareGraphicsExtensions</span>{.bmp,.png,.pdf,.jpg}
<span style='color:#644a9b;'>\theoremstyle</span>{Teo}
<span style='color:#644a9b;'>\newtheorem</span>{Teo}{Teorema}[chapter]
<span style='color:#644a9b;'>\theoremstyle</span>{Def}
<span style='color:#644a9b;'>\newtheorem</span>{Def}{Definición}[chapter]
<span style='color:#644a9b;'>\theoremstyle</span>{Cor}
<span style='color:#644a9b;'>\newtheorem</span>{Cor}{Corolario}[Teo]
<span style='color:#644a9b;'>\theoremstyle</span>{Lem}
<span style='color:#644a9b;'>\newtheorem</span>{Lem}{Lema}[chapter] 
<span style='color:#644a9b;'>\theoremstyle</span>{Prop}
<span style='color:#644a9b;'>\newtheorem</span>{Prop}{Proposición}[chapter]
<span style='color:#644a9b;'>\theoremstyle</span>{PD}
<span style='color:#644a9b;'>\newtheorem</span>*{PD}{Por Demostrar}
<span style='color:#644a9b;'>\theoremstyle</span>{Alg}
<span style='color:#644a9b;'>\newtheorem</span>{Alg}{Algoritmo}

 
<span style='color:#644a9b;'>\renewcommand\qedsymbol</span>{<span style='color:#ff5500;'>$</span><span style='color:#3daee9;'>\rule</span><span style='color:#ff5500;'>{2.5ex}{2.5ex}$</span>}

<span style='color:#644a9b;'>\renewenvironment</span>{proof}{{<span style='color:#644a9b;'>\bfseries</span> <span style='color:#644a9b;'>\textit</span>{Demostración.}}}{<span style='color:#644a9b;'>\hspace</span>{15.65cm}<span style='color:#644a9b;'>\qedsymbol</span>}

<span style='color:#898887;'>%\begin{proof}[\textbf{Demostración.}] si no funciona lo de abajo, usar </span>

<span style='color:#644a9b;'>\makeindex</span> <span style='color:#898887;'>%crear un índice de palabras donde \printindex brinda la posición donde este irá</span>

<span style='color:#644a9b;'>\definecolor</span>{gris}{gray}{0.8}
<span style='color:#644a9b;'>\definecolor</span>{deepblue}{rgb}{0,0,0.5}
<span style='color:#644a9b;'>\definecolor</span>{deepred}{rgb}{0.6,0,0}
<span style='color:#644a9b;'>\definecolor</span>{deepgreen}{rgb}{0,0.5,0}

<b><span style='color:#644a9b;'>\usepackage</span></b>{<b><span style='color:#0095ff;'>listings</span></b>}
<span style='color:#644a9b;'>\lstset</span>{literate=
  {á}{{<span style='color:#644a9b;'>\'</span>a}}1 {é}{{<span style='color:#644a9b;'>\'</span>e}}1 {í}{{<span style='color:#644a9b;'>\'</span>i}}1 {ó}{{<span style='color:#644a9b;'>\'</span>o}}1 {ú}{{<span style='color:#644a9b;'>\'</span>u}}1
  {Á}{{<span style='color:#644a9b;'>\'</span>A}}1 {É}{{<span style='color:#644a9b;'>\'</span>E}}1 {Í}{{<span style='color:#644a9b;'>\'</span>I}}1 {Ó}{{<span style='color:#644a9b;'>\'</span>O}}1 {Ú}{{<span style='color:#644a9b;'>\'</span>U}}1
  {à}{{<span style='color:#644a9b;'>\`</span>a}}1 {è}{{<span style='color:#644a9b;'>\`</span>e}}1 {ì}{{<span style='color:#644a9b;'>\`</span>i}}1 {ò}{{<span style='color:#644a9b;'>\`</span>o}}1 {ù}{{<span style='color:#644a9b;'>\`</span>u}}1
  {À}{{<span style='color:#644a9b;'>\`</span>A}}1 {È}{{<span style='color:#644a9b;'>\'</span>E}}1 {Ì}{{<span style='color:#644a9b;'>\`</span>I}}1 {Ò}{{<span style='color:#644a9b;'>\`</span>O}}1 {Ù}{{<span style='color:#644a9b;'>\`</span>U}}1
  {ä}{{<span style='color:#644a9b;'>\&quot;</span>a}}1 {ë}{{<span style='color:#644a9b;'>\&quot;</span>e}}1 {ï}{{<span style='color:#644a9b;'>\&quot;</span>i}}1 {ö}{{<span style='color:#644a9b;'>\&quot;</span>o}}1 {ü}{{<span style='color:#644a9b;'>\&quot;</span>u}}1
  {Ä}{{<span style='color:#644a9b;'>\&quot;</span>A}}1 {Ë}{{<span style='color:#644a9b;'>\&quot;</span>E}}1 {Ï}{{<span style='color:#644a9b;'>\&quot;</span>I}}1 {Ö}{{<span style='color:#644a9b;'>\&quot;</span>O}}1 {Ü}{{<span style='color:#644a9b;'>\&quot;</span>U}}1
  {â}{{<span style='color:#644a9b;'>\^</span>a}}1 {ê}{{<span style='color:#644a9b;'>\^</span>e}}1 {î}{{<span style='color:#644a9b;'>\^</span>i}}1 {ô}{{<span style='color:#644a9b;'>\^</span>o}}1 {û}{{<span style='color:#644a9b;'>\^</span>u}}1
  {Â}{{<span style='color:#644a9b;'>\^</span>A}}1 {Ê}{{<span style='color:#644a9b;'>\^</span>E}}1 {Î}{{<span style='color:#644a9b;'>\^</span>I}}1 {Ô}{{<span style='color:#644a9b;'>\^</span>O}}1 {Û}{{<span style='color:#644a9b;'>\^</span>U}}1
  {œ}1 {Œ}1 {æ}1 {Æ}1 {ß}1
  {ű}{{<span style='color:#644a9b;'>\H</span>{u}}}1 {Ű}{{<span style='color:#644a9b;'>\H</span>{U}}}1 {ő}{{<span style='color:#644a9b;'>\H</span>{o}}}1 {Ő}{{<span style='color:#644a9b;'>\H</span>{O}}}1
  {ç}{{<span style='color:#644a9b;'>\c</span> c}}1 {Ç}{{<span style='color:#644a9b;'>\c</span> C}}1 {ø}1 {å}{{<span style='color:#644a9b;'>\r</span> a}}1 {Å}{{<span style='color:#644a9b;'>\r</span> A}}1
  {€}1 {£}1 {«}1
  {»}1 {ñ}{{<span style='color:#644a9b;'>\~</span>n}}1 {Ñ}{{<span style='color:#644a9b;'>\~</span>N}}1 {¿}{{?`}}1
}
<span style='color:#644a9b;'>\definecolor</span>{ajustadoverde}{rgb}{0,0.6,0}
<span style='color:#644a9b;'>\definecolor</span>{ajustadovioleta}{rgb}{0.58,0,0.82}
<span style='color:#644a9b;'>\definecolor</span>{ajustadogris}{rgb}{0.5,0.5,0.5}
<span style='color:#644a9b;'>\lstdefinestyle</span>{codigo}{
  keepspaces=true,
  commentstyle=<span style='color:#644a9b;'>\itshape\color</span>{ajustadogris},
  frame=none,
  xleftmargin={0cm},
  breaklines=true,
  language=Python,
  showstringspaces=false,
  basicstyle=<span style='color:#644a9b;'>\scriptsize\ttfamily</span>,
  keywordstyle=<span style='color:#644a9b;'>\color</span>{blue},
  stringstyle=<span style='color:#644a9b;'>\color</span>{ajustadoverde},
  numbers=left,
  numbersep=5pt,
<span style='color:#898887;'>%   numberstyle=\color{black}</span>
}

<span style='color:#644a9b;'>\renewcommand</span>{<b><span style='color:#0095ff;'>\theequation</span></b>}{<span style='color:#644a9b;'>\arabic</span>{equation}}
<span style='color:#644a9b;'>\renewcommand</span>{<b><span style='color:#0095ff;'>\lstlistingname</span></b>}{Código}<span style='color:#898887;'>% Listing -&gt; Algorithm</span>
<span style='color:#644a9b;'>\renewcommand</span>{<b><span style='color:#0095ff;'>\lstlistlistingname</span></b>}{Índice de Códigos}<span style='color:#898887;'>% List of Listings -&gt; List of Algorithms</span>
<span style='color:#644a9b;'>\spanishdecimal</span>{.}<span style='color:#898887;'>%agrega punto en vez de coma</span>

<b><span style='color:#644a9b;'>\usepackage</span></b>{<b><span style='color:#0095ff;'>tikz</span></b>}
<b><span style='color:#644a9b;'>\usepackage</span></b>{<b><span style='color:#0095ff;'>verbatim</span></b>}<span style='color:#898887;'>%comentar lineas mediante \begin{comment}\end{comment}</span>

<b>\begin</b>{<b><span style='color:#0095ff;'>document</span></b>}

<span style='color:#898887;'>%////////////////////////////////////////////////////////////////////////////////////////</span>


<span style='color:#898887;'>%///////////////////////////////////////////////////////////////////////////////////////////////////////////////////////////////////////</span>
<span style='color:#898887;'>%///////////////////////////////////////////////////////////////////////////////////////////////////////////////////////////////////////</span>
<b>\chapter</b>{<b>Realización de sistemas lineales, invariantes y finitos no identificados</b>}
<span style='color:#898887;'>%///////////////////////////////////////////////////////////////////////////////////////////////////////////////////////////////////////</span>
<span style='color:#898887;'>%/////////////////////////////////////////////////////////////////////////////////////////////////////////////////////////////////////// </span>
       
En este capítulo se explica el proceso de realización empleando la aplicación hecha en <b>\citet</b>{<b><span style='color:#0095ff;'>ARGTR</span></b>}. Inicialmente se pormenoriza el modelo usado, el cual representa un subsistema económico dinámico (que se concibió como lineal e invariante), en este caso de base contable. Posteriormente se elucida cómo aplicar la TR a las trayectorias de matrices aleatorias (sucesión de matrices aleatorias), generadas por dicho subsistema, así como sus limitaciones, tanto teóricas, operativas y prácticas. Así mismo, se revisa la noción de cambio estructural en el texto, y se desarrolla un conjunto   de mejoras para corregir sus limitaciones, entre ellas la incorporación de mecanismos de control.<span style='color:#644a9b;'>\-</span>    
 
 Con este ejemplo, se muestra cómo es posible introducir la trayectoria temporal observable de un (sub)sistema económico dentro del proceso de realización, dejando claro que esto requiere especificar al sistema en forma de entrada-salida. Además, se exponen críticamente los planteamientos de la TR en cuanto a su tratamiento de la aleatoriedad, agrupamiento de datos y determinación del cambio estructural, donde se incorpora una revisión crítica de los procesos de optimización, medición y determinación de la bondad de ajuste. <span style='color:#644a9b;'>\-</span> 
 
  Por otro lado, se brindan algunas ideas sobre cómo, a partir de la realización, se puede habilitar el análisis de la propagación de estimulos en el sistema, para despues hablar un poco del control de sistema subyacente, en cuanto a su estructura. 

<b>\begin</b>{<b><span style='color:#0095ff;'>figure</span></b>}[H]
    <span style='color:#644a9b;'>\centering</span>
    <span style='color:#644a9b;'>\caption</span>{<b>\label</b>{<b><span style='color:#0095ff;'>fig11</span></b>}Flujo temático en <b>\citet</b>{<b><span style='color:#0095ff;'>ARGTR</span></b>}}
    <b><span style='color:#644a9b;'>\includegraphics</span></b>[scale=.7]{<b><span style='color:#0095ff;'>TR-Articulo-1.png</span></b>}
    <b>\label</b>{<b><span style='color:#0095ff;'>fig:my</span></b>}
<b>\end</b>{<b><span style='color:#0095ff;'>figure</span></b>}
    
<b>\section</b>{<b>Sistema macroeconómico de determinación del ingreso y la teoría de la realización</b>}
 


El sistema de base contable que se emplea en <b>\citet</b>{<b><span style='color:#0095ff;'>ARGTR</span></b>} es extremadamente simple, pero tiene algunas características relevantes desde el punto de vista conjunto que representan los adjetivos dinámico, estructural y complejo.<span style='color:#644a9b;'>\-</span>

Las relaciones del sistema son aquellas que sostienen, agrupados en dos sectores, los agentes residentes <span style='color:#644a9b;'>\textbf</span>{r} (o internos) y los agentes no residentes <span style='color:#644a9b;'>\textbf</span>{nr} (o externos). Estas relaciones se manifiestan en flujos corrientes, medidos en precios por cantidades y durante cada trimestre. Las transacciones entre ambos sectores se compilan en el siguiente cuadro que satisface los criterios contables del fluir de fondos, del primer principio de Say y de la ley de Walras.<span style='color:#644a9b;'>\-</span>
 
<b>\begin</b>{<b><span style='color:#0095ff;'>figure</span></b>}[H]<span style='color:#644a9b;'>\centering</span>
<span style='color:#644a9b;'>\caption</span>{Criterios contables de flujo de fondos}<span style='color:#644a9b;'>\vspace</span>{1em}
<b>\begin</b>{<b><span style='color:#0095ff;'>tabular</span></b>}{c|cc|c|c}
     <b>&amp;</b><span style='color:#644a9b;'>\textbf</span>{r}<b>&amp;</b><span style='color:#644a9b;'>\textbf</span>{nr}<b>&amp;</b>Ac.<b>&amp;</b>Total  <span style='color:#644a9b;'>\\\hline</span>
<span style='color:#644a9b;'>\textbf</span>{r}    <b>&amp;</b> <span style='color:#ff5500;'>$C_t$</span><b>&amp;</b><span style='color:#ff5500;'>$X_t$</span><b>&amp;</b><span style='color:#ff5500;'>$F_t$</span><b>&amp;</b><span style='color:#ff5500;'>$Y_t$</span><span style='color:#644a9b;'>\\</span>
<span style='color:#644a9b;'>\textbf</span>{nr}   <b>&amp;</b> <span style='color:#ff5500;'>$MC_t$</span><b>&amp;</b><span style='color:#ff5500;'>$0$</span><b>&amp;</b><span style='color:#ff5500;'>$MK_t$</span><b>&amp;</b><span style='color:#ff5500;'>$M_t$</span><span style='color:#644a9b;'>\\</span>
Ah.   <b>&amp;</b> <span style='color:#ff5500;'>$Sr_{t}$</span><b>&amp;</b><span style='color:#ff5500;'>$Sn_{t}$</span><b>&amp;</b><span style='color:#ff5500;'>$0$</span><b>&amp;</b><span style='color:#ff5500;'>$S_t$</span><span style='color:#644a9b;'>\\\hline</span>
      <b>&amp;</b> <span style='color:#ff5500;'>$Y_t$</span><b>&amp;</b><span style='color:#ff5500;'>$M_t$</span><b>&amp;</b><span style='color:#ff5500;'>$I_t$</span><b>&amp;</b><span style='color:#644a9b;'>\\</span>
<b>\end</b>{<b><span style='color:#0095ff;'>tabular</span></b>}
<b>\end</b>{<b><span style='color:#0095ff;'>figure</span></b>}

donde <span style='color:#ff5500;'>$C_t$</span> es el consumo de mercancías intermedias; <span style='color:#ff5500;'>$X_t$</span> exportaciones; <span style='color:#ff5500;'>$F_t$</span> formación bruta de capital; <span style='color:#ff5500;'>$Y_t$</span> ingreso interno bruto; <span style='color:#ff5500;'>$MC_t$</span> consumo intermedio y final de mercancías importadas; <span style='color:#ff5500;'>$MK_t$</span> formación bruta de capital de mercancías importadas; <span style='color:#ff5500;'>$M_t$</span> importaciones, <span style='color:#ff5500;'>$Sr_{t}$</span> ahorro de los residentes; <span style='color:#ff5500;'>$Sn_{t}$</span> ahorro de los no residentes;  <span style='color:#ff5500;'>$S_t$</span> ahorro; e <span style='color:#ff5500;'>$I_t$</span> inversión; el subíndice <span style='color:#ff5500;'>$t$</span> indica el tiempo en que fue hecha la medición respectiva.<span style='color:#644a9b;'>\-</span>

Las dos primeras filas del cuadro son las identidades:
<span style='color:#644a9b;'>\ies</span> <span style='color:#644a9b;'>\footnotesize</span>
Y_t&amp;<span style='color:#644a9b;'>\equiv</span>&amp;C_t+X_t+F_t<span style='color:#644a9b;'>\\</span>
M_t&amp;<span style='color:#644a9b;'>\equiv</span>&amp;MC_t+MK_t
<span style='color:#644a9b;'>\fes</span>
que originan la identidad macroeconómica
<span style='color:#644a9b;'>\ies</span> 
S_t<span style='color:#644a9b;'>\equiv</span> I_t
<span style='color:#644a9b;'>\fes</span>
con la cuál es posible plantear la siguiente igualdad 
<span style='color:#644a9b;'>\ie\footnotesize\label</span>{matsec:1_1}
<span style='color:#644a9b;'>\left</span>(<b>\begin</b>{<b><span style='color:#0095ff;'>array</span></b>}{c}
Y_t<span style='color:#644a9b;'>\\</span>
M_t
<b>\end</b>{<b><span style='color:#0095ff;'>array</span></b>}<span style='color:#644a9b;'>\right</span>)=
<span style='color:#644a9b;'>\left</span>(<b>\begin</b>{<b><span style='color:#0095ff;'>array</span></b>}{cc}
c_t&amp;x_t<span style='color:#644a9b;'>\\</span>
m_t&amp;0
<b>\end</b>{<b><span style='color:#0095ff;'>array</span></b>}<span style='color:#644a9b;'>\right</span>)
<span style='color:#644a9b;'>\left</span>(<b>\begin</b>{<b><span style='color:#0095ff;'>array</span></b>}{c}
Y_t<span style='color:#644a9b;'>\\</span>
M_t
<b>\end</b>{<b><span style='color:#0095ff;'>array</span></b>}<span style='color:#644a9b;'>\right</span>)+
<span style='color:#644a9b;'>\left</span>(<b>\begin</b>{<b><span style='color:#0095ff;'>array</span></b>}{c}
F_t<span style='color:#644a9b;'>\\</span>
MK_t
<b>\end</b>{<b><span style='color:#0095ff;'>array</span></b>}<span style='color:#644a9b;'>\right</span>)
<span style='color:#644a9b;'>\fe</span>
donde 
<span style='color:#644a9b;'>\ies</span> <span style='color:#644a9b;'>\footnotesize</span>
<span style='color:#644a9b;'>\left</span>(<b>\begin</b>{<b><span style='color:#0095ff;'>array</span></b>}{cc}
c_t&amp;x_t<span style='color:#644a9b;'>\\</span>
m_t&amp;0
<b>\end</b>{<b><span style='color:#0095ff;'>array</span></b>}<span style='color:#644a9b;'>\right</span>)=<span style='color:#644a9b;'>\left</span>(<b>\begin</b>{<b><span style='color:#0095ff;'>array</span></b>}{cc}
C_t&amp;X_t<span style='color:#644a9b;'>\\</span>
MC_t&amp;0
<b>\end</b>{<b><span style='color:#0095ff;'>array</span></b>}<span style='color:#644a9b;'>\right</span>)<span style='color:#644a9b;'>\left</span>(<b>\begin</b>{<b><span style='color:#0095ff;'>array</span></b>}{cc}
Y_t^{-1}&amp;0<span style='color:#644a9b;'>\\</span>
0&amp;M_t^{-1}
<b>\end</b>{<b><span style='color:#0095ff;'>array</span></b>}<span style='color:#644a9b;'>\right</span>)
<span style='color:#644a9b;'>\fes</span>
Así, a partir de <b>\eqref</b>{<b><span style='color:#0095ff;'>matsec:1_1</span></b>}, se construye el siguiente modelo entrada-salida
<span style='color:#644a9b;'>\ies\footnotesize</span>
<span style='color:#644a9b;'>\left</span>(<b>\begin</b>{<b><span style='color:#0095ff;'>array</span></b>}{c}
Y_t<span style='color:#644a9b;'>\\</span>
M_t
<b>\end</b>{<b><span style='color:#0095ff;'>array</span></b>}<span style='color:#644a9b;'>\right</span>)&amp;=&amp;<span style='color:#644a9b;'>\left</span>(<b>\begin</b>{<b><span style='color:#0095ff;'>array</span></b>}{cc}
1-c_t&amp;-x_t<span style='color:#644a9b;'>\\</span>
-m_t&amp;1
<b>\end</b>{<b><span style='color:#0095ff;'>array</span></b>}<span style='color:#644a9b;'>\right</span>)^{-1}
<span style='color:#644a9b;'>\left</span>(<b>\begin</b>{<b><span style='color:#0095ff;'>array</span></b>}{c}
F_t<span style='color:#644a9b;'>\\</span>
MK_t
<b>\end</b>{<b><span style='color:#0095ff;'>array</span></b>}<span style='color:#644a9b;'>\right</span>)
<span style='color:#644a9b;'>\fes</span>
Es más, podemos expresar esta igualdad como
<span style='color:#644a9b;'>\ies\label</span>{matsec:1_2} <span style='color:#644a9b;'>\footnotesize</span>
<span style='color:#644a9b;'>\left</span>(<b>\begin</b>{<b><span style='color:#0095ff;'>array</span></b>}{c}
Y_t<span style='color:#644a9b;'>\\</span>
M_t
<b>\end</b>{<b><span style='color:#0095ff;'>array</span></b>}<span style='color:#644a9b;'>\right</span>)&amp;=&amp;<span style='color:#644a9b;'>\frac</span>{1}{<span style='color:#644a9b;'>\abs</span>{A}}<b>\begin</b>{<b><span style='color:#0095ff;'>pmatrix</span></b>}<span style='color:#ff5500;'> </span>
<span style='color:#ff5500;'>1&amp;x_t</span><span style='color:#3daee9;'>\\</span>
<span style='color:#ff5500;'>m_t&amp;1-c_t</span>
<b>\end</b>{<b><span style='color:#0095ff;'>pmatrix</span></b>}<span style='color:#644a9b;'>\left</span>(<b>\begin</b>{<b><span style='color:#0095ff;'>array</span></b>}{c}
F_t<span style='color:#644a9b;'>\\</span>
MK_t
<b>\end</b>{<b><span style='color:#0095ff;'>array</span></b>}<span style='color:#644a9b;'>\right</span>)
<span style='color:#644a9b;'>\fes</span>
donde 
<span style='color:#644a9b;'>\ies\footnotesize</span>
<span style='color:#644a9b;'>\abs</span>{A}&amp;=&amp;det<b>\begin</b>{<b><span style='color:#0095ff;'>pmatrix</span></b>}<span style='color:#ff5500;'> </span>
<span style='color:#ff5500;'>1-c_t&amp;-x_t</span><span style='color:#3daee9;'>\\</span>
<span style='color:#ff5500;'>-m_t&amp;1</span>
<b>\end</b>{<b><span style='color:#0095ff;'>pmatrix</span></b>}<span style='color:#644a9b;'>\\</span>
&amp;=&amp;1-c_t-m_tx_t
<span style='color:#644a9b;'>\fes</span>

Este modelo cuenta con ciertas características sistémicas relevantes que se listan a continuación:
<b>\begin</b>{<b><span style='color:#0095ff;'>enumerate</span></b>}[1.]
    <span style='color:#644a9b;'>\item</span> Interconecta en cada trimestre <span style='color:#ff5500;'>$t$</span> a dos sectores <span style='color:#644a9b;'>\textbf</span>{r} y <span style='color:#644a9b;'>\textbf</span>{nr}, de forma tal que lo que se le vende a <span style='color:#644a9b;'>\textbf</span>{nr} depende de lo que <span style='color:#644a9b;'>\textbf</span>{nr} le compra a <span style='color:#644a9b;'>\textbf</span>{n} y vice-versa;
    <span style='color:#644a9b;'>\item</span> La trayectoria de tres razones clave de las relaciones económicas internas y con el resto del mundo (o externas), son observables: <span style='color:#ff5500;'>$c_t$</span>, la propensión marginal a consumir mercancías de consumo de producción interna; <span style='color:#ff5500;'>$m_t$</span>, la propensión a importar mercancías para producir otras mercancías y para el consumo final; y <span style='color:#ff5500;'>$x_t$</span> la razón del balance en cuenta corriente entre el ingreso por exportaciones y el gasto en importaciones;
    <span style='color:#644a9b;'>\item</span> Relaciona los flujos de entrada (o exógenos) de acumulación de capital <span style='color:#ff5500;'>$</span><span style='color:#3daee9;'>\footnotesize\begin</span><span style='color:#ff5500;'>{pmatrix}</span>
<span style='color:#ff5500;'>F_t</span><span style='color:#3daee9;'>\\</span>
<span style='color:#ff5500;'>MK_t</span>
<b>\end</b>{<b><span style='color:#0095ff;'>pmatrix</span></b>}<span style='color:#ff5500;'>$</span> con los de salida (o endógenos) mediante una descripción observable externa basada en 
<span style='color:#ff5500;'>$</span><span style='color:#3daee9;'>\footnotesize\frac</span><span style='color:#ff5500;'>{1}{</span><span style='color:#3daee9;'>\abs</span><span style='color:#ff5500;'>{A}}</span><b>\begin</b>{<b><span style='color:#0095ff;'>pmatrix</span></b>}<span style='color:#ff5500;'> </span>
<span style='color:#ff5500;'>1&amp;x_t</span><span style='color:#3daee9;'>\\</span>
<span style='color:#ff5500;'>m_t&amp;1-c_t</span>
<b>\end</b>{<b><span style='color:#0095ff;'>pmatrix</span></b>}<span style='color:#ff5500;'>$</span>,
de periodicidad trimestral;
<span style='color:#644a9b;'>\item</span> Goza de una correspondencia entre el grafo del modelo,

<b>\begin</b>{<b><span style='color:#0095ff;'>figure</span></b>}[H]
    <span style='color:#644a9b;'>\centering</span>
    <span style='color:#644a9b;'>\caption</span>{Grafo de intercambios entre residentes (r) y no residentes (nr)}
    <b><span style='color:#644a9b;'>\includegraphics</span></b>{<b><span style='color:#0095ff;'>TR-Articulo.png</span></b>}
    <b>\label</b>{<b><span style='color:#0095ff;'>fig:my_label</span></b>}
<b>\end</b>{<b><span style='color:#0095ff;'>figure</span></b>}
y la matriz
<span style='color:#644a9b;'>\ies</span> <span style='color:#644a9b;'>\footnotesize</span>
<span style='color:#644a9b;'>\frac</span>{1}{<span style='color:#644a9b;'>\abs</span>{A}}<b>\begin</b>{<b><span style='color:#0095ff;'>pmatrix</span></b>}<span style='color:#ff5500;'> </span>
<span style='color:#ff5500;'>1&amp;x_t</span><span style='color:#3daee9;'>\\</span>
<span style='color:#ff5500;'>m_t&amp;1-c_t</span>
<b>\end</b>{<b><span style='color:#0095ff;'>pmatrix</span></b>}
<span style='color:#644a9b;'>\fes</span>
que hace pensar que la retroalimentación entre <span style='color:#644a9b;'>\textbf</span>{r} y <span style='color:#644a9b;'>\textbf</span>{nr} hace emerger una representación que está determinada por una dinámica lineal del sistema subyacente.<span style='color:#644a9b;'>\-</span>
    
<b>\end</b>{<b><span style='color:#0095ff;'>enumerate</span></b>}

El sistema que subyace a la sucesión de grafos, o de otra manera, a la sucesión de sus matrices, es el encargado de gestar la dinámica que existe en dichas sucesiones. Genera en cada tiempo una matriz que permite transformar entradas en salidas, es decir, produce en cada tiempo las condiciones de conversión de estimulos en respuestas.<span style='color:#644a9b;'>\-</span>

El sistema tiene enlazado un conjunto de variables que lo  identifican, asociadas a ciertos procesos clave. Ellas pueden resumir a todas las demás variables, en consonancia, resumen todos los procesos internos llevados a cabo por el sistema. Conociéndolas es posible saber el valor de las restantes en un tiempo dado. Es más, el sistema puede ser concebido como sinónimo de ellas; a las que aludiremos como <span style='color:#644a9b;'>\textbf</span>{variables de estado}. Además,  por circunstancias teórico-contables, la forma en que estas evolucionan se considera que cambia en proporción fija, por lo que el sistema se asume <span style='color:#644a9b;'>\textbf</span>{invariante}.<span style='color:#644a9b;'>\-</span> 

 No obstante, el sistema subyacente no abriga la posibilidad de ser identificado. No es posible establecer teórica o empíricamente reglas sobre su dinámica interna o, equivalentemente, no es posible  determinar sus variables de estado, la interacción entre ellas y  cómo es su evolución. Lo único con lo que se cuenta es con una <span style='color:#644a9b;'>\textbf</span>{cantidad finita} de ``fotografías externas'' que condensan dicha operación, que en nuestro son  100 matrices definidas como
 <span style='color:#644a9b;'>\ies</span>
 <span style='color:#644a9b;'>\mathcal</span>{W}_t=<span style='color:#644a9b;'>\frac</span>{1}{<span style='color:#644a9b;'>\abs</span>{A}}<b>\begin</b>{<b><span style='color:#0095ff;'>pmatrix</span></b>}
<span style='color:#ff5500;'>                   1&amp;x_t</span><span style='color:#3daee9;'>\\</span>
<span style='color:#ff5500;'>                   m_t&amp;1-c_t</span>
<span style='color:#ff5500;'>                  </span><b>\end</b>{<b><span style='color:#0095ff;'>pmatrix</span></b>}
 <span style='color:#644a9b;'>\fes</span>
a las que nombraremos <span style='color:#644a9b;'>\textbf</span>{Descripción Externa (DE)} del sistema, que en nuestro caso se forman trimestre a trimestre.<span style='color:#644a9b;'>\-</span>    


La Teoría de la Realización <span style='color:#644a9b;'>\textbf</span>{TR} trata de identificar al sistema subyacente mediante la DE. Por consiguiente, su uso para la especificación del número de variables de estado, su interacción y evolución,<span style='color:#644a9b;'>\footnote</span>{La TR también se puede emplear en sistemas identificados, pero estos no son muy usuales en Economía, a diferencia de Física o Ingeniería.} <span style='color:#644a9b;'>\textbf</span>{no es un mecanismo de ajuste de datos como lo sería la econometría}; la TR identifica el sistema subyacente que los genera.<span style='color:#644a9b;'>\-</span>


La identificación del sistema invariante culmina al construir mediante la DE una terna de matrices <span style='color:#ff5500;'>$(F,G,H)$</span>, o <span style='color:#644a9b;'>\textbf</span>{Descripción Interna DI}, independientes del tiempo, que puedan generar la DE; a esta construcción se le denominará <span style='color:#644a9b;'>\textbf</span>{realización}. La matriz <span style='color:#ff5500;'>$F$</span> identifica el número, interacción y evolución de las variables de estado. Por su parte, <span style='color:#ff5500;'>$G$</span> es una matriz que codifica el estímulo, o entrada, en información legible para el sistema, mientras que <span style='color:#ff5500;'>$H$</span> hace lo opuesto a <span style='color:#ff5500;'>$G$</span>: decodifica el resultado de los procesos internos (y resumidos por <span style='color:#ff5500;'>$F$</span>), generando una respuesta o salida legibles para el usuario. Así, por un lado tenemos que
<span style='color:#644a9b;'>\ies</span> <span style='color:#644a9b;'>\footnotesize</span>
HF^{t-1}G<span style='color:#644a9b;'>\approx\mathcal</span>{W}_t=<span style='color:#644a9b;'>\frac</span>{1}{<span style='color:#644a9b;'>\abs</span>{A}}<b>\begin</b>{<b><span style='color:#0095ff;'>pmatrix</span></b>}<span style='color:#ff5500;'> </span>
<span style='color:#ff5500;'>1&amp;x_t</span><span style='color:#3daee9;'>\\</span>
<span style='color:#ff5500;'>m_t&amp;1-c_t</span>
<b>\end</b>{<b><span style='color:#0095ff;'>pmatrix</span></b>} <span style='color:#644a9b;'>\ \ \ \ \ </span> t=1,<span style='color:#644a9b;'>\dots</span>,100
<span style='color:#644a9b;'>\fes</span>
donde el símbolo de aproximación <span style='color:#ff5500;'>$(</span><span style='color:#3daee9;'>\approx</span><span style='color:#ff5500;'>)$</span> se coloca ya que la DI no siempre será exacta.<span style='color:#644a9b;'>\footnote</span>{Como se explica adelante, esto es debido a pérdida de información al construirla; por la finitud y calidad de de los datos de la DE; o debido a que el sistema no está bien especificado, por ejemplo, puede ser no lineal o variante temporal.} Esto sustenta que la órbita<span style='color:#644a9b;'>\footnote</span>{La órbita se puede entender como la sucesión generada por una función <span style='color:#ff5500;'>$f$</span>, un valor <span style='color:#ff5500;'>$x_0$</span>  en su dominio y la composición iterada: <span style='color:#ff5500;'>$f(x_0)=f</span><span style='color:#3daee9;'>\circ</span><span style='color:#ff5500;'>^1(x_0),f(f(x_0))  =f</span><span style='color:#3daee9;'>\circ</span><span style='color:#ff5500;'>^2(x_0),</span><span style='color:#3daee9;'>\dots</span><span style='color:#ff5500;'>,f</span><span style='color:#3daee9;'>\circ</span><span style='color:#ff5500;'>^n(x_0),</span><span style='color:#3daee9;'>\dots</span><span style='color:#ff5500;'>$</span> } del sistema subyacente o DI,  ``empata''  con la orbita observada del sistema, la DE. Además, mediante la construcción del sistema subyacente, en cada tiempo podemos analizar cualquier fase del siguiente esquema
 <span style='color:#644a9b;'>\ies</span>
<b>\begin</b>{<b><span style='color:#0095ff;'>matrix</span></b>}
<span style='color:#644a9b;'>\ul</span>{k}&amp;<span style='color:#644a9b;'>\xrightarrow</span>{*}&amp; H&amp; <span style='color:#644a9b;'>\xrightarrow</span>{*}&amp;  FF<span style='color:#644a9b;'>\cdots</span> F=F^{t-1}&amp; <span style='color:#644a9b;'>\xrightarrow</span>{*}&amp; G&amp;  <span style='color:#644a9b;'>\xrightarrow</span>{*}&amp; <span style='color:#644a9b;'>\ul</span>{y} <span style='color:#644a9b;'>\\</span>
<span style='color:#644a9b;'>\mbox</span>{ Estimula }&amp;&amp;<span style='color:#644a9b;'>\mbox</span>{ Codifica }&amp;&amp;<span style='color:#644a9b;'>\mbox</span>{ Evoluciona }&amp;&amp;<span style='color:#644a9b;'>\mbox</span>{ Decodifica}&amp;&amp;<span style='color:#644a9b;'>\mbox</span>{ Responde}
<b>\end</b>{<b><span style='color:#0095ff;'>matrix</span></b>}
 <span style='color:#644a9b;'>\fes</span>
  donde <span style='color:#ff5500;'>$</span><span style='color:#3daee9;'>\ul</span><span style='color:#ff5500;'>{k}_t=</span><b>\begin</b>{<b><span style='color:#0095ff;'>pmatrix</span></b>}
<span style='color:#ff5500;'>                F_t&amp;MK_t</span>
<span style='color:#ff5500;'>               </span><b>\end</b>{<b><span style='color:#0095ff;'>pmatrix</span></b>}<span style='color:#ff5500;'>^T$</span> e <span style='color:#ff5500;'>$</span><span style='color:#3daee9;'>\ul</span><span style='color:#ff5500;'>{y}_t=</span><b>\begin</b>{<b><span style='color:#0095ff;'>pmatrix</span></b>}
<span style='color:#ff5500;'>                Y_t&amp;M_t</span>
<span style='color:#ff5500;'>               </span><b>\end</b>{<b><span style='color:#0095ff;'>pmatrix</span></b>}<span style='color:#ff5500;'>^T$</span>. 
Se está entonces ante la posibilidad de formular más interrogantes sobre el papel del modelo macroeconómico mediante las relaciones estáticas y dinámicas que existen entre <span style='color:#ff5500;'>$</span><span style='color:#3daee9;'>\mathcal</span><span style='color:#ff5500;'>{W}_t$</span> y <span style='color:#ff5500;'>$(F,G,H)$</span>, respecto a:
   <b>\begin</b>{<b><span style='color:#0095ff;'>enumerate</span></b>}[i)]
    <span style='color:#644a9b;'>\item</span> las conexiones registradas en <span style='color:#ff5500;'>$</span><span style='color:#3daee9;'>\mathcal</span><span style='color:#ff5500;'>{W}_t$</span>,
    <span style='color:#644a9b;'>\item</span> las estructuras reveladas por <span style='color:#ff5500;'>$(F,G,H)$</span>, y
    <span style='color:#644a9b;'>\item</span> los cambios en esas conexiones y los correspondientes invariantes que muestran las matrices de parámetros.
   <b>\end</b>{<b><span style='color:#0095ff;'>enumerate</span></b>}
   En consecuencia, el cambio estructural está englobado en la construcción de la realización del sistema subyacente, así como la dinámica  de ajuste que sigue la interacción interna del sistema ante los efectos instigados por cambios intensionales o shocks imprevistos. Esto se irá desvelado con más detenimiento en los siguientes apartados del trabajo.<span style='color:#644a9b;'>\-</span>
  
  
  <b>\section</b>{<b>Construcción de la realización</b>}




Para comprender como construir la DI a través de la DE, incialmente se explicará como elaborarla cuando los datos son finitos y ``bien comportados'', esto último se pormenoriza un poco más adelante.<span style='color:#644a9b;'>\footnote</span>{Existe otra versión cuando los datos son  infinitos y ``bien comportados''. Sin embargo, en la aplicación económica los datos no satisfacen ninguna de estas dos condiciones,  por ello se restringe el análisis  a datos finitos y, eventualmente, a aquellos que no se ``comportan bien''.<span style='color:#644a9b;'>\-</span>} 
<b>\begin</b>{<b><span style='color:#0095ff;'>enumerate</span></b>}[a)]
    <span style='color:#644a9b;'>\item\label</span>{a} Partimos al identificar el tipo de sistema subyacente a modelar. Por ejemplo, en el modelo macroeconómico que tratamos, se asumió un sistema discreto,  lineal, invariante y multientrada-multisalida. Este debe ser expresado de la siguiente forma
    <span style='color:#644a9b;'>\ie\label</span>{sist_lin}
    <span style='color:#644a9b;'>\ul</span>{y}_t=<span style='color:#644a9b;'>\mathcal</span>{W}_t<span style='color:#644a9b;'>\ul</span>{x}_t<span style='color:#644a9b;'>\ \ \ \ \ </span>t=1,<span style='color:#644a9b;'>\dots</span>,k
    <span style='color:#644a9b;'>\fe</span>   
   con <span style='color:#ff5500;'>$</span><span style='color:#3daee9;'>\mathcal</span><span style='color:#ff5500;'>{W}_t</span><span style='color:#3daee9;'>\in\R</span><span style='color:#ff5500;'>^{n</span><span style='color:#3daee9;'>\times</span><span style='color:#ff5500;'> m}$</span> la matriz que condensa y ``encubre'' los procesos internos del sistema subyacente en el tiempo <span style='color:#ff5500;'>$t$</span>; transformando el input o multientrada <span style='color:#ff5500;'>$</span><span style='color:#3daee9;'>\ul</span><span style='color:#ff5500;'>{x}_t$</span> en el output o multisalida <span style='color:#ff5500;'>$</span><span style='color:#3daee9;'>\ul</span><span style='color:#ff5500;'>{y}_t$</span>;
   <span style='color:#644a9b;'>\item</span> A continuación, con las matrices <span style='color:#ff5500;'>$</span><span style='color:#3daee9;'>\mathcal</span><span style='color:#ff5500;'>{W}_t$</span> <span style='color:#ff5500;'>$t=1,</span><span style='color:#3daee9;'>\dots</span><span style='color:#ff5500;'>,k$</span>, que forman la DE del sistema subyacente, construimos la <span style='color:#644a9b;'>\textbf</span>{Matriz de Hankel (MH)}  <span style='color:#ff5500;'>$H_{p,q}(</span><span style='color:#3daee9;'>\mathcal</span><span style='color:#ff5500;'>{W})</span><span style='color:#3daee9;'>\in\M</span><span style='color:#ff5500;'>^{pn</span><span style='color:#3daee9;'>\times</span><span style='color:#ff5500;'> qm}$</span>. Esto se realiza como se ve a continuación:
    <span style='color:#644a9b;'>\ies</span> 
    <b>\begin</b>{<b><span style='color:#0095ff;'>pmatrix</span></b>}
<span style='color:#ff5500;'>    </span><span style='color:#3daee9;'>\mathcal</span><span style='color:#ff5500;'>{W}_1&amp;</span><span style='color:#3daee9;'>\mathcal</span><span style='color:#ff5500;'>{W}_2&amp;</span><span style='color:#3daee9;'>\mathcal</span><span style='color:#ff5500;'>{W}_3&amp;</span><span style='color:#3daee9;'>\mbox</span>{<span style='color:#3daee9;'>\colorbox</span>[rgb]{1,0.41,0.38}{<span style='color:#ff5500;'>$\mathcal{W}_4$</span>}}<span style='color:#ff5500;'>&amp;</span><span style='color:#3daee9;'>\dots</span><span style='color:#ff5500;'>&amp;</span><span style='color:#3daee9;'>\mathcal</span><span style='color:#ff5500;'>{W}_p</span><span style='color:#3daee9;'>\\</span>
<span style='color:#ff5500;'>    </span><span style='color:#3daee9;'>\mathcal</span><span style='color:#ff5500;'>{W}_2&amp;</span><span style='color:#3daee9;'>\mathcal</span><span style='color:#ff5500;'>{W}_3&amp;</span><span style='color:#3daee9;'>\mbox</span>{<span style='color:#3daee9;'>\colorbox</span>[rgb]{1,0.41,0.38}{<span style='color:#ff5500;'>$\mathcal{W}_4$</span>}}<span style='color:#ff5500;'>&amp;</span><span style='color:#3daee9;'>\mathcal</span><span style='color:#ff5500;'>{W}_5&amp;</span><span style='color:#3daee9;'>\dots</span><span style='color:#ff5500;'>&amp;</span><span style='color:#3daee9;'>\mathcal</span><span style='color:#ff5500;'>{W}_{p+1}</span><span style='color:#3daee9;'>\\</span>
<span style='color:#ff5500;'>    </span><span style='color:#3daee9;'>\mathcal</span><span style='color:#ff5500;'>{W}_3&amp;</span><span style='color:#3daee9;'>\mbox</span>{<span style='color:#3daee9;'>\colorbox</span>[rgb]{1,0.41,0.38}{<span style='color:#ff5500;'>$\mathcal{W}_4$</span>}}<span style='color:#ff5500;'>&amp;</span><span style='color:#3daee9;'>\mathcal</span><span style='color:#ff5500;'>{W}_5&amp;</span><span style='color:#3daee9;'>\mathcal</span><span style='color:#ff5500;'>{W}_6&amp;</span><span style='color:#3daee9;'>\dots</span><span style='color:#ff5500;'>&amp;</span><span style='color:#3daee9;'>\mathcal</span><span style='color:#ff5500;'>{W}_{p+2}</span><span style='color:#3daee9;'>\\</span>
<span style='color:#ff5500;'>    </span><span style='color:#3daee9;'>\mbox</span>{<span style='color:#3daee9;'>\colorbox</span>[rgb]{1,0.41,0.38}{<span style='color:#ff5500;'>$\mathcal{W}_4$</span>}}<span style='color:#ff5500;'>&amp;</span><span style='color:#3daee9;'>\mathcal</span><span style='color:#ff5500;'>{W}_5&amp;</span><span style='color:#3daee9;'>\mathcal</span><span style='color:#ff5500;'>{W}_6&amp;</span><span style='color:#3daee9;'>\mathcal</span><span style='color:#ff5500;'>{W}_7&amp;</span><span style='color:#3daee9;'>\dots</span><span style='color:#ff5500;'>&amp;</span><span style='color:#3daee9;'>\mathcal</span><span style='color:#ff5500;'>{W}_{p+3}</span><span style='color:#3daee9;'>\\</span>
<span style='color:#ff5500;'>    </span><span style='color:#3daee9;'>\vdots</span><span style='color:#ff5500;'>&amp;</span><span style='color:#3daee9;'>\vdots</span><span style='color:#ff5500;'>&amp;</span><span style='color:#3daee9;'>\vdots</span><span style='color:#ff5500;'>&amp;</span><span style='color:#3daee9;'>\vdots</span><span style='color:#ff5500;'>&amp;</span><span style='color:#3daee9;'>\ddots</span><span style='color:#ff5500;'>&amp;</span><span style='color:#3daee9;'>\vdots\\</span>
<span style='color:#ff5500;'>    </span><span style='color:#3daee9;'>\mathcal</span><span style='color:#ff5500;'>{W}_q&amp;</span><span style='color:#3daee9;'>\mathcal</span><span style='color:#ff5500;'>{W}_{q+1}&amp;</span><span style='color:#3daee9;'>\mathcal</span><span style='color:#ff5500;'>{W}_{q+2}&amp;</span><span style='color:#3daee9;'>\mathcal</span><span style='color:#ff5500;'>{W}_{q+3}&amp;</span><span style='color:#3daee9;'>\dots</span><span style='color:#ff5500;'>&amp;</span><span style='color:#3daee9;'>\mathcal</span><span style='color:#ff5500;'>{W}_{k}</span>
<span style='color:#ff5500;'>    </span><b>\end</b>{<b><span style='color:#0095ff;'>pmatrix</span></b>}
    <span style='color:#644a9b;'>\fes</span>
    Con esta representación explícita de <span style='color:#ff5500;'>$H_{p,q}(</span><span style='color:#3daee9;'>\mathcal</span><span style='color:#ff5500;'>{W})$</span> se puede ver que su construcción se realiza al colocar el elemento <span style='color:#ff5500;'>$</span><span style='color:#3daee9;'>\mathcal</span><span style='color:#ff5500;'>{W}_s$</span> en cada posición <span style='color:#ff5500;'>$(i,j)$</span> tal que <span style='color:#ff5500;'>$(i+j)-1=s$</span>. Para ejemplificar, observe que para las posiciones <span style='color:#ff5500;'>$(1,4)$</span>, <span style='color:#ff5500;'>$(2,3)$</span>, <span style='color:#ff5500;'>$(3,2)$</span> y <span style='color:#ff5500;'>$(4,1)$</span> se cumple que <span style='color:#ff5500;'>$i+j-1=4$</span>, es decir, <span style='color:#ff5500;'>$(1+4)-1=(2+3)-1=(3+2)-1=(4+1)-1=4$</span>. Así, decimos que el cuarto bloque antidiagonal (en rojo) es homogéneo y está formado por el cuarto elemento de la sucesión, <span style='color:#ff5500;'>$</span><span style='color:#3daee9;'>\mathcal</span><span style='color:#ff5500;'>{W}_4$</span>. De esto se sigue que el <span style='color:#ff5500;'>$k$</span>-ésimo bloque antidiagonal será homogéneo y estará dado por el <span style='color:#ff5500;'>$k$</span>-ésimo elemento de la sucesión. 
    
    <span style='color:#644a9b;'>\item\label</span>{hiptr} En este paso se debe elegir una de todas las matrices de Hankel generadas en el paso anterior. Para ello se selecciona la matriz que cumpla con las condiciones del <span style='color:#644a9b;'>\textbf</span>{Teorema A.15} en <b>\citet</b>[p.50-54]{<b><span style='color:#0095ff;'>ARGTR</span></b>}. Con este fin en mente, dentro de las matrices de Hankel <span style='color:#ff5500;'>$H_{u+1,</span><span style='color:#3daee9;'>\psi</span><span style='color:#ff5500;'>}(</span><span style='color:#3daee9;'>\mathcal</span><span style='color:#ff5500;'>{W})$</span> con <span style='color:#ff5500;'>$</span><span style='color:#3daee9;'>\psi</span><span style='color:#ff5500;'>=k-u$</span>, que satisfacen <span style='color:#ff5500;'>$u+v&lt;100$</span> con <span style='color:#ff5500;'>$v</span><span style='color:#3daee9;'>\in\N</span><span style='color:#ff5500;'>$</span>, y 
    <span style='color:#644a9b;'>\ie\label</span>{condtrp}
     rango(H_{u,v}(<span style='color:#644a9b;'>\mathcal</span>{W}))=rango(H_{u,v+1}(<span style='color:#644a9b;'>\mathcal</span>{W}))=rango(H_{u+1,<span style='color:#644a9b;'>\psi</span>}(<span style='color:#644a9b;'>\mathcal</span>{W}))=r
    <span style='color:#644a9b;'>\fe</span>
    seleccionamos aquella con rango <span style='color:#ff5500;'>$r'$</span> mínimo, digamos <span style='color:#ff5500;'>$H_{p+1,</span><span style='color:#3daee9;'>\psi</span><span style='color:#ff5500;'>}(</span><span style='color:#3daee9;'>\mathcal</span><span style='color:#ff5500;'>{W})$</span>. Cabe indicar que esta matriz no necesariamente es única, sin embargo, todas las matrices con este rango mínimo son equivalentes. Por otro lado, a <span style='color:#ff5500;'>$r'$</span> se denomimna la dimensión de la realización mínima.<span style='color:#644a9b;'>\-</span>  
        
 <span style='color:#644a9b;'>\item\label</span>{d} A partir de <span style='color:#ff5500;'>$H_{p+1,</span><span style='color:#3daee9;'>\psi</span><span style='color:#ff5500;'>}(</span><span style='color:#3daee9;'>\mathcal</span><span style='color:#ff5500;'>{W})$</span> se construyen las matrices <span style='color:#ff5500;'>$P_{p+1}</span><span style='color:#3daee9;'>\in\R</span><span style='color:#ff5500;'>^{n(p+1)}$</span> y <span style='color:#ff5500;'>$Q_{p+1,</span><span style='color:#3daee9;'>\psi</span><span style='color:#ff5500;'>}</span><span style='color:#3daee9;'>\in\R</span><span style='color:#ff5500;'>^{n(p+1)</span><span style='color:#3daee9;'>\times</span><span style='color:#ff5500;'> m</span><span style='color:#3daee9;'>\psi</span><span style='color:#ff5500;'>}$</span>, bajo el <span style='color:#644a9b;'>\textbf</span>{Algoritmo de Factorización}:<span style='color:#644a9b;'>\footnote</span> {Este algoritmo fue expuesto en la <span style='color:#644a9b;'>\textbf</span>{Definición B.1} de <b>\citet</b>{<b><span style='color:#0095ff;'>ARGTR</span></b>}, fundamentándose en lo expuesto en <b>\citet</b>[pp.134-138]{<b><span style='color:#0095ff;'>cas</span></b>}}    
 <span style='color:#898887;'>%#########################################</span>
 
  <span style='color:#0057ae;background:#e0e9f8;'>%BEGIN</span>
  <b>\begin</b>{<b><span style='color:#0095ff;'>enumerate</span></b>}[1.]
   <span style='color:#644a9b;'>\item</span> Definir <span style='color:#ff5500;'>$(Q_{n,m})_{1,:}=(H_{p+1,</span><span style='color:#3daee9;'>\psi</span><span style='color:#ff5500;'>}(</span><span style='color:#3daee9;'>\mathcal</span><span style='color:#ff5500;'>{W}))_{1,:}$</span>;
   <span style='color:#644a9b;'>\item</span> Determinar la condición:
   <b>\begin</b>{<b><span style='color:#0095ff;'>enumerate</span></b>}[2.1.]
    <span style='color:#644a9b;'>\item</span> Si <span style='color:#ff5500;'>$H_{p+1,</span><span style='color:#3daee9;'>\psi</span><span style='color:#ff5500;'>}(</span><span style='color:#3daee9;'>\mathcal</span><span style='color:#ff5500;'>{W})$</span> tiene solo una fila, 
   se define <span style='color:#ff5500;'>$P_{n,n}=[</span><span style='color:#3daee9;'>\hspace</span><span style='color:#ff5500;'>{.1em}1</span><span style='color:#3daee9;'>\hspace</span><span style='color:#ff5500;'>{.1em}]$</span>, 
   y <span style='color:#ff5500;'>$Q_{n,m}=(H_{p+1,</span><span style='color:#3daee9;'>\psi</span><span style='color:#ff5500;'>}(</span><span style='color:#3daee9;'>\mathcal</span><span style='color:#ff5500;'>{W}))_{1,:}$</span>
   <b>\begin</b>{<b><span style='color:#0095ff;'>enumerate</span></b>}[2.1.1] 
    <span style='color:#644a9b;'>\item</span> Fin.
   <b>\end</b>{<b><span style='color:#0095ff;'>enumerate</span></b>}
   <span style='color:#644a9b;'>\item</span> Si <span style='color:#ff5500;'>$H_{p+1,</span><span style='color:#3daee9;'>\psi</span><span style='color:#ff5500;'>}(</span><span style='color:#3daee9;'>\mathcal</span><span style='color:#ff5500;'>{W})$</span> tiene más de una fila, definimos <span style='color:#ff5500;'>$i=1$</span>;
   <b>\end</b>{<b><span style='color:#0095ff;'>enumerate</span></b>}
   <span style='color:#644a9b;'>\item\label</span>{pas3} Si <span style='color:#ff5500;'>$(Q_{n,m})_{i,:}</span><span style='color:#3daee9;'>\neq\ul</span><span style='color:#ff5500;'>{0}$</span>,
   hallar la primer columna <span style='color:#ff5500;'>$j_i$</span> de <span style='color:#ff5500;'>$(Q_{n,m})_{i,:}$</span> tal que <span style='color:#ff5500;'>$q_{i,j_i}</span><span style='color:#3daee9;'>\neq</span><span style='color:#ff5500;'>0$</span> y definir <span style='color:#ff5500;'>$q_{s,j_i}=0$</span>, 
   <span style='color:#ff5500;'>$</span><span style='color:#3daee9;'>\f</span><span style='color:#ff5500;'> s</span><span style='color:#3daee9;'>\in\N</span><span style='color:#ff5500;'>_{i+1:n</span><span style='color:#3daee9;'>\mu</span><span style='color:#ff5500;'>}$</span>;
   <span style='color:#644a9b;'>\item\label</span>{pas4} Determinar <span style='color:#ff5500;'>$p_{u+i,i}$</span> como<span style='color:#644a9b;'>\footnote</span>{Esta expresión es equivalente a   
   <span style='color:#ff5500;'>$(</span><span style='color:#3daee9;'>\mathscr</span><span style='color:#ff5500;'>{H}_{n,m})_{u+i,j_i}=(P_{n,m})_{u+i,:}(Q_{n,m})_{:,j_i}=</span><span style='color:#3daee9;'>\sum</span><span style='color:#ff5500;'>_{s=1}^{n</span><span style='color:#3daee9;'>\mu</span><span style='color:#ff5500;'>}p_{u+i,s}q_{s,j_i}$</span>. Para ello considere el paso 3, que implica 
   <span style='color:#ff5500;'>$</span><span style='color:#3daee9;'>\sum</span><span style='color:#ff5500;'>_{s=i+1}^{n</span><span style='color:#3daee9;'>\mu</span><span style='color:#ff5500;'>}p_{u+i,s}q_{s,j_i}=0$</span>. Además, está bien definida, por el inciso <b>\ref</b>{<b><span style='color:#0095ff;'>pas2</span></b>}, <span style='color:#ff5500;'>$q_{i,j_i}</span><span style='color:#3daee9;'>\neq</span><span style='color:#ff5500;'>0$</span>.
   }
   <span style='color:#644a9b;'>\ies</span>
   p_{u+i,i}=<span style='color:#644a9b;'>\frac</span>{(H_{p+1,<span style='color:#644a9b;'>\psi</span>}(<span style='color:#644a9b;'>\mathcal</span>{W}))_{u+i,j_i}-<span style='color:#644a9b;'>\left</span>(<span style='color:#644a9b;'>\sum</span>_{s=1}^{i-1}p_{u+i,s}q_{s,j_i}<span style='color:#644a9b;'>\right</span>)}{q_{i,j_i}}
   <span style='color:#644a9b;'>\fes</span>
   para toda <span style='color:#ff5500;'>$u</span><span style='color:#3daee9;'>\in\N</span><span style='color:#ff5500;'>_{n</span><span style='color:#3daee9;'>\mu</span><span style='color:#ff5500;'>-i}$</span>;
   <span style='color:#644a9b;'>\item\label</span>{pas5} Determinar <span style='color:#ff5500;'>$q_{i+1,j}$</span> como<span style='color:#644a9b;'>\footnote</span>{Esta expresión es equivalente a 
   <span style='color:#ff5500;'>$(H_{p+1,</span><span style='color:#3daee9;'>\psi</span><span style='color:#ff5500;'>}(</span><span style='color:#3daee9;'>\mathcal</span><span style='color:#ff5500;'>{W}))_{i+1,j}=(P_{n,m})_{i+1,:}(Q_{n,m})_{:,j}=</span><span style='color:#3daee9;'>\sum</span><span style='color:#ff5500;'>_{s</span><span style='color:#3daee9;'>\in\N</span><span style='color:#ff5500;'>_{m</span><span style='color:#3daee9;'>\eta</span><span style='color:#ff5500;'>}}p_{i+1,s}q_{s,j}$</span> Para ello considere que 
   <span style='color:#ff5500;'>$P_{n,n}</span><span style='color:#3daee9;'>\in\textnormal</span><span style='color:#ff5500;'>{It}_{n</span><span style='color:#3daee9;'>\mu</span><span style='color:#ff5500;'>}$</span>, con unos en su diagonal principal. Lo que implica que  
   <span style='color:#ff5500;'>$</span><span style='color:#3daee9;'>\sum</span><span style='color:#ff5500;'>_{s=i+2}^{n</span><span style='color:#3daee9;'>\mu</span><span style='color:#ff5500;'>}p_{i+1,s}q_{s,j_i}=0$</span> y <span style='color:#ff5500;'>$p_{i+1,i+1}=1$</span>. 
   }
   <span style='color:#644a9b;'>\ies</span>
   q_{i+1,j}=(H_{p+1,<span style='color:#644a9b;'>\psi</span>}(<span style='color:#644a9b;'>\mathcal</span>{W}))_{i+1,j}-<span style='color:#644a9b;'>\textstyle\sum</span>_{s=1}^{i-1}p_{i+1,s}q_{s,j}
   <span style='color:#644a9b;'>\fes</span>
   para todo <span style='color:#ff5500;'>$j</span><span style='color:#3daee9;'>\in\N</span><span style='color:#ff5500;'>_{m</span><span style='color:#3daee9;'>\eta</span><span style='color:#ff5500;'>}/</span><span style='color:#3daee9;'>\{</span><span style='color:#ff5500;'>j_r: </span><span style='color:#3daee9;'>\f</span><span style='color:#ff5500;'> r</span><span style='color:#3daee9;'>\in\N</span><span style='color:#ff5500;'>_i</span><span style='color:#3daee9;'>\}</span><span style='color:#ff5500;'>$</span>,<span style='color:#644a9b;'>\footnote</span>{Es decir, todo <span style='color:#ff5500;'>$j$</span> en <span style='color:#ff5500;'>$</span><span style='color:#3daee9;'>\N</span><span style='color:#ff5500;'>_{m</span><span style='color:#3daee9;'>\eta</span><span style='color:#ff5500;'>}$</span> que no corresponda a
   la columna del primer elemento no nulo de la fila <span style='color:#ff5500;'>$1,</span><span style='color:#3daee9;'>\dots</span><span style='color:#ff5500;'>,i$</span> de <span style='color:#ff5500;'>$Q_{n,m}$</span>.}
   <span style='color:#644a9b;'>\item</span> Determinar la condición:
   <b>\begin</b>{<b><span style='color:#0095ff;'>enumerate</span></b>}[6.1.]<b>\label</b>{<b><span style='color:#0095ff;'>6</span></b>}
    <span style='color:#644a9b;'>\item</span> Si <span style='color:#ff5500;'>$(Q_{n,m})_{i+1,:}=</span><span style='color:#3daee9;'>\ul</span><span style='color:#ff5500;'>{0}$</span>, entonces con los valores obtenidos en las <span style='color:#ff5500;'>$i$</span> 
   repeticiones precedentes del AF, y haciendo <span style='color:#ff5500;'>$(Q_{n,m})_{i+1+k,:}=</span><span style='color:#3daee9;'>\ul</span><span style='color:#ff5500;'>{0}$</span>, <span style='color:#ff5500;'>$</span><span style='color:#3daee9;'>\f</span><span style='color:#ff5500;'> r</span><span style='color:#3daee9;'>\in\N</span><span style='color:#ff5500;'>_{n</span><span style='color:#3daee9;'>\mu</span><span style='color:#ff5500;'>-(i+1)}$</span>,
   conformamos las matrices <span style='color:#ff5500;'>$P_{n,n}$</span> y <span style='color:#ff5500;'>$Q_{n,m}$</span>, y terminamos.
    <span style='color:#644a9b;'>\item</span> Si <span style='color:#ff5500;'>$H_{p+1,</span><span style='color:#3daee9;'>\psi</span><span style='color:#ff5500;'>}(</span><span style='color:#3daee9;'>\mathcal</span><span style='color:#ff5500;'>{W})$</span> tiene sólo <span style='color:#ff5500;'>$i+1$</span> filas, 
   entonces con los valores obtenidos en las <span style='color:#ff5500;'>$i$</span> repeticiones precedentes del AF   
   conformamos las matrices <span style='color:#ff5500;'>$P_{n,n}$</span> y <span style='color:#ff5500;'>$Q_{n,m}$</span>, y  terminamos.
   <span style='color:#644a9b;'>\item</span> Si <span style='color:#ff5500;'>$H_{p+1,</span><span style='color:#3daee9;'>\psi</span><span style='color:#ff5500;'>}(</span><span style='color:#3daee9;'>\mathcal</span><span style='color:#ff5500;'>{W})$</span> tiene más de <span style='color:#ff5500;'>$i+1$</span> filas y 
   <span style='color:#ff5500;'>$(Q_{n,m})_{i+1,:}</span><span style='color:#3daee9;'>\neq\ul</span><span style='color:#ff5500;'>{0}$</span>, entonces definimos <span style='color:#ff5500;'>$i=i+1$</span>, 
   es decir, tomamos una nueva <span style='color:#ff5500;'>$i$</span> igual a la <span style='color:#ff5500;'>$i$</span> inmediatemente 
   anterior, más <span style='color:#ff5500;'>$1$</span>, y repetimos el AF desde el inciso 
   3.
   <b>\end</b>{<b><span style='color:#0095ff;'>enumerate</span></b>} 
   <b>\end</b>{<b><span style='color:#0095ff;'>enumerate</span></b>}
   <span style='color:#0057ae;background:#e0e9f8;'>%END </span>
 
 
 
 
 <span style='color:#898887;'>%#########################################</span>
 La primera matriz es triangular inferior con diagonal unitaria y la segunda una matriz con sus filas mayores a <span style='color:#ff5500;'>$r'$</span> nulas.<span style='color:#644a9b;'>\footnote</span>{Esto se sigue del Lema B.1 en <b>\citet</b>{<b><span style='color:#0095ff;'>ARGTR</span></b>}}
 
  <span style='color:#644a9b;'>\item\label</span>{e} Por último, se genera la realización mínima <span style='color:#ff5500;'>$(F,G,H)$</span>. Construimos las matrices <span style='color:#ff5500;'>$P'_{r'}</span><span style='color:#3daee9;'>\in\R</span><span style='color:#ff5500;'>^{r'(p+1)}$</span>, <span style='color:#ff5500;'>$P^*_{r'}</span><span style='color:#3daee9;'>\in\R</span><span style='color:#ff5500;'>^{r'(p+1)}$</span>, definidas como
  <span style='color:#644a9b;'>\ies</span>
P'_{r'}=<span style='color:#644a9b;'>\big</span>(P_{u+1}<span style='color:#644a9b;'>\big</span>)_{1:r',1:r'}<span style='color:#644a9b;'>\hspace</span>{1em}
P^*_{r'}=<span style='color:#644a9b;'>\big</span>(P_{u+1}<span style='color:#644a9b;'>\big</span>)_{n+1:n+r',1:r'}
 <span style='color:#644a9b;'>\fes</span>
 Luego, se define la realización mínima como se muestra a continuación, 
  <span style='color:#644a9b;'>\ies</span>
  (F,G,H)=(F_{r'},G_{r'},H_{r'})
  <span style='color:#644a9b;'>\fes</span>
  donde <span style='color:#ff5500;'>$F_{r'}=(P'_{r'})^{-1}P^*_{r'}$</span>, <span style='color:#ff5500;'>$G_{r'}=(Q_{p+1,</span><span style='color:#3daee9;'>\psi</span><span style='color:#ff5500;'>}</span><span style='color:#3daee9;'>\big</span><span style='color:#ff5500;'>)_{1:r',1:m}$</span> y <span style='color:#ff5500;'>$H_{r'}=</span><span style='color:#3daee9;'>\big</span><span style='color:#ff5500;'>(P_{p+1}</span><span style='color:#3daee9;'>\big</span><span style='color:#ff5500;'>)_{1:n,1:r'}$</span>
  
  <b>\end</b>{<b><span style='color:#0095ff;'>enumerate</span></b>}

  Explicado lo anterior, podemos evocar aquello de datos ``bien comportados'' para dilucidarlo. Dicha expresión denota que la DE (datos) puede constituir al menos una MH que satisfaga la condición <b>\eqref</b>{<b><span style='color:#0095ff;'>condtrp</span></b>} en el paso <b>\eqref</b>{<b><span style='color:#0095ff;'>hiptr</span></b>}. Una DE que corrobore dicha condición se corresponde con que una parte consecutiva de ella pueda explicarla a toda, o equivalentemente, todo elemento de la DE es una combinación lineal de una subsucesión consecutiva de la DE.<span style='color:#644a9b;'>\-</span>   
  

  Existen sistemas que al ser no lineales o poseer una dimensión infinita (requieren una cantidad infinita de variables de estado para representar la información interna) no permitirán que una parte consecutiva de su DE pueda representar a toda ella. Los cambios no lineales en la DE o la dimensión infinita requieren en este sentido una cantidad infinita para representar a la DE. Así mismo, la DE puede no estar ``bien comportada'' por algo ajeno al sistema subyacente: choques aleatorios o errores al generar la DE; o debido al tamaño de la DE, ya que si es insuficiente hace independientes a los elementos. Estas situaciones ponen un obstáculo a la realización que hasta ahora se ha explicado. <span style='color:#644a9b;'>\-</span>
  
  No obstante, realizar una DE franqueando el hecho de que no satisface la condición <b>\eqref</b>{<b><span style='color:#0095ff;'>condtrp</span></b>} para ninguna MH, es decir, que no es ``bien comportada'', es posible. Esto puede hacerse al rectificar el paso <b>\eqref</b>{<b><span style='color:#0095ff;'>hiptr</span></b>} mediante la siguiente secuencia de acciones:
  <b>\begin</b>{<b><span style='color:#0095ff;'>enumerate</span></b>}
   <span style='color:#644a9b;'>\item</span> Cada una de las matrices de Hankel <span style='color:#ff5500;'>$H_{p,q}$</span> generadas por medio de la DE de tamaño <span style='color:#ff5500;'>$k$</span>, con <span style='color:#ff5500;'>$2</span><span style='color:#3daee9;'>\leq</span><span style='color:#ff5500;'> p</span><span style='color:#3daee9;'>\leq</span><span style='color:#ff5500;'> k-1$</span>,  se factorizan mediante la Descomposición en Valores Singulares (SVD). Es decir, para cada <span style='color:#ff5500;'>$H_{p,q}$</span> se encuentran tres matrices: <span style='color:#ff5500;'>$U_{_{pq}}</span><span style='color:#3daee9;'>\in\R</span><span style='color:#ff5500;'>^{n</span><span style='color:#3daee9;'>\times</span><span style='color:#ff5500;'> n}$</span> y <span style='color:#ff5500;'>$V_{_{pq}}</span><span style='color:#3daee9;'>\in\R</span><span style='color:#ff5500;'>^{m</span><span style='color:#3daee9;'>\times</span><span style='color:#ff5500;'> m}$</span> ortogonales; y <span style='color:#ff5500;'>$</span><span style='color:#3daee9;'>\S</span><span style='color:#ff5500;'>_{_{pq}}$</span> diagonal con <span style='color:#ff5500;'>$</span><span style='color:#3daee9;'>\s</span><span style='color:#ff5500;'>_1</span><span style='color:#3daee9;'>\geq\cdots\geq\s</span><span style='color:#ff5500;'>_{r_{_{pq}}}</span><span style='color:#3daee9;'>\geq</span><span style='color:#ff5500;'>0$</span> valores singulares de <span style='color:#ff5500;'>$H_{p,q}$</span> en su diagonal principal, tales que
<span style='color:#644a9b;'>\ies</span>
H_{p,q}=U_{_{pq}}<span style='color:#644a9b;'>\S</span>_{_{pq}} V'_{_{pq}}=
        <span style='color:#644a9b;'>\sum</span>_{i=1}^n<span style='color:#644a9b;'>\sigma</span>_i<span style='color:#644a9b;'>\ul</span>{u}_i<span style='color:#644a9b;'>\ul</span>{v}_i^T=<span style='color:#644a9b;'>\sum</span>_{i=1}^n<span style='color:#644a9b;'>\sigma</span>_i
<b>\begin</b>{<b><span style='color:#0095ff;'>pmatrix</span></b>}
<span style='color:#ff5500;'>u_{i1}v_{i1}&amp;</span><span style='color:#3daee9;'>\cdots</span><span style='color:#ff5500;'>&amp; u_{i1}v_{i(</span><span style='color:#3daee9;'>\a\eta</span><span style='color:#ff5500;'>)}</span><span style='color:#3daee9;'>\\</span>
<span style='color:#3daee9;'>\vdots</span><span style='color:#ff5500;'>&amp;</span><span style='color:#3daee9;'>\ddots</span><span style='color:#ff5500;'>&amp;</span><span style='color:#3daee9;'>\vdots\\</span>
<span style='color:#ff5500;'>u_{i(</span><span style='color:#3daee9;'>\b\mu</span><span style='color:#ff5500;'>)}v_{i1}&amp;</span><span style='color:#3daee9;'>\cdots</span><span style='color:#ff5500;'>&amp; u_{i(</span><span style='color:#3daee9;'>\b\mu</span><span style='color:#ff5500;'>)}v_{i(</span><span style='color:#3daee9;'>\a\eta</span><span style='color:#ff5500;'>)}</span>
<b>\end</b>{<b><span style='color:#0095ff;'>pmatrix</span></b>}
<span style='color:#644a9b;'>\fes</span>
   <span style='color:#644a9b;'>\item</span> Debido a que el número de valores singulares de cada <span style='color:#ff5500;'>$H_{p,q}$</span> es igual a su rango <span style='color:#ff5500;'>$r_{_{pq}}$</span>, para cada una creamos las aproximaciones <span style='color:#ff5500;'>$H_s$</span> dadas como
   <span style='color:#644a9b;'>\ies</span>
   H_{p,q}<span style='color:#644a9b;'>\approx</span> H_s=U_s <span style='color:#644a9b;'>\S</span>_s V'_s
   <span style='color:#644a9b;'>\fes</span>
   donde <span style='color:#ff5500;'>$s$</span> satisface <span style='color:#ff5500;'>$n</span><span style='color:#3daee9;'>\leq</span><span style='color:#ff5500;'> s</span><span style='color:#3daee9;'>\leq</span><span style='color:#ff5500;'> n(p-1)$</span>, y expresa que en <span style='color:#ff5500;'>$</span><span style='color:#3daee9;'>\S</span><span style='color:#ff5500;'>$</span> solo he dejado los primero <span style='color:#ff5500;'>$s$</span> valores singulares, volviendo el resto 0     
   
   Se crea el conjunto índice definido como
   <span style='color:#644a9b;'>\ies</span> 
   S=<span style='color:#644a9b;'>\{</span>s<span style='color:#644a9b;'>\in\N</span>:s&lt;r <span style='color:#644a9b;'>\wedge</span> n|s<span style='color:#644a9b;'>\}</span>
   <span style='color:#644a9b;'>\fes</span>
   <span style='color:#644a9b;'>\item\label</span>{mredpaso} Para cada <span style='color:#ff5500;'>$s</span><span style='color:#3daee9;'>\in</span><span style='color:#ff5500;'> S$</span>, hay que construir la matriz reducida <span style='color:#ff5500;'>$H_s$</span> dada por
   <span style='color:#644a9b;'>\ies</span>
   H_s=U_s <span style='color:#644a9b;'>\S</span>_s V'_s
   <span style='color:#644a9b;'>\fes</span>
   donde <span style='color:#ff5500;'>$</span><span style='color:#3daee9;'>\S</span><span style='color:#ff5500;'>_s=(</span><span style='color:#3daee9;'>\S</span><span style='color:#ff5500;'>)_{1:s,1:s}$</span>, <span style='color:#ff5500;'>$U_s=(U)_{:,1:s}$</span> y <span style='color:#ff5500;'>$V_s=(V)_{:,1:s}$</span>. Esta matriz, en general, pierde la estructura antidiagonal de hankel, es decir, ya no cuenta con bloques antidiagonales homogéneos de <span style='color:#ff5500;'>$</span><span style='color:#3daee9;'>\mathcal</span><span style='color:#ff5500;'>{W}$</span>.<span style='color:#644a9b;'>\-</span>
   
   En el paso <b>\eqref</b>{<b><span style='color:#0095ff;'>tra1</span></b>} se hizo una elección de la MH con el mayor rango. La justificación de tal elección es debido al presente paso. Como la cantidad de valores singulares es igual al rango de la matriz, entonces, a mayor rango más valores singulares, lo cuál permite tener más matrices reducidas y, por tanto, un conjunto mayor de aproximaciones para optimizar la elección de una realización mínima, como se ve en los proximos pasos.<span style='color:#644a9b;'>\-</span>
   
   La justificación, empero, no es teórica y por lo tanto, no necesariamente imbatible por otra condición. En una sección de este capitulo se hablará de ella con más rigurosidad y, en otra, se trabajará en su formalización.
   
   
   <span style='color:#644a9b;'>\item</span> Obtenemos la realización <span style='color:#ff5500;'>$(F_s,G_s,H_s)$</span> de <span style='color:#ff5500;'>$H_s$</span>, para cada <span style='color:#ff5500;'>$s</span><span style='color:#3daee9;'>\in</span><span style='color:#ff5500;'> S$</span>. Esto se consigue al aplicar los pasos <b>\eqref</b>{<b><span style='color:#0095ff;'>d</span></b>} y <b>\eqref</b>{<b><span style='color:#0095ff;'>e</span></b>} a <span style='color:#ff5500;'>$H_s$</span>.
   
   <span style='color:#644a9b;'>\item</span> Se construye la sucesión estimada 
   <span style='color:#644a9b;'>\ies</span>
   <span style='color:#644a9b;'>\widehat</span>{<span style='color:#644a9b;'>\mathcal</span>{W}}_s=<span style='color:#644a9b;'>\left\{</span>H_s(F_s)^{t-1}G_s|_{t<span style='color:#644a9b;'>\in\N</span>_k}<span style='color:#644a9b;'>\right\}</span>
   <span style='color:#644a9b;'>\fes</span>
   para cada <span style='color:#ff5500;'>$s</span><span style='color:#3daee9;'>\in</span><span style='color:#ff5500;'> S$</span>. El atributo ``estimada'' es debido a que posee un grado de proximidad a la representación externa <span style='color:#ff5500;'>$</span><span style='color:#3daee9;'>\mathcal</span><span style='color:#ff5500;'>{W}$</span>.
   
  
   Mediante la realización de la matriz reducida no se puede reconstruir los elementos que componen la matriz reducida, debido a que no cuenta con estructura de hankel. Así, dado lo que se dijo en <b>\eqref</b>{<b><span style='color:#0095ff;'>mredpaso</span></b>}, cada sucesión estimada <span style='color:#ff5500;'>$</span><span style='color:#3daee9;'>\widehat</span><span style='color:#ff5500;'>{</span><span style='color:#3daee9;'>\mathcal</span><span style='color:#ff5500;'>{W}}_s$</span>  acumula dos distanciamientos de <span style='color:#ff5500;'>$</span><span style='color:#3daee9;'>\mathcal</span><span style='color:#ff5500;'>{W}$</span>: el generado al crear la matriz reducida y el derivado de la realización. El primero es necesario para retirar el ruido en los datos, el segundo, sin embargo, aun queda abierto a ser investigado.<span style='color:#644a9b;'>\footnote</span>{Hay información en las antidiagonales de la matriz reducida que se pierden en el proceso de realización debido a que no son homogéneas. Esta información podría brindar un mejor ajuste a la matriz <span style='color:#ff5500;'>$</span><span style='color:#3daee9;'>\mathcal</span><span style='color:#ff5500;'>{W}$</span>.}
   
   <span style='color:#644a9b;'>\item</span> La proximidad de <span style='color:#ff5500;'>$</span><span style='color:#3daee9;'>\widehat</span><span style='color:#ff5500;'>{</span><span style='color:#3daee9;'>\mathcal</span><span style='color:#ff5500;'>{W}}_s$</span> a <span style='color:#ff5500;'>$</span><span style='color:#3daee9;'>\mathcal</span><span style='color:#ff5500;'>{W}$</span>, para cada <span style='color:#ff5500;'>$s</span><span style='color:#3daee9;'>\in</span><span style='color:#ff5500;'> S$</span>, está definida como el promedio de la distancia entre el elemento <span style='color:#ff5500;'>$i$</span> de <span style='color:#ff5500;'>$</span><span style='color:#3daee9;'>\widehat</span><span style='color:#ff5500;'>{</span><span style='color:#3daee9;'>\mathcal</span><span style='color:#ff5500;'>{W}}_s$</span> y el elemento <span style='color:#ff5500;'>$i$</span> de <span style='color:#ff5500;'>$</span><span style='color:#3daee9;'>\mathcal</span><span style='color:#ff5500;'>{W}$</span>, para cada <span style='color:#ff5500;'>$i=1,</span><span style='color:#3daee9;'>\dots</span><span style='color:#ff5500;'>,k$</span>. 
   
   Esta definición se logra expresar matemáticamente utilizando la distancia de Frobenius, es decir, la función <span style='color:#ff5500;'>$d(</span><span style='color:#3daee9;'>\cdot</span><span style='color:#ff5500;'>,</span><span style='color:#3daee9;'>\cdot</span><span style='color:#ff5500;'>)_F:</span><span style='color:#3daee9;'>\R</span><span style='color:#ff5500;'>^{n</span><span style='color:#3daee9;'>\times</span><span style='color:#ff5500;'> m}</span><span style='color:#3daee9;'>\times\R</span><span style='color:#ff5500;'>^{n</span><span style='color:#3daee9;'>\times</span><span style='color:#ff5500;'> m}</span><span style='color:#3daee9;'>\to</span><span style='color:#ff5500;'> </span><span style='color:#3daee9;'>\R</span><span style='color:#ff5500;'>$</span> inducida por la norma de Frobenius,
   <span style='color:#644a9b;'>\ies</span>
   d(A,B)_F=<span style='color:#644a9b;'>\norm</span>{B-A}_F
   <span style='color:#644a9b;'>\fes</span>
   A partir de ella, la proximidad entre <span style='color:#ff5500;'>$</span><span style='color:#3daee9;'>\widehat</span><span style='color:#ff5500;'>{</span><span style='color:#3daee9;'>\mathcal</span><span style='color:#ff5500;'>{W}}_s$</span> y <span style='color:#ff5500;'>$</span><span style='color:#3daee9;'>\mathcal</span><span style='color:#ff5500;'>{W}$</span>, como fue definida, está dada por
   <span style='color:#644a9b;'>\ies</span>
   <span style='color:#644a9b;'>\overline</span>{d(<span style='color:#644a9b;'>\mathcal</span>{W},<span style='color:#644a9b;'>\widehat</span>{<span style='color:#644a9b;'>\mathcal</span>{W}}_s)}=<span style='color:#644a9b;'>\frac</span>{<span style='color:#644a9b;'>\sum</span>_{i=1}^{k}d(W_i,<span style='color:#644a9b;'>\widehat</span>{W}s_i)}{k}
   <span style='color:#644a9b;'>\fes</span>
   Definimos el conjunto de proximidad <span style='color:#ff5500;'>$M_p$</span> como
   <span style='color:#644a9b;'>\ies</span>
   M_p=<span style='color:#644a9b;'>\left\{\overline</span>{d(<span style='color:#644a9b;'>\mathcal</span>{W},<span style='color:#644a9b;'>\widehat</span>{<span style='color:#644a9b;'>\mathcal</span>{W}}_s)}: s<span style='color:#644a9b;'>\in</span> S <span style='color:#644a9b;'>\wedge</span> i=1,<span style='color:#644a9b;'>\dots</span>,k <span style='color:#644a9b;'>\right\}</span>
   <span style='color:#644a9b;'>\fes</span>
   <span style='color:#644a9b;'>\item</span> Por último, se elige la realización que induce el mejor ajuste <span style='color:#ff5500;'>$</span><span style='color:#3daee9;'>\widehat</span><span style='color:#ff5500;'>{</span><span style='color:#3daee9;'>\mathcal</span><span style='color:#ff5500;'>{W}}_{s}$</span> a <span style='color:#ff5500;'>$</span><span style='color:#3daee9;'>\mathcal</span><span style='color:#ff5500;'>{W}$</span>. Es decir,  se toma  <span style='color:#ff5500;'>$(F_{s^*},G_{s^*},H_{s^*})$</span> tal que 
   <span style='color:#644a9b;'>\ies</span>
   <span style='color:#644a9b;'>\overline</span>{d(<span style='color:#644a9b;'>\mathcal</span>{W},<span style='color:#644a9b;'>\widehat</span>{<span style='color:#644a9b;'>\mathcal</span>{W}}_{s^*})}=<span style='color:#644a9b;'>\min</span> M_p
   <span style='color:#644a9b;'>\fes</span>
    <b>\end</b>{<b><span style='color:#0095ff;'>enumerate</span></b>}


<b>\section</b>{<b>SVD, aleatoriedad y métodos de agrupación</b>}


El Teorema SVD es aplicado en varias metodológicas. Los resultados y características en diversos usos en los que queda urdido heredan propiedades a la TR; aunque no se puede decir que el trasbase se dé cabalmente. Sin embargo, aun de forma parcial puede generar la aparición incipiente de las cualidades de una metodología en otra. Esto justamente es lo que se explora en esta sección: la forma y grado en que metodologías para el tratamiento de aleatoriedad y agrupación mediante SVD transmiten características a la TR.<span style='color:#644a9b;'>\-</span> 

El Teorema SVD nos dice que para una matriz <span style='color:#ff5500;'>$X</span><span style='color:#3daee9;'>\in\R</span><span style='color:#ff5500;'>^{n</span><span style='color:#3daee9;'>\times</span><span style='color:#ff5500;'> m}$</span>, con rango <span style='color:#ff5500;'>$r$</span>, existen las matrices <span style='color:#ff5500;'>$U</span><span style='color:#3daee9;'>\in\R</span><span style='color:#ff5500;'>^{n</span><span style='color:#3daee9;'>\times</span><span style='color:#ff5500;'> n}$</span> y <span style='color:#ff5500;'>$V</span><span style='color:#3daee9;'>\in\R</span><span style='color:#ff5500;'>^{m</span><span style='color:#3daee9;'>\times</span><span style='color:#ff5500;'> m}$</span> ortogonales, y <span style='color:#ff5500;'>$</span><span style='color:#3daee9;'>\S</span><span style='color:#ff5500;'>$</span> diagonal con <span style='color:#ff5500;'>$</span><span style='color:#3daee9;'>\s</span><span style='color:#ff5500;'>_1</span><span style='color:#3daee9;'>\geq\cdots\geq\s</span><span style='color:#ff5500;'>_{r}</span><span style='color:#3daee9;'>\geq</span><span style='color:#ff5500;'>0$</span>, tales que
<span style='color:#644a9b;'>\ies</span>
X=U<span style='color:#644a9b;'>\S</span> V'
<span style='color:#644a9b;'>\fes</span>
A partir de las tres matrices de la factorización, podemos definir las siguientes matrices: <span style='color:#ff5500;'>$</span><span style='color:#3daee9;'>\S</span><span style='color:#ff5500;'>_p=(</span><span style='color:#3daee9;'>\S</span><span style='color:#ff5500;'>)_{1:p,1:p}$</span>, <span style='color:#ff5500;'>$U_p=(U)_{:,1:p}$</span> y <span style='color:#ff5500;'>$V_p=(V)_{1:p,:}$</span>, con <span style='color:#ff5500;'>$p</span><span style='color:#3daee9;'>\leq</span><span style='color:#ff5500;'> r$</span>. Si multiplicamos cada una como sigue 
<span style='color:#644a9b;'>\ies</span>
X_p=U_p<span style='color:#644a9b;'>\S</span>_pV'_p
<span style='color:#644a9b;'>\fes</span>
obtenemos la matriz SVD truncada en <span style='color:#ff5500;'>$p$</span> valores singulares, <span style='color:#ff5500;'>$X_p$</span>. Esta matriz tiene una relación con la matriz original <span style='color:#ff5500;'>$X$</span>. Tomando <span style='color:#ff5500;'>$E_p$</span>  como la matriz de error generada al truncar la SVD completa, tenemos que
<span style='color:#644a9b;'>\ie\label</span>{svdp}
X=U_p<span style='color:#644a9b;'>\S</span>_p V_p+E_p=X_p+E_p
<span style='color:#644a9b;'>\fe</span>
Considerando el caso particular donde <span style='color:#ff5500;'>$p=r$</span>, entonces  <span style='color:#ff5500;'>$E_p$</span> sería identicamente nula, esto debido a que toda fila o columna mayor a <span style='color:#ff5500;'>$r$</span> en <span style='color:#ff5500;'>$</span><span style='color:#3daee9;'>\S</span><span style='color:#ff5500;'>$</span> es nula. En este sentido, <span style='color:#ff5500;'>$X$</span> sería igual a <span style='color:#ff5500;'>$X_r$</span> y la factorización <span style='color:#ff5500;'>$U_r$</span>, <span style='color:#ff5500;'>$</span><span style='color:#3daee9;'>\S</span><span style='color:#ff5500;'>_r$</span> <span style='color:#ff5500;'>$V_r$</span> de <span style='color:#ff5500;'>$X_r$</span> se pasa a llamar la SVD compacta de <span style='color:#ff5500;'>$X$</span>.<span style='color:#644a9b;'>\-</span>

Otro rasgo sobre la matriz SVD truncada en <span style='color:#ff5500;'>$p$</span>, es que el error que induce <span style='color:#ff5500;'>$E_p$</span>, es el mínimo entre todas las matrices en <span style='color:#ff5500;'>$</span><span style='color:#3daee9;'>\R</span><span style='color:#ff5500;'>^{n</span><span style='color:#3daee9;'>\times</span><span style='color:#ff5500;'> m}$</span> de rango menor o igual a <span style='color:#ff5500;'>$p$</span>. En efecto, a partir de las demostraciones constructivas elaboradas por <b>\citet</b>{<b><span style='color:#0095ff;'>eckart</span></b>} y  <b>\citet</b>{<b><span style='color:#0095ff;'>mirsky</span></b>}, podemos afirmar que dada una matriz <span style='color:#ff5500;'>$X$</span>, la minimización de <span style='color:#ff5500;'>$d_{_F}(X,</span><span style='color:#3daee9;'>\widehat</span><span style='color:#ff5500;'>{X})$</span>,<span style='color:#644a9b;'>\footnote</span>{La función <span style='color:#ff5500;'>$d_{_F}(</span><span style='color:#3daee9;'>\cdot</span><span style='color:#ff5500;'>,</span><span style='color:#3daee9;'>\cdot</span><span style='color:#ff5500;'>):</span><span style='color:#3daee9;'>\R</span><span style='color:#ff5500;'>^{n</span><span style='color:#3daee9;'>\times</span><span style='color:#ff5500;'> m}</span><span style='color:#3daee9;'>\times\R</span><span style='color:#ff5500;'>^{n</span><span style='color:#3daee9;'>\times</span><span style='color:#ff5500;'> m}</span><span style='color:#3daee9;'>\to</span><span style='color:#ff5500;'> </span><span style='color:#3daee9;'>\R</span><span style='color:#ff5500;'>$</span> es la distancia de Frobenius, Definición <b>\ref</b>{<b><span style='color:#0095ff;'>disfrob</span></b>}.} dentro del conjunto de matrices <span style='color:#ff5500;'>$</span><span style='color:#3daee9;'>\widehat</span><span style='color:#ff5500;'>{X}</span><span style='color:#3daee9;'>\in\R</span><span style='color:#ff5500;'>^{n</span><span style='color:#3daee9;'>\times</span><span style='color:#ff5500;'> m}$</span> sujetas a <span style='color:#ff5500;'>$rango(</span><span style='color:#3daee9;'>\widehat</span><span style='color:#ff5500;'>{X})</span><span style='color:#3daee9;'>\leq</span><span style='color:#ff5500;'> p$</span>, es tal que 
<span style='color:#644a9b;'>\ie\label</span>{trmc}
d_{_F}(X,X_p)=<span style='color:#644a9b;'>\inf\limits</span>_{rango(X')<span style='color:#644a9b;'>\leq</span> p}d_{_F}(X',X)
<span style='color:#644a9b;'>\fe</span>
Es decir, <span style='color:#ff5500;'>$X_p$</span> es la matriz más proxima a <span style='color:#ff5500;'>$X$</span> dentro de las matrices en <span style='color:#ff5500;'>$n</span><span style='color:#3daee9;'>\times</span><span style='color:#ff5500;'> m$</span> con rango menor o igual a <span style='color:#ff5500;'>$p$</span>. Además, es única cuando y sólo cuando <span style='color:#ff5500;'>$</span><span style='color:#3daee9;'>\s</span><span style='color:#ff5500;'>_p</span><span style='color:#3daee9;'>\neq\s</span><span style='color:#ff5500;'>_{p+1}$</span>. Esto permite asegurar que en el proceso de realización, las matrices de Hankel reducidas son las de mínima distancia y únicas, dentro de las matrices con igual o menor rango al de la matriz de Hankel original. <span style='color:#644a9b;'>\-</span>

 Por otro lado, el proceso de minimización afirmado en la desigualdad <b>\eqref</b>{<b><span style='color:#0095ff;'>trmc</span></b>} cuenta con la siguiente expresión <span style='color:#ff5500;'>$</span><span style='color:#3daee9;'>\n</span><span style='color:#ff5500;'>{X_p}_{_F}^{^2}/</span><span style='color:#3daee9;'>\n</span><span style='color:#ff5500;'>{X}^{^2}$</span> para medir su bondad de ajuste <b>\citep</b>[p.2]{<b><span style='color:#0095ff;'>gonval</span></b>}. Aunque, por causa del Teorema <b>\ref</b>{<b><span style='color:#0095ff;'>vsfrob</span></b>}, la norma de Frobenius <span style='color:#ff5500;'>$</span><span style='color:#3daee9;'>\n</span><span style='color:#ff5500;'>{</span><span style='color:#3daee9;'>\cdot</span><span style='color:#ff5500;'>}_{_F}$</span> es equivalente a la suma cudrática de los valores singulares <span style='color:#ff5500;'>$</span><span style='color:#3daee9;'>\s</span><span style='color:#ff5500;'>_i$</span>, o lo que es lo mismo, a la suma de los valores propios <span style='color:#ff5500;'>$</span><span style='color:#3daee9;'>\l</span><span style='color:#ff5500;'>_i=</span><span style='color:#3daee9;'>\s</span><span style='color:#ff5500;'>_i^2$</span>  de la matriz <span style='color:#ff5500;'>$X'X$</span> (Definición <b>\ref</b>{<b><span style='color:#0095ff;'>valorsingular</span></b>}). Por lo que la bondad de ajuste puede ser vista como
<span style='color:#644a9b;'>\ies</span>
<span style='color:#644a9b;'>\frac</span>{<span style='color:#644a9b;'>\n</span>{X_p}_{_F}^{^2}}{<span style='color:#644a9b;'>\n</span>{X}^{^2}_{F}}=
<span style='color:#644a9b;'>\frac</span>{<span style='color:#644a9b;'>\sum</span>_{i=1}^p<span style='color:#644a9b;'>\s</span>_i^2}{<span style='color:#644a9b;'>\sum</span>_{i=1}^r<span style='color:#644a9b;'>\s</span>_i^2}=
<span style='color:#644a9b;'>\frac</span>{<span style='color:#644a9b;'>\sum</span>_{i=1}^p<span style='color:#644a9b;'>\l</span>_i}{<span style='color:#644a9b;'>\sum</span>_{i=1}^r<span style='color:#644a9b;'>\l</span>_i}=<span style='color:#644a9b;'>\frac</span>{1}{1+<span style='color:#644a9b;'>\frac</span>{<span style='color:#644a9b;'>\sum</span>_{i=p+1}^{r}<span style='color:#644a9b;'>\l</span>_i}{<span style='color:#644a9b;'>\sum</span>_{i=1}^{p}<span style='color:#644a9b;'>\l</span>_i}}
<span style='color:#644a9b;'>\fes</span>



<span style='color:#644a9b;'>\-</span>


<b>\section</b>{<b>El mecanismo de agrupación en la TR: acercamiento al concepto de cambio estructural</b>}

El modelo contable visto con anterioridad tomando l

obtuvo el siguiente resultado para la descripción interna
<span style='color:#644a9b;'>\ies\footnotesize</span>
HF^{t-1}G&amp;=&amp;<b>\begin</b>{<b><span style='color:#0095ff;'>pmatrix</span></b>}
<span style='color:#ff5500;'>1&amp;0</span><span style='color:#3daee9;'>\\</span>
<span style='color:#ff5500;'>0&amp;1</span>
<b>\end</b>{<b><span style='color:#0095ff;'>pmatrix</span></b>}
<b>\begin</b>{<b><span style='color:#0095ff;'>pmatrix</span></b>}
<span style='color:#ff5500;'>1&amp;0</span><span style='color:#3daee9;'>\\</span>
<span style='color:#ff5500;'>0&amp;1</span>
<b>\end</b>{<b><span style='color:#0095ff;'>pmatrix</span></b>}^{t-1}
<b>\begin</b>{<b><span style='color:#0095ff;'>pmatrix</span></b>}
<span style='color:#ff5500;'>5&amp;5</span><span style='color:#3daee9;'>\\</span>
<span style='color:#ff5500;'>0&amp;1</span>
<b>\end</b>{<b><span style='color:#0095ff;'>pmatrix</span></b>}<span style='color:#644a9b;'>\\</span>
&amp;=&amp;<b>\begin</b>{<b><span style='color:#0095ff;'>pmatrix</span></b>}
<span style='color:#ff5500;'>5&amp;5</span><span style='color:#3daee9;'>\\</span>
<span style='color:#ff5500;'>0&amp;1</span>
<b>\end</b>{<b><span style='color:#0095ff;'>pmatrix</span></b>}
<span style='color:#644a9b;'>\fes</span>
para cada <span style='color:#ff5500;'>$t=1,</span><span style='color:#3daee9;'>\dots</span><span style='color:#ff5500;'>,100$</span>, así tenemos que el sistema subyacente expresa que las 100 matrices, retirando el ruido, son muy parecidas y se pueden representar mediante una única matriz:
<span style='color:#644a9b;'>\ies\footnotesize</span>
<span style='color:#644a9b;'>\frac</span>{1}{<span style='color:#644a9b;'>\abs</span>{A}}<b>\begin</b>{<b><span style='color:#0095ff;'>pmatrix</span></b>}
<span style='color:#ff5500;'>1&amp;x_t</span><span style='color:#3daee9;'>\\</span>
<span style='color:#ff5500;'>m_t&amp;1-c_t</span>
<b>\end</b>{<b><span style='color:#0095ff;'>pmatrix</span></b>}&amp;=&amp;<b>\begin</b>{<b><span style='color:#0095ff;'>pmatrix</span></b>}
<span style='color:#ff5500;'>5&amp;5</span><span style='color:#3daee9;'>\\</span>
<span style='color:#ff5500;'>0&amp;1</span>
<b>\end</b>{<b><span style='color:#0095ff;'>pmatrix</span></b>}<span style='color:#644a9b;'>\\</span>
&amp;=&amp;<span style='color:#644a9b;'>\frac</span>{1}{0.2}<b>\begin</b>{<b><span style='color:#0095ff;'>pmatrix</span></b>}
<span style='color:#ff5500;'>1&amp;1</span><span style='color:#3daee9;'>\\</span>
<span style='color:#ff5500;'>0&amp;0.2</span>
<b>\end</b>{<b><span style='color:#0095ff;'>pmatrix</span></b>}
<span style='color:#644a9b;'>\fes</span>
lo cuál implica que 
<span style='color:#644a9b;'>\ies</span> <span style='color:#644a9b;'>\footnotesize</span>
<b>\begin</b>{<b><span style='color:#0095ff;'>pmatrix</span></b>}
<span style='color:#ff5500;'>Y_t</span><span style='color:#3daee9;'>\\</span>
<span style='color:#ff5500;'>M_t</span>
<b>\end</b>{<b><span style='color:#0095ff;'>pmatrix</span></b>}&amp;=&amp;<span style='color:#644a9b;'>\frac</span>{1}{0.2}<b>\begin</b>{<b><span style='color:#0095ff;'>pmatrix</span></b>}
<span style='color:#ff5500;'>1&amp;1</span><span style='color:#3daee9;'>\\</span>
<span style='color:#ff5500;'>0&amp;0.2</span>
<b>\end</b>{<b><span style='color:#0095ff;'>pmatrix</span></b>}<b>\begin</b>{<b><span style='color:#0095ff;'>pmatrix</span></b>}
<span style='color:#ff5500;'>F_t</span><span style='color:#3daee9;'>\\</span>
<span style='color:#ff5500;'>MK_t</span>
<b>\end</b>{<b><span style='color:#0095ff;'>pmatrix</span></b>}<span style='color:#644a9b;'>\\</span>
&amp;=&amp;<span style='color:#644a9b;'>\frac</span>{1}{0.2}<b>\begin</b>{<b><span style='color:#0095ff;'>pmatrix</span></b>}
<span style='color:#ff5500;'>F_t+MK_t</span><span style='color:#3daee9;'>\\</span>
<span style='color:#ff5500;'>0.2MK_t</span>
<b>\end</b>{<b><span style='color:#0095ff;'>pmatrix</span></b>}<span style='color:#644a9b;'>\\</span>
&amp;=&amp;5<b>\begin</b>{<b><span style='color:#0095ff;'>pmatrix</span></b>}
<span style='color:#ff5500;'>I_t</span><span style='color:#3daee9;'>\\</span>
<span style='color:#ff5500;'>0.2MK_t</span>
<b>\end</b>{<b><span style='color:#0095ff;'>pmatrix</span></b>}
<span style='color:#644a9b;'>\fes</span>
Esta relación indica que los trimestres entre 1993 y 2018 mantienen una relación de incremento de <span style='color:#ff5500;'>$5=</span><span style='color:#3daee9;'>\frac</span><span style='color:#ff5500;'>{1}{0.2}$</span> por 1 entre inversión  <span style='color:#ff5500;'>$Y_t$</span> e <span style='color:#ff5500;'>$I_t$</span>, respectivamente. Además, en este proceso multiplicador, cada unidad de incremento en la formación bruta de capital <span style='color:#ff5500;'>$F_t$</span> pesa lo mismo que cada unidad de incremento en la formación bruta de capital de mercantivas importadas <span style='color:#ff5500;'>$MK_t$</span>. Así mismo, el monto de importaciones <span style='color:#ff5500;'>$M_t$</span> es uno a uno con respecto a la formación bruta de mercancías importadas <span style='color:#ff5500;'>$MK_t$</span>, lo que implica que el consumo intermedio y final de mercancías importadas <span style='color:#ff5500;'>$MC_t$</span> no tiene impacto sobre las importaciones totales <span style='color:#ff5500;'>$M_t$</span>.<span style='color:#644a9b;'>\-</span>

El análisis económico tradicional tiene su propia especificación de la matriz del modelo contable para el periodo, la cual se ve debajo
<span style='color:#644a9b;'>\ies\footnotesize</span>
<b>\begin</b>{<b><span style='color:#0095ff;'>pmatrix</span></b>}
<span style='color:#ff5500;'>c_t&amp;x_t</span><span style='color:#3daee9;'>\\</span>
<span style='color:#ff5500;'>m_t&amp;0</span>
<b>\end</b>{<b><span style='color:#0095ff;'>pmatrix</span></b>}=
<b>\begin</b>{<b><span style='color:#0095ff;'>pmatrix</span></b>}
<span style='color:#ff5500;'>0.6&amp;1</span><span style='color:#3daee9;'>\\</span>
<span style='color:#ff5500;'>0.1&amp;0</span>
<b>\end</b>{<b><span style='color:#0095ff;'>pmatrix</span></b>}
<span style='color:#644a9b;'>\fes</span>
y si la expresamos como un modelo de entrada-salida tendriamos la siguiente forma
<span style='color:#644a9b;'>\ies\footnotesize</span> 
<span style='color:#644a9b;'>\frac</span>{1}{<span style='color:#644a9b;'>\abs</span>{A}}<b>\begin</b>{<b><span style='color:#0095ff;'>pmatrix</span></b>}<span style='color:#ff5500;'> </span>
<span style='color:#ff5500;'>1&amp;x_t</span><span style='color:#3daee9;'>\\</span>
<span style='color:#ff5500;'>m_t&amp;1-c_t</span>
<b>\end</b>{<b><span style='color:#0095ff;'>pmatrix</span></b>}=<span style='color:#644a9b;'>\frac</span>{1}{0.3}<b>\begin</b>{<b><span style='color:#0095ff;'>pmatrix</span></b>}<span style='color:#ff5500;'> </span>
<span style='color:#ff5500;'>1&amp;1</span><span style='color:#3daee9;'>\\</span>
<span style='color:#ff5500;'>0.1&amp;0.4</span>
<b>\end</b>{<b><span style='color:#0095ff;'>pmatrix</span></b>}
<span style='color:#644a9b;'>\fes</span>
Observe que esta matriz y la obtenida mediante la TR,  <span style='color:#ff5500;'>$</span><span style='color:#3daee9;'>\footnotesize\frac</span><span style='color:#ff5500;'>{1}{0.2}</span><b>\begin</b>{<b><span style='color:#0095ff;'>pmatrix</span></b>}<span style='color:#ff5500;'> </span>
<span style='color:#ff5500;'>1&amp;1</span><span style='color:#3daee9;'>\\</span>
<span style='color:#ff5500;'>0&amp;0.2</span>
<span style='color:#ff5500;'> </span><b>\end</b>{<b><span style='color:#0095ff;'>pmatrix</span></b>}<span style='color:#ff5500;'>$</span>, tienen una gran similitud estructural.<span style='color:#644a9b;'>\-</span>
 
 La estructura subyacente de un sistema lineal e invariante bajo la TR está dada por <span style='color:#ff5500;'>$F$</span>, <span style='color:#ff5500;'>$G$</span> y <span style='color:#ff5500;'>$H$</span>, matrices invariantes. Entonces, el cambio estructural viene influenciado por la variación en los parámetros, la forma en que son  transformados en la descripción externa, es decir, en la manera de ser asociados y dispuestos en cada entrada de
<span style='color:#644a9b;'>\ies</span> <span style='color:#644a9b;'>\footnotesize</span>
<span style='color:#644a9b;'>\frac</span>{1}{<span style='color:#644a9b;'>\abs</span>{A}}<b>\begin</b>{<b><span style='color:#0095ff;'>pmatrix</span></b>}<span style='color:#ff5500;'> </span>
<span style='color:#ff5500;'>1&amp;x_t</span><span style='color:#3daee9;'>\\</span>
<span style='color:#ff5500;'>m_t&amp;1-c_t</span>
<b>\end</b>{<b><span style='color:#0095ff;'>pmatrix</span></b>}
<span style='color:#644a9b;'>\fes</span>
y por la relación de igualdad entre esta sucesión de matrices y el sistema subyacente a través de 
<span style='color:#644a9b;'>\ies</span> <span style='color:#644a9b;'>\footnotesize</span>
<b>\begin</b>{<b><span style='color:#0095ff;'>pmatrix</span></b>}<span style='color:#ff5500;'> </span>
<span style='color:#ff5500;'>h_{11}(t)&amp;h_{12}(t)</span><span style='color:#3daee9;'>\\</span>
<span style='color:#ff5500;'>h_{21}(t)&amp;h_{22}(t)</span>
<b>\end</b>{<b><span style='color:#0095ff;'>pmatrix</span></b>}
<b>\begin</b>{<b><span style='color:#0095ff;'>pmatrix</span></b>}<span style='color:#ff5500;'> </span>
<span style='color:#ff5500;'>f_{11}(t)&amp;f_{12}(t)</span><span style='color:#3daee9;'>\\</span>
<span style='color:#ff5500;'>f_{21}(t)&amp;f_{22}(t)</span>
<b>\end</b>{<b><span style='color:#0095ff;'>pmatrix</span></b>}^{t-1}<b>\begin</b>{<b><span style='color:#0095ff;'>pmatrix</span></b>}<span style='color:#ff5500;'> </span>
<span style='color:#ff5500;'>g_{11}(t)&amp;g_{12}(t)</span><span style='color:#3daee9;'>\\</span>
<span style='color:#ff5500;'>g_{21}(t)&amp;g_{22}(t)</span>
<b>\end</b>{<b><span style='color:#0095ff;'>pmatrix</span></b>}
<span style='color:#644a9b;'>\fes</span>
especificamente, por los valores de la matriz <span style='color:#ff5500;'>$F$</span>, pues estos, en conjunto, al ser elevados a la potencia <span style='color:#ff5500;'>$t-1$</span>, pueden mostrar un crecimiento que en ciertos cortes de tiempo genere un diferencial amplio.<span style='color:#644a9b;'>\-</span>

El modelo contable especificado remite directamente al núcleo básico de una economía abierta y permite una extensión que en varios sentidos pone a prueba a la teoría de la realización mediante:
<b>\begin</b>{<b><span style='color:#0095ff;'>enumerate</span></b>}[i)]
<span style='color:#644a9b;'>\item</span> adicionar otros sectores institucionales como el gobierno y la autoridad monetaria;
<span style='color:#644a9b;'>\item</span> especificar el correspondiente modelo determinado por la oferta de ahorro en lugar de la demanda de inversión:
<span style='color:#644a9b;'>\ies\footnotesize</span>
<b>\begin</b>{<b><span style='color:#0095ff;'>pmatrix</span></b>}
<span style='color:#ff5500;'>Y_t</span><span style='color:#3daee9;'>\\</span><span style='color:#ff5500;'>M_t</span>
<b>\end</b>{<b><span style='color:#0095ff;'>pmatrix</span></b>}=<b>\begin</b>{<b><span style='color:#0095ff;'>pmatrix</span></b>}
<span style='color:#ff5500;'>c_t&amp;mc_t</span><span style='color:#3daee9;'>\\</span>
<span style='color:#ff5500;'>a_{xt}&amp;0</span>
<b>\end</b>{<b><span style='color:#0095ff;'>pmatrix</span></b>}<b>\begin</b>{<b><span style='color:#0095ff;'>pmatrix</span></b>}
<span style='color:#ff5500;'>Y_t</span><span style='color:#3daee9;'>\\</span><span style='color:#ff5500;'>M_t</span>
<b>\end</b>{<b><span style='color:#0095ff;'>pmatrix</span></b>}+<b>\begin</b>{<b><span style='color:#0095ff;'>pmatrix</span></b>}
<span style='color:#ff5500;'>Sr_t</span><span style='color:#3daee9;'>\\</span><span style='color:#ff5500;'>Snr_t</span>
<b>\end</b>{<b><span style='color:#0095ff;'>pmatrix</span></b>}
<span style='color:#644a9b;'>\fes</span>
donde
<span style='color:#644a9b;'>\ies\footnotesize</span>
<b>\begin</b>{<b><span style='color:#0095ff;'>pmatrix</span></b>}
<span style='color:#ff5500;'>c_t&amp;mc_t</span><span style='color:#3daee9;'>\\</span>
<span style='color:#ff5500;'>a_{xt}&amp;0</span>
<b>\end</b>{<b><span style='color:#0095ff;'>pmatrix</span></b>}=<b>\begin</b>{<b><span style='color:#0095ff;'>pmatrix</span></b>}
<span style='color:#ff5500;'>C_t&amp;MC_t</span><span style='color:#3daee9;'>\\</span><span style='color:#ff5500;'>  </span>
<span style='color:#ff5500;'>X_t&amp;0</span>
<b>\end</b>{<b><span style='color:#0095ff;'>pmatrix</span></b>}<b>\begin</b>{<b><span style='color:#0095ff;'>pmatrix</span></b>}
<span style='color:#ff5500;'>Y^{-1}_t&amp;0</span><span style='color:#3daee9;'>\\</span>
<span style='color:#ff5500;'>0&amp;M^{-1}_t</span><span style='color:#3daee9;'>\\</span>
<b>\end</b>{<b><span style='color:#0095ff;'>pmatrix</span></b>}
<span style='color:#644a9b;'>\fes</span>
y con lo cuál se obtiene
<span style='color:#644a9b;'>\ies\footnotesize</span>
<b>\begin</b>{<b><span style='color:#0095ff;'>pmatrix</span></b>}
<span style='color:#ff5500;'>Y_t</span><span style='color:#3daee9;'>\\</span><span style='color:#ff5500;'>M_t</span>
<b>\end</b>{<b><span style='color:#0095ff;'>pmatrix</span></b>}=<span style='color:#644a9b;'>\left</span>[<b>\begin</b>{<b><span style='color:#0095ff;'>pmatrix</span></b>}
<span style='color:#ff5500;'>1&amp;0</span><span style='color:#3daee9;'>\\</span>
<span style='color:#ff5500;'>0&amp;1</span>
<b>\end</b>{<b><span style='color:#0095ff;'>pmatrix</span></b>}-<b>\begin</b>{<b><span style='color:#0095ff;'>pmatrix</span></b>}
<span style='color:#ff5500;'>c_t&amp;mc_t</span><span style='color:#3daee9;'>\\</span>
<span style='color:#ff5500;'>a_{xt}&amp;0</span>
<b>\end</b>{<b><span style='color:#0095ff;'>pmatrix</span></b>}<span style='color:#644a9b;'>\right</span>]^{-1}<b>\begin</b>{<b><span style='color:#0095ff;'>pmatrix</span></b>}
<span style='color:#ff5500;'>Sr_t</span><span style='color:#3daee9;'>\\</span><span style='color:#ff5500;'>Snr_t</span>
<b>\end</b>{<b><span style='color:#0095ff;'>pmatrix</span></b>}
<span style='color:#644a9b;'>\fes</span>
<span style='color:#644a9b;'>\item</span> aumentar la frecuencia de la información disponible en virtud de su consistencia contable y alta confiabilidad.
<b>\end</b>{<b><span style='color:#0095ff;'>enumerate</span></b>}<span style='color:#644a9b;'>\-</span>

En este marco es interesante pensar de qué dependen los elementos de <span style='color:#ff5500;'>$F$</span> y <span style='color:#ff5500;'>$G$</span>, que pueden cambiar, por ejemplo, el multiplicador <span style='color:#ff5500;'>$</span><span style='color:#3daee9;'>\frac</span><span style='color:#ff5500;'>{1}{</span><span style='color:#3daee9;'>\abs</span><span style='color:#ff5500;'>{A_t}}$</span>.<span style='color:#644a9b;'>\-</span>

Considerando que la matriz <span style='color:#ff5500;'>$F$</span> tenga 2 valores propios distintos, podemos descomponerla como 
<span style='color:#644a9b;'>\ies</span>
F=PDP^{-1}
<span style='color:#644a9b;'>\fes</span>
donde <span style='color:#ff5500;'>$P$</span> tiene por columnas los vectores propios de <span style='color:#ff5500;'>$F$</span>, es decir, <span style='color:#ff5500;'>$v_1=</span><b>\begin</b>{<b><span style='color:#0095ff;'>pmatrix</span></b>}
<span style='color:#ff5500;'>v_{11}&amp;</span>
<span style='color:#ff5500;'>v_{12}</span>
<b>\end</b>{<b><span style='color:#0095ff;'>pmatrix</span></b>}<span style='color:#ff5500;'>'$</span> y <span style='color:#ff5500;'>$v_2=</span><b>\begin</b>{<b><span style='color:#0095ff;'>pmatrix</span></b>}
<span style='color:#ff5500;'>v_{21}&amp;</span>
<span style='color:#ff5500;'>v_{22}</span>
<b>\end</b>{<b><span style='color:#0095ff;'>pmatrix</span></b>}<span style='color:#ff5500;'>'$</span>; <span style='color:#ff5500;'>$P^{-1}$</span> tiene por columnas los vectores <span style='color:#ff5500;'>$</span><span style='color:#3daee9;'>\frac</span><span style='color:#ff5500;'>{1}{</span><span style='color:#3daee9;'>\det</span><span style='color:#ff5500;'>(P)}</span><b>\begin</b>{<b><span style='color:#0095ff;'>pmatrix</span></b>}
<span style='color:#ff5500;'>   v_{22}&amp;</span>
<span style='color:#ff5500;'>   -v_{12}</span>
<span style='color:#ff5500;'>  </span><b>\end</b>{<b><span style='color:#0095ff;'>pmatrix</span></b>}<span style='color:#ff5500;'>'$</span> y <span style='color:#ff5500;'>$</span><span style='color:#3daee9;'>\frac</span><span style='color:#ff5500;'>{1}{</span><span style='color:#3daee9;'>\det</span><span style='color:#ff5500;'>(P)}</span><b>\begin</b>{<b><span style='color:#0095ff;'>pmatrix</span></b>}
<span style='color:#ff5500;'>   -v_{21}&amp;</span>
<span style='color:#ff5500;'>   v_{11}</span>
<span style='color:#ff5500;'>  </span><b>\end</b>{<b><span style='color:#0095ff;'>pmatrix</span></b>}<span style='color:#ff5500;'>'$</span>; y <span style='color:#ff5500;'>$D$</span> es una matriz diagonal con <span style='color:#ff5500;'>$d_1$</span> y <span style='color:#ff5500;'>$d_2$</span> en su diagonal principal.<span style='color:#644a9b;'>\-</span>
  
  De esta forma, se sigue que 
<span style='color:#644a9b;'>\ies</span> 
<span style='color:#644a9b;'>\frac</span>{1}{<span style='color:#644a9b;'>\abs</span>{A_t}}=a_1d_1^t+a_2d_2^t
<span style='color:#644a9b;'>\fes</span>
donde 
<span style='color:#644a9b;'>\ies</span>
a_1&amp;=&amp;<span style='color:#644a9b;'>\frac</span>{v_{11}v_{22}g_{11}-v_{11}v_{21}g_{21}}{<span style='color:#644a9b;'>\det</span>(P)}<span style='color:#644a9b;'>\\</span> 
a_2&amp;=&amp;<span style='color:#644a9b;'>\frac</span>{v_{11}v_{21}g_{21}-v_{12}v_{21}g_{21}}{<span style='color:#644a9b;'>\det</span>(P)}
<span style='color:#644a9b;'>\fes</span>


<b>\end</b>{<b><span style='color:#0095ff;'>document</span></b>}

<span style='color:#898887;'>%///////////////////////////////////////////////////////////////////////////////////////////////////////////////////////////////////////</span>
<span style='color:#898887;'>%///////////////////////////////////////////////////////////////////////////////////////////////////////////////////////////////////////</span>
<b>\chapter</b>{<b>Aspecto sobre el proceso de realización y cambio estructural</b>}
<span style='color:#898887;'>%///////////////////////////////////////////////////////////////////////////////////////////////////////////////////////////////////////</span>
<span style='color:#898887;'>%///////////////////////////////////////////////////////////////////////////////////////////////////////////////////////////////////////</span>

Aleatoriedad, dimensión y realización

Multiples realizaciones y selección óptima

Algorimo de factorización, estructura de Hankel y caminata del sistema

Conservación de la estructura de Hankel 

Modelo invarinte y variante


Estructura:

En una descomposición SVD de la matriz <span style='color:#ff5500;'>$X$</span> que venimos manejando, como <span style='color:#ff5500;'>$</span><span style='color:#3daee9;'>\S</span><span style='color:#ff5500;'>_r</span><span style='color:#3daee9;'>\in\R</span><span style='color:#ff5500;'>^{r</span><span style='color:#3daee9;'>\times</span><span style='color:#ff5500;'> r}$</span> es diagonal, tenemos que
<span style='color:#644a9b;'>\ies</span> 
<span style='color:#644a9b;'>\S</span>_r^k=<b>\begin</b>{<b><span style='color:#0095ff;'>pmatrix</span></b>}
<span style='color:#ff5500;'> </span><span style='color:#3daee9;'>\s</span><span style='color:#ff5500;'>_1^k&amp;0&amp;</span><span style='color:#3daee9;'>\cdots</span><span style='color:#ff5500;'>&amp;0</span><span style='color:#3daee9;'>\\</span>
<span style='color:#ff5500;'> 0&amp;</span><span style='color:#3daee9;'>\s</span><span style='color:#ff5500;'>_2^k&amp;</span><span style='color:#3daee9;'>\ddots</span><span style='color:#ff5500;'>&amp;</span><span style='color:#3daee9;'>\vdots\\</span>
<span style='color:#ff5500;'> </span><span style='color:#3daee9;'>\vdots</span><span style='color:#ff5500;'>&amp;</span><span style='color:#3daee9;'>\ddots</span><span style='color:#ff5500;'>&amp;</span><span style='color:#3daee9;'>\ddots</span><span style='color:#ff5500;'>&amp;0</span><span style='color:#3daee9;'>\\</span>
<span style='color:#ff5500;'> 0&amp;</span><span style='color:#3daee9;'>\cdots</span><span style='color:#ff5500;'>&amp;0&amp;</span><span style='color:#3daee9;'>\s</span><span style='color:#ff5500;'>_r^k</span>
<span style='color:#ff5500;'> </span><b>\end</b>{<b><span style='color:#0095ff;'>pmatrix</span></b>}
<span style='color:#644a9b;'>\fes</span>
En consecuencia, podemos expresar <span style='color:#ff5500;'>$X$</span> de la siguiente manera
<span style='color:#644a9b;'>\ies</span>
X=U_r<span style='color:#644a9b;'>\S</span>^<span style='color:#644a9b;'>\a</span>_r<span style='color:#644a9b;'>\S</span>^{1-<span style='color:#644a9b;'>\a</span>}_r V_r'
<span style='color:#644a9b;'>\fes</span>

<span style='color:#644a9b;'>\-</span>
   

A pesar que <b>\citet</b>{<b><span style='color:#0095ff;'>eckart</span></b>} y <b>\citet</b>{<b><span style='color:#0095ff;'>mirsky</span></b>} nos hallan permitido afirmar que <span style='color:#ff5500;'>$X_p$</span> es la matriz más proxima a <span style='color:#ff5500;'>$X$</span> dentro de las matrices de <span style='color:#ff5500;'>$n</span><span style='color:#3daee9;'>\times</span><span style='color:#ff5500;'> m$</span> de rango menor o igual a <span style='color:#ff5500;'>$p$</span>, al  truncar <span style='color:#ff5500;'>$</span><span style='color:#3daee9;'>\H</span><span style='color:#ff5500;'>_{</span><span style='color:#3daee9;'>\a</span><span style='color:#ff5500;'>,</span><span style='color:#3daee9;'>\b</span><span style='color:#ff5500;'>}$</span> en el proceso de realización, la matriz reducida que se obtiene pierde, en general, su estructura de bloques antidiagonales homogéneos. Sin embargo, debido a <b>\citet</b>{<b><span style='color:#0095ff;'>gohost</span></b>}, existe una manera de preservar  parte de la estructura de la MH original. Si <span style='color:#ff5500;'>$X$</span> en vez de ser pensada como una matriz completa, es asumida como una matriz compuesta por dos matrices, es decir, <span style='color:#ff5500;'>$X=(A_1</span><span style='color:#3daee9;'>\ \ </span><span style='color:#ff5500;'>A_2)$</span> <span style='color:#644a9b;'>\-</span>     
 

Métodos para la selección de la matriz reducida óptima...




<span style='color:#644a9b;'>\vspace</span>{10cm}

<b>\section</b>{<b>Realización, valores singulares y valores propios</b>}

La realización <span style='color:#ff5500;'>$(F,G,H)$</span> de una sucesión de matrices <span style='color:#ff5500;'>$</span><span style='color:#3daee9;'>\{</span><span style='color:#ff5500;'>L_t</span><span style='color:#3daee9;'>\}</span><span style='color:#ff5500;'>$</span>, <span style='color:#ff5500;'>$L_t</span><span style='color:#3daee9;'>\in\R</span><span style='color:#ff5500;'>^{</span><span style='color:#3daee9;'>\mu\times\eta</span><span style='color:#ff5500;'>}$</span>, permite que <span style='color:#ff5500;'>$L_t</span><span style='color:#3daee9;'>\approx</span><span style='color:#ff5500;'> HF^{t-1}G$</span>. Para esto, se vio que debemos acomodar los elementos <span style='color:#ff5500;'>$</span><span style='color:#3daee9;'>\{</span><span style='color:#ff5500;'>L_t</span><span style='color:#3daee9;'>\}</span><span style='color:#ff5500;'>$</span> en una matriz <span style='color:#ff5500;'>$</span><span style='color:#3daee9;'>\mathscr</span><span style='color:#ff5500;'>{H}$</span> con estructura de Hankel y de rango <span style='color:#ff5500;'>$n$</span>. Una vez hecho, se descompone la matriz <span style='color:#ff5500;'>$</span><span style='color:#3daee9;'>\mathscr</span><span style='color:#ff5500;'>{H}$</span> en valores <span style='color:#ff5500;'>$</span><span style='color:#3daee9;'>\sigma</span><span style='color:#ff5500;'>_i$</span> y vectores <span style='color:#ff5500;'>$</span><span style='color:#3daee9;'>\vec</span><span style='color:#ff5500;'>{v}_i$</span>, <span style='color:#ff5500;'>$</span><span style='color:#3daee9;'>\vec</span><span style='color:#ff5500;'>{u}_i$</span> singulares, lo cual permite que 
<span style='color:#644a9b;'>\ies</span>
<span style='color:#644a9b;'>\mathscr</span>{H}=<span style='color:#644a9b;'>\sum</span>_{i=1}^n<span style='color:#644a9b;'>\sigma</span>_i<span style='color:#644a9b;'>\vec</span>{u}_i<span style='color:#644a9b;'>\vec</span>{v}_i^T=<span style='color:#644a9b;'>\sum</span>_{i=1}^n<span style='color:#644a9b;'>\sigma</span>_i
<b>\begin</b>{<b><span style='color:#0095ff;'>pmatrix</span></b>}
<span style='color:#ff5500;'>u_{i1}v_{i1}&amp;</span><span style='color:#3daee9;'>\cdots</span><span style='color:#ff5500;'>&amp; u_{i1}v_{i(</span><span style='color:#3daee9;'>\a\eta</span><span style='color:#ff5500;'>)}</span><span style='color:#3daee9;'>\\</span>
<span style='color:#3daee9;'>\vdots</span><span style='color:#ff5500;'>&amp;</span><span style='color:#3daee9;'>\ddots</span><span style='color:#ff5500;'>&amp;</span><span style='color:#3daee9;'>\vdots\\</span>
<span style='color:#ff5500;'>u_{i(</span><span style='color:#3daee9;'>\b\mu</span><span style='color:#ff5500;'>)}v_{i1}&amp;</span><span style='color:#3daee9;'>\cdots</span><span style='color:#ff5500;'>&amp; u_{i(</span><span style='color:#3daee9;'>\b\mu</span><span style='color:#ff5500;'>)}v_{i(</span><span style='color:#3daee9;'>\a\eta</span><span style='color:#ff5500;'>)}</span>
<b>\end</b>{<b><span style='color:#0095ff;'>pmatrix</span></b>}
<span style='color:#644a9b;'>\fes</span>
es decir, podemos ver a la matriz <span style='color:#ff5500;'>$</span><span style='color:#3daee9;'>\mathscr</span><span style='color:#ff5500;'>{H}$</span> como una combinación lineal de las matrices generadas por los vectores singulares y cuyos coeficientes son los valores singulares.<span style='color:#644a9b;'>\-</span>


Continuando con el proceso, se debe truncar la matriz <span style='color:#ff5500;'>$</span><span style='color:#3daee9;'>\mathscr</span><span style='color:#ff5500;'>{H}$</span>. Esto se logra al ir q´uitanto valores singulares <span style='color:#ff5500;'>$</span><span style='color:#3daee9;'>\sigma</span><span style='color:#ff5500;'>_i$</span>, lo cual redunda en un ajuste   la suma izquierda hasta <span style='color:#ff5500;'>$r$</span>, obtenemos la matriz <span style='color:#ff5500;'>$</span><span style='color:#3daee9;'>\mathscr</span><span style='color:#ff5500;'>{H}_r</span><span style='color:#3daee9;'>\approx</span><span style='color:#ff5500;'> </span><span style='color:#3daee9;'>\mathscr</span><span style='color:#ff5500;'>{H}$</span>,
<span style='color:#644a9b;'>\ies</span>
<span style='color:#644a9b;'>\mathscr</span>{H}_r=<span style='color:#644a9b;'>\sum</span>_{i=1}^r<span style='color:#644a9b;'>\sigma</span>_i<span style='color:#644a9b;'>\vec</span>{u}_i<span style='color:#644a9b;'>\vec</span>{v}_i^T
<span style='color:#644a9b;'>\fes</span>
Además, si descomponemos <span style='color:#ff5500;'>$</span><span style='color:#3daee9;'>\mathscr</span><span style='color:#ff5500;'>{H}_r$</span> bajo el Algoritmo de Factorización de la Definición B.1 en el Apéndice, obtenemos la relación
<span style='color:#644a9b;'>\ies</span>
<span style='color:#644a9b;'>\mathscr</span>{H}-PQ=<span style='color:#644a9b;'>\mathscr</span>{H}-<span style='color:#644a9b;'>\mathscr</span>{H}_r=<span style='color:#644a9b;'>\sum</span>_{i=r+1}^n<span style='color:#644a9b;'>\sigma</span>_i<span style='color:#644a9b;'>\vec</span>{u}_i<span style='color:#644a9b;'>\vec</span>{v}_i^T
<span style='color:#644a9b;'>\fes</span>
por consiguiente, se tiene que
<span style='color:#644a9b;'>\ie\label</span>{e1rvvs}
<span style='color:#644a9b;'>\mathscr</span>{H}-<span style='color:#644a9b;'>\sum</span>_{i=r+1}^n<span style='color:#644a9b;'>\sigma</span>_i<span style='color:#644a9b;'>\vec</span>{u}_i<span style='color:#644a9b;'>\vec</span>{v}_i^T=PQ
<span style='color:#644a9b;'>\fe</span>
Luego, debido a la Definición B.2 y al Teorema de Realización Parcial Mínima, sabemos que la realización dada por <span style='color:#ff5500;'>$(F_r,G_r,H_r)$</span>, está definida como
<span style='color:#644a9b;'>\ies</span>
F_r&amp;=&amp;(P_{1:r})^{-1}P_{<span style='color:#644a9b;'>\mu+</span>1:<span style='color:#644a9b;'>\mu+</span>r,1:r}
<span style='color:#644a9b;'>\\</span>
G_r&amp;=&amp;Q_{1:r ,1:<span style='color:#644a9b;'>\eta</span>}
<span style='color:#644a9b;'>\\</span>
H_r&amp;=&amp;P_{1:<span style='color:#644a9b;'>\mu</span>,1:r}
<span style='color:#644a9b;'>\fes</span>
por consiguiente, al emplear <b>\eqref</b>{<b><span style='color:#0095ff;'>e1rvvs</span></b>} con <span style='color:#ff5500;'>$r</span><span style='color:#3daee9;'>\geq\mu</span><span style='color:#ff5500;'>$</span> (condición del Teorema de Realización Parcial Mínima) y dado que <span style='color:#ff5500;'>$P$</span> es una matriz triangular inferior, se tiene
<span style='color:#644a9b;'>\ies</span>
L_1-<span style='color:#644a9b;'>\sum</span>_{i=1}^r<span style='color:#644a9b;'>\sigma</span>_i(<span style='color:#644a9b;'>\vec</span>{u}_{1:<span style='color:#644a9b;'>\mu</span>})_i(<span style='color:#644a9b;'>\vec</span>{v}_{1:<span style='color:#644a9b;'>\mu</span>})_i^T=H_rG_r=H_<span style='color:#644a9b;'>\mu</span> G_<span style='color:#644a9b;'>\mu</span>
<span style='color:#644a9b;'>\fes</span>
 donde la última igualdad estriba en que las columnas de <span style='color:#ff5500;'>$H_r$</span> posteriores a <span style='color:#ff5500;'>$</span><span style='color:#3daee9;'>\mu</span><span style='color:#ff5500;'>$</span> son nulas, por definición y que <span style='color:#ff5500;'>$P$</span> es triangular inferior con diagonal principal de unos. Además, aunque el orden de <span style='color:#ff5500;'>$H_</span><span style='color:#3daee9;'>\mu</span><span style='color:#ff5500;'>$</span> y <span style='color:#ff5500;'>$G_</span><span style='color:#3daee9;'>\mu</span><span style='color:#ff5500;'>$</span> no depende de <span style='color:#ff5500;'>$r$</span>, sus entradas sí.<span style='color:#644a9b;'>\-</span>

 Por otro lado, de la premultiplicación de <span style='color:#ff5500;'>$F_r$</span> por <span style='color:#ff5500;'>$G_r$</span> observamos que, 
<span style='color:#644a9b;'>\ies</span>
F_rG_r&amp;=&amp;   (P_{1:r})^{-1}P_{<span style='color:#644a9b;'>\mu+</span>1:<span style='color:#644a9b;'>\mu+</span>r,1:r}G_r
=
(P_{1:r})^{-1}
<b>\begin</b>{<b><span style='color:#0095ff;'>pmatrix</span></b>}
<span style='color:#ff5500;'>L_2</span><span style='color:#3daee9;'>\\</span>
<span style='color:#ff5500;'>L^{2</span><span style='color:#3daee9;'>\mu</span><span style='color:#ff5500;'>+1}_{</span><span style='color:#3daee9;'>\mu</span><span style='color:#ff5500;'>+r}</span>
<b>\end</b>{<b><span style='color:#0095ff;'>pmatrix</span></b>}
<span style='color:#644a9b;'>\\</span>
&amp;=&amp;<b>\begin</b>{<b><span style='color:#0095ff;'>pmatrix</span></b>}
<span style='color:#ff5500;'>   P_{1:</span><span style='color:#3daee9;'>\mu</span><span style='color:#ff5500;'>}&amp;0_M</span><span style='color:#3daee9;'>\\</span>
<span style='color:#ff5500;'>   P_{21}&amp;P_{22}</span>
<span style='color:#ff5500;'>  </span><b>\end</b>{<b><span style='color:#0095ff;'>pmatrix</span></b>}^{-1}<b>\begin</b>{<b><span style='color:#0095ff;'>pmatrix</span></b>}
<span style='color:#ff5500;'>L_2</span><span style='color:#3daee9;'>\\</span>
<span style='color:#ff5500;'>L^{2</span><span style='color:#3daee9;'>\mu</span><span style='color:#ff5500;'>+1}_{</span><span style='color:#3daee9;'>\mu</span><span style='color:#ff5500;'>+r} </span>
<b>\end</b>{<b><span style='color:#0095ff;'>pmatrix</span></b>}
=
<b>\begin</b>{<b><span style='color:#0095ff;'>pmatrix</span></b>}
<span style='color:#ff5500;'>   (P_{1:</span><span style='color:#3daee9;'>\mu</span><span style='color:#ff5500;'>})^{-1}&amp;0_M</span><span style='color:#3daee9;'>\\</span>
<span style='color:#ff5500;'>   -P^{-1}_{22}P_{21}(P_{1:</span><span style='color:#3daee9;'>\mu</span><span style='color:#ff5500;'>})^{-1} &amp;P^{-1}_{22}</span>
<span style='color:#ff5500;'>  </span><b>\end</b>{<b><span style='color:#0095ff;'>pmatrix</span></b>}<b>\begin</b>{<b><span style='color:#0095ff;'>pmatrix</span></b>}
<span style='color:#ff5500;'>L_2</span><span style='color:#3daee9;'>\\</span><span style='color:#ff5500;'> </span>
<span style='color:#ff5500;'>L^{2</span><span style='color:#3daee9;'>\mu</span><span style='color:#ff5500;'>+1}_{</span><span style='color:#3daee9;'>\mu</span><span style='color:#ff5500;'>+r} </span>
<b>\end</b>{<b><span style='color:#0095ff;'>pmatrix</span></b>}<span style='color:#644a9b;'>\\</span>
&amp;=&amp;
<b>\begin</b>{<b><span style='color:#0095ff;'>pmatrix</span></b>}
<span style='color:#ff5500;'>(P_{1:</span><span style='color:#3daee9;'>\mu</span><span style='color:#ff5500;'>})^{-1}L_2</span><span style='color:#3daee9;'>\\</span>
<span style='color:#ff5500;'>-P^{-1}_{22}P_{21}(P_{1:</span><span style='color:#3daee9;'>\mu</span><span style='color:#ff5500;'>})^{-1}L_2+P^{-1}_{22}L^{2</span><span style='color:#3daee9;'>\mu</span><span style='color:#ff5500;'>+1}_{</span><span style='color:#3daee9;'>\mu</span><span style='color:#ff5500;'>+r} </span>
<b>\end</b>{<b><span style='color:#0095ff;'>pmatrix</span></b>}<span style='color:#644a9b;'>\\</span>
&amp;=&amp;
<b>\begin</b>{<b><span style='color:#0095ff;'>pmatrix</span></b>}
<span style='color:#ff5500;'>(P_{1:</span><span style='color:#3daee9;'>\mu</span><span style='color:#ff5500;'>})^{-1}P_{</span><span style='color:#3daee9;'>\mu</span><span style='color:#ff5500;'>+1:2</span><span style='color:#3daee9;'>\mu</span><span style='color:#ff5500;'>,1:r}G_r</span><span style='color:#3daee9;'>\\</span>
<span style='color:#ff5500;'>-P^{-1}_{22}P_{21}(P_{1:</span><span style='color:#3daee9;'>\mu</span><span style='color:#ff5500;'>})^{-1}L_2+P^{-1}_{22}L^{2</span><span style='color:#3daee9;'>\mu</span><span style='color:#ff5500;'>+1}_{</span><span style='color:#3daee9;'>\mu</span><span style='color:#ff5500;'>+r} </span>
<b>\end</b>{<b><span style='color:#0095ff;'>pmatrix</span></b>}
<span style='color:#644a9b;'>\fes</span>
y debido a lo que se comentó sobre las columnas nulas de <span style='color:#ff5500;'>$H_r$</span> posteriores a  <span style='color:#ff5500;'>$</span><span style='color:#3daee9;'>\mu</span><span style='color:#ff5500;'>$</span>, y tomando <span style='color:#ff5500;'>$H=H_</span><span style='color:#3daee9;'>\mu</span><span style='color:#ff5500;'>$</span>, entonces tenemos que
<span style='color:#644a9b;'>\ies</span>
H_rF_rG_r=<b>\begin</b>{<b><span style='color:#0095ff;'>pmatrix</span></b>}
<span style='color:#ff5500;'>           H&amp;0_M</span>
<span style='color:#ff5500;'>          </span><b>\end</b>{<b><span style='color:#0095ff;'>pmatrix</span></b>}
<b>\begin</b>{<b><span style='color:#0095ff;'>pmatrix</span></b>}
<span style='color:#ff5500;'>(P_{1:</span><span style='color:#3daee9;'>\mu</span><span style='color:#ff5500;'>})^{-1}P_{</span><span style='color:#3daee9;'>\mu</span><span style='color:#ff5500;'>+1:2</span><span style='color:#3daee9;'>\mu</span><span style='color:#ff5500;'>,1:r}G_r</span><span style='color:#3daee9;'>\\</span>
<span style='color:#ff5500;'>-P^{-1}_{22}P_{21}(P_{1:</span><span style='color:#3daee9;'>\mu</span><span style='color:#ff5500;'>})^{-1}L_2+P^{-1}_{22}L^{2</span><span style='color:#3daee9;'>\mu</span><span style='color:#ff5500;'>+1}_{</span><span style='color:#3daee9;'>\mu</span><span style='color:#ff5500;'>+r} </span>
<b>\end</b>{<b><span style='color:#0095ff;'>pmatrix</span></b>}=H(P_{1:<span style='color:#644a9b;'>\mu</span>})^{-1}P_{<span style='color:#644a9b;'>\mu+</span>1:2<span style='color:#644a9b;'>\mu</span>,1:r}G_r
<span style='color:#644a9b;'>\fes</span>
y por consiguiente, como el rango de <span style='color:#ff5500;'>$</span><span style='color:#3daee9;'>\mathscr</span><span style='color:#ff5500;'>{H}_r$</span> es <span style='color:#ff5500;'>$r$</span>,  podemos definir la realización mínima estocástica de una forma más compacta
<span style='color:#644a9b;'>\ies</span>
F&amp;=&amp; (P_{1:<span style='color:#644a9b;'>\mu</span>})^{-1}P_{<span style='color:#644a9b;'>\mu+</span>1:2<span style='color:#644a9b;'>\mu</span>,1:r}<span style='color:#644a9b;'>\\</span>
G&amp;=&amp; Q_{1:r,1:<span style='color:#644a9b;'>\eta</span>}<span style='color:#644a9b;'>\\</span>
H&amp;=&amp; P_{1:<span style='color:#644a9b;'>\mu</span>}
<span style='color:#644a9b;'>\fes</span>
para la cual se cumple que 
<span style='color:#644a9b;'>\ies</span>
H_rF_rG_r=HFG
<span style='color:#644a9b;'>\fes</span>

Así mismo, para <span style='color:#ff5500;'>$r&gt;</span><span style='color:#3daee9;'>\mu</span><span style='color:#ff5500;'>$</span> y tomando <span style='color:#ff5500;'>$P_1=P_{1:</span><span style='color:#3daee9;'>\mu</span><span style='color:#ff5500;'>}$</span>, <span style='color:#ff5500;'>$0_1=0</span><span style='color:#3daee9;'>\in\R</span><span style='color:#ff5500;'>^{</span><span style='color:#3daee9;'>\mu\times</span><span style='color:#ff5500;'>(r-</span><span style='color:#3daee9;'>\mu</span><span style='color:#ff5500;'>)}$</span>, <span style='color:#ff5500;'>$0_2=0</span><span style='color:#3daee9;'>\in\R</span><span style='color:#ff5500;'>^{(r-</span><span style='color:#3daee9;'>\mu</span><span style='color:#ff5500;'>)</span><span style='color:#3daee9;'>\times</span><span style='color:#ff5500;'> </span><span style='color:#3daee9;'>\mu</span><span style='color:#ff5500;'>}$</span>,  <span style='color:#ff5500;'>$P_2=P_{</span><span style='color:#3daee9;'>\mu</span><span style='color:#ff5500;'>+1:r,1:</span><span style='color:#3daee9;'>\mu</span><span style='color:#ff5500;'>}$</span>, <span style='color:#ff5500;'>$P_3=P_{</span><span style='color:#3daee9;'>\mu</span><span style='color:#ff5500;'>+1:r,</span><span style='color:#3daee9;'>\mu</span><span style='color:#ff5500;'>+1:r}$</span>, <span style='color:#ff5500;'>$P_4=P_{1+r:</span><span style='color:#3daee9;'>\mu</span><span style='color:#ff5500;'>+r,1:</span><span style='color:#3daee9;'>\mu</span><span style='color:#ff5500;'>}$</span> y <span style='color:#ff5500;'>$P_5=P_{1+r:</span><span style='color:#3daee9;'>\mu</span><span style='color:#ff5500;'>+r,</span><span style='color:#3daee9;'>\mu</span><span style='color:#ff5500;'>+1:r}$</span>
<span style='color:#644a9b;'>\ies</span>
F^2&amp;=&amp;FF=(P_{r})^{-1}P_{<span style='color:#644a9b;'>\mu+</span>1:<span style='color:#644a9b;'>\mu+</span>r,1:r}(P_{1:r})^{-1}P_{<span style='color:#644a9b;'>\mu+</span>1:<span style='color:#644a9b;'>\mu+</span>r,1:r}<span style='color:#644a9b;'>\\</span>
&amp;=&amp;<b>\begin</b>{<b><span style='color:#0095ff;'>pmatrix</span></b>}
<span style='color:#ff5500;'>   P_1&amp;0_1</span><span style='color:#3daee9;'>\\</span>
<span style='color:#ff5500;'>   P_2&amp;P_3</span>
<span style='color:#ff5500;'>  </span><b>\end</b>{<b><span style='color:#0095ff;'>pmatrix</span></b>}^{-1}
  P_{<span style='color:#644a9b;'>\mu+</span>1:<span style='color:#644a9b;'>\mu+</span>r,1:r}
  <b>\begin</b>{<b><span style='color:#0095ff;'>pmatrix</span></b>}
<span style='color:#ff5500;'>   P_1&amp;0_1</span><span style='color:#3daee9;'>\\</span>
<span style='color:#ff5500;'>   P_2&amp;P_3</span>
<span style='color:#ff5500;'>  </span><b>\end</b>{<b><span style='color:#0095ff;'>pmatrix</span></b>}^{-1}
  P_{<span style='color:#644a9b;'>\mu+</span>1:<span style='color:#644a9b;'>\mu+</span>r,1:r}<span style='color:#644a9b;'>\\</span>
  &amp;=&amp;<b>\begin</b>{<b><span style='color:#0095ff;'>pmatrix</span></b>}
<span style='color:#ff5500;'>   P_1&amp;0_1</span><span style='color:#3daee9;'>\\</span>
<span style='color:#ff5500;'>   P_2&amp;P_3</span>
<span style='color:#ff5500;'>  </span><b>\end</b>{<b><span style='color:#0095ff;'>pmatrix</span></b>}^{-1}<b>\begin</b>{<b><span style='color:#0095ff;'>pmatrix</span></b>}
<span style='color:#ff5500;'>   P_2&amp;P_3</span><span style='color:#3daee9;'>\\</span>
<span style='color:#ff5500;'>   P_4&amp;P_5 </span>
<span style='color:#ff5500;'>  </span><b>\end</b>{<b><span style='color:#0095ff;'>pmatrix</span></b>}<b>\begin</b>{<b><span style='color:#0095ff;'>pmatrix</span></b>}
<span style='color:#ff5500;'>   P_{1}&amp;0_1</span><span style='color:#3daee9;'>\\</span>
<span style='color:#ff5500;'>   P_{2}&amp;P_{3}</span>
<span style='color:#ff5500;'>  </span><b>\end</b>{<b><span style='color:#0095ff;'>pmatrix</span></b>}^{-1}P_{<span style='color:#644a9b;'>\mu+</span>1:<span style='color:#644a9b;'>\mu+</span>r,1:r}<span style='color:#644a9b;'>\\</span> 
  &amp;=&amp;<b>\begin</b>{<b><span style='color:#0095ff;'>pmatrix</span></b>}
<span style='color:#ff5500;'>   (P_1)^{-1}&amp;0_1</span><span style='color:#3daee9;'>\\</span>
<span style='color:#ff5500;'>   -P^{-1}_3P_2(P1)^{-1} &amp;P^{-1}_3 </span>
<span style='color:#ff5500;'>  </span><b>\end</b>{<b><span style='color:#0095ff;'>pmatrix</span></b>}<b>\begin</b>{<b><span style='color:#0095ff;'>pmatrix</span></b>}
<span style='color:#ff5500;'>   P_2&amp;P_3</span><span style='color:#3daee9;'>\\</span>
<span style='color:#ff5500;'>   P_4&amp;P_5 </span>
<span style='color:#ff5500;'>  </span><b>\end</b>{<b><span style='color:#0095ff;'>pmatrix</span></b>}<b>\begin</b>{<b><span style='color:#0095ff;'>pmatrix</span></b>}
<span style='color:#ff5500;'>   (P_1)^{-1}&amp;0_1</span><span style='color:#3daee9;'>\\</span>
<span style='color:#ff5500;'>   -P^{-1}_3P_2(P1)^{-1} &amp;P^{-1}_3</span>
<span style='color:#ff5500;'>  </span><b>\end</b>{<b><span style='color:#0095ff;'>pmatrix</span></b>}P_{<span style='color:#644a9b;'>\mu+</span>1:<span style='color:#644a9b;'>\mu+</span>r,1:r}<span style='color:#644a9b;'>\\</span>
  &amp;=&amp;<b>\begin</b>{<b><span style='color:#0095ff;'>pmatrix</span></b>}
<span style='color:#ff5500;'>   (P_1)^{-1}&amp;0_1</span><span style='color:#3daee9;'>\\</span>
<span style='color:#ff5500;'>   -P^{-1}_3P_2(P1)^{-1} &amp;P^{-1}_3</span>
<span style='color:#ff5500;'>  </span><b>\end</b>{<b><span style='color:#0095ff;'>pmatrix</span></b>}<b>\begin</b>{<b><span style='color:#0095ff;'>pmatrix</span></b>}
<span style='color:#ff5500;'>   P_{2}(P_{1})^{-1}-P_{2}(P_{1})^{-1}&amp;I_{r-</span><span style='color:#3daee9;'>\mu</span><span style='color:#ff5500;'>}</span><span style='color:#3daee9;'>\\</span>
<span style='color:#ff5500;'>   P_{4}(P_{1})^{-1}-P_{5}P_3^{-1}P_{2}(P_{1})^{-1}&amp;P_{5}P^{-1}_3</span>
<span style='color:#ff5500;'>  </span><b>\end</b>{<b><span style='color:#0095ff;'>pmatrix</span></b>}P_{<span style='color:#644a9b;'>\mu+</span>1:<span style='color:#644a9b;'>\mu+</span>r,1:r}<span style='color:#644a9b;'>\\</span>
  &amp;=&amp;<b>\begin</b>{<b><span style='color:#0095ff;'>pmatrix</span></b>}
<span style='color:#ff5500;'>   (P_1)^{-1}&amp;0_1</span><span style='color:#3daee9;'>\\</span>
<span style='color:#ff5500;'>   -P^{-1}_3P_2(P1)^{-1} &amp;P^{-1}_3</span>
<span style='color:#ff5500;'>  </span><b>\end</b>{<b><span style='color:#0095ff;'>pmatrix</span></b>}<b>\begin</b>{<b><span style='color:#0095ff;'>pmatrix</span></b>}
<span style='color:#ff5500;'>   0_2&amp;I_{r-</span><span style='color:#3daee9;'>\mu</span><span style='color:#ff5500;'>}</span><span style='color:#3daee9;'>\\</span>
<span style='color:#ff5500;'>   [P_{4}-P_{5}P_3^{-1}P_{2}](P_{1})^{-1}&amp;P_{5}P^{-1}_3</span>
<span style='color:#ff5500;'>  </span><b>\end</b>{<b><span style='color:#0095ff;'>pmatrix</span></b>}P_{<span style='color:#644a9b;'>\mu+</span>1:<span style='color:#644a9b;'>\mu+</span>r,1:r}
<span style='color:#644a9b;'>\fes</span> 

Por consiguiente se tiene que
<span style='color:#644a9b;'>\ies</span>
H_rF_r^2G_r&amp;=&amp;<b>\begin</b>{<b><span style='color:#0095ff;'>pmatrix</span></b>}
<span style='color:#ff5500;'>             P_1&amp;0_1 </span>
<span style='color:#ff5500;'>            </span><b>\end</b>{<b><span style='color:#0095ff;'>pmatrix</span></b>}<b>\begin</b>{<b><span style='color:#0095ff;'>pmatrix</span></b>}
<span style='color:#ff5500;'>   (P_1)^{-1}&amp;0_1</span><span style='color:#3daee9;'>\\</span>
<span style='color:#ff5500;'>   -P^{-1}_3P_2(P1)^{-1} &amp;P^{-1}_3</span>
<span style='color:#ff5500;'>  </span><b>\end</b>{<b><span style='color:#0095ff;'>pmatrix</span></b>}<b>\begin</b>{<b><span style='color:#0095ff;'>pmatrix</span></b>}
<span style='color:#ff5500;'>   0_2&amp;I_{r-</span><span style='color:#3daee9;'>\mu</span><span style='color:#ff5500;'>}</span><span style='color:#3daee9;'>\\</span>
<span style='color:#ff5500;'>   [P_{4}-P_{5}P_3^{-1}P_{2}](P_{1})^{-1}&amp;P_{5}P^{-1}_3</span>
<span style='color:#ff5500;'>  </span><b>\end</b>{<b><span style='color:#0095ff;'>pmatrix</span></b>}P_{<span style='color:#644a9b;'>\mu+</span>1:<span style='color:#644a9b;'>\mu+</span>r,1:r}Q_{1:r,1:<span style='color:#644a9b;'>\eta</span>}<span style='color:#644a9b;'>\\</span>
  &amp;=&amp;<b>\begin</b>{<b><span style='color:#0095ff;'>pmatrix</span></b>}
<span style='color:#ff5500;'>   I_{</span><span style='color:#3daee9;'>\mu</span><span style='color:#ff5500;'>}&amp;0_1</span>
<span style='color:#ff5500;'>  </span><b>\end</b>{<b><span style='color:#0095ff;'>pmatrix</span></b>}<b>\begin</b>{<b><span style='color:#0095ff;'>pmatrix</span></b>}
<span style='color:#ff5500;'>   0_2&amp;I_{r-</span><span style='color:#3daee9;'>\mu</span><span style='color:#ff5500;'>}</span><span style='color:#3daee9;'>\\</span>
<span style='color:#ff5500;'>   [P_{4}-P_{5}P_3^{-1}P_{2}](P_{1})^{-1}&amp;P_{5}P^{-1}_3</span>
<span style='color:#ff5500;'>  </span><b>\end</b>{<b><span style='color:#0095ff;'>pmatrix</span></b>}<b>\begin</b>{<b><span style='color:#0095ff;'>pmatrix</span></b>}
<span style='color:#ff5500;'>                L_2</span><span style='color:#3daee9;'>\\</span>
<span style='color:#ff5500;'>                L_{</span><span style='color:#3daee9;'>\mu</span><span style='color:#ff5500;'>+r}^{2</span><span style='color:#3daee9;'>\mu</span><span style='color:#ff5500;'>+1} </span>
<span style='color:#ff5500;'>               </span><b>\end</b>{<b><span style='color:#0095ff;'>pmatrix</span></b>}
<span style='color:#644a9b;'>\fes</span>
y de esto se sigue que para <span style='color:#ff5500;'>$t</span><span style='color:#3daee9;'>\geq</span><span style='color:#ff5500;'>0$</span>, 
<span style='color:#644a9b;'>\ies</span>
H_rF^tG_r=<b>\begin</b>{<b><span style='color:#0095ff;'>pmatrix</span></b>}
<span style='color:#ff5500;'>   I_{</span><span style='color:#3daee9;'>\mu</span><span style='color:#ff5500;'>}&amp;0_1</span>
<span style='color:#ff5500;'>  </span><b>\end</b>{<b><span style='color:#0095ff;'>pmatrix</span></b>}
  <b>\begin</b>{<b><span style='color:#0095ff;'>pmatrix</span></b>}
<span style='color:#ff5500;'>   0_2&amp;I_{r-</span><span style='color:#3daee9;'>\mu</span><span style='color:#ff5500;'>}</span><span style='color:#3daee9;'>\\</span>
<span style='color:#ff5500;'>   [P_{4}-P_{5}P_3^{-1}P_{2}](P_{1})^{-1}&amp;P_{5}P^{-1}_3</span>
<span style='color:#ff5500;'>  </span><b>\end</b>{<b><span style='color:#0095ff;'>pmatrix</span></b>}^{t}
  <b>\begin</b>{<b><span style='color:#0095ff;'>pmatrix</span></b>}
<span style='color:#ff5500;'>                L_2</span><span style='color:#3daee9;'>\\</span>
<span style='color:#ff5500;'>                L_{</span><span style='color:#3daee9;'>\mu</span><span style='color:#ff5500;'>+r}^{2</span><span style='color:#3daee9;'>\mu</span><span style='color:#ff5500;'>+1} </span>
<span style='color:#ff5500;'>               </span><b>\end</b>{<b><span style='color:#0095ff;'>pmatrix</span></b>}
<span style='color:#644a9b;'>\fes</span>
Es decir, ya que la matriz extrema izquierda  en <span style='color:#ff5500;'>$F^t$</span> es constante e independiente de los valores singulares elegidos, entonces para cada truncamiento de la matriz de Hankel en <span style='color:#ff5500;'>$r&gt;</span><span style='color:#3daee9;'>\mu</span><span style='color:#ff5500;'>$</span> valores singulares, lo que determina la sucesión estimada es la órbita de
<span style='color:#644a9b;'>\ies</span>
D_r=<b>\begin</b>{<b><span style='color:#0095ff;'>pmatrix</span></b>}
<span style='color:#ff5500;'>   0_2&amp;I_{r-</span><span style='color:#3daee9;'>\mu</span><span style='color:#ff5500;'>}</span><span style='color:#3daee9;'>\\</span>
<span style='color:#ff5500;'>   [P_{4}-P_{5}P_3^{-1}P_{2}](P_{1})^{-1}&amp;P_{5}P^{-1}_3</span>
<span style='color:#ff5500;'>  </span><b>\end</b>{<b><span style='color:#0095ff;'>pmatrix</span></b>}<span style='color:#644a9b;'>\in\R</span>^{r<span style='color:#644a9b;'>\times</span> r}
<span style='color:#644a9b;'>\fes</span>
y la matriz <span style='color:#ff5500;'>$</span><b>\begin</b>{<b><span style='color:#0095ff;'>pmatrix</span></b>}
<span style='color:#ff5500;'>                L_2</span><span style='color:#3daee9;'>\\</span>
<span style='color:#ff5500;'>                L_{</span><span style='color:#3daee9;'>\mu</span><span style='color:#ff5500;'>+r}^{2</span><span style='color:#3daee9;'>\mu</span><span style='color:#ff5500;'>+1} </span>
<span style='color:#ff5500;'>               </span><b>\end</b>{<b><span style='color:#0095ff;'>pmatrix</span></b>}<span style='color:#ff5500;'>$</span>, ambas dependientes del número de valores singulares. Así mismo, como <span style='color:#ff5500;'>$P_5P_3^{-1}</span><span style='color:#3daee9;'>\in\R</span><span style='color:#ff5500;'>^{</span><span style='color:#3daee9;'>\mu\times</span><span style='color:#ff5500;'> r-</span><span style='color:#3daee9;'>\mu</span><span style='color:#ff5500;'>}$</span>, y por consiguiente <span style='color:#ff5500;'>$P_6=(P_5P_3^{-1})_{:,r-</span><span style='color:#3daee9;'>\mu</span><span style='color:#ff5500;'>+2:r-</span><span style='color:#3daee9;'>\mu</span><span style='color:#ff5500;'>}$</span> es la submatriz cuadrada de <span style='color:#ff5500;'>$</span><span style='color:#3daee9;'>\mu\times\mu</span><span style='color:#ff5500;'>$</span> en la esquina inferior de <span style='color:#ff5500;'>$D_r$</span>, se tiene que
<span style='color:#644a9b;'>\ies</span>
tr(P_6)=tr(D_r)=<span style='color:#644a9b;'>\sum</span>_{i=1}^r<span style='color:#644a9b;'>\lambda</span>_i
<span style='color:#644a9b;'>\fes</span>
Por otro lado, tenemos que la matriz de Hankel de rango <span style='color:#ff5500;'>$n$</span> es tal que
<span style='color:#644a9b;'>\ies</span>
PQ=<span style='color:#644a9b;'>\mathscr</span>{H}_{<span style='color:#644a9b;'>\b</span>,<span style='color:#644a9b;'>\a</span>}=<span style='color:#644a9b;'>\sum</span>_{i=1}^r<span style='color:#644a9b;'>\sigma</span>_i<span style='color:#644a9b;'>\vec</span>{u}_i<span style='color:#644a9b;'>\vec</span>{v}_i^T=<span style='color:#644a9b;'>\sum</span>_{i=1}^n<span style='color:#644a9b;'>\sigma</span>_i
<b>\begin</b>{<b><span style='color:#0095ff;'>pmatrix</span></b>}
<span style='color:#ff5500;'>u_{i1}v_{i1}&amp;</span><span style='color:#3daee9;'>\cdots</span><span style='color:#ff5500;'>&amp; u_{i1}v_{i(</span><span style='color:#3daee9;'>\a\eta</span><span style='color:#ff5500;'>)}</span><span style='color:#3daee9;'>\\</span>
<span style='color:#3daee9;'>\vdots</span><span style='color:#ff5500;'>&amp;</span><span style='color:#3daee9;'>\ddots</span><span style='color:#ff5500;'>&amp;</span><span style='color:#3daee9;'>\vdots\\</span>
<span style='color:#ff5500;'>u_{i(</span><span style='color:#3daee9;'>\b\mu</span><span style='color:#ff5500;'>)}v_{i1}&amp;</span><span style='color:#3daee9;'>\cdots</span><span style='color:#ff5500;'>&amp; u_{i(</span><span style='color:#3daee9;'>\b\mu</span><span style='color:#ff5500;'>)}v_{i(</span><span style='color:#3daee9;'>\a\eta</span><span style='color:#ff5500;'>)}</span>
<b>\end</b>{<b><span style='color:#0095ff;'>pmatrix</span></b>}
<span style='color:#644a9b;'>\fes</span> 
y debido al Algoritmo de Factorización,
<span style='color:#644a9b;'>\ies</span>
P=<b>\begin</b>{<b><span style='color:#0095ff;'>matrix</span></b>} 
                        <b>\begin</b>{<b><span style='color:#0095ff;'>tikzpicture</span></b>}
<span style='color:#644a9b;'>\draw</span> (0,0) rectangle (6,6);
<span style='color:#644a9b;'>\draw</span> (0,5) rectangle (1,6) node[pos=.5] {<span style='color:#ff5500;'>$</span><span style='color:#3daee9;'>\mathbf</span><span style='color:#ff5500;'>{P_1}$</span>};
<span style='color:#644a9b;'>\draw</span> (1,2) rectangle (4,5) node[pos=.5] {<span style='color:#ff5500;'>$</span><span style='color:#3daee9;'>\mathbf</span><span style='color:#ff5500;'>{P_3}$</span>};
<span style='color:#644a9b;'>\draw</span> (0,2) rectangle (1,5) node[pos=.5] {<span style='color:#ff5500;'>$</span><span style='color:#3daee9;'>\mathbf</span><span style='color:#ff5500;'>{P_2}$</span>};
<span style='color:#644a9b;'>\draw</span> (0,1) rectangle (1,2) node[pos=.5] {<span style='color:#ff5500;'>$</span><span style='color:#3daee9;'>\mathbf</span><span style='color:#ff5500;'>{P_4}$</span>};
<span style='color:#644a9b;'>\draw</span> (1,1) rectangle (4,2) node[pos=.5]{<span style='color:#ff5500;'>$</span><span style='color:#3daee9;'>\mathbf</span><span style='color:#ff5500;'>{P_5}$</span>};
<span style='color:#644a9b;'>\draw</span> (1,6.2) node{<span style='color:#ff5500;'>$</span><span style='color:#3daee9;'>\mu</span><span style='color:#ff5500;'>$</span>} (4,6.2) node{<span style='color:#ff5500;'>$r$</span>} (-0.5,5) node{<span style='color:#ff5500;'>$</span><span style='color:#3daee9;'>\mu</span><span style='color:#ff5500;'>$</span>} (-0.5,1) node{<span style='color:#ff5500;'>$</span><span style='color:#3daee9;'>\mu</span><span style='color:#ff5500;'>+r$</span>} (-0.5,2) node{<span style='color:#ff5500;'>$r$</span>};
<b>\end</b>{<b><span style='color:#0095ff;'>tikzpicture</span></b>}  
                       <b>\end</b>{<b><span style='color:#0095ff;'>matrix</span></b>}                        
<span style='color:#644a9b;'>\fes</span>
Debido a que por el Algoritmo de Factorización no permite funcionalizar las entradas en <span style='color:#ff5500;'>$P_3$</span> y <span style='color:#ff5500;'>$P_5$</span> debido a que
 
 <span style='color:#644a9b;'>\ies</span>
 <span style='color:#644a9b;'>\framebox</span>[2cm][c]{<span style='color:#ff5500;'>$Q_{1,:}$</span>}&amp;=&amp;H_{1,:}<span style='color:#644a9b;'>\\</span>  
 <span style='color:#644a9b;'>\\</span>
 <span style='color:#644a9b;'>\framebox</span>[2cm][c]{<span style='color:#ff5500;'>$P_{:,1}$</span>}&amp;=&amp;<span style='color:#644a9b;'>\frac</span>{1}{Q_{1,1}}H_{:,1}<span style='color:#644a9b;'>\\</span>
 <span style='color:#644a9b;'>\framebox</span>[2cm][c]{<span style='color:#ff5500;'>$Q_{2,2:}$</span>}&amp;=&amp;H_{2,2:}<span style='color:#644a9b;'>\\</span>
 <span style='color:#644a9b;'>\\</span>
 P_{2:,2}&amp;=&amp;<span style='color:#644a9b;'>\frac</span>{1}{Q_{2,2}}(H_{2:,2}-<span style='color:#644a9b;'>\framebox</span>[2cm][c]{<span style='color:#ff5500;'>$P_{2:,1}Q_{1,2}$</span>})<span style='color:#644a9b;'>\\</span>
 Q_{3,3:}&amp;=&amp;H_{3,3:}-<span style='color:#644a9b;'>\framebox</span>[2cm][c]{<span style='color:#ff5500;'>$P_{3,1}Q_{1,3:}$</span>}<span style='color:#644a9b;'>\\</span>
 <span style='color:#644a9b;'>\\</span>
 P_{3:,3}&amp;=&amp;<span style='color:#644a9b;'>\frac</span>{1}{Q_{3,3}}(H_{3:,3}-<span style='color:#644a9b;'>\framebox</span>[2cm][c]{<span style='color:#ff5500;'>$P_{3:,1}Q_{1,3}$</span>}-P_{3:,2}Q_{2,3}))<span style='color:#644a9b;'>\\</span>
 &amp;=&amp;
 <span style='color:#644a9b;'>\frac</span>{1}{Q_{3,3}}<span style='color:#644a9b;'>\left</span>(H_{3:,3}-<span style='color:#644a9b;'>\framebox</span>[2cm][c]{<span style='color:#ff5500;'>$P_{3:,1}Q_{1,3}$</span>}-<span style='color:#644a9b;'>\frac</span>{1}{Q_{2,2}}(H_{3:,2}-<span style='color:#644a9b;'>\framebox</span>[2cm][c]{<span style='color:#ff5500;'>$P_{3:,1}Q_{1,2}$</span>})<span style='color:#644a9b;'>\framebox</span>[2cm][c]{<span style='color:#ff5500;'>$Q_{2,3}$</span>}<span style='color:#644a9b;'>\right</span>)<span style='color:#644a9b;'>\\</span>
 &amp;=&amp;<span style='color:#644a9b;'>\frac</span>{1}{Q_{3,3}}<span style='color:#644a9b;'>\left</span>(H_{3:,3}-<span style='color:#644a9b;'>\framebox</span>[2cm][c]{<span style='color:#ff5500;'>$P_{3:,1}Q_{1,3}$</span>}-<span style='color:#644a9b;'>\frac</span>{<span style='color:#644a9b;'>\framebox</span>[2cm][c]{<span style='color:#ff5500;'>$Q_{2,3}$</span>}}{Q_{2,2}}(H_{3:,2}-<span style='color:#644a9b;'>\framebox</span>[2cm][c]{<span style='color:#ff5500;'>$P_{3:,1}Q_{1,2}$</span>})<span style='color:#644a9b;'>\right</span>)<span style='color:#644a9b;'>\\</span> 
 Q_{4,4:}&amp;=&amp; H_{4,4:}-<span style='color:#644a9b;'>\framebox</span>[2cm][c]{<span style='color:#ff5500;'>$P_{4,1}Q_{1,4:}$</span>} -P_{4,2}Q_{2,4:}<span style='color:#644a9b;'>\\</span>
 &amp;=&amp; H_{4,4:}-<span style='color:#644a9b;'>\framebox</span>[2cm][c]{<span style='color:#ff5500;'>$P_{4,1}Q_{1,4:}$</span>} -<span style='color:#644a9b;'>\frac</span>{Q_{2,4,:}}{Q_{2,2}}(H_{4,2}-<span style='color:#644a9b;'>\framebox</span>[2cm][c]{<span style='color:#ff5500;'>$P_{4,1}Q_{1,2}$</span>})
 <span style='color:#644a9b;'>\fes</span>
 se propone el siguiente formato de descomposición como forma alternativa. Inicialmente, tomamos una matriz triangular inferior <span style='color:#ff5500;'>$P$</span> con diagonal principal de <span style='color:#ff5500;'>$1$</span> y hacemos lo siguiente
 <span style='color:#644a9b;'>\ies</span>
 <span style='color:#644a9b;'>\mathscr</span>{H}=P*Q_0-<span style='color:#644a9b;'>\sum</span>_{i=1}^{}Q_i
 <span style='color:#644a9b;'>\fes</span> 
 y así, para construir una descomposición <span style='color:#ff5500;'>$P$</span>, <span style='color:#ff5500;'>$Q$</span> 
 <span style='color:#644a9b;'>\ies</span>
 PQ=<span style='color:#644a9b;'>\mathscr</span>{H}=P*Q_0-<span style='color:#644a9b;'>\sum</span>_{i=1}^{}Q_i
 <span style='color:#644a9b;'>\fes</span>
y como <span style='color:#ff5500;'>$P$</span> es invertible,
<span style='color:#644a9b;'>\ies</span>
Q=Q_0-P^{-1}<span style='color:#644a9b;'>\sum</span>_{i=1}Q_i
<span style='color:#644a9b;'>\fes</span>
Ahora, para construir <span style='color:#ff5500;'>$P$</span> tomamos la matriz triangular inferior de <span style='color:#ff5500;'>$</span><span style='color:#3daee9;'>\mathscr</span><span style='color:#ff5500;'>{H}$</span>, y dividimos cada columna entre el valor del elemento en su diagonal
<span style='color:#644a9b;'>\ies</span>
P=<b>\begin</b>{<b><span style='color:#0095ff;'>pmatrix</span></b>}
<span style='color:#3daee9;'>\frac</span><span style='color:#ff5500;'>{h_{11}}{h_{11}}&amp;0&amp;</span><span style='color:#3daee9;'>\cdots</span><span style='color:#ff5500;'>&amp; 0</span><span style='color:#3daee9;'>\\</span><span style='color:#ff5500;'>[.2cm]</span>
<span style='color:#3daee9;'>\frac</span><span style='color:#ff5500;'>{h_{21}}{h_{11}}&amp;1&amp;</span><span style='color:#3daee9;'>\cdots</span><span style='color:#ff5500;'>&amp;0</span><span style='color:#3daee9;'>\\</span><span style='color:#ff5500;'>[.2cm]</span>
<span style='color:#3daee9;'>\vdots</span><span style='color:#ff5500;'>&amp;</span><span style='color:#3daee9;'>\ddots</span><span style='color:#ff5500;'>&amp;</span><span style='color:#3daee9;'>\ddots</span><span style='color:#ff5500;'>&amp;</span><span style='color:#3daee9;'>\vdots\\</span><span style='color:#ff5500;'>[.2cm]</span>
<span style='color:#3daee9;'>\frac</span><span style='color:#ff5500;'>{h_{:1}}{h_{11}}&amp;</span><span style='color:#3daee9;'>\cdots</span><span style='color:#ff5500;'>&amp;</span><span style='color:#3daee9;'>\frac</span><span style='color:#ff5500;'>{h_{::-1}}{h_{:-1:-1}}&amp; </span><span style='color:#3daee9;'>\frac</span><span style='color:#ff5500;'>{h_{::}}{h_{::}}</span>
<b>\end</b>{<b><span style='color:#0095ff;'>pmatrix</span></b>}
=
<b>\begin</b>{<b><span style='color:#0095ff;'>pmatrix</span></b>}
<span style='color:#ff5500;'>1&amp;0&amp;</span><span style='color:#3daee9;'>\cdots</span><span style='color:#ff5500;'>&amp; 0</span><span style='color:#3daee9;'>\\</span><span style='color:#ff5500;'>[.2cm]</span>
<span style='color:#3daee9;'>\frac</span><span style='color:#ff5500;'>{h_{21}}{h_{11}}&amp;1&amp;</span><span style='color:#3daee9;'>\cdots</span><span style='color:#ff5500;'>&amp;0</span><span style='color:#3daee9;'>\\</span><span style='color:#ff5500;'>[.2cm]</span>
<span style='color:#3daee9;'>\vdots</span><span style='color:#ff5500;'>&amp;</span><span style='color:#3daee9;'>\ddots</span><span style='color:#ff5500;'>&amp;</span><span style='color:#3daee9;'>\ddots</span><span style='color:#ff5500;'>&amp;</span><span style='color:#3daee9;'>\vdots\\</span><span style='color:#ff5500;'>[.2cm]</span>
<span style='color:#3daee9;'>\frac</span><span style='color:#ff5500;'>{h_{:1}}{h_{11}}&amp;</span><span style='color:#3daee9;'>\cdots</span><span style='color:#ff5500;'>&amp;</span><span style='color:#3daee9;'>\frac</span><span style='color:#ff5500;'>{h_{::-1}}{h_{:-1:-1}}&amp; 1</span>
<b>\end</b>{<b><span style='color:#0095ff;'>pmatrix</span></b>}
<span style='color:#644a9b;'>\fes</span> 
por otro lado, con la matriz triangular superior de <span style='color:#ff5500;'>$</span><span style='color:#3daee9;'>\mathscr</span><span style='color:#ff5500;'>{H}$</span> construimos la matriz <span style='color:#ff5500;'>$Q_0$</span>
<span style='color:#644a9b;'>\ies</span>
Q_0=
<b>\begin</b>{<b><span style='color:#0095ff;'>pmatrix</span></b>}
<span style='color:#ff5500;'>h_{11}&amp;h_{12}&amp;</span><span style='color:#3daee9;'>\cdots</span><span style='color:#ff5500;'>&amp; h_{1:-1}&amp;h_{1:}</span><span style='color:#3daee9;'>\\</span><span style='color:#ff5500;'>[.2cm]</span>
<span style='color:#ff5500;'>0&amp;h_{22}&amp;</span><span style='color:#3daee9;'>\cdots</span><span style='color:#ff5500;'>&amp;h_{2:}&amp;h_{2:}</span><span style='color:#3daee9;'>\\</span><span style='color:#ff5500;'>[.2cm]</span>
<span style='color:#3daee9;'>\vdots</span><span style='color:#ff5500;'>&amp;</span><span style='color:#3daee9;'>\ddots</span><span style='color:#ff5500;'>&amp;</span><span style='color:#3daee9;'>\ddots</span><span style='color:#ff5500;'>&amp;</span><span style='color:#3daee9;'>\vdots\\</span><span style='color:#ff5500;'>[.2cm]</span>
<span style='color:#ff5500;'>0&amp;</span><span style='color:#3daee9;'>\cdots</span><span style='color:#ff5500;'>&amp;0&amp; h_{::_1}&amp;h_{::}</span>
<b>\end</b>{<b><span style='color:#0095ff;'>pmatrix</span></b>}
<span style='color:#644a9b;'>\fes</span>
y las matrices <span style='color:#ff5500;'>$Q_i$</span> se construyen de la siguiente manera
<span style='color:#644a9b;'>\ies</span>
Q_i=<span style='color:#644a9b;'>\frac</span>{1}{h_{ii}}
<span style='color:#644a9b;'>\fes</span>

<span style='color:#898887;'>% \subsection{Extra}</span>
<span style='color:#898887;'>% </span>
<span style='color:#898887;'>% \ies</span>
<span style='color:#898887;'>% \begin{pmatrix}</span>
<span style='color:#898887;'>%  \lambda_{v}v_1\\</span>
<span style='color:#898887;'>%  \cdots\\</span>
<span style='color:#898887;'>%  \lambda_{v}v_{r-\mu}\\</span>
<span style='color:#898887;'>%  \lambda_{v}v_n</span>
<span style='color:#898887;'>% \end{pmatrix}=</span>
<span style='color:#898887;'>% \lambda_v\vec{v}=\begin{pmatrix}</span>
<span style='color:#898887;'>%    0_2&amp;I_{r-\mu}\\</span>
<span style='color:#898887;'>%    [P_{4}-P_{5}P_3^{-1}P_{2}](P_{1})^{-1}&amp;P_{5}P^{-1}_3</span>
<span style='color:#898887;'>%   \end{pmatrix}\vec{v}=\begin{pmatrix}</span>
<span style='color:#898887;'>%    v_{\mu+1}\\</span>
<span style='color:#898887;'>%    \cdots\\</span>
<span style='color:#898887;'>%    v_{r}\\</span>
<span style='color:#898887;'>%    [P_{4}-P_{5}P_3^{-1}P_{2}](P_{1})^{-1}&amp;P_{5}P^{-1}_3</span>
<span style='color:#898887;'>%   \end{pmatrix}</span>
<span style='color:#898887;'>% \fes</span>
<span style='color:#898887;'>% </span>
<span style='color:#898887;'>% $\lambda_{v}v_i=v_{\mu+i}$ $i=1,\cdots,r-\mu$, $r\geq 2\mu$, por consiguiente los vectores propios son tales que</span>
<span style='color:#898887;'>% \ies</span>
<span style='color:#898887;'>% \begin{pmatrix}</span>
<span style='color:#898887;'>%  v_1\\</span>
<span style='color:#898887;'>%  \vdots\\</span>
<span style='color:#898887;'>%  v_\mu</span>
<span style='color:#898887;'>% \end{pmatrix}\hspace{.5cm}\rightrightarrows\hspace{.5cm}</span>
<span style='color:#898887;'>% \begin{matrix}</span>
<span style='color:#898887;'>% \lambda_v\begin{pmatrix}</span>
<span style='color:#898887;'>%  v_{1}\\</span>
<span style='color:#898887;'>%  \vdots\\</span>
<span style='color:#898887;'>%  v_{\mu}</span>
<span style='color:#898887;'>% \end{pmatrix}&amp;=&amp;</span>
<span style='color:#898887;'>% \begin{pmatrix}</span>
<span style='color:#898887;'>%  v_{\mu+1}\\</span>
<span style='color:#898887;'>%  \vdots\\</span>
<span style='color:#898887;'>%  v_{2\mu}</span>
<span style='color:#898887;'>% \end{pmatrix}\\[.5cm]</span>
<span style='color:#898887;'>% \lambda^2_v\begin{pmatrix}</span>
<span style='color:#898887;'>%  v_{1}\\</span>
<span style='color:#898887;'>%  \vdots\\</span>
<span style='color:#898887;'>%  v_{\mu}</span>
<span style='color:#898887;'>% \end{pmatrix}&amp;=&amp;\lambda_v</span>
<span style='color:#898887;'>% \begin{pmatrix}</span>
<span style='color:#898887;'>%  v_{\mu+1}\\</span>
<span style='color:#898887;'>%  \vdots\\</span>
<span style='color:#898887;'>%  v_{2\mu}</span>
<span style='color:#898887;'>% \end{pmatrix}&amp;=&amp;\begin{pmatrix}</span>
<span style='color:#898887;'>%  v_{_{2\mu+1}}\\</span>
<span style='color:#898887;'>%  \vdots\\</span>
<span style='color:#898887;'>%  v_{_{3\mu}}</span>
<span style='color:#898887;'>% \end{pmatrix}\\</span>
<span style='color:#898887;'>% \hspace{1em}\vdots&amp;&amp;\vdots&amp;&amp;\vdots&amp;&amp;\ddots</span>
<span style='color:#898887;'>% \\</span>
<span style='color:#898887;'>% \lambda^{r-1}_v\begin{pmatrix}</span>
<span style='color:#898887;'>%  v_{_{\mu+1}}\\</span>
<span style='color:#898887;'>%  \vdots\\</span>
<span style='color:#898887;'>%  v_{_{2\mu}}</span>
<span style='color:#898887;'>% \end{pmatrix}&amp;=&amp;\lambda^{r-2}_v</span>
<span style='color:#898887;'>% \begin{pmatrix}</span>
<span style='color:#898887;'>%  v_{\mu+1}\\</span>
<span style='color:#898887;'>%  \vdots\\</span>
<span style='color:#898887;'>%  v_{2\mu}</span>
<span style='color:#898887;'>% \end{pmatrix}&amp;=&amp;\lambda^{r-3}_v\begin{pmatrix}</span>
<span style='color:#898887;'>%  v_{_{2\mu+1}}\\</span>
<span style='color:#898887;'>%  \vdots\\</span>
<span style='color:#898887;'>%  v_{_{3\mu}}</span>
<span style='color:#898887;'>% \end{pmatrix}&amp;=&amp;\cdots&amp;=&amp;</span>
<span style='color:#898887;'>% \begin{pmatrix}</span>
<span style='color:#898887;'>%  v_{_{(r-1)\mu+1}}\\</span>
<span style='color:#898887;'>%  \vdots\\</span>
<span style='color:#898887;'>%  v_{_{r\mu}}</span>
<span style='color:#898887;'>% \end{pmatrix}</span>
<span style='color:#898887;'>% \end{matrix}</span>
<span style='color:#898887;'>% \fes</span>
<span style='color:#898887;'>% así tenemos que cada vector propio se puede ver como</span>
<span style='color:#898887;'>% \ies</span>
<span style='color:#898887;'>% \vec{v}=\begin{pmatrix}</span>
<span style='color:#898887;'>%          I_\mu&amp;0&amp;\cdots&amp;0\\</span>
<span style='color:#898887;'>%          0&amp;\lambda_vI_\mu&amp;\ddots&amp;\vdots\\ \vdots&amp;\ddots&amp;\ddots&amp;0\\</span>
<span style='color:#898887;'>%          0&amp;\cdots&amp;0&amp;\lambda^{r-1}_{v}I_\mu</span>
<span style='color:#898887;'>%         \end{pmatrix}\vec{v}_{_{\mu^r}}</span>
<span style='color:#898887;'>% \fes</span>
<span style='color:#898887;'>% donde $\vec{v}_{_{\mu^r}}$ es el vector formado al colocar $r$ veces el vector de los primeros $\mu$ elementos de $\vec{v}$.\-</span>
<span style='color:#898887;'>% </span>
<span style='color:#898887;'>% </span>
<span style='color:#898887;'>% Además, si descomponemos la matriz $D_r$ en valores propios, tenemos que </span>
<span style='color:#898887;'>% \ies</span>
<span style='color:#898887;'>% D_r=Q\Sigma Q^{-1}</span>
<span style='color:#898887;'>% \fes</span>
<span style='color:#898887;'>% y de esta igualdad (recordando que $r\geq2\mu$) tenemos que</span>
<span style='color:#898887;'>% \ies</span>
<span style='color:#898887;'>% \begin{pmatrix}</span>
<span style='color:#898887;'>%  0_3&amp;I_\mu&amp;0_4</span>
<span style='color:#898887;'>% \end{pmatrix}</span>
<span style='color:#898887;'>% \fes</span>


Primeras adaptaciones del sistema que muestran la existencia del cambio estructural se dan en variables subyacentes o en variables que no son tomadas en cuenta en el modelo. 



<b>\begin</b>{<b><span style='color:#0095ff;'>enumerate</span></b>}
<span style='color:#644a9b;'>\item</span> Rediseñando el concepto de cambio estructural y su determinación
<b>\begin</b>{<b><span style='color:#0095ff;'>enumerate</span></b>}[i.]
<span style='color:#644a9b;'>\item</span>  Enfoques sobre el cambio estructural
<span style='color:#644a9b;'>\item</span>  El cambio estructural y la definición del sistema
<span style='color:#644a9b;'>\item</span>  Operacionalización de la DE del sistema para la determinación
del CE
<b>\end</b>{<b><span style='color:#0095ff;'>enumerate</span></b>}
<b>\end</b>{<b><span style='color:#0095ff;'>enumerate</span></b>}





Un sistema puede ser entendido como un cúmulo de elementos relacionados,  las cuáles poseen relativa permanencia <b>\citep</b>[p.8]{}. 
Las relaciones son lo que se denomina interacción,  cuando <span style='color:#ff5500;'>$a_i$</span> y <span style='color:#ff5500;'>$a_j$</span> se engarzan bajo la relación <span style='color:#ff5500;'>$R_k$</span>


Los modelos matemáticos para el estudio de sistemas pueden representar un gran desafío. Existen implementaciones analíticas que permiten un tratamiento sencillo de un tipo particular de sistemas (físicos o conceptuales): aquellos que son un agregado simple (aditividad) de sus partes constitutivas; concepción por antonomasia de la ciencia clásica <b>\citet</b>[pp.17]{}. Estos sistemas cuentan con elementos cuyas interacciones no son muy fuertes, y donde al ser estimulados generan una reacción proporcional (linealidad).
Sin embargo, existe un espectro de sistemas que no son de este tipo. Estos sistemas requieren modelos matemáticos más complicados debido a la interdependencia fuerte entre sus elementos y su reacción ante estimulos no es proporcional  que en ocasiones son irresolubles (Figura <b>\ref</b>{})





<span style='color:#644a9b;'>\-</span>


La estructura económica es parte de lo que puede denominarse el sistema económico. En cuanto a ser estructura se concibe como  un entramado de relaciones . Autores como <b>\citet</b>{} 
expresan que los elementos que intervienen en la estructura son

<span style='color:#644a9b;'>\-</span>

La economía como sistema también cuenta con una estructura, la cuál, sin embargo, tiene una amplitud de elementos y relaciónes que se subordina a la corriente de pensamiento. Esto es, cada una de ellas coloca una frontera al sistema económico en ocasión a sus perspectivas sobre lo que debe ser enfatizado. Esto produce una colección de estructuras teóricas sobre el sistema económico empírico. Una clasificación indirecta de ellas la brinda <b>\citet</b>{<b><span style='color:#0095ff;'>peirano</span></b>}. Aduce que existen 5 nociones de cambio estructural, cada una sembrada en ideas de elementos que interactuan <span style='color:#644a9b;'>\-</span>



``Estudiar a la economía como un SC es una alternativa para explicar
aspectos que no han sido suficientemente entendidos, por ejemplo, a nivel micro, el
comportamiento de elección de consumidores y productores; a nivel macro, el ciclo
económico y el fenómeno del desarrollo; y a nivel intermedio, el comportamiento de los
mercados, en particular las bolsas de valores.'' <b>\citep</b>[p.4]{}







El cambio estructural se puede entender como una sensible modificación en el número de elementos, sus características, las relaciones de unos con otro, y  el resultado de sus interacciones. En particular, 



 Cambio estructural:
  <b>\begin</b>{<b><span style='color:#0095ff;'>center</span></b>}{<span style='color:#644a9b;'>\small</span>
  <b>\begin</b>{<b><span style='color:#0095ff;'>tabular</span></b>}{cc}
  Autor(es)<b>&amp;</b>Enfoque<span style='color:#644a9b;'>\\\hline\vspace</span>{-1em}<span style='color:#644a9b;'>\\</span>
   Karl Polanyi<b>&amp;</b>Transformación social<span style='color:#644a9b;'>\\</span>
   Lewis, Kutznets y Kaldor<b>&amp;</b>Redistribución laboral (+ productividad)<span style='color:#644a9b;'>\\</span>
   Schumpeter<b>&amp;</b>Renovación tecnológica<span style='color:#644a9b;'>\\</span>
   Hopkins, Wallerstein y Gereffi<b>&amp;</b>Actualización de cadenas de valor<span style='color:#644a9b;'>\\</span>
   Fanelli y Frenkel<b>&amp;</b> Diversificación
  <b>\end</b>{<b><span style='color:#0095ff;'>tabular</span></b>}}
  <b>\begin</b>{<b><span style='color:#0095ff;'>flushleft</span></b>}
  {<span style='color:#644a9b;'>\tiny</span> Elaboración propia con datos de <b>\citet</b>{<b><span style='color:#0095ff;'>peirano</span></b>}, <b>\citet</b>{<b><span style='color:#0095ff;'>yoguel</span></b>} y <b>\citet</b>{<b><span style='color:#0095ff;'>saspo</span></b>}} 
  <b>\end</b>{<b><span style='color:#0095ff;'>flushleft</span></b>} 
  <b>\end</b>{<b><span style='color:#0095ff;'>center</span></b>}



<span style='color:#898887;'>%NUEVO 17/12/20</span>


Supongamos que tenemos dos Matrices de Leontief <span style='color:#ff5500;'>$L_1,L_2$</span> las cuales se conformaron con datos comparables, es decir, deflactados, y donde el ruido existente en ellos fue controlado. Además, si tenemos las demandas <span style='color:#ff5500;'>$</span><span style='color:#3daee9;'>\mathbf</span><span style='color:#ff5500;'>{d}_t$</span> y productos <span style='color:#ff5500;'>$</span><span style='color:#3daee9;'>\mathbf</span><span style='color:#ff5500;'>{y}_t$</span> reales y comparables de las matrices <span style='color:#ff5500;'>$L_t$</span>, <span style='color:#ff5500;'>$t=1,2$</span>, podemos desarrolar un método para aislar el cambio técnico y los rendimientos a escala en las funciones de producción estática, a partir de la TR.<span style='color:#644a9b;'>\-</span>

Como las matrices <span style='color:#ff5500;'>$L_1$</span> y <span style='color:#ff5500;'>$L_2$</span> son comparables, es decir, sus diferencias no se deben ni a variaciones en los niveles de precios ni a aleatoriedad de los datos, entonces una propuesta es que sus fluctuaciones estriben en el cambio técnico y/o en economías de escala. Estos altibajos los podemos observar como sigue
<span style='color:#644a9b;'>\ies</span>
<span style='color:#644a9b;'>\mathbf</span>{y}_2=(L_1+C_{1,2})<span style='color:#644a9b;'>\mathbf</span>{y}_2<span style='color:#644a9b;'>\hspace</span>{2em} y <span style='color:#644a9b;'>\hspace</span>{2em} <span style='color:#644a9b;'>\mathbf</span>{y}_2=T_{1,2}L_1<span style='color:#644a9b;'>\mathbf</span>{y}_2
<span style='color:#644a9b;'>\fes</span>
En la ecuación izquierda, <span style='color:#ff5500;'>$C_{1,2}$</span> refleja los cambios técnicos y/o las economías de escala que se incorporarón o que poseía el sistema productivo, y con lo cuales  para que la matriz <span style='color:#ff5500;'>$L_1$</span> igualará a la matriz <span style='color:#ff5500;'>$L_2$</span>. Por su parte, la matriz <span style='color:#ff5500;'>$T_{1,2}$</span> refleja el como el subsistema  

<span style='color:#644a9b;'>\ies</span>
<span style='color:#644a9b;'>\mathbf</span>{d}_1
<span style='color:#644a9b;'>\mathbf</span>{y}_2 = L_2<span style='color:#644a9b;'>\mathbf</span>{d}_2
<span style='color:#644a9b;'>\fes</span>

<span style='color:#644a9b;'>\textbf</span>{Estructura de la matriz de Hankel} <span style='color:#644a9b;'>\-</span>


Si tenemos una sucesión de matrices <span style='color:#ff5500;'>$</span><span style='color:#3daee9;'>\{</span><span style='color:#ff5500;'>W_n</span><span style='color:#3daee9;'>\}</span><span style='color:#ff5500;'>_</span><span style='color:#3daee9;'>\N</span><span style='color:#ff5500;'>$</span>, la matriz de Hankel de <span style='color:#ff5500;'>$q</span><span style='color:#3daee9;'>\times</span><span style='color:#ff5500;'> p$</span> bloques, construida a partir de ella, es
<span style='color:#644a9b;'>\ies</span>
   <span style='color:#644a9b;'>\mathscr</span>{H}_{q,p}(W_n)=<b>\begin</b>{<b><span style='color:#0095ff;'>pmatrix</span></b>}
<span style='color:#ff5500;'>    W_1&amp;W_2&amp;W_3&amp;</span><span style='color:#3daee9;'>\mbox</span>{<span style='color:#3daee9;'>\colorbox</span>[rgb]{1,0.41,0.38}{<span style='color:#ff5500;'>$W_4$</span>}}<span style='color:#ff5500;'>&amp;</span><span style='color:#3daee9;'>\dots</span><span style='color:#ff5500;'>&amp;W_p</span><span style='color:#3daee9;'>\\</span>
<span style='color:#ff5500;'>    W_2&amp;W_3&amp;</span><span style='color:#3daee9;'>\mbox</span>{<span style='color:#3daee9;'>\colorbox</span>[rgb]{1,0.41,0.38}{<span style='color:#ff5500;'>$W_4$</span>}}<span style='color:#ff5500;'>&amp;W_5&amp;</span><span style='color:#3daee9;'>\dots</span><span style='color:#ff5500;'>&amp;W_{p+1}</span><span style='color:#3daee9;'>\\</span>
<span style='color:#ff5500;'>    W_3&amp;</span><span style='color:#3daee9;'>\mbox</span>{<span style='color:#3daee9;'>\colorbox</span>[rgb]{1,0.41,0.38}{<span style='color:#ff5500;'>$W_4$</span>}}<span style='color:#ff5500;'>&amp;W_5&amp;W_6&amp;</span><span style='color:#3daee9;'>\dots</span><span style='color:#ff5500;'>&amp;W_{p+2}</span><span style='color:#3daee9;'>\\</span>
<span style='color:#ff5500;'>    </span><span style='color:#3daee9;'>\mbox</span>{<span style='color:#3daee9;'>\colorbox</span>[rgb]{1,0.41,0.38}{<span style='color:#ff5500;'>$W_4$</span>}}<span style='color:#ff5500;'>&amp;W_5&amp;W_6&amp;W_7&amp;</span><span style='color:#3daee9;'>\dots</span><span style='color:#ff5500;'>&amp;W_{p+3}</span><span style='color:#3daee9;'>\\</span>
<span style='color:#ff5500;'>    </span><span style='color:#3daee9;'>\vdots</span><span style='color:#ff5500;'>&amp;</span><span style='color:#3daee9;'>\vdots</span><span style='color:#ff5500;'>&amp;</span><span style='color:#3daee9;'>\vdots</span><span style='color:#ff5500;'>&amp;</span><span style='color:#3daee9;'>\vdots</span><span style='color:#ff5500;'>&amp;</span><span style='color:#3daee9;'>\ddots</span><span style='color:#ff5500;'>&amp;</span><span style='color:#3daee9;'>\vdots\\</span>
<span style='color:#ff5500;'>    W_q&amp;W_{q+1}&amp;W_{q+2}&amp;W_{q+3}&amp;</span><span style='color:#3daee9;'>\dots</span><span style='color:#ff5500;'>&amp;W_n</span>
<span style='color:#ff5500;'>    </span><b>\end</b>{<b><span style='color:#0095ff;'>pmatrix</span></b>} 
<span style='color:#644a9b;'>\fes</span>
Bajo la descomposición en valores singulares podemos expresar esta matriz de Hankel de la siguiente manera
<span style='color:#644a9b;'>\ies</span>
<span style='color:#644a9b;'>\mathscr</span>{H}_{q,p}(W_n)=U<span style='color:#644a9b;'>\Sigma</span>_s V^T
<span style='color:#644a9b;'>\fes</span>
donde <span style='color:#ff5500;'>$s$</span> es el número total de valores singulares en la diagonal principal de <span style='color:#ff5500;'>$</span><span style='color:#3daee9;'>\Sigma</span><span style='color:#ff5500;'>$</span>. Si tomamos <span style='color:#ff5500;'>$</span><span style='color:#3daee9;'>\Sigma</span><span style='color:#ff5500;'>_{s-r}$</span>, la matriz se obtiene al retirar<span style='color:#ff5500;'>$ r$</span> valores singulares a <span style='color:#ff5500;'>$</span><span style='color:#3daee9;'>\Sigma</span><span style='color:#ff5500;'>_s$</span>, entonces debido a <b>\citet</b>{<b><span style='color:#0095ff;'>eckart</span></b>} la matriz truncada <span style='color:#ff5500;'>$</span><span style='color:#3daee9;'>\mathscr</span><span style='color:#ff5500;'>{H}_{r}=U</span><span style='color:#3daee9;'>\Sigma</span><span style='color:#ff5500;'>_{s-r} V^T$</span> bajo la norma de Frobenius o por <b>\citet</b>{<b><span style='color:#0095ff;'>mirsky</span></b>} para cualquier norma invariante unitariamente,  satisface que
<span style='color:#644a9b;'>\ies</span>
<span style='color:#644a9b;'>\norm</span>{<span style='color:#644a9b;'>\mathscr</span>{H}_{r}-<span style='color:#644a9b;'>\mathscr</span>{H}_{q,p}(W_n)}=<span style='color:#644a9b;'>\inf\limits</span>_{rank(H)<span style='color:#644a9b;'>\leq</span> s-r}<span style='color:#644a9b;'>\norm</span>{H-<span style='color:#644a9b;'>\mathscr</span>{H}_{q,p}(W_n)}
<span style='color:#644a9b;'>\fes</span>
No obstante, una matriz truncada de esta forma generalmente tiene sus entradas distintas a la matriz original. En nuestro caso, <span style='color:#ff5500;'>$</span><span style='color:#3daee9;'>\mathscr</span><span style='color:#ff5500;'>{H}_{s-r}$</span> pierde la homogeneidad en sus bloques antidiagonales. Sin embargo, <b>\citet</b>{<b><span style='color:#0095ff;'>gohost</span></b>} indica una forma para poder preservar parte de la estructura de la matriz original. Se expresa la matriz que se quiere truncar de la siguiente manera 
<span style='color:#644a9b;'>\ies</span>
<span style='color:#644a9b;'>\mathscr</span>{H}_{q,p}(W_n)=<b>\begin</b>{<b><span style='color:#0095ff;'>pmatrix</span></b>}
<span style='color:#ff5500;'>H_1&amp;H_2                                                                                                                                                                                                                                                                                                                                                                                                            </span><b>\end</b>{<b><span style='color:#0095ff;'>pmatrix</span></b>}
<span style='color:#644a9b;'>\fes</span>
donde las columnas de <span style='color:#ff5500;'>$H_1$</span> son <span style='color:#ff5500;'>$s-r$</span> y las de <span style='color:#ff5500;'>$H_2$</span> son <span style='color:#ff5500;'>$r$</span>. Posteriormente, sea <span style='color:#ff5500;'>$l=rank(H_1)$</span>; <span style='color:#ff5500;'>$P$</span> es la proyección ortogonal sobre el espacio de columnas de <span style='color:#ff5500;'>$</span><span style='color:#3daee9;'>\mathscr</span><span style='color:#ff5500;'>{H}_{q,p}(W_n)$</span>; y <span style='color:#ff5500;'>$K_{s}$</span> es el operador que toma una matriz <span style='color:#ff5500;'>$X$</span> y la mapea en <span style='color:#ff5500;'>$X_r$</span>, la matriz truncada en <span style='color:#ff5500;'>$r$</span> valores singulares. Entonces, si <span style='color:#ff5500;'>$l</span><span style='color:#3daee9;'>\leq</span><span style='color:#ff5500;'> r$</span> y 
<span style='color:#644a9b;'>\ies</span>
<span style='color:#644a9b;'>\hat</span>{H}_2=PH_2+K_{r-l}(P^{<span style='color:#644a9b;'>\bot</span>}H_2) 
<span style='color:#644a9b;'>\fes</span>
tenemos que <span style='color:#ff5500;'>$</span><span style='color:#3daee9;'>\mathscr</span><span style='color:#ff5500;'>{</span><span style='color:#3daee9;'>\hat</span><span style='color:#ff5500;'>{H}}_{s-r}=</span><b>\begin</b>{<b><span style='color:#0095ff;'>pmatrix</span></b>}<span style='color:#ff5500;'>H_1&amp;</span><span style='color:#3daee9;'>\hat</span><span style='color:#ff5500;'>{H}_2</span><b>\end</b>{<b><span style='color:#0095ff;'>pmatrix</span></b>}<span style='color:#ff5500;'>$</span> satisface
<span style='color:#644a9b;'>\ies</span>
<span style='color:#644a9b;'>\norm</span>{<span style='color:#644a9b;'>\mathscr</span>{<span style='color:#644a9b;'>\hat</span>{H}}_{r}-<span style='color:#644a9b;'>\mathscr</span>{H}_{q,p}(W_n)}=<span style='color:#644a9b;'>\inf\limits</span>_{rank(H)<span style='color:#644a9b;'>\leq</span> s-r}<span style='color:#644a9b;'>\norm</span>{H-<span style='color:#644a9b;'>\mathscr</span>{H}_{q,p}(W_n)}
<span style='color:#644a9b;'>\fes</span>
así, la matriz <span style='color:#ff5500;'>$</span><span style='color:#3daee9;'>\mathscr</span><span style='color:#ff5500;'>{</span><span style='color:#3daee9;'>\hat</span><span style='color:#ff5500;'>{H}}_{r}$</span> preserva la parte <span style='color:#ff5500;'>$H_1$</span>.<span style='color:#644a9b;'>\-</span>

<span style='color:#644a9b;'>\textbf</span>{Sistema Tiempo-Variante}<span style='color:#644a9b;'>\-</span>

Asúmase <span style='color:#ff5500;'>$</span><span style='color:#3daee9;'>\{</span><span style='color:#ff5500;'>L_i</span><span style='color:#3daee9;'>\}</span><span style='color:#ff5500;'>_{</span><span style='color:#3daee9;'>\N</span><span style='color:#ff5500;'>_k}$</span>, una sucesión de matrices de Leontief con <span style='color:#ff5500;'>$L_i</span><span style='color:#3daee9;'>\in</span><span style='color:#ff5500;'> </span><span style='color:#3daee9;'>\textbf</span>{GL}<span style='color:#ff5500;'>_n(</span><span style='color:#3daee9;'>\R</span><span style='color:#ff5500;'>)$</span>, el grupo general lineal de matrices de rango <span style='color:#ff5500;'>$n$</span>. Ahora, a partir de estas matrices podemos establecer su dinámica como sigue
<span style='color:#644a9b;'>\ies</span>
L_2&amp;=&amp;f_1(L_1)<span style='color:#644a9b;'>\\</span>
L_3&amp;=&amp;f_2(L_2)<span style='color:#644a9b;'>\\</span>
<span style='color:#644a9b;'>\vdots</span>&amp;=&amp;<span style='color:#644a9b;'>\vdots\\</span>
L_k&amp;=&amp;f_{k-1}(L_{k-1})
<span style='color:#644a9b;'>\fes</span>
Ahora, si <span style='color:#ff5500;'>$f_i$</span> fuese lineal, tendriamos que
<span style='color:#644a9b;'>\ies</span>
L_2&amp;=&amp;A_1L_1<span style='color:#644a9b;'>\\</span>
L_3&amp;=&amp;A_2L_2<span style='color:#644a9b;'>\\</span>
<span style='color:#644a9b;'>\vdots</span>&amp;=&amp;<span style='color:#644a9b;'>\vdots\\</span>
L_k&amp;=&amp;A_{k-1}L_{k-1}
<span style='color:#644a9b;'>\fes</span>
donde <span style='color:#ff5500;'>$A_i</span><span style='color:#3daee9;'>\in\textbf</span><span style='color:#ff5500;'>{GL}_n(</span><span style='color:#3daee9;'>\R</span><span style='color:#ff5500;'>)$</span> queda automáticamente determinada por la sucesión <span style='color:#ff5500;'>$</span><span style='color:#3daee9;'>\{</span><span style='color:#ff5500;'>L_i</span><span style='color:#3daee9;'>\}</span><span style='color:#ff5500;'>_{</span><span style='color:#3daee9;'>\N</span><span style='color:#ff5500;'>_k}$</span>,
<span style='color:#644a9b;'>\ies</span>
L_1^{-1}L_2&amp;=&amp;A_1<span style='color:#644a9b;'>\\</span>
L_2^{-1}L_3&amp;=&amp;A_2<span style='color:#644a9b;'>\\</span>
<span style='color:#644a9b;'>\vdots</span>&amp;=&amp;<span style='color:#644a9b;'>\vdots\\</span>
L_{k-1}^{-1}L_k&amp;=&amp;A_{k-1} 
<span style='color:#644a9b;'>\fes</span>

Así, tenemos una nueva sucesión de matrices a la cual atribuirle una noción de cambio de una estructura de Leontief a otra, <span style='color:#ff5500;'>$</span><span style='color:#3daee9;'>\{</span><span style='color:#ff5500;'>A_i</span><span style='color:#3daee9;'>\}</span><span style='color:#ff5500;'>_{</span><span style='color:#3daee9;'>\N</span><span style='color:#ff5500;'>_{k-1}}$</span>. Ahora, observemos que para todo <span style='color:#ff5500;'>$j&gt;1$</span> se puede expresar <span style='color:#ff5500;'>$L_j$</span> en función <span style='color:#ff5500;'>$L_1$</span> de la siguiente manera
<span style='color:#644a9b;'>\ies</span>
L_j=<span style='color:#644a9b;'>\left</span>(<span style='color:#644a9b;'>\prod</span>_{r=1}^{j-1}A_r<span style='color:#644a9b;'>\right</span>)L_1
<span style='color:#644a9b;'>\fes</span>
Si en esta última igualdad tuvieramos que <span style='color:#ff5500;'>$A_i=A</span><span style='color:#3daee9;'>\in\M</span><span style='color:#ff5500;'>_n$</span> para toda <span style='color:#ff5500;'>$i</span><span style='color:#3daee9;'>\in\N</span><span style='color:#ff5500;'>_n$</span>, entonces podriamos expresar la dinámica de las <span style='color:#ff5500;'>$L_i$</span> como sigue
<span style='color:#644a9b;'>\ies</span>
L_j=A^{j-1}L_1 
<span style='color:#644a9b;'>\fes</span>
Sin embargo, proverbialmente los datos de las <span style='color:#ff5500;'>$L_i$</span> no permiten tener una tal matriz <span style='color:#ff5500;'>$A$</span>. Una forma de solventar este tema es generar una aproximación mediante la TR. En vez de determinar una <span style='color:#ff5500;'>$A$</span> constante, construiremos matrices <span style='color:#ff5500;'>$F$</span>, <span style='color:#ff5500;'>$G$</span> y <span style='color:#ff5500;'>$H$</span> constantes tales que <span style='color:#ff5500;'>$HF^{s}G</span><span style='color:#3daee9;'>\approx</span><span style='color:#ff5500;'> A_{s+1}$</span>; obtenemos de esta manera una aproximación a la dinámica de las <span style='color:#ff5500;'>$L_i$</span>.


Nuevamente, para ejemplificar, empleamos las 100 matrices con que contamos. A partir de ellas, generamos la sucesión <span style='color:#ff5500;'>$</span><span style='color:#3daee9;'>\{</span><span style='color:#ff5500;'>A_n</span><span style='color:#3daee9;'>\}</span><span style='color:#ff5500;'>_{</span><span style='color:#3daee9;'>\N</span><span style='color:#ff5500;'>_{99}}$</span> (Tabla <b>\ref</b>{}), y a partir de aplicar el algoritmo sobre esta, se obtiene que la mejor aproximación bajo la métrica de Frobenius es 
<span style='color:#644a9b;'>\ies</span>
<span style='color:#644a9b;'>\{</span>A_n<span style='color:#644a9b;'>\}</span>_{<span style='color:#644a9b;'>\N</span>_{99}}=<span style='color:#644a9b;'>\{</span>I_2<span style='color:#644a9b;'>\}</span>_{<span style='color:#644a9b;'>\N</span>_{99}}
<span style='color:#644a9b;'>\fes</span>
es decir, la mejor aproximación es la sucesión  constante con la identidad como elemento. Esto quiere decir que bajo el método usado, no existe un cambio significativo entre las matrices <span style='color:#ff5500;'>$L_i$</span>, cualquiera puede representar a todo el periodo.


<b>\section</b>{<b>Análisis de un sistema variante temporal</b>}

<b>\section</b>{<b>Realización por Acortamiento de Series</b>}

La propuesta de realización por acortamiento de series RAS se resume en la siguiente tabla:

<b>\begin</b>{<b><span style='color:#0095ff;'>figure</span></b>}[H]<span style='color:#644a9b;'>\centering</span>
<span style='color:#644a9b;'>\caption</span>{<b>\label</b>{<b><span style='color:#0095ff;'>MVRAS</span></b>} Método RAS}
<span style='color:#ff5500;'>$</span><b>\begin</b>{<b><span style='color:#0095ff;'>matrix</span></b>}
<span style='color:#ff5500;'>L_n&amp;</span><span style='color:#3daee9;'>\hat</span><span style='color:#ff5500;'>{L}_{n,1}&amp;</span><span style='color:#3daee9;'>\hat</span><span style='color:#ff5500;'>{L}_{n-1,2}&amp;</span><span style='color:#3daee9;'>\cdots</span><span style='color:#ff5500;'>&amp;</span><span style='color:#3daee9;'>\hat</span><span style='color:#ff5500;'>{L}_{2,n-1}</span>
<span style='color:#ff5500;'>&amp;(F'_{_{n}},G'_{_{n}},H'_{_{n}})</span><span style='color:#3daee9;'>\\</span>
<span style='color:#ff5500;'>L_{n-1}&amp;</span><span style='color:#3daee9;'>\hat</span><span style='color:#ff5500;'>{L}_{{n-1},1}&amp;</span><span style='color:#3daee9;'>\hat</span><span style='color:#ff5500;'>{L}_{n-2,2}&amp;</span><span style='color:#3daee9;'>\cdots</span><span style='color:#ff5500;'>&amp;</span><span style='color:#3daee9;'>\hat</span><span style='color:#ff5500;'>{L}_{1,n-1}&amp;(F'_{_{n-1}},G'_{_{n-1}},H'_{_{n-1}})</span><span style='color:#3daee9;'>\\</span>
<span style='color:#ff5500;'>L_{n-2}&amp;</span><span style='color:#3daee9;'>\hat</span><span style='color:#ff5500;'>{L}_{{n-2},1}&amp;</span><span style='color:#3daee9;'>\hat</span><span style='color:#ff5500;'>{L}_{n-3,2}&amp;</span><span style='color:#3daee9;'>\cdots</span><span style='color:#ff5500;'>&amp;{</span><span style='color:#3daee9;'>\color</span><span style='color:#ff5500;'>{red}</span><span style='color:#3daee9;'>\blacksquare</span><span style='color:#ff5500;'>}&amp;(F'_{_{n-2}},G'_{_{n-2}},H'_{_{n-2}})</span><span style='color:#3daee9;'>\\</span>
<span style='color:#3daee9;'>\vdots</span><span style='color:#ff5500;'>&amp;</span><span style='color:#3daee9;'>\vdots</span><span style='color:#ff5500;'>&amp;</span><span style='color:#3daee9;'>\vdots</span><span style='color:#ff5500;'>&amp;{</span><span style='color:#3daee9;'>\color</span><span style='color:#ff5500;'>{red}</span><span style='color:#3daee9;'>\cdots</span><span style='color:#ff5500;'>}&amp;{</span><span style='color:#3daee9;'>\color</span><span style='color:#ff5500;'>{red}</span><span style='color:#3daee9;'>\blacksquare</span><span style='color:#ff5500;'>}&amp;</span><span style='color:#3daee9;'>\vdots\\</span>
<span style='color:#ff5500;'>L_2&amp;</span><span style='color:#3daee9;'>\hat</span><span style='color:#ff5500;'>{L}_{2,1}&amp;</span><span style='color:#3daee9;'>\hat</span><span style='color:#ff5500;'>{L}_{1,2}&amp;{</span><span style='color:#3daee9;'>\color</span><span style='color:#ff5500;'>{red}</span><span style='color:#3daee9;'>\cdots</span><span style='color:#ff5500;'>}&amp;{</span><span style='color:#3daee9;'>\color</span><span style='color:#ff5500;'>{red}</span><span style='color:#3daee9;'>\blacksquare</span><span style='color:#ff5500;'>}&amp;(F'_2,G'_2,H'_2)</span><span style='color:#3daee9;'>\\</span>
<span style='color:#ff5500;'>L_1&amp;</span><span style='color:#3daee9;'>\hat</span><span style='color:#ff5500;'>{L}_{1,1}&amp;{</span><span style='color:#3daee9;'>\color</span><span style='color:#ff5500;'>{red}</span><span style='color:#3daee9;'>\blacksquare</span><span style='color:#ff5500;'>}&amp;{</span><span style='color:#3daee9;'>\color</span><span style='color:#ff5500;'>{red}</span><span style='color:#3daee9;'>\cdots</span><span style='color:#ff5500;'>}&amp;{</span><span style='color:#3daee9;'>\color</span><span style='color:#ff5500;'>{red}</span><span style='color:#3daee9;'>\blacksquare</span><span style='color:#ff5500;'>}&amp;(F'_1,G'_1,H'_1)</span><span style='color:#3daee9;'>\\</span><span style='color:#ff5500;'>[.1cm]</span><span style='color:#3daee9;'>\hline\\</span><span style='color:#ff5500;'>[-.37cm]</span>
<span style='color:#ff5500;'>&amp;(F_1,G_1,H_1)&amp;(F_2,G_2,H_2)&amp;</span><span style='color:#3daee9;'>\cdots</span><span style='color:#ff5500;'>&amp;(F_{_{n-1}},G_{_{n-1}},H_{_{n-1}})&amp; _{[(F_{_{_{F'}}},G_{_{_{F'}}},H_{_{_{F'}}}),(F_{_{_{G'}}},G_{_{_{G'}}},H_{_{_{G'}}}),(F_{_{_{H'}}},G_{_{_{H'}}},H_{_{_{H'}}})]}</span>
<b>\end</b>{<b><span style='color:#0095ff;'>matrix</span></b>}<span style='color:#ff5500;'>$</span>
<b>\end</b>{<b><span style='color:#0095ff;'>figure</span></b>}  

 
Para efectuar esta propuesta de realización variante, tomando una DE  <span style='color:#ff5500;'>$L_1, </span><span style='color:#3daee9;'>\dots</span><span style='color:#ff5500;'>, L_n$</span>, debemos empezar a realizar los siguientes pasos:  
<b>\begin</b>{<b><span style='color:#0095ff;'>enumerate</span></b>}
 <span style='color:#644a9b;'>\item</span> Tomando <span style='color:#ff5500;'>$m=1$</span> hacemos la realización <span style='color:#ff5500;'>$(F_m,G_m,H_m)$</span> de la DE;
 <span style='color:#644a9b;'>\item</span> Creamos la descripción externa acortada DE<span style='color:#ff5500;'>$_{-m}$</span> retirando los primeros <span style='color:#ff5500;'>$m$</span> elementos de DE;
 <span style='color:#644a9b;'>\item</span> Hacemos la realización <span style='color:#ff5500;'>$(F_{_{m+1}},G_{_{m+1}},H_{_{m+1}})$</span> de DE<span style='color:#ff5500;'>$_{-m}$</span>;
 <span style='color:#644a9b;'>\item</span> Si <span style='color:#ff5500;'>$m&lt;n-1$</span>, tomamos <span style='color:#ff5500;'>$m=m+1$</span> y repetimos desde el paso 2, de otro modo terminamos el proceso.
<b>\end</b>{<b><span style='color:#0095ff;'>enumerate</span></b>}
Este proceso crea las realizaciones <span style='color:#ff5500;'>$(F_t,G_t,H_t)$</span> en la base de la figura <b>\ref</b>{<b><span style='color:#0095ff;'>MVRAS</span></b>}, y a partir de ellas podemos obtener la estimación de las sucesiones acortadas como se ve en la misma figura. Todas las sucesiones para una misma realización <span style='color:#ff5500;'>$(F_t,G_t,H_t)$</span> son longitudinales (verticales), indican un estado en el tiempo, pero para el siguiente proceso vamos a considerar las sucesiones generadas con las matrices estimadas y la DE original en un mismo tiempo, en corte transversal, es decir, las sucesiones formadas por las filas en la figura <b>\ref</b>{<b><span style='color:#0095ff;'>MVRAS</span></b>}: <span style='color:#ff5500;'>$</span><span style='color:#3daee9;'>\hat</span><span style='color:#ff5500;'>{L}_{1,k},</span><span style='color:#3daee9;'>\hat</span><span style='color:#ff5500;'>{L}_{2,k-1},</span><span style='color:#3daee9;'>\dots</span><span style='color:#ff5500;'>,</span><span style='color:#3daee9;'>\hat</span><span style='color:#ff5500;'>{L}_{k,1},L_{k}$</span>. Estas sucesiones también pueden ser realizadas, y en cada una se obtienen las realizaciones en la última columna de la figura <b>\ref</b>{<b><span style='color:#0095ff;'>MVRAS</span></b>}: <span style='color:#ff5500;'>$(F'_1, G'_1, H'_1),</span><span style='color:#3daee9;'>\dots</span><span style='color:#ff5500;'>,(F'_{_{n-1}}, G'_{_{n-1}}, H'_{_{n-1}}),(F'_n, G'_n, H'_n)$</span>.<span style='color:#644a9b;'>\-</span>

A partir de las realizaciones de corte transversal podemos generar tres sucesiones <span style='color:#ff5500;'>$F'_1,</span><span style='color:#3daee9;'>\dots</span><span style='color:#ff5500;'>,F'_{_{n-1}},F'_n$</span>; <span style='color:#ff5500;'>$G'_1,</span><span style='color:#3daee9;'>\dots</span><span style='color:#ff5500;'>,G'_{_{n-1}},G'_n$</span> y <span style='color:#ff5500;'>$H'_1,</span><span style='color:#3daee9;'>\dots</span><span style='color:#ff5500;'>,H'_{_{n-1}},H'_n$</span>, y a ellas les podemos encontrar su realización, las cuales se pueden ver en la esquina inferior derecha de la figura <b>\ref</b>{<b><span style='color:#0095ff;'>MVRAS</span></b>}, donde <span style='color:#ff5500;'>$(F_{_{_{F'}}},G_{_{_{F'}}},H_{_{_{F'}}})$</span> es la realización de la sucesión <span style='color:#ff5500;'>$F'_T$</span>; <span style='color:#ff5500;'>$(F_{_{_{G'}}},G_{_{_{G'}}},H_{_{_{G'}}})$</span> de la sucesión <span style='color:#ff5500;'>$G'_T$</span>; y <span style='color:#ff5500;'>$(F_{_{_{H'}}},G_{_{_{H'}}},H_{_{_{H'}}})$</span> de la sucesión <span style='color:#ff5500;'>$H'_T$</span>. A partir de estas realizaciones nosotros sabemos que 
<span style='color:#644a9b;'>\ies</span>
F'_i=H_{_{F'}}F_{_{F'}}^{i-1}G_{_{F'}}<span style='color:#644a9b;'>\hspace</span>{1cm} 
G'_i=H_{_{G'}}F_{_{G'}}^{i-1}G_{_{G'}}<span style='color:#644a9b;'>\hspace</span>{1cm} 
H'_i=H_{_{H'}}F_{_{H'}}^{i-1}G_{_{H'}}
<span style='color:#644a9b;'>\fes</span>
Además, como <span style='color:#ff5500;'>$L_t=H'_t(F'_t)^{t}G'_t$</span> si <span style='color:#ff5500;'>$t&lt;n$</span>; y <span style='color:#ff5500;'>$L_t=H'_t(F'_t)^{t-1}G'_t$</span>, se tiene que
<span style='color:#644a9b;'>\ie\label</span>{igualdad_ras}<span style='color:#644a9b;'>\\</span>
<span style='color:#644a9b;'>\nonumber</span> L_t=<span style='color:#644a9b;'>\left\{</span><b>\begin</b>{<b><span style='color:#0095ff;'>matrix</span></b>}
            H_{_{H'}}F_{_{H'}}^{t-1}G_{_{H'}}(H_{_{F'}}F_{_{F'}}^{t-1}G_{_{F'}})^{t}H_{_{G'}}F_{_{G'}}^{t-1}G_{_{G'}}&amp;si&amp; 1<span style='color:#644a9b;'>\leq</span> t&lt;n<span style='color:#644a9b;'>\\</span>
            <span style='color:#644a9b;'>\ \\</span>
<span style='color:#644a9b;'>\nonumber</span> H_{_{H'}}F_{_{H'}}^{t-1}G_{_{H'}}(H_{_{F'}}F_{_{F'}}^{t-1}G_{_{F'}})^{t-1}H_{_{G'}}F_{_{G'}}^{t-1}G_{_{G'}}&amp;si&amp;t=n
           <b>\end</b>{<b><span style='color:#0095ff;'>matrix</span></b>}<span style='color:#644a9b;'>\right</span>.<span style='color:#644a9b;'>\\</span>
<span style='color:#644a9b;'>\fe</span>
y si escribimos <span style='color:#ff5500;'>$F(t)=H_{_{F'}}F_{_{F'}}^{t-1}G_{_{F'}}$</span>; <span style='color:#ff5500;'>$G(t)=H_{_{G'}}F_{_{G'}}^{t-1}G_{_{G'}}$</span> ; y <span style='color:#ff5500;'>$H(t)=H_{_{H'}}F_{_{H'}}^{t-1}G_{_{H'}}$</span>, y tomando <span style='color:#ff5500;'>$</span><span style='color:#3daee9;'>\delta</span><span style='color:#ff5500;'>(t-n)=1$</span> si <span style='color:#ff5500;'>$t=n$</span> y  <span style='color:#ff5500;'>$</span><span style='color:#3daee9;'>\delta</span><span style='color:#ff5500;'>(t-n)=0$</span> si <span style='color:#ff5500;'>$t</span><span style='color:#3daee9;'>\neq</span><span style='color:#ff5500;'>0$</span> que la igualdad <b>\eqref</b>{<b><span style='color:#0095ff;'>igualdad_ras</span></b>} queda como
<span style='color:#644a9b;'>\ies</span>
L_t=H(t)F(t)^{t-<span style='color:#644a9b;'>\delta</span>(t-n)}H(t),<span style='color:#644a9b;'>\ </span> <span style='color:#644a9b;'>\ </span> <span style='color:#644a9b;'>\ </span> t=1,<span style='color:#644a9b;'>\dots</span>,n
<span style='color:#644a9b;'>\fes</span>
así, tenemos tres matrices <span style='color:#ff5500;'>$(F(t),G(t),H(t))$</span>, tiempo dependientes, que estiman el comportamiento de la DE; esta terna es la propuesta de realización tiempo variante.


<b>\section</b>{<b>Matriz de rotación y factor de escalamiento</b>}

Bajo un modelo de la forma 
<span style='color:#644a9b;'>\ies</span> <span style='color:#644a9b;'>\footnotesize</span>
<span style='color:#644a9b;'>\vec</span>{y}=<span style='color:#644a9b;'>\left</span>(<b>\begin</b>{<b><span style='color:#0095ff;'>array</span></b>}{c}
y_1<span style='color:#644a9b;'>\\</span>
y_2
<b>\end</b>{<b><span style='color:#0095ff;'>array</span></b>}<span style='color:#644a9b;'>\right</span>)=<span style='color:#644a9b;'>\frac</span>{1}{<span style='color:#644a9b;'>\abs</span>{A}}<b>\begin</b>{<b><span style='color:#0095ff;'>pmatrix</span></b>}<span style='color:#ff5500;'> </span>
<span style='color:#ff5500;'>a_{22}&amp;-a_{21}</span><span style='color:#3daee9;'>\\</span>
<span style='color:#ff5500;'>-a_{12}&amp;a_{11}</span>
<b>\end</b>{<b><span style='color:#0095ff;'>pmatrix</span></b>}<span style='color:#644a9b;'>\left</span>(<b>\begin</b>{<b><span style='color:#0095ff;'>array</span></b>}{c}
x_1<span style='color:#644a9b;'>\\</span>
x_2
<b>\end</b>{<b><span style='color:#0095ff;'>array</span></b>}<span style='color:#644a9b;'>\right</span>)=<span style='color:#644a9b;'>\frac</span>{1}{<span style='color:#644a9b;'>\abs</span>{A}}adj(A)^T<span style='color:#644a9b;'>\vec</span>{x}
<span style='color:#644a9b;'>\fes</span>
podemos ver que el vector <span style='color:#ff5500;'>$</span><span style='color:#3daee9;'>\vec</span><span style='color:#ff5500;'>{x}$</span> es transformado en el  vector <span style='color:#ff5500;'>$</span><span style='color:#3daee9;'>\vec</span><span style='color:#ff5500;'>{y}$</span> mediante una deformación dada por la matriz <span style='color:#ff5500;'>$adj(A)^T$</span> y el escalamiento generado por <span style='color:#ff5500;'>$</span><span style='color:#3daee9;'>\frac</span><span style='color:#ff5500;'>{1}{</span><span style='color:#3daee9;'>\abs</span><span style='color:#ff5500;'>{A}}$</span>. No obstante, otra forma de ver este proceso es que dicha matriz y el factor de escalamiento no deforman el vector, si no al espacio circundante, rotándolo y/o dilatándolo (contrayéndolo). Por consiguiente, si el espacio es deformado mediante la matriz de rotación <span style='color:#ff5500;'>$R(</span><span style='color:#3daee9;'>\theta</span><span style='color:#ff5500;'>)$</span> y el factor de escalamiento <span style='color:#ff5500;'>$</span><span style='color:#3daee9;'>\Delta</span><span style='color:#ff5500;'> e$</span>, entonces el modelo anterior lo podemos ver también como
<span style='color:#644a9b;'>\ie\label</span>{mod_ER} <span style='color:#644a9b;'>\footnotesize</span>
<span style='color:#644a9b;'>\vec</span>{y}=<span style='color:#644a9b;'>\Delta</span> eR(<span style='color:#644a9b;'>\theta</span>) <span style='color:#644a9b;'>\vec</span>{x}=<span style='color:#644a9b;'>\Delta</span> e<b>\begin</b>{<b><span style='color:#0095ff;'>pmatrix</span></b>}<span style='color:#ff5500;'> </span>
<span style='color:#3daee9;'>\cos\theta</span><span style='color:#ff5500;'>&amp;-</span><span style='color:#3daee9;'>\sin\theta\\</span>
<span style='color:#3daee9;'>\sin\theta</span><span style='color:#ff5500;'>&amp;</span><span style='color:#3daee9;'>\cos\theta</span>
<b>\end</b>{<b><span style='color:#0095ff;'>pmatrix</span></b>}<span style='color:#644a9b;'>\vec</span>{x}
<span style='color:#644a9b;'>\fe</span>  
ahora, si tenemos una serie de vectores <span style='color:#ff5500;'>$</span><span style='color:#3daee9;'>\vec</span><span style='color:#ff5500;'>{x}_t$</span>, <span style='color:#ff5500;'>$</span><span style='color:#3daee9;'>\vec</span><span style='color:#ff5500;'>{y}_t$</span> para cada tiempo <span style='color:#ff5500;'>$t$</span>, tamién existe un ángulo <span style='color:#ff5500;'>$</span><span style='color:#3daee9;'>\theta</span><span style='color:#ff5500;'>_t$</span> entre dichos vectores asociado al tiempo <span style='color:#ff5500;'>$t$</span>, el cual podemos cálcularlo mediante la expresión 
<span style='color:#644a9b;'>\ies\footnotesize</span>
<span style='color:#644a9b;'>\theta</span>_t=<span style='color:#644a9b;'>\frac</span>{<span style='color:#644a9b;'>\vec</span>{x}<span style='color:#644a9b;'>\bullet\vec</span>{y}}{<span style='color:#644a9b;'>\norm</span>{<span style='color:#644a9b;'>\vec</span>{x}}<span style='color:#644a9b;'>\norm</span>{<span style='color:#644a9b;'>\vec</span>{y}}}
<span style='color:#644a9b;'>\fes</span>
Luego, si obtenemos la realización <span style='color:#ff5500;'>$(F_</span><span style='color:#3daee9;'>\theta</span><span style='color:#ff5500;'>, G_</span><span style='color:#3daee9;'>\theta</span><span style='color:#ff5500;'>, H_</span><span style='color:#3daee9;'>\theta</span><span style='color:#ff5500;'>)$</span> de esta serie de ángulos (que tienen ruido), podemos expresar su dinámica como
<span style='color:#644a9b;'>\ies</span>
<span style='color:#644a9b;'>\theta</span>_t<span style='color:#644a9b;'>\approx</span> H_<span style='color:#644a9b;'>\theta</span> F^{t-1}_<span style='color:#644a9b;'>\theta</span> G_<span style='color:#644a9b;'>\theta</span>
<span style='color:#644a9b;'>\fes</span>
y por consiguiente <b>\eqref</b>{<b><span style='color:#0095ff;'>mod_ER</span></b>} puede ser aproximado de la siguiente manera
<span style='color:#644a9b;'>\ies</span>
<span style='color:#644a9b;'>\footnotesize</span> 
<span style='color:#644a9b;'>\left</span>(<b>\begin</b>{<b><span style='color:#0095ff;'>array</span></b>}{c}
Y_t<span style='color:#644a9b;'>\\</span>
M_t
<b>\end</b>{<b><span style='color:#0095ff;'>array</span></b>}<span style='color:#644a9b;'>\right</span>)<span style='color:#644a9b;'>\approx\Delta</span> e_t<b>\begin</b>{<b><span style='color:#0095ff;'>pmatrix</span></b>}<span style='color:#ff5500;'> </span>
<span style='color:#3daee9;'>\cos</span><span style='color:#ff5500;'>(H_</span><span style='color:#3daee9;'>\theta</span><span style='color:#ff5500;'> F^{t-1}_</span><span style='color:#3daee9;'>\theta</span><span style='color:#ff5500;'> G_</span><span style='color:#3daee9;'>\theta</span><span style='color:#ff5500;'>)&amp;-</span><span style='color:#3daee9;'>\sin</span><span style='color:#ff5500;'>(H_</span><span style='color:#3daee9;'>\theta</span><span style='color:#ff5500;'> F^{t-1}_</span><span style='color:#3daee9;'>\theta</span><span style='color:#ff5500;'> G_</span><span style='color:#3daee9;'>\theta</span><span style='color:#ff5500;'>)</span><span style='color:#3daee9;'>\\</span><span style='color:#ff5500;'> </span><span style='color:#3daee9;'>\\</span>
<span style='color:#3daee9;'>\sin</span><span style='color:#ff5500;'>(H_</span><span style='color:#3daee9;'>\theta</span><span style='color:#ff5500;'> F^{t-1}_</span><span style='color:#3daee9;'>\theta</span><span style='color:#ff5500;'> G_</span><span style='color:#3daee9;'>\theta</span><span style='color:#ff5500;'>)&amp;</span><span style='color:#3daee9;'>\cos</span><span style='color:#ff5500;'>(H_</span><span style='color:#3daee9;'>\theta</span><span style='color:#ff5500;'> F^{t-1}_</span><span style='color:#3daee9;'>\theta</span><span style='color:#ff5500;'> G_</span><span style='color:#3daee9;'>\theta</span><span style='color:#ff5500;'>)</span>
<b>\end</b>{<b><span style='color:#0095ff;'>pmatrix</span></b>}<span style='color:#644a9b;'>\left</span>(<b>\begin</b>{<b><span style='color:#0095ff;'>array</span></b>}{c}
F_t<span style='color:#644a9b;'>\\</span>
MK_t
<b>\end</b>{<b><span style='color:#0095ff;'>array</span></b>}<span style='color:#644a9b;'>\right</span>)
<span style='color:#644a9b;'>\fes</span>
así, la matriz de rotación ahora está en función del tiempo <span style='color:#ff5500;'>$R(t)=(R</span><span style='color:#3daee9;'>\circ\theta</span><span style='color:#ff5500;'>)(t)$</span>, es decir, tenemos una matriz variante temporal. Además, obteniendo la realización del factor de expansión <span style='color:#ff5500;'>$(F_</span><span style='color:#3daee9;'>\Delta</span><span style='color:#ff5500;'>,G_</span><span style='color:#3daee9;'>\Delta</span><span style='color:#ff5500;'>,H_</span><span style='color:#3daee9;'>\Delta</span><span style='color:#ff5500;'>)$</span>, se sigue que el modelo queda como
<span style='color:#644a9b;'>\ie\label</span>{mod_ER_1}  
<span style='color:#644a9b;'>\footnotesize</span> 
<span style='color:#644a9b;'>\left</span>(<b>\begin</b>{<b><span style='color:#0095ff;'>array</span></b>}{c}
Y_t<span style='color:#644a9b;'>\\</span>
M_t
<b>\end</b>{<b><span style='color:#0095ff;'>array</span></b>}<span style='color:#644a9b;'>\right</span>)<span style='color:#644a9b;'>\approx</span> H_<span style='color:#644a9b;'>\Delta</span> F^{t-1}_<span style='color:#644a9b;'>\Delta</span> G_<span style='color:#644a9b;'>\Delta\begin</span>{pmatrix} 
<span style='color:#644a9b;'>\cos</span>(H_<span style='color:#644a9b;'>\theta</span> F^{t-1}_<span style='color:#644a9b;'>\theta</span> G_<span style='color:#644a9b;'>\theta</span>)&amp;-<span style='color:#644a9b;'>\sin</span>(H_<span style='color:#644a9b;'>\theta</span> F^{t-1}_<span style='color:#644a9b;'>\theta</span> G_<span style='color:#644a9b;'>\theta</span>)<span style='color:#644a9b;'>\\</span> <span style='color:#644a9b;'>\\</span>
<span style='color:#644a9b;'>\sin</span>(H_<span style='color:#644a9b;'>\theta</span> F^{t-1}_<span style='color:#644a9b;'>\theta</span> G_<span style='color:#644a9b;'>\theta</span>)&amp;<span style='color:#644a9b;'>\cos</span>(H_<span style='color:#644a9b;'>\theta</span> F^{t-1}_<span style='color:#644a9b;'>\theta</span> G_<span style='color:#644a9b;'>\theta</span>)
<b>\end</b>{<b><span style='color:#0095ff;'>pmatrix</span></b>}<span style='color:#644a9b;'>\left</span>(<b>\begin</b>{<b><span style='color:#0095ff;'>array</span></b>}{c}
F_t<span style='color:#644a9b;'>\\</span>
MK_t
<b>\end</b>{<b><span style='color:#0095ff;'>array</span></b>}<span style='color:#644a9b;'>\right</span>)
<span style='color:#644a9b;'>\fe</span>
Observemos que aquí se independizó la reconstrucción de la dinámica de <span style='color:#ff5500;'>$</span><span style='color:#3daee9;'>\theta</span><span style='color:#ff5500;'>$</span> de la dinámica de <span style='color:#ff5500;'>$</span><span style='color:#3daee9;'>\Delta</span><span style='color:#ff5500;'> e$</span>, no obstante, ambos números pertencen al mismo mecanismo subyacente que reconfigura el espacio circundante del vector de entrada, por lo que tratarlos como una pareja ordenada hace sentido. Ahora, si construimos la realización de está sucesión de parejas ordenadas  <span style='color:#ff5500;'>$(F_</span><span style='color:#3daee9;'>\alpha</span><span style='color:#ff5500;'>,G_</span><span style='color:#3daee9;'>\alpha</span><span style='color:#ff5500;'>,H_</span><span style='color:#3daee9;'>\alpha</span><span style='color:#ff5500;'>)$</span>, tendriamos que
<span style='color:#644a9b;'>\ies</span>
<b>\begin</b>{<b><span style='color:#0095ff;'>pmatrix</span></b>}
<span style='color:#ff5500;'> </span><span style='color:#3daee9;'>\theta</span><span style='color:#ff5500;'>_t</span><span style='color:#3daee9;'>\\</span>
<span style='color:#ff5500;'> </span><span style='color:#3daee9;'>\Delta</span><span style='color:#ff5500;'> e_t</span>
<b>\end</b>{<b><span style='color:#0095ff;'>pmatrix</span></b>}=H_<span style='color:#644a9b;'>\alpha</span> F^t_<span style='color:#644a9b;'>\alpha</span> G_<span style='color:#644a9b;'>\alpha</span>
=<b>\begin</b>{<b><span style='color:#0095ff;'>pmatrix</span></b>}
<span style='color:#ff5500;'>  H_{</span><span style='color:#3daee9;'>\alpha</span><span style='color:#ff5500;'>_1}</span><span style='color:#3daee9;'>\\</span>
<span style='color:#ff5500;'>  H_{</span><span style='color:#3daee9;'>\alpha</span><span style='color:#ff5500;'>_2}</span>
<span style='color:#ff5500;'> </span><b>\end</b>{<b><span style='color:#0095ff;'>pmatrix</span></b>}F^t_<span style='color:#644a9b;'>\alpha</span> G_<span style='color:#644a9b;'>\alpha</span>=<b>\begin</b>{<b><span style='color:#0095ff;'>pmatrix</span></b>}
<span style='color:#ff5500;'>  H_{</span><span style='color:#3daee9;'>\alpha</span><span style='color:#ff5500;'>_1}F^t_</span><span style='color:#3daee9;'>\alpha</span><span style='color:#ff5500;'> G_</span><span style='color:#3daee9;'>\alpha\\</span>
<span style='color:#ff5500;'>  H_{</span><span style='color:#3daee9;'>\alpha</span><span style='color:#ff5500;'>_2}F^t_</span><span style='color:#3daee9;'>\alpha</span><span style='color:#ff5500;'> G_</span><span style='color:#3daee9;'>\alpha</span>
<span style='color:#ff5500;'> </span><b>\end</b>{<b><span style='color:#0095ff;'>pmatrix</span></b>}
<span style='color:#644a9b;'>\fes</span>
y de esto se desprende que el modelo quedaría como
<span style='color:#644a9b;'>\ie\label</span>{mod_ER_2}
<span style='color:#644a9b;'>\footnotesize</span> 
<span style='color:#644a9b;'>\left</span>(<b>\begin</b>{<b><span style='color:#0095ff;'>array</span></b>}{c}
Y_t<span style='color:#644a9b;'>\\</span>
M_t
<b>\end</b>{<b><span style='color:#0095ff;'>array</span></b>}<span style='color:#644a9b;'>\right</span>)&amp;<span style='color:#644a9b;'>\approx</span>&amp; H_{<span style='color:#644a9b;'>\alpha</span>_2}F^t_<span style='color:#644a9b;'>\alpha</span> G_<span style='color:#644a9b;'>\alpha\begin</span>{pmatrix} 
<span style='color:#644a9b;'>\cos</span>(H_{<span style='color:#644a9b;'>\alpha</span>_1}F^t_<span style='color:#644a9b;'>\alpha</span> G_<span style='color:#644a9b;'>\alpha</span>)&amp;-<span style='color:#644a9b;'>\sin</span>(H_{<span style='color:#644a9b;'>\alpha</span>_1}F^t_<span style='color:#644a9b;'>\alpha</span> G_<span style='color:#644a9b;'>\alpha</span>)<span style='color:#644a9b;'>\\</span> <span style='color:#644a9b;'>\\</span>
<span style='color:#644a9b;'>\sin</span>(H_{<span style='color:#644a9b;'>\alpha</span>_1}F^t_<span style='color:#644a9b;'>\alpha</span> G_<span style='color:#644a9b;'>\alpha</span>)&amp;<span style='color:#644a9b;'>\cos</span>(H_{<span style='color:#644a9b;'>\alpha</span>_1}F^t_<span style='color:#644a9b;'>\alpha</span> G_<span style='color:#644a9b;'>\alpha</span>)
<b>\end</b>{<b><span style='color:#0095ff;'>pmatrix</span></b>}<span style='color:#644a9b;'>\left</span>(<b>\begin</b>{<b><span style='color:#0095ff;'>array</span></b>}{c}
F_t<span style='color:#644a9b;'>\\</span>
MK_t
<b>\end</b>{<b><span style='color:#0095ff;'>array</span></b>}<span style='color:#644a9b;'>\right</span>)
<span style='color:#644a9b;'>\fe</span>
donde la matriz del lado derecho es dependiente temporal, con lo cuál se obtiene mediante la realización una aproximación variante temporal del cambio estructural de la economía.<span style='color:#644a9b;'>\-</span>

<b>\section</b>{<b>Implementación métodos variantes</b>}

Bajo el método por acortamiento de series, 


Para ir ejemplificando el método, y también para poder llegar a contrastar con el método invariante, utilizamos las 100 matrices usadas por <b>\cite</b>{<b><span style='color:#0095ff;'>ARGTR</span></b>}, las cuales intentan determinar la realización invariante de un modelo de entrada y salida basado en los siguientes
<b>\begin</b>{<b><span style='color:#0095ff;'>figure</span></b>}[H]<span style='color:#644a9b;'>\centering</span>
<span style='color:#644a9b;'>\caption</span>*{Criterios contables de flujo de fondos}<span style='color:#644a9b;'>\vspace</span>{1em}
<b>\begin</b>{<b><span style='color:#0095ff;'>tabular</span></b>}{c|cc|c|c}
     <b>&amp;</b><span style='color:#644a9b;'>\textbf</span>{r}<b>&amp;</b><span style='color:#644a9b;'>\textbf</span>{nr}<b>&amp;</b>Ac.<b>&amp;</b>Total  <span style='color:#644a9b;'>\\\hline</span>
<span style='color:#644a9b;'>\textbf</span>{r}    <b>&amp;</b> <span style='color:#ff5500;'>$C_t$</span><b>&amp;</b><span style='color:#ff5500;'>$X_t$</span><b>&amp;</b><span style='color:#ff5500;'>$F_t$</span><b>&amp;</b><span style='color:#ff5500;'>$Y_t$</span><span style='color:#644a9b;'>\\</span>
<span style='color:#644a9b;'>\textbf</span>{nr}   <b>&amp;</b> <span style='color:#ff5500;'>$MC_t$</span><b>&amp;</b><span style='color:#ff5500;'>$0$</span><b>&amp;</b><span style='color:#ff5500;'>$Mk_t$</span><b>&amp;</b><span style='color:#ff5500;'>$M_t$</span><span style='color:#644a9b;'>\\</span>
Ah.   <b>&amp;</b> <span style='color:#ff5500;'>$S_{rt}$</span><b>&amp;</b><span style='color:#ff5500;'>$S_{nrt}$</span><b>&amp;</b><span style='color:#ff5500;'>$0$</span><b>&amp;</b><span style='color:#ff5500;'>$S_t$</span><span style='color:#644a9b;'>\\\hline</span>
      <b>&amp;</b> <span style='color:#ff5500;'>$Y_t$</span><b>&amp;</b><span style='color:#ff5500;'>$M_t$</span><b>&amp;</b><span style='color:#ff5500;'>$I_t$</span><b>&amp;</b><span style='color:#644a9b;'>\\</span>
<b>\end</b>{<b><span style='color:#0095ff;'>tabular</span></b>}
<b>\end</b>{<b><span style='color:#0095ff;'>figure</span></b>}

donde <span style='color:#ff5500;'>$C_t$</span> es el consumo de mercancías intermedias, <span style='color:#ff5500;'>$X_t$</span> exportaciones, <span style='color:#ff5500;'>$F_t$</span> formación bruta de capital, <span style='color:#ff5500;'>$Y_t$</span> ingreso interno bruto, <span style='color:#ff5500;'>$MC_t$</span> consumo intermedio y final de mercancías importadas, <span style='color:#ff5500;'>$MK_t$</span> formación bruta de capital de mercancías importadas, <span style='color:#ff5500;'>$M_t$</span> importaciones, <span style='color:#ff5500;'>$S_{rt}$</span> ahorro de los residentes, <span style='color:#ff5500;'>$S_{nrt}$</span> ahorro de los no residentes,  <span style='color:#ff5500;'>$S_t$</span> ahorro, e <span style='color:#ff5500;'>$I_t$</span> inversión.<span style='color:#644a9b;'>\-</span>

Este modelo utiliza los criterios (contables) anteriores expresados de la siguiente manera
<span style='color:#644a9b;'>\ies</span> <span style='color:#644a9b;'>\footnotesize</span>
Y_t&amp;<span style='color:#644a9b;'>\equiv</span>&amp;C_t+X_t+F_t<span style='color:#644a9b;'>\\</span>
M_t&amp;<span style='color:#644a9b;'>\equiv</span>&amp;MC_t+MK_t
<span style='color:#644a9b;'>\fes</span>
y a partir de ellos plantea la siguiente igualdad 
<span style='color:#644a9b;'>\ie\footnotesize\label</span>{primexp}
<span style='color:#644a9b;'>\left</span>(<b>\begin</b>{<b><span style='color:#0095ff;'>array</span></b>}{c}
Y_t<span style='color:#644a9b;'>\\</span>
M_t
<b>\end</b>{<b><span style='color:#0095ff;'>array</span></b>}<span style='color:#644a9b;'>\right</span>)=
<span style='color:#644a9b;'>\left</span>(<b>\begin</b>{<b><span style='color:#0095ff;'>array</span></b>}{cc}
c_t&amp;x_t<span style='color:#644a9b;'>\\</span>
m_t&amp;0
<b>\end</b>{<b><span style='color:#0095ff;'>array</span></b>}<span style='color:#644a9b;'>\right</span>)
<span style='color:#644a9b;'>\left</span>(<b>\begin</b>{<b><span style='color:#0095ff;'>array</span></b>}{c}
Y_t<span style='color:#644a9b;'>\\</span>
M_t
<b>\end</b>{<b><span style='color:#0095ff;'>array</span></b>}<span style='color:#644a9b;'>\right</span>)+
<span style='color:#644a9b;'>\left</span>(<b>\begin</b>{<b><span style='color:#0095ff;'>array</span></b>}{c}
F_t<span style='color:#644a9b;'>\\</span>
MK_t
<b>\end</b>{<b><span style='color:#0095ff;'>array</span></b>}<span style='color:#644a9b;'>\right</span>)
<span style='color:#644a9b;'>\fe</span>
donde 
<span style='color:#644a9b;'>\ies</span> <span style='color:#644a9b;'>\footnotesize</span>
<span style='color:#644a9b;'>\left</span>(<b>\begin</b>{<b><span style='color:#0095ff;'>array</span></b>}{cc}
c_t&amp;x_t<span style='color:#644a9b;'>\\</span>
m_t&amp;0
<b>\end</b>{<b><span style='color:#0095ff;'>array</span></b>}<span style='color:#644a9b;'>\right</span>)=<span style='color:#644a9b;'>\left</span>(<b>\begin</b>{<b><span style='color:#0095ff;'>array</span></b>}{cc}
C_t&amp;X_t<span style='color:#644a9b;'>\\</span>
MC_t&amp;0
<b>\end</b>{<b><span style='color:#0095ff;'>array</span></b>}<span style='color:#644a9b;'>\right</span>)<span style='color:#644a9b;'>\left</span>(<b>\begin</b>{<b><span style='color:#0095ff;'>array</span></b>}{cc}
Y_t^{-1}&amp;0<span style='color:#644a9b;'>\\</span>
0&amp;M_t^{-1}
<b>\end</b>{<b><span style='color:#0095ff;'>array</span></b>}<span style='color:#644a9b;'>\right</span>)
<span style='color:#644a9b;'>\fes</span>
Luego, haciendo uso de <b>\eqref</b>{<b><span style='color:#0095ff;'>primexp</span></b>} se construye el modelo final, en su formato entrada-salida:
<span style='color:#644a9b;'>\ies\footnotesize</span>
<span style='color:#644a9b;'>\left</span>(<b>\begin</b>{<b><span style='color:#0095ff;'>array</span></b>}{c}
Y_t<span style='color:#644a9b;'>\\</span>
M_t
<b>\end</b>{<b><span style='color:#0095ff;'>array</span></b>}<span style='color:#644a9b;'>\right</span>)&amp;=&amp;<span style='color:#644a9b;'>\left</span>(<b>\begin</b>{<b><span style='color:#0095ff;'>array</span></b>}{cc}
1-c_t&amp;-x_t<span style='color:#644a9b;'>\\</span>
-m_t&amp;1
<b>\end</b>{<b><span style='color:#0095ff;'>array</span></b>}<span style='color:#644a9b;'>\right</span>)^{-1}
<span style='color:#644a9b;'>\left</span>(<b>\begin</b>{<b><span style='color:#0095ff;'>array</span></b>}{c}
F_t<span style='color:#644a9b;'>\\</span>
MK_t
<b>\end</b>{<b><span style='color:#0095ff;'>array</span></b>}<span style='color:#644a9b;'>\right</span>)
<span style='color:#644a9b;'>\fes</span>
Donde se puede reexpresar de la siguiente forma
<span style='color:#644a9b;'>\ies</span> <span style='color:#644a9b;'>\footnotesize</span>
<span style='color:#644a9b;'>\left</span>(<b>\begin</b>{<b><span style='color:#0095ff;'>array</span></b>}{c}
Y_t<span style='color:#644a9b;'>\\</span>
M_t
<b>\end</b>{<b><span style='color:#0095ff;'>array</span></b>}<span style='color:#644a9b;'>\right</span>)&amp;=&amp;<span style='color:#644a9b;'>\frac</span>{1}{<span style='color:#644a9b;'>\abs</span>{A}}<b>\begin</b>{<b><span style='color:#0095ff;'>pmatrix</span></b>}<span style='color:#ff5500;'> </span>
<span style='color:#ff5500;'>1&amp;x_t</span><span style='color:#3daee9;'>\\</span>
<span style='color:#ff5500;'>m_t&amp;1-c_t</span>
<b>\end</b>{<b><span style='color:#0095ff;'>pmatrix</span></b>}<span style='color:#644a9b;'>\left</span>(<b>\begin</b>{<b><span style='color:#0095ff;'>array</span></b>}{c}
F_t<span style='color:#644a9b;'>\\</span>
MK_t
<b>\end</b>{<b><span style='color:#0095ff;'>array</span></b>}<span style='color:#644a9b;'>\right</span>)
<span style='color:#644a9b;'>\fes</span>
donde 
<span style='color:#644a9b;'>\ies\footnotesize</span>
<span style='color:#644a9b;'>\abs</span>{A}&amp;=&amp;det<b>\begin</b>{<b><span style='color:#0095ff;'>pmatrix</span></b>}<span style='color:#ff5500;'> </span>
<span style='color:#ff5500;'>1-c_t&amp;-x_t</span><span style='color:#3daee9;'>\\</span>
<span style='color:#ff5500;'>-m_t&amp;1</span>
<b>\end</b>{<b><span style='color:#0095ff;'>pmatrix</span></b>}<span style='color:#644a9b;'>\\</span>
&amp;=&amp;1-c_t-m_tx_t
<span style='color:#644a9b;'>\fes</span>

Este modelo cuenta con ciertas características sistémicas relevantes que se listan a continuación:
<b>\begin</b>{<b><span style='color:#0095ff;'>enumerate</span></b>}[i)]
    <span style='color:#644a9b;'>\item</span> interconecta en cada <span style='color:#ff5500;'>$t$</span> a dos sectores <span style='color:#644a9b;'>\textbf</span>{r} y <span style='color:#644a9b;'>\textbf</span>{nr}, de forma tal que lo que se le vende a <span style='color:#644a9b;'>\textbf</span>{nr} depende de lo que <span style='color:#644a9b;'>\textbf</span>{nr} le compra a <span style='color:#644a9b;'>\textbf</span>{n} y vice-versa;
    <span style='color:#644a9b;'>\item</span> la trayectoria de tres razones clave de las relaciones económicas internas y con el resto del mundo (o externas), son observables: <span style='color:#ff5500;'>$c_t$</span>, la propensión marginal a consumir mercancías de consumo de producción interna; <span style='color:#ff5500;'>$m_t$</span>, la propensión a importar mercancías para producir otras mercancías y para el consumo final; y <span style='color:#ff5500;'>$x_t$</span> la razón del balance en cuenta corriente entre el ingreso por exportaciones y el gasto en importaciones;
    <span style='color:#644a9b;'>\item</span> relaciona los flujos de entrada (o exógenos) de acumulación de capital <span style='color:#ff5500;'>$</span><span style='color:#3daee9;'>\footnotesize\begin</span><span style='color:#ff5500;'>{pmatrix}</span>
<span style='color:#ff5500;'>F_t</span><span style='color:#3daee9;'>\\</span>
<span style='color:#ff5500;'>MK_t</span>
<b>\end</b>{<b><span style='color:#0095ff;'>pmatrix</span></b>}<span style='color:#ff5500;'>$</span> con los de salida (o endógenos) mediante una descripción observable externa basada en 
<span style='color:#ff5500;'>$</span><span style='color:#3daee9;'>\footnotesize\frac</span><span style='color:#ff5500;'>{1}{</span><span style='color:#3daee9;'>\abs</span><span style='color:#ff5500;'>{A}}</span><b>\begin</b>{<b><span style='color:#0095ff;'>pmatrix</span></b>}<span style='color:#ff5500;'> </span>
<span style='color:#ff5500;'>1&amp;x_t</span><span style='color:#3daee9;'>\\</span>
<span style='color:#ff5500;'>m_t&amp;1-c_t</span>
<b>\end</b>{<b><span style='color:#0095ff;'>pmatrix</span></b>}<span style='color:#ff5500;'>$</span>,
de periodicidad trimestral;
<span style='color:#644a9b;'>\item</span> posee una descripción no observable interna de las variables de estado dada por la terna de matrices <span style='color:#ff5500;'>$(F,G,H)$</span>, las que coinciden con las variables de salida mediante la siguiente formalización dinámica:
<span style='color:#644a9b;'>\ies</span> <span style='color:#644a9b;'>\footnotesize</span>
<span style='color:#644a9b;'>\frac</span>{1}{<span style='color:#644a9b;'>\abs</span>{A}}<b>\begin</b>{<b><span style='color:#0095ff;'>pmatrix</span></b>}<span style='color:#ff5500;'> </span>
<span style='color:#ff5500;'>1&amp;x_t</span><span style='color:#3daee9;'>\\</span>
<span style='color:#ff5500;'>m_t&amp;1-c_t</span>
<b>\end</b>{<b><span style='color:#0095ff;'>pmatrix</span></b>}=HF^{t-1}G
<span style='color:#644a9b;'>\fes</span>
donde esta última matriz es la que corresponde a un sistema dinámico lineal, discreto, finito, de coeficientes invariables que transforma la inversión <span style='color:#ff5500;'>$</span><span style='color:#3daee9;'>\footnotesize\begin</span><span style='color:#ff5500;'>{pmatrix}</span>
<span style='color:#ff5500;'>F_t</span><span style='color:#3daee9;'>\\</span>
<span style='color:#ff5500;'>MK_t</span>
<b>\end</b>{<b><span style='color:#0095ff;'>pmatrix</span></b>}<span style='color:#ff5500;'>$</span> en ingreso interno y del resto del mundo <span style='color:#ff5500;'>$</span><span style='color:#3daee9;'>\footnotesize\begin</span><span style='color:#ff5500;'>{pmatrix}</span>
<span style='color:#ff5500;'>Y_t</span><span style='color:#3daee9;'>\\</span>
<span style='color:#ff5500;'>M_t</span>
<b>\end</b>{<b><span style='color:#0095ff;'>pmatrix</span></b>}<span style='color:#ff5500;'>$</span>

<span style='color:#644a9b;'>\ies</span> <span style='color:#644a9b;'>\footnotesize</span>
<b>\begin</b>{<b><span style='color:#0095ff;'>pmatrix</span></b>}<span style='color:#ff5500;'> </span>
<span style='color:#ff5500;'>Y_t</span><span style='color:#3daee9;'>\\</span>
<span style='color:#ff5500;'>M_t</span>
<b>\end</b>{<b><span style='color:#0095ff;'>pmatrix</span></b>}=HF^{t-1}G<b>\begin</b>{<b><span style='color:#0095ff;'>pmatrix</span></b>}<span style='color:#ff5500;'> </span>
<span style='color:#ff5500;'>F_t</span><span style='color:#3daee9;'>\\</span>
<span style='color:#ff5500;'>MK_t</span>
<b>\end</b>{<b><span style='color:#0095ff;'>pmatrix</span></b>}
<span style='color:#644a9b;'>\fes</span>

Aplicar la propuesta variante a las 100 matrices de este modelo, comienza creando <span style='color:#ff5500;'>$n-1=100-1=99$</span> realizaciones, cada una surge de quitar información a la serie original; quitar la última matriz, luego las dos últimas, y continuar así hasta solo dejar las dos más recientes; y en cada ocasión se obtiene un patrón distinto que depende del tiempo inicial.<span style='color:#644a9b;'>\-</span>

Al usar el cógido () se obtienen las siguientes realizaciones
<span style='color:#644a9b;'>\ies</span>
PONER GRÁFICA
<span style='color:#644a9b;'>\fes</span>

    
<b>\end</b>{<b><span style='color:#0095ff;'>enumerate</span></b>}


<b>\chapter</b>{<b>Metodología para el control del cambio estructural del sistema económico</b>}




 Preguntas discutidas con mi tutor:
    <b>\begin</b>{<b><span style='color:#0095ff;'>enumerate</span></b>}
        <span style='color:#644a9b;'>\item</span> ¿Cuáles son las variables de estado?
        <span style='color:#644a9b;'>\item</span> ¿Cuál será la matriz objetivo? (Escenario excelente, regular y deficiente, matriz manejable <span style='color:#ff5500;'>$5</span><span style='color:#3daee9;'>\times</span><span style='color:#ff5500;'>5$</span>)
        <span style='color:#644a9b;'>\item</span> ¿cómo fabricar una red? (indicadora de trayectoria, periodo de implementación, periodo de ajuste, restricciones, optimización...)
        <span style='color:#644a9b;'>\item</span> ¿como determinar las trayectorias?
    <b>\end</b>{<b><span style='color:#0095ff;'>enumerate</span></b>}


    Para dirigir un sistema económico requeririamos (i) identificar la estructura objetivo a donde llevarlo, (ii) delimitar la sucesión de estructuras intermedias, (iii) definir los intrumetos de control, (iv) tener un indicador de ruta y (v) definir acciones de rectificación.<span style='color:#644a9b;'>\-</span>

    <b>\section</b>{<b>Estructura objetivo y capas de aproximación</b>}

    La estructura objetivo estará en función  de las industrias preponderantes en el sistema económico de los países desarrollados o en el umbral del desarrollo. La jerarquización
    
    <span style='color:#644a9b;'>\-</span>

    Las industrias  de alto valor agregado en países desarrollados, por su parte, utilizan y/o producen bienes que pueden catalogarse de alta tecnología, por lo cual, una clasificación que podemos útilizar en lo siguiente es mediante intensidad tecnológica. Un trabajo hecho en este sentido fue el construido por la CEPAL para el periodo 1980-1990.<span style='color:#644a9b;'>\footnote</span>{ Debo de reproducir la metología para el periodo presente, hallar un documento que la implementó más recientemente, encontrar otra metodología, o crear una. } Utilizando  2 indicadores directos para la medición de la intensidad en investigación y desarrollo, y 1 indirecto basado en los coeficientes técnicos de matrices insumo-producto, obtuvo la clasificación en la Tabla  <b>\ref</b>{<b><span style='color:#0095ff;'>int_tec</span></b>}.<span style='color:#644a9b;'>\-</span>

    <b>\begin</b>{<b><span style='color:#0095ff;'>table</span></b>}[H]<span style='color:#644a9b;'>\centering\footnotesize</span>
    <span style='color:#644a9b;'>\caption</span>{<b>\label</b>{<b><span style='color:#0095ff;'>int_tec</span></b>} Intensidad Tecnológica}
    <b>\begin</b>{<b><span style='color:#0095ff;'>tabular</span></b>}{cccc}
    <span style='color:#644a9b;'>\textbf</span>{Alta}                                                                       <b>&amp;</b> <span style='color:#644a9b;'>\textbf</span>{Media alta}                                                 <b>&amp;</b> <span style='color:#644a9b;'>\textbf</span>{Media baja}                                                        <b>&amp;</b> <span style='color:#644a9b;'>\textbf</span>{Baja}                                                          <span style='color:#644a9b;'>\\</span> <span style='color:#644a9b;'>\hline</span>
    Aeroespacial                                                                        <b>&amp;</b> <b>\begin</b>{<b><span style='color:#0095ff;'>tabular</span></b>}[c]{<span style='color:#924c9d;'>@{}</span>c<span style='color:#924c9d;'>@{}</span>}Instrumentos<span style='color:#644a9b;'>\\</span> científicos<b>\end</b>{<b><span style='color:#0095ff;'>tabular</span></b>}  <b>&amp;</b> <b>\begin</b>{<b><span style='color:#0095ff;'>tabular</span></b>}[c]{<span style='color:#924c9d;'>@{}</span>c<span style='color:#924c9d;'>@{}</span>}Productos de <span style='color:#644a9b;'>\\</span> plástico y hule<b>\end</b>{<b><span style='color:#0095ff;'>tabular</span></b>}    <b>&amp;</b> <b>\begin</b>{<b><span style='color:#0095ff;'>tabular</span></b>}[c]{<span style='color:#924c9d;'>@{}</span>c<span style='color:#924c9d;'>@{}</span>}Impresión de<span style='color:#644a9b;'>\\</span> papel<b>\end</b>{<b><span style='color:#0095ff;'>tabular</span></b>}           <span style='color:#644a9b;'>\vspace</span>{.2em}<span style='color:#644a9b;'>\\</span> 
    <b>\begin</b>{<b><span style='color:#0095ff;'>tabular</span></b>}[c]{<span style='color:#924c9d;'>@{}</span>c<span style='color:#924c9d;'>@{}</span>}Equipo de cómputo y<span style='color:#644a9b;'>\\</span> maquinaria de oficina<b>\end</b>{<b><span style='color:#0095ff;'>tabular</span></b>} <b>&amp;</b> <b>\begin</b>{<b><span style='color:#0095ff;'>tabular</span></b>}[c]{<span style='color:#924c9d;'>@{}</span>c<span style='color:#924c9d;'>@{}</span>}Vehículos de <span style='color:#644a9b;'>\\</span> motor<b>\end</b>{<b><span style='color:#0095ff;'>tabular</span></b>}       <b>&amp;</b> <b>\begin</b>{<b><span style='color:#0095ff;'>tabular</span></b>}[c]{<span style='color:#924c9d;'>@{}</span>c<span style='color:#924c9d;'>@{}</span>}Fabricación de <span style='color:#644a9b;'>\\</span> barcos<b>\end</b>{<b><span style='color:#0095ff;'>tabular</span></b>}           <b>&amp;</b> <b>\begin</b>{<b><span style='color:#0095ff;'>tabular</span></b>}[c]{<span style='color:#924c9d;'>@{}</span>c<span style='color:#924c9d;'>@{}</span>}Textiles y prendas<span style='color:#644a9b;'>\\</span> de vestir<b>\end</b>{<b><span style='color:#0095ff;'>tabular</span></b>} <span style='color:#644a9b;'>\vspace</span>{.2em}<span style='color:#644a9b;'>\\</span> 
    <b>\begin</b>{<b><span style='color:#0095ff;'>tabular</span></b>}[c]{<span style='color:#924c9d;'>@{}</span>c<span style='color:#924c9d;'>@{}</span>}Electrónica y<span style='color:#644a9b;'>\\</span> comunicaciones<b>\end</b>{<b><span style='color:#0095ff;'>tabular</span></b>}              <b>&amp;</b> <b>\begin</b>{<b><span style='color:#0095ff;'>tabular</span></b>}[c]{<span style='color:#924c9d;'>@{}</span>c<span style='color:#924c9d;'>@{}</span>}Maquinaria <span style='color:#644a9b;'>\\</span> eléctrica<b>\end</b>{<b><span style='color:#0095ff;'>tabular</span></b>}     <b>&amp;</b> Otras manufacturas                                                         <b>&amp;</b> <b>\begin</b>{<b><span style='color:#0095ff;'>tabular</span></b>}[c]{<span style='color:#924c9d;'>@{}</span>c<span style='color:#924c9d;'>@{}</span>}Alimentos, bebidas<span style='color:#644a9b;'>\\</span> y tabaco<b>\end</b>{<b><span style='color:#0095ff;'>tabular</span></b>}  <span style='color:#644a9b;'>\vspace</span>{.2em}<span style='color:#644a9b;'>\\</span> 
    Farmacéutica                                                                        <b>&amp;</b> Químicos                                                            <b>&amp;</b> Metales no ferrosos                                                        <b>&amp;</b> Madera y muebles                                                       <span style='color:#644a9b;'>\vspace</span>{.2em}<span style='color:#644a9b;'>\\</span> 
                                                                                        <b>&amp;</b> <b>\begin</b>{<b><span style='color:#0095ff;'>tabular</span></b>}[c]{<span style='color:#924c9d;'>@{}</span>c<span style='color:#924c9d;'>@{}</span>}Otro equipo de<span style='color:#644a9b;'>\\</span> transporte<b>\end</b>{<b><span style='color:#0095ff;'>tabular</span></b>} <b>&amp;</b> <b>\begin</b>{<b><span style='color:#0095ff;'>tabular</span></b>}[c]{<span style='color:#924c9d;'>@{}</span>c<span style='color:#924c9d;'>@{}</span>}Productos minerales<span style='color:#644a9b;'>\\</span> no metálicos<b>\end</b>{<b><span style='color:#0095ff;'>tabular</span></b>} <b>&amp;</b>                                                                        <span style='color:#644a9b;'>\vspace</span>{.2em}<span style='color:#644a9b;'>\\</span>
                                                                                        <b>&amp;</b> <b>\begin</b>{<b><span style='color:#0095ff;'>tabular</span></b>}[c]{<span style='color:#924c9d;'>@{}</span>c<span style='color:#924c9d;'>@{}</span>}Maquinaria no <span style='color:#644a9b;'>\\</span> eléctrica<b>\end</b>{<b><span style='color:#0095ff;'>tabular</span></b>}  <b>&amp;</b> <b>\begin</b>{<b><span style='color:#0095ff;'>tabular</span></b>}[c]{<span style='color:#924c9d;'>@{}</span>c<span style='color:#924c9d;'>@{}</span>}Productos metálicos<span style='color:#644a9b;'>\\</span> fabricados<b>\end</b>{<b><span style='color:#0095ff;'>tabular</span></b>}   <b>&amp;</b>                                                                        <span style='color:#644a9b;'>\vspace</span>{.2em}<span style='color:#644a9b;'>\\</span> 
                                                                                        <b>&amp;</b>                                                                     <b>&amp;</b> <b>\begin</b>{<b><span style='color:#0095ff;'>tabular</span></b>}[c]{<span style='color:#924c9d;'>@{}</span>c<span style='color:#924c9d;'>@{}</span>}Refinación de <span style='color:#644a9b;'>\\</span> petróleo<b>\end</b>{<b><span style='color:#0095ff;'>tabular</span></b>}          <b>&amp;</b>                                                                        <span style='color:#644a9b;'>\\</span> 
    <b>\end</b>{<b><span style='color:#0095ff;'>tabular</span></b>}
    <b>\end</b>{<b><span style='color:#0095ff;'>table</span></b>}
    Por consiguiente, si <span style='color:#ff5500;'>$S_{1},</span><span style='color:#3daee9;'>\dots</span><span style='color:#ff5500;'>,S_{n}$</span> son todas las industrias, entonces siguiendo una metodología como la de la CEPAL los podemos ordenar según su intensidad tecnológica: las que se hallen en <span style='color:#ff5500;'>$r_0=1,</span><span style='color:#3daee9;'>\dots</span><span style='color:#ff5500;'>,r_1$</span> serán de alta tecnología; aquellas en las posiciones <span style='color:#ff5500;'>$r_1+1, </span><span style='color:#3daee9;'>\dots</span><span style='color:#ff5500;'>, r_2$</span> serán de media-alta tecnología; las que se ubiquen en <span style='color:#ff5500;'>$r_2+1,</span><span style='color:#3daee9;'>\dots</span><span style='color:#ff5500;'>,r_3$</span> de media-baja tecnología; y, por último, las que se encuentren en <span style='color:#ff5500;'>$r_3+1, </span><span style='color:#3daee9;'>\dots</span><span style='color:#ff5500;'>,r_4=n$</span> de baja tecnología.<span style='color:#644a9b;'>\-</span>
    
    Tomemos <span style='color:#ff5500;'>$A^{(k)}$</span> la matriz de coeficientes técnicos del país desarrollado <span style='color:#ff5500;'>$k$</span>-ésimo, donde las filas y columnas están ordenadas bajo la clasificación dada arriba.
    Entonces, inicialmente, a partir de las <span style='color:#ff5500;'>$A^{(k)}$</span> tendríamos que encontrar los sectores estructura objetivo. Tomando las columnas <span style='color:#ff5500;'>$1,</span><span style='color:#3daee9;'>\dots</span><span style='color:#ff5500;'>,r_1$</span> de <span style='color:#ff5500;'>$A^{(k)}$</span> estaríamos frente a   <span style='color:#644a9b;'>\-</span>
    
    Elementos en revisión para la construcción de la matriz objetivo mediante las <span style='color:#ff5500;'>$A^{(k)}$</span> y elementos extra.
    
    <b>\begin</b>{<b><span style='color:#0095ff;'>enumerate</span></b>}
    <span style='color:#644a9b;'>\item</span> Las funciones de producción que subyacen en las columnas de <span style='color:#ff5500;'>$A^{(k)}$</span>. Diferenciales de productividad pueden  afectar la competitividad y supervivencia  del sector dentro del comercio internacional. 

    <span style='color:#644a9b;'>\item</span> Interdependencia y jerarquías sectoriales<span style='color:#644a9b;'>\footnote</span>{Lamel et. al. (1971) Patterns of industrial structure and economic development. Triangulation of input-output tables of ECE countries. European Economic Review. <span style='color:#644a9b;'>\url</span>{https://reader.elsevier.com/reader/sd/pii/0014292172900232?token=E2A6BA0DD58642A5AF94E87A329BB3AB5A34E84130CA497AD4D7CD960CA4B82D137FE6AEE4BFD4EBFB2FB39D971E65FB&amp;originRegion=us-east-1&amp;originCreation=20210511082258} } 
    
    <span style='color:#644a9b;'>\item</span> Estructura de la demanda: <span style='color:#ff5500;'>$Y=a_cY+a_IY+a_GG+a_{x_n}(X-M)=C+I+G+X-M$</span>
    <span style='color:#644a9b;'>\item</span> Estructura factorial del ingreso
    <span style='color:#644a9b;'>\item</span> Estructura de importaciones
    <span style='color:#644a9b;'>\item</span> Estructura tecnológica
    <span style='color:#644a9b;'>\item</span> Estructura del producto total 
    <span style='color:#644a9b;'>\item</span> Estructura costos industriales
    <span style='color:#644a9b;'>\item</span> Dispersión de la productividad y homogeneidad estructural<span style='color:#644a9b;'>\footnote</span>{
    CEPAL. Parte 4 La dimensión productiva. Capítulo X. Las brechas de productividad y la heterogeneidad estructural. 
    <span style='color:#644a9b;'>\url</span>{https://repositorio.cepal.org/bitstream/handle/11362/43540/66/cap10_Desarrollo_e_igualdad_es.pdf}}
    <b>\end</b>{<b><span style='color:#0095ff;'>enumerate</span></b>}
    
    Matriz de triangulación 
    
    El sector <span style='color:#ff5500;'>$S_i$</span> participa directamente en la producción de <span style='color:#ff5500;'>$S_j$</span> siempre y cuando <span style='color:#ff5500;'>$a_{ij}&gt;0$</span>, y un sector <span style='color:#ff5500;'>$S_k$</span> participa indirectamente en la producción de <span style='color:#ff5500;'>$S_k$</span> siempre que existan <span style='color:#ff5500;'>$a_{kk_1}, a_{k_1k_2}, </span><span style='color:#3daee9;'>\dots</span><span style='color:#ff5500;'>, a_{k_{s-1}k_s}, a_{k_{s}k}&gt;0$</span>.  

    <span style='color:#644a9b;'>\ies</span>
    X^{(k)}=L^{(k)}Y^{(k)}=<span style='color:#644a9b;'>\sum</span>_{i=1}^{<span style='color:#644a9b;'>\infty</span>}<span style='color:#644a9b;'>\left</span>(A^{(k)}<span style='color:#644a9b;'>\right</span>)^iY^{(k)}=Y^{(k)}+A^{(k)}Y^{(k)}+<span style='color:#644a9b;'>\left</span>(A^{(k)}<span style='color:#644a9b;'>\right</span>)^2Y^{(k)}+<span style='color:#644a9b;'>\cdots</span>
    <span style='color:#644a9b;'>\fes</span>
    
    Clasificación de los sectores de los países desarrollados.<span style='color:#644a9b;'>\footnote</span>{Tarancón (2003). Técnicas de análisis económico input-output. p.73. Clasificación de sectores por triangulaciones o coeficientes importantes.}
    
    
    <b>\section</b>{<b>Identificación de sectores importantes mediante la TR</b>}
       

    El proceso de transición será autoregulado y dependerá de la asimilación que la economía pueda hacer de la variación de coeficientes técnicos, la cuál se estudia en la sección 4 del presente capítulo.<span style='color:#644a9b;'>\-</span>
    
    Una vez teniendo la matriz dínamica objetivo, hay que determinar la sucesión de matrices intermedias, y para ello, se deben de identificar y ordenar los sectores por capas. Mientras más alta la capa en la que se encuentra un sector, es más dependiente de sectores altá y media-alta tecnología.<span style='color:#644a9b;'>\-</span>
    
    Supongamos que el sector de media-baja tecnología <span style='color:#ff5500;'>$S_{r_4}$</span> es el único que se encuentra en la primera capa. Es decir, él es indispensable para otros sectores de media-alta y alta tecnología. Transformar la estructura en la primera fase significará llevar la función de producción <span style='color:#ff5500;'>$A_{:r_4}$</span> hacia <span style='color:#ff5500;'>$A'_{:r_4}$</span>. Esto, a su vez, impactará los sectores 
    
   
   <span style='color:#644a9b;'>\-</span>
    
    Por consiguiente, la identificación de fases de transición hacia la matriz objetivo se basará en la identificación <span style='color:#644a9b;'>\textbf</span>{identificación dinámica de los coeficientes técnicos importantes mediante la TR}.<span style='color:#644a9b;'>\-</span>
    
    
     A manera de preámbulo, en la Tabla <b>\ref</b>{<b><span style='color:#0095ff;'>coef_imp</span></b>} he colocado a los autores relevantes en el desarrollo de métodos de identificación de coeficientes técnicos significativos. Además, se puede observar que he colocado su respectiva fórmula de identificación, así como definir el tipo de desarrollo, una especia de impronta del método. Una vez revisada la tabla, el lector podrá encontrar en el resto del sub-apartado: <span style='color:#644a9b;'>\textbf</span>{(i)} la explicación de cada uno de dichos métodos; <span style='color:#644a9b;'>\textbf</span>{(ii)} la agrupación de los métodos según la similaridad en sus resultados; y <span style='color:#644a9b;'>\textbf</span>{(iii)} la generalización de un representante de cada grupo mediante la Tería de la Realización, es decir, la conversión dinámica de dichos métodos.

    <b>\begin</b>{<b><span style='color:#0095ff;'>table</span></b>}[H]<span style='color:#644a9b;'>\centering</span>
    <span style='color:#644a9b;'>\caption</span>{<b>\label</b>{<b><span style='color:#0095ff;'>coef_imp</span></b>}Métodos de clasificación de sectores importantes}
    <b>\begin</b>{<b><span style='color:#0095ff;'>tabular</span></b>}{lll}
    Autor(es)<b>&amp;</b><span style='color:#644a9b;'>\hspace</span>{2cm} Fórmula<b>&amp;</b><span style='color:#644a9b;'>\hspace</span>{1cm}Tipo<span style='color:#644a9b;'>\\\hline\vspace</span>{-.5em} 
    <span style='color:#644a9b;'>\\</span>
    <span style='color:#644a9b;'>\vspace</span>{-.2em}
    Evans<b>&amp;</b><span style='color:#ff5500;'>$</span><span style='color:#3daee9;'>\Delta</span><span style='color:#ff5500;'> L_{ij}=</span><span style='color:#3daee9;'>\dfrac</span><span style='color:#ff5500;'>{L{ir}E_{rs}L_{sj}}{1-E_{rs}L_{sr}}$</span><b>&amp;</b><span style='color:#644a9b;'>\multicolumn</span>{1}{p{5cm}}{Efectos indeseables de pequeñas variaciones}<span style='color:#644a9b;'>\\</span>
    <span style='color:#644a9b;'>\vspace</span>{-.2em}
    <span style='color:#644a9b;'>\\</span>
    Sekulic<b>&amp;</b><span style='color:#ff5500;'>$r_{ij}(p)=</span><span style='color:#3daee9;'>\frac</span><span style='color:#ff5500;'>{100</span><span style='color:#3daee9;'>\rho</span><span style='color:#ff5500;'>}{A_{ij}</span><span style='color:#3daee9;'>\left</span><span style='color:#ff5500;'>(X_j</span><span style='color:#3daee9;'>\max\limits</span><span style='color:#ff5500;'>_k</span><span style='color:#3daee9;'>\frac</span><span style='color:#ff5500;'>{L_{ki}}{X_k}+</span><span style='color:#3daee9;'>\rho</span><span style='color:#ff5500;'> L_{ji}</span><span style='color:#3daee9;'>\right</span><span style='color:#ff5500;'>)}$</span><b>&amp;</b>Límite tolerable <span style='color:#ff5500;'>$</span><span style='color:#3daee9;'>\rho</span><span style='color:#ff5500;'>$</span><span style='color:#644a9b;'>\\</span>
    <span style='color:#644a9b;'>\vspace</span>{-.2em}
    <span style='color:#644a9b;'>\\</span>
    Jílek<b>&amp;</b> <span style='color:#ff5500;'>$r_{ij}=</span><span style='color:#3daee9;'>\frac</span><span style='color:#ff5500;'>{1}{A_{ij}</span><span style='color:#3daee9;'>\left</span><span style='color:#ff5500;'>(X_j</span><span style='color:#3daee9;'>\max\limits</span><span style='color:#ff5500;'>_k</span><span style='color:#3daee9;'>\frac</span><span style='color:#ff5500;'>{L_{ki}}{W_k}X_j+0.01L_{ij}</span><span style='color:#3daee9;'>\right</span><span style='color:#ff5500;'>)}$</span><b>&amp;</b>Límite tolerable <span style='color:#ff5500;'>$1</span><span style='color:#3daee9;'>\%</span><span style='color:#ff5500;'>$</span>
    <span style='color:#644a9b;'>\\</span>
    <span style='color:#644a9b;'>\vspace</span>{-.2em}
    <span style='color:#644a9b;'>\\</span>
    Sebald y Bullard <b>&amp;</b>
    <span style='color:#ff5500;'>$p_{ij}=</span><span style='color:#3daee9;'>\sum\limits</span><span style='color:#ff5500;'>_{m, n}f_{</span><span style='color:#3daee9;'>\tau</span><span style='color:#ff5500;'>}((</span><span style='color:#3daee9;'>\Delta</span><span style='color:#ff5500;'> J)_{m</span><span style='color:#3daee9;'>\hspace</span><span style='color:#ff5500;'>{.1em}n}))$</span><b>&amp;</b><span style='color:#644a9b;'>\multicolumn</span>{1}{p{5cm}}{<span style='color:#644a9b;'>\textbf</span>{U}mbral <span style='color:#ff5500;'>$</span><span style='color:#3daee9;'>\tau</span><span style='color:#ff5500;'>$</span>; <span style='color:#644a9b;'>\textbf</span>{N}úmero de condición; <span style='color:#644a9b;'>\textbf</span>{M}étodo de   
    reducción de pruebas}<span style='color:#644a9b;'>\\</span>
    <span style='color:#644a9b;'>\vspace</span>{-.2em}
    <span style='color:#644a9b;'>\\</span>
     West<b>&amp;</b><span style='color:#ff5500;'>$(E_1)_{ij}=</span><span style='color:#3daee9;'>\sum</span><span style='color:#ff5500;'>_s</span><span style='color:#3daee9;'>\sum</span><span style='color:#ff5500;'>_rL_{ir}a_{rs}p_{rs}L_{sj}$</span>
    <b>&amp;</b><span style='color:#644a9b;'>\multicolumn</span>{1}{p{5cm}}{<span style='color:#ff5500;'>$</span><span style='color:#3daee9;'>\Delta</span><span style='color:#ff5500;'> A_{ij}$</span> mayor al <span style='color:#ff5500;'>$</span><span style='color:#3daee9;'>\Delta</span><span style='color:#ff5500;'>$</span> del multiplicador}<span style='color:#644a9b;'>\\\</span>
    <span style='color:#644a9b;'>\vspace</span>{-.2em}
    <span style='color:#644a9b;'>\\</span>
    Schintke y Stäglin<b>&amp;</b><span style='color:#ff5500;'>$r_{ij}(p)=</span><span style='color:#3daee9;'>\frac</span><span style='color:#ff5500;'>{100p}{A_{ij}</span><span style='color:#3daee9;'>\left</span><span style='color:#ff5500;'>(L_{ji}</span><span style='color:#3daee9;'>\rho</span><span style='color:#ff5500;'>+100L_{ii}</span><span style='color:#3daee9;'>\frac</span><span style='color:#ff5500;'>{X_j}{X_i}</span><span style='color:#3daee9;'>\right</span><span style='color:#ff5500;'>)}$</span><b>&amp;</b><span style='color:#644a9b;'>\multicolumn</span>{1}{p{5cm}}{<span style='color:#644a9b;'>\textbf</span>{Si} <span style='color:#ff5500;'>$100</span><span style='color:#3daee9;'>\Delta</span><span style='color:#ff5500;'> E_{ij}/A_{ij}&lt;100</span><span style='color:#3daee9;'>\%</span><span style='color:#ff5500;'>$</span> y <span style='color:#ff5500;'>$</span><span style='color:#3daee9;'>\Delta</span><span style='color:#ff5500;'> X_j&gt;</span><span style='color:#3daee9;'>\rho</span><span style='color:#ff5500;'>$</span>, entonces <span style='color:#ff5500;'>$A_{ij}$</span> será importante}<span style='color:#644a9b;'>\\</span>
    <span style='color:#644a9b;'>\vspace</span>{-.2em}
    <span style='color:#644a9b;'>\\</span>
    <span style='color:#644a9b;'>\multicolumn</span>{1}{p{3cm}}{Hewings, Fonseca, Guilhoto y Sonis}<b>&amp;</b><span style='color:#644a9b;'>\multicolumn</span>{1}{p{3cm}}{<span style='color:#644a9b;'>\hspace</span>{3cm}<span style='color:#ff5500;'>$F(e)=</span><span style='color:#3daee9;'>\dfrac</span><span style='color:#ff5500;'>{Z(e)-Z}{e}$</span>}
    <b>&amp;</b><span style='color:#644a9b;'>\multicolumn</span>{1}{p{5cm}}{<span style='color:#644a9b;'>\textbf</span>{C}ampos de influencia; <span style='color:#644a9b;'>\textbf</span>{F}unción de aproximación a la matriz de campos de influencia}
    <span style='color:#644a9b;'>\\</span>
    <span style='color:#644a9b;'>\vspace</span>{-.2em}
    <span style='color:#644a9b;'>\\</span>
    Songling y Gould<b>&amp;</b> {<span style='color:#644a9b;'>\footnotesize</span><span style='color:#ff5500;'>$</span><span style='color:#3daee9;'>\Delta</span><span style='color:#ff5500;'> L =</span><span style='color:#3daee9;'>\sum\limits</span><span style='color:#ff5500;'>_{r,s=1}^{n}</span><span style='color:#3daee9;'>\mu</span><span style='color:#ff5500;'>_{rs}´</span><b>\begin</b>{<b><span style='color:#0095ff;'>pmatrix</span></b>}
<span style='color:#ff5500;'>    L_{1r}L_{s1}&amp;</span><span style='color:#3daee9;'>\cdots</span><span style='color:#ff5500;'>&amp;L_{1r}L_{sn}</span><span style='color:#3daee9;'>\\</span>
<span style='color:#ff5500;'>    </span><span style='color:#3daee9;'>\vdots</span><span style='color:#ff5500;'>&amp;</span><span style='color:#3daee9;'>\ddots</span><span style='color:#ff5500;'>&amp;</span><span style='color:#3daee9;'>\vdots\\</span>
<span style='color:#ff5500;'>    L_{nr}L_{s1}&amp;</span><span style='color:#3daee9;'>\cdots</span><span style='color:#ff5500;'>&amp;L_{nr}L_{sn}</span>
<span style='color:#ff5500;'>    </span><b>\end</b>{<b><span style='color:#0095ff;'>pmatrix</span></b>}
<span style='color:#ff5500;'>$</span>}
    <b>&amp;</b>
    <span style='color:#644a9b;'>\multicolumn</span>{1}{p{5cm}}{<span style='color:#644a9b;'>\textbf</span>{M}ultiplicador <span style='color:#ff5500;'>$</span><span style='color:#3daee9;'>\partial</span><span style='color:#ff5500;'> L_{ij}/</span><span style='color:#3daee9;'>\partial</span><span style='color:#ff5500;'> A_{rs}$</span>;{<span style='color:#644a9b;'>\color</span>{white}..} <span style='color:#644a9b;'>\textbf</span>{P}roducto potencial <span style='color:#ff5500;'>$</span><span style='color:#3daee9;'>\Delta</span><span style='color:#ff5500;'> X_i$</span>}
    <span style='color:#644a9b;'>\\</span>
    <span style='color:#644a9b;'>\vspace</span>{-.6em}
    <span style='color:#644a9b;'>\\</span>
    Cassetti<b>&amp;</b><span style='color:#ff5500;'>$</span><span style='color:#3daee9;'>\a\left</span><span style='color:#ff5500;'>(A^{(k-l)}</span><span style='color:#3daee9;'>\right</span><span style='color:#ff5500;'>)=</span><span style='color:#3daee9;'>\min\limits</span><span style='color:#ff5500;'>_j</span><span style='color:#3daee9;'>\left\abs</span><span style='color:#ff5500;'>{</span><span style='color:#3daee9;'>\frac</span><span style='color:#ff5500;'>{c_j^{(k)}}{c_j}</span><span style='color:#3daee9;'>\right</span><span style='color:#ff5500;'>}$</span><b>&amp;</b><span style='color:#644a9b;'>\multicolumn</span>{1}{p{5cm}}{<span style='color:#644a9b;'>\textbf</span>{C}ompara estructuras de países;  <span style='color:#644a9b;'>\textbf</span>{T}ransacciones intermedias significativas}<span style='color:#644a9b;'>\\</span>
    <span style='color:#644a9b;'>\vspace</span>{-.2em}
    Siebe<b>&amp;</b><span style='color:#644a9b;'>\multicolumn</span>{1}{p{4cm}}{<span style='color:#ff5500;'>$</span><span style='color:#3daee9;'>\max</span><span style='color:#ff5500;'>_{i,j}=</span><span style='color:#3daee9;'>\max\limits</span><span style='color:#ff5500;'>_k</span><span style='color:#3daee9;'>\left\abs</span><span style='color:#ff5500;'>{</span><span style='color:#3daee9;'>\frac</span><span style='color:#ff5500;'>{</span><span style='color:#3daee9;'>\Delta</span><span style='color:#ff5500;'> X^{i,j}_k}{X_k}</span><span style='color:#3daee9;'>\right</span><span style='color:#ff5500;'>}$</span> <span style='color:#ff5500;'>$sum_{i,j}=</span><span style='color:#3daee9;'>\sum\limits</span><span style='color:#ff5500;'>_{k=1}^n</span><span style='color:#3daee9;'>\left\abs</span><span style='color:#ff5500;'>{</span><span style='color:#3daee9;'>\frac</span><span style='color:#ff5500;'>{</span><span style='color:#3daee9;'>\Delta</span><span style='color:#ff5500;'> X^{i,j}_k}{X_k}</span><span style='color:#3daee9;'>\right</span><span style='color:#ff5500;'>}$</span>}<b>&amp;</b><span style='color:#644a9b;'>\multicolumn</span>{1}{p{5cm}}{<span style='color:#644a9b;'>\textbf</span>{M}áxima <span style='color:#ff5500;'>$</span><span style='color:#3daee9;'>\Delta</span><span style='color:#ff5500;'>$</span> de la producción, inducida por una variación transaccional; <span style='color:#644a9b;'>\textbf</span>{A}gregación de las <span style='color:#ff5500;'>$</span><span style='color:#3daee9;'>\Delta</span><span style='color:#ff5500;'>$</span> del producto, inducidas por una variación transaccional}<span style='color:#644a9b;'>\\</span>
    <span style='color:#644a9b;'>\vspace</span>{-.2em}
    <span style='color:#644a9b;'>\\</span> 
    <b>\end</b>{<b><span style='color:#0095ff;'>tabular</span></b>} 
    <b>\end</b>{<b><span style='color:#0095ff;'>table</span></b>}

    
    Para empezar el estudio de los coeficientes técnicos importantes, observamos que el modelo de insumo-producto relaciona el producto y la demanda mediante la expresión 
    <span style='color:#644a9b;'>\ie\label</span>{ce}
    X=LY
    <span style='color:#644a9b;'>\fe</span>
    con <span style='color:#ff5500;'>$X$</span> el vector producto de la economía; <span style='color:#ff5500;'>$Y$</span> la demanda de la economía; y <span style='color:#ff5500;'>$L=(I-A)^{-1}$</span>  la matriz de Leontief, con <span style='color:#ff5500;'>$A$</span> la matriz de coeficientes técnicos e <span style='color:#ff5500;'>$I$</span> la matriz identidad. Así, la incorporación de las variaciones en <span style='color:#ff5500;'>$A$</span> puede considerarse dentro de la matriz <span style='color:#ff5500;'>$L'=(I-A')^{-1}$</span> con <span style='color:#ff5500;'>$A'=A+E$</span>, donde <span style='color:#ff5500;'>$E$</span> es quien adiciona la variación en algunos coeficientes de <span style='color:#ff5500;'>$A$</span>, y a partir de ello si <span style='color:#ff5500;'>$I-A'$</span> es invertible, tenemos que
    <span style='color:#644a9b;'>\ie\label</span>{ce_1}
    X'=L'Y
    <span style='color:#644a9b;'>\fe</span>
    además, siguiendo la idea en la demostración del teorema de Sherman y Morrison (1950) (un caso particular del Teorema de Woodbury), es posible identificar <span style='color:#ff5500;'>$L'$</span> bajo cambios <span style='color:#ff5500;'>$E$</span> en <span style='color:#ff5500;'>$A$</span>, donde <span style='color:#ff5500;'>$E$</span> es no nula solo en una o en una fila,
    <span style='color:#644a9b;'>\ies</span>
    L'=(L^{-1}-E)^{-1}=(L^{-1}-ELL^{-1})^{-1}
    =((I-EL)L^{-1})^{-1}
    =L(I-EL)^{-1}
    <span style='color:#644a9b;'>\fes</span> 
    inclusive es posible identificar la matriz de variación <span style='color:#ff5500;'>$</span><span style='color:#3daee9;'>\Delta</span><span style='color:#ff5500;'> L=L'-L$</span>, 
    <span style='color:#644a9b;'>\ie\label</span>{eva}
    <span style='color:#644a9b;'>\Delta</span> L=L(I-EL)^{-1}-L=L(I-EL)^{-1}-L(I-EL)(I-EL)^{-1}=LEL(I-EL)^{-1}
    <span style='color:#644a9b;'>\fe</span>
    ahora, si hacemos que unicamente cambien los coeficientes de <span style='color:#ff5500;'>$A$</span> en la fila <span style='color:#ff5500;'>$r$</span>, unicamente <span style='color:#ff5500;'>$E_{r:}$</span> será no nula. Así, <span style='color:#ff5500;'>$EL$</span> será no nula solo en su fila <span style='color:#ff5500;'>$r$</span>, y <span style='color:#ff5500;'>$EL_{r:}=E_{r1}A_{1:}+</span><span style='color:#3daee9;'>\cdots</span><span style='color:#ff5500;'>+E_{rn}A_{n:}$</span>. Si consideramos <span style='color:#ff5500;'>$I-EL$</span> como la matriz por bloques dada como
    <span style='color:#644a9b;'>\ies\footnotesize</span>
    I-EL=<b>\begin</b>{<b><span style='color:#0095ff;'>pmatrix</span></b>}
<span style='color:#ff5500;'>    B_1&amp;B_2</span><span style='color:#3daee9;'>\\</span>
<span style='color:#ff5500;'>    B_3&amp;B_4      </span>
<span style='color:#ff5500;'>        </span><b>\end</b>{<b><span style='color:#0095ff;'>pmatrix</span></b>}=<b>\begin</b>{<b><span style='color:#0095ff;'>pmatrix</span></b>}
<span style='color:#ff5500;'>        </span><b>\begin</b>{<b><span style='color:#0095ff;'>pmatrix</span></b>}
<span style='color:#ff5500;'>        1&amp;0&amp;</span><span style='color:#3daee9;'>\cdots</span><span style='color:#ff5500;'>&amp;0</span><span style='color:#3daee9;'>\\</span>
<span style='color:#ff5500;'>        0&amp;1&amp;</span><span style='color:#3daee9;'>\cdots</span><span style='color:#ff5500;'>&amp;0</span><span style='color:#3daee9;'>\\</span>
<span style='color:#ff5500;'>        </span><span style='color:#3daee9;'>\vdots</span><span style='color:#ff5500;'>&amp;</span><span style='color:#3daee9;'>\vdots</span><span style='color:#ff5500;'>&amp;</span><span style='color:#3daee9;'>\ddots</span><span style='color:#ff5500;'>&amp;</span><span style='color:#3daee9;'>\vdots\\</span>
<span style='color:#ff5500;'>        -(EL)_{r1}&amp;-(EL)_{r2}&amp;</span><span style='color:#3daee9;'>\cdots</span><span style='color:#ff5500;'>&amp;1-(EL)_{rr}</span>
<span style='color:#ff5500;'>        </span><b>\end</b>{<b><span style='color:#0095ff;'>pmatrix</span></b>}&amp;<b>\begin</b>{<b><span style='color:#0095ff;'>pmatrix</span></b>}
<span style='color:#ff5500;'>        0&amp;</span><span style='color:#3daee9;'>\cdots</span><span style='color:#ff5500;'>&amp;0</span><span style='color:#3daee9;'>\\</span>
<span style='color:#ff5500;'>        </span><span style='color:#3daee9;'>\vdots</span><span style='color:#ff5500;'>&amp;</span><span style='color:#3daee9;'>\ddots</span><span style='color:#ff5500;'>&amp;</span><span style='color:#3daee9;'>\vdots\\</span>
<span style='color:#ff5500;'>        0&amp;</span><span style='color:#3daee9;'>\cdots</span><span style='color:#ff5500;'>&amp;0</span><span style='color:#3daee9;'>\\</span>
<span style='color:#ff5500;'>        -(EL)_{r(r+1)}&amp;</span><span style='color:#3daee9;'>\cdots</span><span style='color:#ff5500;'>&amp;-(EL)_{rn}</span>
<span style='color:#ff5500;'>        </span><b>\end</b>{<b><span style='color:#0095ff;'>pmatrix</span></b>}<span style='color:#644a9b;'>\\</span>
        <b>\begin</b>{<b><span style='color:#0095ff;'>pmatrix</span></b>}
<span style='color:#ff5500;'>        0&amp;0&amp;</span><span style='color:#3daee9;'>\cdots</span><span style='color:#ff5500;'>&amp;0</span><span style='color:#3daee9;'>\\</span>
<span style='color:#ff5500;'>        </span><span style='color:#3daee9;'>\vdots</span><span style='color:#ff5500;'>&amp;</span><span style='color:#3daee9;'>\vdots</span><span style='color:#ff5500;'>&amp;</span><span style='color:#3daee9;'>\ddots</span><span style='color:#ff5500;'>&amp;</span><span style='color:#3daee9;'>\vdots\\</span>
<span style='color:#ff5500;'>        </span><span style='color:#3daee9;'>\hspace</span><span style='color:#ff5500;'>{.6cm}0</span><span style='color:#3daee9;'>\hspace</span><span style='color:#ff5500;'>{.6cm}</span><span style='color:#3daee9;'>\ </span><span style='color:#ff5500;'>&amp;</span><span style='color:#3daee9;'>\hspace</span><span style='color:#ff5500;'>{.6cm}0</span><span style='color:#3daee9;'>\hspace</span><span style='color:#ff5500;'>{.6cm}</span><span style='color:#3daee9;'>\ </span><span style='color:#ff5500;'>&amp;</span><span style='color:#3daee9;'>\cdots</span><span style='color:#ff5500;'>&amp;</span><span style='color:#3daee9;'>\hspace</span><span style='color:#ff5500;'>{.6cm}0</span><span style='color:#3daee9;'>\hspace</span><span style='color:#ff5500;'>{.6cm}</span><span style='color:#3daee9;'>\ </span>
<span style='color:#ff5500;'>        </span><b>\end</b>{<b><span style='color:#0095ff;'>pmatrix</span></b>}
    &amp; <b>\begin</b>{<b><span style='color:#0095ff;'>pmatrix</span></b>}
<span style='color:#ff5500;'>        1&amp;</span><span style='color:#3daee9;'>\cdots</span><span style='color:#ff5500;'>&amp;0</span><span style='color:#3daee9;'>\\</span>
<span style='color:#ff5500;'>        </span><span style='color:#3daee9;'>\vdots</span><span style='color:#ff5500;'>&amp;</span><span style='color:#3daee9;'>\ddots</span><span style='color:#ff5500;'>&amp;</span><span style='color:#3daee9;'>\vdots\\</span>
<span style='color:#ff5500;'>        </span><span style='color:#3daee9;'>\hspace</span><span style='color:#ff5500;'>{.6cm}0</span><span style='color:#3daee9;'>\hspace</span><span style='color:#ff5500;'>{.6cm}</span><span style='color:#3daee9;'>\ </span><span style='color:#ff5500;'>&amp;</span><span style='color:#3daee9;'>\cdots</span><span style='color:#ff5500;'>&amp;</span><span style='color:#3daee9;'>\hspace</span><span style='color:#ff5500;'>{.6cm}1</span><span style='color:#3daee9;'>\hspace</span><span style='color:#ff5500;'>{.6cm}</span><span style='color:#3daee9;'>\ </span>
<span style='color:#ff5500;'>        </span><b>\end</b>{<b><span style='color:#0095ff;'>pmatrix</span></b>}
        <b>\end</b>{<b><span style='color:#0095ff;'>pmatrix</span></b>} 
    <span style='color:#644a9b;'>\fes</span>
    y mediante una expresión conocida,<span style='color:#644a9b;'>\footnote</span>{Esta expresión relaciona la inversa de una matriz por bloques con sus bloques e inversas de sus bloques. Es muy conocida y puede hallarse en varios libros de álgebra lineal y artículos, por ejemplo, en la igualdad 10 del artículo de <b>\citet</b>{<b><span style='color:#0095ff;'>cal</span></b>}.}<span style='color:#898887;'>% o en el libro de \citet{horn}.} </span>
    podemos expresar la relación entre <span style='color:#ff5500;'>$(I-EL)^{-1}$</span> con <span style='color:#ff5500;'>$B_1, B_2, B_3=0_M,B_4=I_{n-1}, B_1^{-1}, B_4^{-1}=I_{n-r}$</span> como   
    <span style='color:#644a9b;'>\ies</span>
    (I-EL)^{-1}=<b>\begin</b>{<b><span style='color:#0095ff;'>pmatrix</span></b>}
<span style='color:#ff5500;'>                B_1^{-1}&amp;-B_1^{-1}B_2</span><span style='color:#3daee9;'>\\</span>
<span style='color:#ff5500;'>                0_M&amp;I_{n-r}</span>
<span style='color:#ff5500;'>                </span><b>\end</b>{<b><span style='color:#0095ff;'>pmatrix</span></b>}
    <span style='color:#644a9b;'>\fes</span>
    es más, debido a que podemos ver a <span style='color:#ff5500;'>$B_1$</span> de la siguiente manera
    <span style='color:#644a9b;'>\ies</span>
    B_1=<b>\begin</b>{<b><span style='color:#0095ff;'>pmatrix</span></b>}
<span style='color:#ff5500;'>        B_{11}&amp;B_{12}</span><span style='color:#3daee9;'>\\</span>
<span style='color:#ff5500;'>        B_{21}&amp;B_{22}</span>
<span style='color:#ff5500;'>        </span><b>\end</b>{<b><span style='color:#0095ff;'>pmatrix</span></b>}=
    <b>\begin</b>{<b><span style='color:#0095ff;'>pmatrix</span></b>}
<span style='color:#ff5500;'>        </span><b>\begin</b>{<b><span style='color:#0095ff;'>pmatrix</span></b>}
<span style='color:#ff5500;'>        1&amp;</span><span style='color:#3daee9;'>\cdots</span><span style='color:#ff5500;'>&amp;0</span><span style='color:#3daee9;'>\\</span>
<span style='color:#ff5500;'>        </span><span style='color:#3daee9;'>\vdots</span><span style='color:#ff5500;'>&amp;</span><span style='color:#3daee9;'>\ddots</span><span style='color:#ff5500;'>&amp;</span><span style='color:#3daee9;'>\vdots\\</span>
<span style='color:#ff5500;'>        </span><span style='color:#3daee9;'>\hspace</span><span style='color:#ff5500;'>{.8cm}0</span><span style='color:#3daee9;'>\hspace</span><span style='color:#ff5500;'>{.8cm}</span><span style='color:#3daee9;'>\ </span><span style='color:#ff5500;'>&amp;</span><span style='color:#3daee9;'>\cdots</span><span style='color:#ff5500;'>&amp;</span><span style='color:#3daee9;'>\hspace</span><span style='color:#ff5500;'>{.8cm}1</span><span style='color:#3daee9;'>\hspace</span><span style='color:#ff5500;'>{.8cm}</span><span style='color:#3daee9;'>\ </span>
<span style='color:#ff5500;'>        </span><b>\end</b>{<b><span style='color:#0095ff;'>pmatrix</span></b>}&amp;<b>\begin</b>{<b><span style='color:#0095ff;'>pmatrix</span></b>}
<span style='color:#ff5500;'>        0</span><span style='color:#3daee9;'>\\</span>
<span style='color:#ff5500;'>        </span><span style='color:#3daee9;'>\vdots\\</span>
<span style='color:#ff5500;'>        0 </span>
<span style='color:#ff5500;'>        </span><b>\end</b>{<b><span style='color:#0095ff;'>pmatrix</span></b>}<span style='color:#644a9b;'>\\</span>
        <b>\begin</b>{<b><span style='color:#0095ff;'>pmatrix</span></b>}<span style='color:#ff5500;'>-(EL)_{r1}&amp;</span><span style='color:#3daee9;'>\cdots</span><span style='color:#ff5500;'>&amp;-(EL)_{r(r-1)}</span>
<span style='color:#ff5500;'>        </span><b>\end</b>{<b><span style='color:#0095ff;'>pmatrix</span></b>}&amp;<b>\begin</b>{<b><span style='color:#0095ff;'>pmatrix</span></b>}
<span style='color:#ff5500;'>        1-(EL)_{rr}</span>
<span style='color:#ff5500;'>        </span><b>\end</b>{<b><span style='color:#0095ff;'>pmatrix</span></b>}
        <b>\end</b>{<b><span style='color:#0095ff;'>pmatrix</span></b>}
    <span style='color:#644a9b;'>\fes</span>
    aplicando de nueva cuenta la expresión usada anteriormente tenemos que
    <span style='color:#644a9b;'>\ies</span>
    B_1^{-1}=<b>\begin</b>{<b><span style='color:#0095ff;'>pmatrix</span></b>}
<span style='color:#ff5500;'>            I_{r-1}&amp;</span><span style='color:#3daee9;'>\vec</span><span style='color:#ff5500;'>{0}</span><span style='color:#3daee9;'>\\</span>
<span style='color:#ff5500;'>            -</span><span style='color:#3daee9;'>\frac</span><span style='color:#ff5500;'>{1}{1-(EL)_{rr}}B_{21}&amp;</span><span style='color:#3daee9;'>\frac</span><span style='color:#ff5500;'>{1}{1-(EL)_{rr}}</span>
<span style='color:#ff5500;'>            </span><b>\end</b>{<b><span style='color:#0095ff;'>pmatrix</span></b>}
    <span style='color:#644a9b;'>\fes</span>
    por consiguiente, 
    <span style='color:#644a9b;'>\ie\label</span>{var_E}
    (I-EL)^{-1}=
    <b>\begin</b>{<b><span style='color:#0095ff;'>pmatrix</span></b>}
<span style='color:#ff5500;'>        </span><b>\begin</b>{<b><span style='color:#0095ff;'>pmatrix</span></b>}
<span style='color:#ff5500;'>            I_{r-1}&amp;</span><span style='color:#3daee9;'>\vec</span><span style='color:#ff5500;'>{0}</span><span style='color:#3daee9;'>\\</span>
<span style='color:#ff5500;'>            -</span><span style='color:#3daee9;'>\frac</span><span style='color:#ff5500;'>{1}{1-(EL)_{rr}}B_{21}&amp;</span><span style='color:#3daee9;'>\frac</span><span style='color:#ff5500;'>{1}{1-(EL)_{rr}}</span>
<span style='color:#ff5500;'>            </span><b>\end</b>{<b><span style='color:#0095ff;'>pmatrix</span></b>}&amp;<span style='color:#644a9b;'>\frac</span>{1}{1-(EL)_{rr}}<b>\begin</b>{<b><span style='color:#0095ff;'>pmatrix</span></b>}
<span style='color:#ff5500;'>        </span><span style='color:#3daee9;'>\vec</span><span style='color:#ff5500;'>{0}&amp;</span><span style='color:#3daee9;'>\cdots</span><span style='color:#ff5500;'>&amp;</span><span style='color:#3daee9;'>\vec</span><span style='color:#ff5500;'>{0}</span><span style='color:#3daee9;'>\\</span>
<span style='color:#ff5500;'>        (EL)_{r(r+1)}&amp;</span><span style='color:#3daee9;'>\cdots</span><span style='color:#ff5500;'>&amp;(EL)_{rn}</span>
<span style='color:#ff5500;'>        </span><b>\end</b>{<b><span style='color:#0095ff;'>pmatrix</span></b>}<span style='color:#644a9b;'>\\</span>
        <b>\begin</b>{<b><span style='color:#0095ff;'>pmatrix</span></b>}
<span style='color:#ff5500;'>        0&amp;</span><span style='color:#3daee9;'>\cdots</span><span style='color:#ff5500;'>&amp;0</span><span style='color:#3daee9;'>\\</span>
<span style='color:#ff5500;'>        </span><span style='color:#3daee9;'>\vdots</span><span style='color:#ff5500;'>&amp;</span><span style='color:#3daee9;'>\ddots</span><span style='color:#ff5500;'>&amp;</span><span style='color:#3daee9;'>\vdots\\</span>
<span style='color:#ff5500;'>        </span><span style='color:#3daee9;'>\hspace</span><span style='color:#ff5500;'>{.6cm}0</span><span style='color:#3daee9;'>\hspace</span><span style='color:#ff5500;'>{.6cm}</span><span style='color:#3daee9;'>\ </span><span style='color:#ff5500;'>&amp;</span><span style='color:#3daee9;'>\cdots</span><span style='color:#ff5500;'>&amp;</span><span style='color:#3daee9;'>\hspace</span><span style='color:#ff5500;'>{.6cm}0</span><span style='color:#3daee9;'>\hspace</span><span style='color:#ff5500;'>{.6cm}</span><span style='color:#3daee9;'>\ </span>
<span style='color:#ff5500;'>        </span><b>\end</b>{<b><span style='color:#0095ff;'>pmatrix</span></b>}&amp;<span style='color:#644a9b;'>\hspace</span>{3.5em}<b>\begin</b>{<b><span style='color:#0095ff;'>pmatrix</span></b>}
<span style='color:#ff5500;'>        1&amp;</span><span style='color:#3daee9;'>\cdots</span><span style='color:#ff5500;'>&amp;0</span><span style='color:#3daee9;'>\\</span>
<span style='color:#ff5500;'>        </span><span style='color:#3daee9;'>\vdots</span><span style='color:#ff5500;'>&amp;</span><span style='color:#3daee9;'>\ddots</span><span style='color:#ff5500;'>&amp;</span><span style='color:#3daee9;'>\vdots\\</span>
<span style='color:#ff5500;'>        </span><span style='color:#3daee9;'>\hspace</span><span style='color:#ff5500;'>{.6cm}0</span><span style='color:#3daee9;'>\hspace</span><span style='color:#ff5500;'>{.6cm}</span><span style='color:#3daee9;'>\ </span><span style='color:#ff5500;'>&amp;</span><span style='color:#3daee9;'>\cdots</span><span style='color:#ff5500;'>&amp;</span><span style='color:#3daee9;'>\hspace</span><span style='color:#ff5500;'>{.6cm}1</span><span style='color:#3daee9;'>\hspace</span><span style='color:#ff5500;'>{.6cm}</span><span style='color:#3daee9;'>\ </span>
<span style='color:#ff5500;'>        </span><b>\end</b>{<b><span style='color:#0095ff;'>pmatrix</span></b>}
        <b>\end</b>{<b><span style='color:#0095ff;'>pmatrix</span></b>}
    <span style='color:#644a9b;'>\fe</span>
    Por otro lado, como <span style='color:#ff5500;'>$E$</span> es nula en toda fila distinta de la <span style='color:#ff5500;'>$r$</span>-ésima, tenemos que 
    <span style='color:#644a9b;'>\ies</span>
    (LEL)_{ik}=<span style='color:#644a9b;'>\sum</span>_{s,u}L_{is}E_{su}L_{uj}=<span style='color:#644a9b;'>\sum</span>_{u}L_{ir}E_{ru}L_{uk}=L_{ir}<span style='color:#644a9b;'>\sum</span>_{u}E_{ru}L_{uk}=L_{ir}(EL)_{rk} 
    <span style='color:#644a9b;'>\fes</span>
    por consiguiente, de esta igualdad tenemos
    <span style='color:#644a9b;'>\ies</span>
    (I-EL)^{-1}_{ij}=(LEL(I-EL)^{-1})_{ij}
    =<span style='color:#644a9b;'>\sum</span>_{k}(LEL)_{ik}((I-EL)^{-1})_{kj}
    =L_{ir}<span style='color:#644a9b;'>\sum</span>_{k}(EL)_{rk}((I-EL)^{-1})_{kj}
    <span style='color:#644a9b;'>\fes</span>
    y <span style='color:#644a9b;'>\textbf</span>{(i)} si <span style='color:#ff5500;'>$j=r$</span>, entonces a partir de <b>\eqref</b>{<b><span style='color:#0095ff;'>var_E</span></b>} 
    <span style='color:#644a9b;'>\ies</span>
    L_{ir}<span style='color:#644a9b;'>\sum</span>_{k}(EL)_{rk}((I-EL)^{-1})_{kj}=<span style='color:#644a9b;'>\frac</span>{L_{ir}(EL)_{rj}}{1-(EL)_{rj}}
    <span style='color:#644a9b;'>\fes</span>
    a su vez, <span style='color:#644a9b;'>\textbf</span>{(ii)} si <span style='color:#ff5500;'>$j</span><span style='color:#3daee9;'>\neq</span><span style='color:#ff5500;'> r$</span>, a partir de <b>\eqref</b>{<b><span style='color:#0095ff;'>var_E</span></b>}
    <span style='color:#644a9b;'>\ies</span>
    L_{ir}<span style='color:#644a9b;'>\sum</span>_{k}(EL)_{rk}((I-EL)^{-1})_{kj}&amp;=&amp;L_{ir}(EL)_{rj}+<span style='color:#644a9b;'>\frac</span>{L_{ir}(EL)_{rr}}{1-(EL)_{rr}}(EL)_{rj}<span style='color:#644a9b;'>\\</span>
    &amp;=&amp;L_{ir}<span style='color:#644a9b;'>\left</span>(1-<span style='color:#644a9b;'>\frac</span>{(EL)_{rr}}{1-(EL)_{rr}}<span style='color:#644a9b;'>\right</span>)(EL)_{rj}<span style='color:#644a9b;'>\\</span>
    &amp;=&amp;L_{ir}<span style='color:#644a9b;'>\left</span>(<span style='color:#644a9b;'>\frac</span>{1-(EL)_{rr}+(EL)_{rr}}{1-(EL)_{rr}}<span style='color:#644a9b;'>\right</span>)(EL)_{rj}<span style='color:#644a9b;'>\\</span>
    &amp;=&amp;L_{ir}<span style='color:#644a9b;'>\left</span>(<span style='color:#644a9b;'>\frac</span>{1}{1-(EL)_{rr}}<span style='color:#644a9b;'>\right</span>)(EL)_{rj}<span style='color:#644a9b;'>\\</span>
    &amp;=&amp;<span style='color:#644a9b;'>\frac</span>{L_{ir}(EL)_{rj}}{1-(EL)_{rr}}
    <span style='color:#644a9b;'>\fes</span>
    <span style='color:#644a9b;'>\-</span>
    es decir, para todo <span style='color:#ff5500;'>$i, j=1,</span><span style='color:#3daee9;'>\dots</span><span style='color:#ff5500;'>,n$</span>, tenemos que 
    <span style='color:#644a9b;'>\ies</span>
    <span style='color:#644a9b;'>\Delta</span> L_{ij}=<span style='color:#644a9b;'>\frac</span>{L_{ir}(EL)_{rj}}{1-(EL)_{rr}}
    <span style='color:#644a9b;'>\fes</span>
    Por medio de este desarrollo, podemos ver como fue que Evans(1954), tomando lo generado por Sherman y Morrison (1950), llegó a su fórmula de medición del efecto de la variación de una fila de <span style='color:#ff5500;'>$A$</span> en los elementos de la matriz de Leontief
    <span style='color:#644a9b;'>\ies</span>
    <span style='color:#644a9b;'>\Delta</span> L_{ij}=
    <span style='color:#644a9b;'>\frac</span>{L_{ir}<span style='color:#644a9b;'>\sum</span>_uE_{ru}L_{uj}}{1-<span style='color:#644a9b;'>\sum</span>_uE_{ru}L_{ur}}
    <span style='color:#644a9b;'>\fes</span>
    incluso, entender cómo es que llegó a perfilar el efecto del cambio en un sólo elemento de la matriz <span style='color:#ff5500;'>$A$</span>, digamos, en <span style='color:#ff5500;'>$r, s$</span>, 
    <span style='color:#644a9b;'>\ies</span>
    <span style='color:#644a9b;'>\Delta</span> L_{ij}=
    <span style='color:#644a9b;'>\frac</span>{L_{ir}E_{rs}L_{sj }}{1-E_{rs}L_{sr}}
    <span style='color:#644a9b;'>\fes</span>
    A partir de lo generado por Evans, y si solo cambiamos el elemento <span style='color:#ff5500;'>$r$</span>, <span style='color:#ff5500;'>$s$</span> de <span style='color:#ff5500;'>$A$</span>, primero vemos que 
    <span style='color:#644a9b;'>\ies</span>
    <span style='color:#644a9b;'>\Delta</span> X_i=<span style='color:#644a9b;'>\Delta</span> (L Y)_{i}=<span style='color:#644a9b;'>\sum</span>_{u}<span style='color:#644a9b;'>\Delta</span> L_{iu}Y_u=<span style='color:#644a9b;'>\sum</span>_{u}<span style='color:#644a9b;'>\frac</span>{L_{ir}E_{rs}L_{su }}{1-E_{rs}L_{sr}}Y_u
    =<span style='color:#644a9b;'>\frac</span>{L_{ir}E_{rs}<span style='color:#644a9b;'>\sum</span>_{u}L_{su}Y_u}{1-E_{rs}L_{sr}}=
    <span style='color:#644a9b;'>\frac</span>{L_{ir}E_{rs}X_s}{1-E_{rs}L_{sr}}
    <span style='color:#644a9b;'>\fes</span>
    así podemos obtener el máximo de los porcentajes de crecimiento del producto de cada sector dada una variación <span style='color:#ff5500;'>$E_{rs}$</span> en  <span style='color:#ff5500;'>$A_{rs}$</span> 
    <span style='color:#644a9b;'>\ies</span>
    <span style='color:#644a9b;'>\max\limits</span>_i<span style='color:#644a9b;'>\frac</span>{<span style='color:#644a9b;'>\Delta</span> X_i}{X_i}=<span style='color:#644a9b;'>\max\limits</span>_i<span style='color:#644a9b;'>\frac</span>{L_{ir}E_{rs}X_s}{X_i(1-E_{rs}L_{sr})}=
    <span style='color:#644a9b;'>\frac</span>{E_{rs}X_s}{1-E_{rs}L_{sr}}<span style='color:#644a9b;'>\max\limits</span>_i<span style='color:#644a9b;'>\frac</span>{L_{ir}}{X_i}
    <span style='color:#644a9b;'>\fes</span>
    Luego, lo que podemos buscar es identificar el tamaño de <span style='color:#ff5500;'>$E_{rs}$</span> tal que <span style='color:#ff5500;'>$</span><span style='color:#3daee9;'>\max\limits</span><span style='color:#ff5500;'>_i</span><span style='color:#3daee9;'>\frac</span><span style='color:#ff5500;'>{</span><span style='color:#3daee9;'>\Delta</span><span style='color:#ff5500;'> X_i}{X_i}</span><span style='color:#3daee9;'>\leq</span><span style='color:#ff5500;'> </span><span style='color:#3daee9;'>\rho</span><span style='color:#ff5500;'>$</span>, para un valor <span style='color:#ff5500;'>$</span><span style='color:#3daee9;'>\rho</span><span style='color:#ff5500;'>$</span> dado. Esto se logra aplicando un poco de álgebra, y lo que se obtiene es 
    <span style='color:#644a9b;'>\ies</span>
    E_{rs}<span style='color:#644a9b;'>\leq\frac</span>{<span style='color:#644a9b;'>\rho</span>}{X_s<span style='color:#644a9b;'>\max\limits</span>_i<span style='color:#644a9b;'>\frac</span>{L_{ir}}{X_i}+<span style='color:#644a9b;'>\rho</span> L_{sr}}
    <span style='color:#644a9b;'>\fes</span>
    Ahora, si esa desigualdad la dividimos entre <span style='color:#ff5500;'>$A_{rs}$</span> y la multiplicamos por <span style='color:#ff5500;'>$100$</span>, obtenemos 
    <span style='color:#644a9b;'>\ies</span>
    100<span style='color:#644a9b;'>\frac</span>{E_{rs}}{A_{rs}}<span style='color:#644a9b;'>\leq\frac</span>{100<span style='color:#644a9b;'>\rho</span>}{A_{rs}(X_sL_{rr}+<span style='color:#644a9b;'>\rho</span> L_{sr})}=
    <span style='color:#644a9b;'>\frac</span>{100<span style='color:#644a9b;'>\rho</span>}{A_{rs}(X_s<span style='color:#644a9b;'>\max\limits</span>_i<span style='color:#644a9b;'>\frac</span>{L_{ir}}{X_i}+<span style='color:#644a9b;'>\rho</span> L_{sr})}
    <span style='color:#644a9b;'>\fes</span>
    lo cuál significa que <span style='color:#ff5500;'>$100</span><span style='color:#3daee9;'>\frac</span><span style='color:#ff5500;'>{E_{rs}}{A_{rs}}$</span>, el porcentaje de crecimiento del coeficiente <span style='color:#ff5500;'>$A_{rs}$</span>, debe de ser a lo más igual a la expresión derecha para que pueda tenerse que <span style='color:#ff5500;'>$</span><span style='color:#3daee9;'>\max\limits</span><span style='color:#ff5500;'>_i</span><span style='color:#3daee9;'>\Delta</span><span style='color:#ff5500;'> X_i</span><span style='color:#3daee9;'>\leq</span><span style='color:#ff5500;'> </span><span style='color:#3daee9;'>\rho</span><span style='color:#ff5500;'>$</span>. A partir de esto podemos ver como Sekulic (1968) definió la siguiente función que identifica los coeficientes más significativos
    <span style='color:#644a9b;'>\ies</span>
    r_{ij}(p)=<span style='color:#644a9b;'>\frac</span>{100<span style='color:#644a9b;'>\rho</span>}{A_{ij}<span style='color:#644a9b;'>\left</span>(X_j<span style='color:#644a9b;'>\max\limits</span>_k<span style='color:#644a9b;'>\frac</span>{L_{ki}}{X_k}+<span style='color:#644a9b;'>\rho</span> L_{ji}<span style='color:#644a9b;'>\right</span>)}
    <span style='color:#644a9b;'>\fes</span>
    Entre más pequeño <span style='color:#ff5500;'>$r_{ij}$</span>, el porcentaje de crecimiento máximo de <span style='color:#ff5500;'>$A_{ij}$</span> sera menor, pero su efecto sobre la tasa de crecimiento del producto de los sectores es tal que el máximo de ellos igualará <span style='color:#ff5500;'>$</span><span style='color:#3daee9;'>\rho</span><span style='color:#ff5500;'>$</span>, es decir, <span style='color:#644a9b;'>\textbf</span>{genera la misma variación <span style='color:#ff5500;'>$</span><span style='color:#3daee9;'>\rho</span><span style='color:#ff5500;'>$</span> del producto con menor esfuerzo}. Así mismo, Jilek (1971) definió la función  
    <span style='color:#644a9b;'>\ies</span>
    r_{ij}=<span style='color:#644a9b;'>\frac</span>{1}{A_{ij}<span style='color:#644a9b;'>\left</span>(X_j<span style='color:#644a9b;'>\max\limits</span>_k<span style='color:#644a9b;'>\frac</span>{L_{ki}}{W_k}X_j+0.01L_{ij}<span style='color:#644a9b;'>\right</span>)}
    <span style='color:#644a9b;'>\fes</span>
    la cuál sigue la misma idea planteada anteriormente con <span style='color:#ff5500;'>$p=1</span><span style='color:#3daee9;'>\%</span><span style='color:#ff5500;'> (=0.01)$</span>, con la sola desviación de que en la construcción usa el producto efectivo <span style='color:#ff5500;'>$W_k$</span> en vez del producto total <span style='color:#ff5500;'>$X_k$</span>.<span style='color:#644a9b;'>\-</span>
    
    Otra forma de deslindar la importancia de ciertos sectores<span style='color:#644a9b;'>\footnote</span>{Aunque Viet (1980) argumenta que matemáticamente otorgan resultados similares.} es mediante la propuesta hecha por Sebald (1974) y después retomada por Bullard y Sebald (1977). Parte de explicitar la función en  
    <span style='color:#644a9b;'>\ies</span>
    J=g((I-A)^{-1},Y)
    <span style='color:#644a9b;'>\fes</span>
    y así, a partir de variar en <span style='color:#ff5500;'>$E$</span> a la matriz <span style='color:#ff5500;'>$A$</span>, se analiza el cambio que sucitado en <span style='color:#ff5500;'>$J$</span>, donde se puede aplicar el desarrollo de Sherman y Morrison (1950) previamente desvelado, es decir, se busca la magnitud
    <span style='color:#644a9b;'>\ies</span>
    <span style='color:#644a9b;'>\Delta</span> J=g((I-(A+E))^{-1},Y)
    <span style='color:#644a9b;'>\fes</span>
    Luego, para la clasificación de coeficientes técnicos, se toma un valor <span style='color:#ff5500;'>$</span><span style='color:#3daee9;'>\tau</span><span style='color:#ff5500;'>$</span>, el cual es el umbral que deben atravesar las entradas <span style='color:#ff5500;'>$m,n$</span>-ésimas de <span style='color:#ff5500;'>$</span><span style='color:#3daee9;'>\Delta</span><span style='color:#ff5500;'> J$</span>. Atravesarlo significa que dichas entradas de <span style='color:#ff5500;'>$</span><span style='color:#3daee9;'>\Delta</span><span style='color:#ff5500;'> J$</span> fueron alteradas de manera significativa mediante la variación  <span style='color:#ff5500;'>$E$</span>. Ahora, para clasificar el coeficiente <span style='color:#ff5500;'>$A_{ij}$</span> (alterado por <span style='color:#ff5500;'>$E$</span>) primero definimos la función
    <span style='color:#644a9b;'>\ies</span>
    f_<span style='color:#644a9b;'>\tau</span>(x)=<span style='color:#644a9b;'>\left\{</span><b>\begin</b>{<b><span style='color:#0095ff;'>matrix</span></b>}x<span style='color:#644a9b;'>\tau</span>^{-1}&amp;x<span style='color:#644a9b;'>\geq\tau\\</span>
    0&amp;otro<span style='color:#644a9b;'>\ </span>caso
    <b>\end</b>{<b><span style='color:#0095ff;'>matrix</span></b>}<span style='color:#644a9b;'>\right</span>.
    <span style='color:#644a9b;'>\fes</span>
    y luego se obtiene la posición de <span style='color:#ff5500;'>$A_{ij}$</span> mediante<span style='color:#644a9b;'>\footnote</span>{Esta expresión es equivalente a la expresada por Bullard y Sebald (1977).}
    <span style='color:#644a9b;'>\ies</span>
    p_{ij}=<span style='color:#644a9b;'>\sum</span>_{m, n}f_<span style='color:#644a9b;'>\tau</span>((<span style='color:#644a9b;'>\Delta</span> J)_{m<span style='color:#644a9b;'>\hspace</span>{.1em} n})
    <span style='color:#644a9b;'>\fes</span>
    donde <span style='color:#ff5500;'>$p_{ij}$</span> es la posición del coeficiente <span style='color:#ff5500;'>$A_{i,j}$</span> medido mediante los valores de <span style='color:#ff5500;'>$</span><span style='color:#3daee9;'>\Delta</span><span style='color:#ff5500;'> J$</span> que ha acrecentado por arriba del umbral seleccionado.<span style='color:#644a9b;'>\-</span>
    
    El siguiente método parte de tomar la matriz de error <span style='color:#ff5500;'>$E$</span> que altera el coeficiente <span style='color:#ff5500;'>$A_{ij}$</span> y utilizar lo expuesto por Sherman y Morrison (), pero de manera diferente a lo confeccionado por  Evans (1954), ya que en este la adición del error se trabajaba de manera meramente aditiva durante el proceso, sin embargo, en este método ese error se tranformaba en un choque aleatorio de la siguiente manera
    <span style='color:#644a9b;'>\ies</span>
    I-(A+E)=(I-A)(I-R)
    <span style='color:#644a9b;'>\fes</span>
    Encontrar quien era la matriz <span style='color:#ff5500;'>$R$</span> que permitía tal separación se puede seguir al suponer su existencia y resolver la igualdad anterior, así
    <span style='color:#644a9b;'>\ies</span>
    R=(I-A)^{-1}(I-A-(I-(A+E)))=(I-A)^{-1}E=LE
    <span style='color:#644a9b;'>\fes</span>
    Además, como <span style='color:#ff5500;'>$E$</span> es no nula en su coeficiente <span style='color:#ff5500;'>$ij$</span>-ésimo 
    <span style='color:#644a9b;'>\-</span>

  


    
    La variación en <span style='color:#ff5500;'>$E$</span> se efectua de <span style='color:#ff5500;'>$r$</span> en <span style='color:#ff5500;'>$r$</span> elementos en su columna <span style='color:#ff5500;'>$E_j$</span>, <span style='color:#ff5500;'>$</span><span style='color:#3daee9;'>\f</span><span style='color:#ff5500;'> r</span><span style='color:#3daee9;'>\in\{</span><span style='color:#ff5500;'>1,</span><span style='color:#3daee9;'>\cdots</span><span style='color:#ff5500;'> n</span><span style='color:#3daee9;'>\}</span><span style='color:#ff5500;'>$</span>, y  bajo una variación pequeña y homogénea de estos (0.01 o 0.005). La idea fundamental es que a pequeños  incrementos en uno o un grupo de coeficientes técnicos de <span style='color:#ff5500;'>$A_j$</span>, se busca encontrar <span style='color:#644a9b;'>\textbf</span>{(i)} el mayor incremento <span style='color:#ff5500;'>$(</span><span style='color:#3daee9;'>\Delta</span><span style='color:#ff5500;'> X)_j$</span> y <span style='color:#644a9b;'>\textbf</span>{(ii)} el mayor incremento en <span style='color:#ff5500;'>$(</span><span style='color:#3daee9;'>\Delta</span><span style='color:#ff5500;'> X)_k$</span>, <span style='color:#ff5500;'>$</span><span style='color:#3daee9;'>\f</span><span style='color:#ff5500;'> k</span><span style='color:#3daee9;'>\neq</span><span style='color:#ff5500;'> j$</span>. Así mismo, el vector de demanda <span style='color:#ff5500;'>$Y$</span>, es una señal <span style='color:#644a9b;'>\-</span>


    El sector <span style='color:#ff5500;'>$S_j$</span> de cada país desarrollado o en el umbral del desarrollo, tiene un vector de coeficientes técnicos <span style='color:#ff5500;'>$A_j$</span>, ahora, empleando la fórmula expuesta por Schintke y Stäglin (1985)<span style='color:#644a9b;'>\footnote</span>{Revisar a West para ver si se utiliza su fórmula, ya que identifica la importancia de los coeficientes de forma amplia.}
    <span style='color:#ff5500;'>$$</span>
<span style='color:#ff5500;'>    w_{ij}= a_{ij}</span><span style='color:#3daee9;'>\left</span><span style='color:#ff5500;'>(b_{ji}p+b_{ii}</span><span style='color:#3daee9;'>\frac</span><span style='color:#ff5500;'>{X_j }{X_i}</span><span style='color:#3daee9;'>\right</span><span style='color:#ff5500;'>)</span>
<span style='color:#ff5500;'>    $$</span>
    donde <span style='color:#ff5500;'>$a_{ij}$</span> es el coeficiente técnico <span style='color:#ff5500;'>$ij$</span>-ésimo; <span style='color:#ff5500;'>$b_{ji}$</span> y <span style='color:#ff5500;'>$b_{ii}$</span> las entradas <span style='color:#ff5500;'>$ji$</span>-ésima e <span style='color:#ff5500;'>$ii$</span>-ésima de la matriz inversa de Leontief; <span style='color:#ff5500;'>$X_j$</span> e <span style='color:#ff5500;'>$X_i$</span> los montos de producción del sector <span style='color:#ff5500;'>$j$</span> e <span style='color:#ff5500;'>$i$</span> respectivamente, se determinaa la importancia de los coeficientes técnicos sobresalientes de <span style='color:#ff5500;'>$S_j$</span> cuando <span style='color:#ff5500;'>$w_{ij}$</span> es grande.<span style='color:#644a9b;'>\-</span>


    Dentro del conjunto de industrias que conceden insumos a <span style='color:#ff5500;'>$A_i$</span>, unicamente nos podemos enfocar en las de media-alta, media-baja y baja, ya que si nos enfocamos en las industrias estratégicas dentro de estas, todas apoyarian aportarían
    
    <b>\section</b>{<b>Identificación de sectores importantes mediante la Teoría de la Realización</b>}
    
    Si nosotros poseemos las matrices <span style='color:#ff5500;'>$L_1,</span><span style='color:#3daee9;'>\dots</span><span style='color:#ff5500;'>,L_n$</span>, las metodologías anteriores tendrían dificultades para darnos información sobre los efectos futuros  en <span style='color:#ff5500;'>$L$</span> e <span style='color:#ff5500;'>$Y$</span>, de variar uno o algunos de los elementos de <span style='color:#ff5500;'>$A$</span>, es decir, no podrian decirnos qué sector sería más importante en el tiempo <span style='color:#ff5500;'>$t&gt;n$</span>, conociendo las estructuras <span style='color:#ff5500;'>$L_1,</span><span style='color:#3daee9;'>\dots</span><span style='color:#ff5500;'>,L_n$</span>.<span style='color:#644a9b;'>\-</span>
    
    Una manera de poder incorporar una dinámica a los métodos estáticos sería a través de la TR. Observemos que esta teoría nos permite conocer las matrices desconocidas <span style='color:#ff5500;'>$L_t$</span>, con <span style='color:#ff5500;'>$t&gt;n$</span>, utilizando las matrices observadas <span style='color:#ff5500;'>$L_1,</span><span style='color:#3daee9;'>\dots</span><span style='color:#ff5500;'>,L_n$</span>. Aún más, permite encontrar la manera en que cambiamos de una a otra en el tiempo mediante la expresión
    <span style='color:#644a9b;'>\ies</span>
    L_t=GF^{t-1}H<span style='color:#644a9b;'>\hspace</span>{1cm} t=1,<span style='color:#644a9b;'>\dots</span>,n,<span style='color:#644a9b;'>\dots</span>,n+k,<span style='color:#644a9b;'>\dots</span>
    <span style='color:#644a9b;'>\fes</span>
    teniendo en cuenta que, como todo método de proyección, posee un umbral de precisión posterior al tamaño muestral, donde a partir de él es más amplia la desigualdad entre la estimación y la información empírica.<span style='color:#644a9b;'>\-</span>
    
    
    
    <span style='color:#644a9b;'>\textbf</span>{PRELIMINAR.} A partir de las matrices de Leontief <span style='color:#ff5500;'>$L_1,</span><span style='color:#3daee9;'>\dots</span><span style='color:#ff5500;'>,L_n$</span> más recientes del país que busca encaminarse a la estructura objetivo, se genera una realización <span style='color:#ff5500;'>$(F,G,H)$</span>.<span style='color:#644a9b;'>\footnote</span>{por revisar si es variante o invariante.} Luego, partiendo de los métodos de identificación de sectores importantes por medio de variaciones en los coeficientes técnicos, se tienen dos propuestas 
    <b>\begin</b>{<b><span style='color:#0095ff;'>enumerate</span></b>}
    <span style='color:#644a9b;'>\item</span> Variar los coeficientes técnicos de un sector en <span style='color:#ff5500;'>$A_1$</span> en una magnitud matricial <span style='color:#ff5500;'>$E$</span>; generar una realización sustituyendo la primer matriz por <span style='color:#ff5500;'>$L'_1=(I-A_1-E)^{-1}$</span>; evaluar el efecto dinámico de tal sustitución, es decir, el efecto sobre las matrices <span style='color:#ff5500;'>$</span><span style='color:#3daee9;'>\widehat</span><span style='color:#ff5500;'>{L'_t}=HF^{t-1}G$</span>; y determinar una función indicadora de la importancia dinámica del sector que fue variado, con el fin de jerarquizar los sectores.
    
    <span style='color:#644a9b;'>\item</span> Variar los coeficientes técnicos de un sector en cada <span style='color:#ff5500;'>$A_1,</span><span style='color:#3daee9;'>\dots</span><span style='color:#ff5500;'>,A_n$</span> en una idéntica magnitud matricial <span style='color:#ff5500;'>$E$</span>; generar una realización sustituyendo cada <span style='color:#ff5500;'>$L_i$</span> por <span style='color:#ff5500;'>$L'_i=(I-A_i-E)^{-1}$</span>; evaluar el efecto dinámico de tal sustitución, es decir, el efecto sobre las matrices <span style='color:#ff5500;'>$</span><span style='color:#3daee9;'>\widehat</span><span style='color:#ff5500;'>{L'_t}=HF^{t-1}G$</span>; y determinar una función indicadora de la importancia dinámica del sector que fue variado, con el fin de jerarquizar los sectores.
    <b>\end</b>{<b><span style='color:#0095ff;'>enumerate</span></b>}
    
    
    
    
    <b>\section</b>{<b>Determinación de la matriz objetivo (dinámica)</b>}
    
    <b>\section</b>{<b>Jerarquización de sectores y determinación</b><b><span style='color:#644a9b;'>\\</span></b><b> de capas de aproximación.</b>}

    
    
    
    <b>\section</b>{<b>Elementos de control estructural</b>}

    Las medidas para dirigir la estructura económica serán..
    <span style='color:#644a9b;'>\-</span>

    Explicaciones como las de Rostow nos hablaban sobre fases para el desarrollo; en cada cual habría que afectar al sistema de una manera e intensidad específica para poder inducir cambios estructurales encaminados a la conformación de una estructura emparejada a las de los países desarrollados.<span style='color:#644a9b;'>\-</span>




    <b>\begin</b>{<b><span style='color:#0095ff;'>table</span></b>}[H]<span style='color:#644a9b;'>\centering\footnotesize</span>
    <span style='color:#644a9b;'>\caption</span>{Teorías del desarrollo}
    <b>\begin</b>{<b><span style='color:#0095ff;'>tabular</span></b>}{ccl}
    Denominación <b>&amp;</b> Representantes <b>&amp;</b><span style='color:#644a9b;'>\hspace</span>{3em} Características<span style='color:#644a9b;'>\\\hline</span>
    <b>\begin</b>{<b><span style='color:#0095ff;'>tabular</span></b>}[c]{c}Teoría de la<span style='color:#644a9b;'>\\</span> Modernización <b>\end</b>{<b><span style='color:#0095ff;'>tabular</span></b>} <b>&amp;</b> <b>\begin</b>{<b><span style='color:#0095ff;'>tabular</span></b>}[c]{c}
    Rostow<span style='color:#644a9b;'>\\</span>
    Huntington<span style='color:#644a9b;'>\\</span>
    Watanuki<span style='color:#644a9b;'>\\</span>
    Crozier
    <b>\end</b>{<b><span style='color:#0095ff;'>tabular</span></b>}<b>&amp;</b> <b>\begin</b>{<b><span style='color:#0095ff;'>tabular</span></b>}[l]{l}1. Existen fases definidas, linealidad<span style='color:#644a9b;'>\\</span>
    2. Convergencia entre sociedades<span style='color:#644a9b;'>\\</span>
    3. Modelo americanizador y eurocéntrico<span style='color:#644a9b;'>\\</span>
    4. Al iniciarse es irreversible<span style='color:#644a9b;'>\\</span>
    5. Es un proceso deseable<span style='color:#644a9b;'>\\</span>
    6. El proceso es paulatino y largo<span style='color:#644a9b;'>\\</span>
    7. Énfasis en el individualismo
    <b>\end</b>{<b><span style='color:#0095ff;'>tabular</span></b>}
    <span style='color:#644a9b;'>\\</span>
    <b>\begin</b>{<b><span style='color:#0095ff;'>tabular</span></b>}[l]{l}Teoría de la<span style='color:#644a9b;'>\\</span> Dependencia <b>\end</b>{<b><span style='color:#0095ff;'>tabular</span></b>}<b>&amp;</b><b>\begin</b>{<b><span style='color:#0095ff;'>tabular</span></b>}[c]{c}Raúl Prebish<span style='color:#644a9b;'>\\</span>
    Gunder Frank<span style='color:#644a9b;'>\\</span>
    Dos Santos<span style='color:#644a9b;'>\\</span>
    Enrique Cardoso<span style='color:#644a9b;'>\\</span>
    E. Torres-Rivas<span style='color:#644a9b;'>\\</span>
    Samir Amin<span style='color:#644a9b;'>\\</span>
    Baran<span style='color:#644a9b;'>\\</span>
    Ladsberg
    <b>\end</b>{<b><span style='color:#0095ff;'>tabular</span></b>}<b>&amp;</b> 
    <b>\begin</b>{<b><span style='color:#0095ff;'>tabular</span></b>}[l]{l}1. Estudia a la periferia y el estado-nación<span style='color:#644a9b;'>\\</span>
    2. Aumenta la demanda, mayor salario<span style='color:#644a9b;'>\\</span>
    3. Desarrollo supeditado al centro<span style='color:#644a9b;'>\\</span>
    4. Crisis del Centro, más desarrollo<span style='color:#644a9b;'>\\</span>
    5. Cercanía al centro, menos desarrollo<span style='color:#644a9b;'>\\</span>
    6. Más necesidad tecnológica, menos desarrollo<span style='color:#644a9b;'>\\</span>
    7. Las transnacionales perjudican el desarrollo
    <b>\end</b>{<b><span style='color:#0095ff;'>tabular</span></b>}<span style='color:#644a9b;'>\\</span>
    <b>\begin</b>{<b><span style='color:#0095ff;'>tabular</span></b>}[c]{c}Teoría de los<span style='color:#644a9b;'>\\</span> Sistemas Mundiales <b>\end</b>{<b><span style='color:#0095ff;'>tabular</span></b>}<b>&amp;</b><b>\begin</b>{<b><span style='color:#0095ff;'>tabular</span></b>}[c]{c}
    Wallerstein<span style='color:#644a9b;'>\\</span>
    <b>\end</b>{<b><span style='color:#0095ff;'>tabular</span></b>}<b>&amp;</b> 
    <b>\begin</b>{<b><span style='color:#0095ff;'>tabular</span></b>}[l]{l}1. Sistemas mundo, no solo estado-nación <span style='color:#644a9b;'>\\</span>
    2. Clasificación centro, periferia y semiperiferia<span style='color:#644a9b;'>\\</span>
    3. Analiza sistemas sociales amplios<span style='color:#644a9b;'>\\</span>
    4. Hay movilidad arriba y abajo
    <span style='color:#644a9b;'>\vspace</span>{.6em}
    <b>\end</b>{<b><span style='color:#0095ff;'>tabular</span></b>}<span style='color:#644a9b;'>\\</span>
    <b>\begin</b>{<b><span style='color:#0095ff;'>tabular</span></b>}[c]{c}Teoría de la<span style='color:#644a9b;'>\\</span> Globalización <b>\end</b>{<b><span style='color:#0095ff;'>tabular</span></b>} <b>&amp;</b> <b>\begin</b>{<b><span style='color:#0095ff;'>tabular</span></b>}[c]{c}
    R
    <b>\end</b>{<b><span style='color:#0095ff;'>tabular</span></b>}<b>&amp;</b> <b>\begin</b>{<b><span style='color:#0095ff;'>tabular</span></b>}[l]{l}1. <span style='color:#644a9b;'>\\</span>
    2. 
    <b>\end</b>{<b><span style='color:#0095ff;'>tabular</span></b>}
    <b>\end</b>{<b><span style='color:#0095ff;'>tabular</span></b>}
    <b>\end</b>{<b><span style='color:#0095ff;'>table</span></b>}

    , dentro de las industrias nacionales se constituyen en estratégicas aquellas con mayor intensidad tecnológica. Sin embargo, existen otros criterios que restringen las industrias de un país sobre la cuales elegir bajo el criterio del valor agregado, entre estos contamos con los siguientes:

    <b>\begin</b>{<b><span style='color:#0095ff;'>table</span></b>}[H]<span style='color:#644a9b;'>\centering</span>
    <span style='color:#644a9b;'>\caption</span>{Elementos de control y restricciones}
    <b>\begin</b>{<b><span style='color:#0095ff;'>tabular</span></b>}{cl}
    <span style='color:#644a9b;'>\textbf</span>{Tipo}                                                                       <b>&amp;</b> <span style='color:#644a9b;'>\textbf</span>{Restricción} <span style='color:#644a9b;'>\\\hline</span>
    <b>\begin</b>{<b><span style='color:#0095ff;'>tabular</span></b>}[c]{<span style='color:#924c9d;'>@{}</span>c<span style='color:#924c9d;'>@{}</span>}
    Importación<b>\end</b>{<b><span style='color:#0095ff;'>tabular</span></b>}<b>&amp;</b> Suficiencia de insumos<span style='color:#644a9b;'>\\</span>
    Exportación<b>&amp;</b> Facilidad de inserción en mercados internacionales<span style='color:#644a9b;'>\\</span>
    Crédito<b>&amp;</b> Disponibilidad de financiamiento privado y/o público<span style='color:#644a9b;'>\\</span>
    Técnica<b>&amp;</b> Grado de conocimientos, información, habilidades y Know How<span style='color:#644a9b;'>\\</span>
    Gerencia<b>&amp;</b>  Organización e innovación<span style='color:#644a9b;'>\\</span>
    Institucional<b>&amp;</b> Coerción institucional y reglamentación<span style='color:#644a9b;'>\\</span>
    Política<b>&amp;</b>Control de la corrupción y mediación activa del gobierno<span style='color:#644a9b;'>\\</span>
    <b>\end</b>{<b><span style='color:#0095ff;'>tabular</span></b>}
    <b>\end</b>{<b><span style='color:#0095ff;'>table</span></b>}




    La determinación del plazo puede disponerse de forma tentativa al considerar que los países desarrollados o que están en el umbral del desarrollo, han tendió un periodo de industrialización en el rango de 20 a 120 años y que, además, es perceptible que entre más tardío el proceso de cambio estructural, más rápido ha sido el proceso de desarrollo.<span style='color:#644a9b;'>\-</span>
    
    <b>\section</b>{<b>Construcción de la sucesión de estructuras intermedias</b>}
    
    Lo visto anteriormente sobre las capas de aproximación muestra que ellas tienen el objetivo de brindar un orden de inserción de nuevos sectores en la economía, con el fin de converger a la matriz objetivo. Además, adosa una idea de nuevos sectores fuera de nuestros objetivos por capas que puede inducir la inserción. 
    
    <span style='color:#644a9b;'>\-</span>
    
    Por ejemplo, si se tienen 6 capas de aproximación, podremos agrupar los sectores de la economía en sectores clave de <span style='color:#ff5500;'>$i$</span>-ésima capa y sectores de acompañamiento. Los primeros son aquellos que deben de fortalecerse pues son indispensables para la inserción de los nuevos sectores determinados en las capas de aproximación. Por su parte, los sectores de acompañamiento son sectores que son fáciles de potenciar o que permiten enfrentar contratiempos macroeconómicos (inestabilidad en el tipo de cambio, por ejemplo)  
    
    <span style='color:#644a9b;'>\-</span>
     
    

    <b>\section</b>{<b>Evaluación y corrección</b>}

    Los efectos de las medidas pueden ser caprichosos y por ello se debe de contar con una herramienta que permita ``vigilar'' el comportamiento del proceso. 

    <span style='color:#644a9b;'>\-</span>


    Sea <span style='color:#ff5500;'>$L$</span> la estructura económica de un país en un cierto tiempo y sea <span style='color:#ff5500;'>$L'$</span> la matriz objetivo dinámica de dicho país. Entonces, conducir la estructura <span style='color:#ff5500;'>$L$</span> hacia <span style='color:#ff5500;'>$L'$</span> involucra teóricamente transitar una sucesión finita de estructuras <span style='color:#ff5500;'>$</span><span style='color:#3daee9;'>\{</span><span style='color:#ff5500;'>L_n</span><span style='color:#3daee9;'>\}</span><span style='color:#ff5500;'>_{n</span><span style='color:#3daee9;'>\in\N</span><span style='color:#ff5500;'>_k}$</span>, con <span style='color:#ff5500;'>$L_0=L$</span> y <span style='color:#ff5500;'>$L_k=L$</span>. A su vez, el recorrido debe considerar la atenuación de los efectos no deseados en la economía, los cuáles se definieron en la sección anterior, evitando el acortamiento excesivo del tamaño de la sucesión, así como los saltos abruptos y no asimilables por la economía entre uno y otro de los elementos en ella.

    <span style='color:#644a9b;'>\-</span>

    Debido a situaciones económicas y técnicas, los efectos reales sobre la estructura económica (o sucesión empírica) que surgen al aplicar los instrumentos de control pueden ser conocidos solo en ciertos lapsos de tiempo, los cuales no necesariamente son los deseables. Así, el sesgo entre la sucesión teórica y la empírica en tiempos más adecuados serían desconocidos, lo cual no permite ajustar los instrumentos de control. Sin embargo, como se explicó también en la sección anterior, existen variables observables en un tiempo adecuado las cuales están asociadas fuertemente con variaciones en la estructura económica, y las cuales pueden ser usadas como un indicador de la desviación  

    El impulso a sectores clave con el fin de converger a una estructura objetivo pueden traer efectos sobre la estructura de la demanda, del ingreso, de las importaciones, de costos industriales, así como en la heterogeneidad estructural, los niveles de vida, la contaminación, etc.<span style='color:#644a9b;'>\-</span>
    
    Para identificar... se desarrolla una red neuronal que pueda tomar por entradas esos elementos, y que permita evaluar la marcha del proceso e implementar correcciones. Aquí se recuerda que ...


<span style='color:#644a9b;'>\-</span>


<span style='color:#898887;'>%///////////////////////////////////////////////////////////////////////////////////////////////////////////////////////////////////////</span>
<span style='color:#898887;'>%///////////////////////////////////////////////////////////////////////////////////////////////////////////////////////////////////////</span>
<b>\chapter</b>{<b>Aplicación de variantes de la TR a un modelo de tasas de incidencia mensual de la inflación de México, 2013-2022</b>}
<span style='color:#898887;'>%///////////////////////////////////////////////////////////////////////////////////////////////////////////////////////////////////////</span>
<span style='color:#898887;'>%///////////////////////////////////////////////////////////////////////////////////////////////////////////////////////////////////////</span>
Se buscó un modelo IO de prueba que tuviera una alta exigencia para el método de la realización. Esto significa que describir su dinámica mediante la identificación del sistema subyacente es complicado. Aclarando, complicado comporta las siguientes características:

<b>\begin</b>{<b><span style='color:#0095ff;'>enumerate</span></b>}
 <span style='color:#644a9b;'>\item</span> Cantidad de datos grande <span style='color:#ff5500;'>$&gt;100$</span>
 <span style='color:#644a9b;'>\item</span> Valores pequeños
 <span style='color:#644a9b;'>\item</span> Alta variabilidad en algunos datos
 <span style='color:#644a9b;'>\item</span> Baja variabilidad en algunos datos
 <span style='color:#644a9b;'>\item</span> Alta correlación entre datos
 <span style='color:#644a9b;'>\item</span> Ortogonalidad en variables
 <span style='color:#644a9b;'>\item</span> DE no cuadrada
 <span style='color:#644a9b;'>\item</span> Inconvenientes en la estructura de la DE
<b>\end</b>{<b><span style='color:#0095ff;'>enumerate</span></b>}

<b>\begin</b>{<b><span style='color:#0095ff;'>itemize</span></b>}
<span style='color:#644a9b;'>\item</span> El objetivo de obtener un modelo con el índole previo es encontrar cualidades a mejorar en la Teoría de la Realización, así como sus límites y excepciones de aplicación. 
 
<span style='color:#644a9b;'>\item</span> Para la consecución de este objetivo, ocupé un modelo derivado de los componentes del INPC. El INPC tiene dos grandes componentes: la <span style='color:#644a9b;'>\textbf</span>{Subyacente} y <span style='color:#644a9b;'>\textbf</span>{No Subyacente}. Así mismo, cada una engloba dos subcomponentes más: <span style='color:#644a9b;'>\textbf</span>{Mercancías} y <span style='color:#644a9b;'>\textbf</span>{Servicios}, y <span style='color:#644a9b;'>\textbf</span>{Agropecuarios} y <span style='color:#644a9b;'>\textbf</span>{Energéticos y tarifas autorizadas por el gobierno}, respectivamente.

<span style='color:#644a9b;'>\item</span> Cada componente (e incluso el INPC) posee una valor estadístico llamado <span style='color:#644a9b;'>\textbf</span>{Incidencia}. Permite establecer que  tanto influye cada componente del INPC en la inflación general.

<span style='color:#644a9b;'>\item</span> La Incidencia puede pensarse acumulativa (aunque hay variaciones estadísticas). Las Incidencias de las componentes Subyacente y No Subyacente son, respectivamente, la suma de las incidencias de sus subcomponentes. 

<b>\end</b>{<b><span style='color:#0095ff;'>itemize</span></b>} 

Por lo dicho antes, las igualdades que tenemos son
 <span style='color:#644a9b;'>\ies</span>
 C_SI_{S}&amp;=&amp;I_mC_m+I_sC_s<span style='color:#644a9b;'>\\</span>
 C_NI_{N}&amp;=&amp;I_aC_a+I_eC_e
 <span style='color:#644a9b;'>\fes</span>
 donde <span style='color:#ff5500;'>$C_S$</span>, <span style='color:#ff5500;'>$I_S$</span>; <span style='color:#ff5500;'>$C_N$</span>, <span style='color:#ff5500;'>$I_N$</span>;  <span style='color:#ff5500;'>$C_m$</span>, <span style='color:#ff5500;'>$I_m$</span>; <span style='color:#ff5500;'>$C_s$</span>, <span style='color:#ff5500;'>$I_s$</span>; <span style='color:#ff5500;'>$C_a$</span>, <span style='color:#ff5500;'>$I_a$</span>; y <span style='color:#ff5500;'>$C_e$</span>, <span style='color:#ff5500;'>$I_e$</span>, son los componentes e incidencias mencionados anteriormente: Subyacente, No subyacente, Mercancias, Servicios, Agropecuarios, y Energéticos y tarifas autorizadas por el gobierno, respectivamente. Por consiguiente, el modelo IO puede verse como
 <span style='color:#644a9b;'>\ies</span>
 <b>\begin</b>{<b><span style='color:#0095ff;'>pmatrix</span></b>}
<span style='color:#ff5500;'>  C_S</span><span style='color:#3daee9;'>\\</span>
<span style='color:#ff5500;'>  C_N</span>
<span style='color:#ff5500;'> </span><b>\end</b>{<b><span style='color:#0095ff;'>pmatrix</span></b>}=<b>\begin</b>{<b><span style='color:#0095ff;'>pmatrix</span></b>}
<span style='color:#ff5500;'> </span><span style='color:#3daee9;'>\frac</span><span style='color:#ff5500;'>{I_m}{I_S}&amp;</span><span style='color:#3daee9;'>\frac</span><span style='color:#ff5500;'>{I_s}{I_S}&amp;0&amp;0</span><span style='color:#3daee9;'>\\</span>
<span style='color:#ff5500;'> 0&amp;0&amp;</span><span style='color:#3daee9;'>\frac</span><span style='color:#ff5500;'>{I_a}{I_S}&amp;</span><span style='color:#3daee9;'>\frac</span><span style='color:#ff5500;'>{I_e}{I_N}</span>
<span style='color:#ff5500;'> </span><b>\end</b>{<b><span style='color:#0095ff;'>pmatrix</span></b>}<b>\begin</b>{<b><span style='color:#0095ff;'>pmatrix</span></b>}
<span style='color:#ff5500;'> C_m</span><span style='color:#3daee9;'>\\</span>
<span style='color:#ff5500;'> C_s</span><span style='color:#3daee9;'>\\</span>
<span style='color:#ff5500;'> C_a</span><span style='color:#3daee9;'>\\</span>
<span style='color:#ff5500;'> C_e</span>
<span style='color:#ff5500;'> </span><b>\end</b>{<b><span style='color:#0095ff;'>pmatrix</span></b>}
  <span style='color:#644a9b;'>\fes</span> 
  No obstante, una primera problemática con la forma estructural de la <span style='color:#644a9b;'>\textbf</span>{descripción externa} (<span style='color:#644a9b;'>\textbf</span>{DE}) (<span style='color:#644a9b;'>\textbf</span>{matrices con las tasas de incidencias}), es de carácter operativo, se requiere que su diagonal sea no nula. Esto puede franquearse simplemente al reorganizar la estructura como se ve abajo:
  <span style='color:#644a9b;'>\ies</span>
 <b>\begin</b>{<b><span style='color:#0095ff;'>pmatrix</span></b>}
<span style='color:#ff5500;'>  C_S</span><span style='color:#3daee9;'>\\</span>
<span style='color:#ff5500;'>  C_N</span>
<span style='color:#ff5500;'> </span><b>\end</b>{<b><span style='color:#0095ff;'>pmatrix</span></b>}=<b>\begin</b>{<b><span style='color:#0095ff;'>pmatrix</span></b>}
<span style='color:#ff5500;'> </span><span style='color:#3daee9;'>\frac</span><span style='color:#ff5500;'>{I_m}{I_S}&amp;0&amp;</span><span style='color:#3daee9;'>\frac</span><span style='color:#ff5500;'>{I_s}{I_S}&amp;0</span><span style='color:#3daee9;'>\\</span>
<span style='color:#ff5500;'> 0&amp;</span><span style='color:#3daee9;'>\frac</span><span style='color:#ff5500;'>{I_a}{I_S}&amp;0&amp;</span><span style='color:#3daee9;'>\frac</span><span style='color:#ff5500;'>{I_e}{I_N}</span>
<span style='color:#ff5500;'> </span><b>\end</b>{<b><span style='color:#0095ff;'>pmatrix</span></b>}<b>\begin</b>{<b><span style='color:#0095ff;'>pmatrix</span></b>}
<span style='color:#ff5500;'> C_m</span><span style='color:#3daee9;'>\\</span>
<span style='color:#ff5500;'> C_a</span><span style='color:#3daee9;'>\\</span>
<span style='color:#ff5500;'> C_s</span><span style='color:#3daee9;'>\\</span>
<span style='color:#ff5500;'> C_e</span>
<span style='color:#ff5500;'> </span><b>\end</b>{<b><span style='color:#0095ff;'>pmatrix</span></b>}
  <span style='color:#644a9b;'>\fes</span>
  Una vez con este modelo, aunque hay datos del INPC desde 1969,  la Incidencia se calculó para todos los componentes requeridos  hasta junio de 2013. Por consiguiente, la DE mensual consta de 112 matrices con datos desde Junio de 2013 y hasta Octubre de 2022: <span style='color:#644a9b;'>\url</span>{https://www.inegi.org.mx/app/tabulados/default.aspx?nc=ca55_2018&amp;idrt=137&amp;opc=t}
  

  <b>\begin</b>{<b><span style='color:#0095ff;'>figure</span></b>}[H]
  <span style='color:#644a9b;'>\centering</span>
  <b><span style='color:#644a9b;'>\includegraphics</span></b>[width=11.5cm, height=8.1cm]{<b><span style='color:#0095ff;'>INPC_DE.png</span></b>}
  <b>\end</b>{<b><span style='color:#0095ff;'>figure</span></b>}
 
 <b>\begin</b>{<b><span style='color:#0095ff;'>itemize</span></b>}
       <span style='color:#644a9b;'>\item</span> La suma de los valores singulares (ponderados)  hasta aquel en rojo es más del <span style='color:#ff5500;'>$90</span><span style='color:#3daee9;'>\%</span><span style='color:#ff5500;'>$</span> del total de ellos.
       
       <span style='color:#644a9b;'>\item</span> Los valores singulares en este caso parece que no poseen valores dominantes grandes. 
 <b>\end</b>{<b><span style='color:#0095ff;'>itemize</span></b>}
 
 <b>\begin</b>{<b><span style='color:#0095ff;'>figure</span></b>}[H]
 <span style='color:#644a9b;'>\centering</span>
 <span style='color:#644a9b;'>\caption</span>{Valores singulares de la Matriz de Hankel}
 <b><span style='color:#644a9b;'>\includegraphics</span></b>[width=11.5cm, height=4cm]{<b><span style='color:#0095ff;'>sin0vs_inpc.png</span></b>}
 <b>\end</b>{<b><span style='color:#0095ff;'>figure</span></b>}





<span style='color:#898887;'>%///////////////////////////////////////////////////////////////////////////////////////////////////////////////////</span>
<span style='color:#898887;'>%\\\\\\\\\\\\\\\\\\\\\\\\\\\\\\\\\\\\\\\\\\\\\\\\\\\\\\\\\\\\\\\\\\\\\\\\\\\\\\\\\\\\\\\\\\\\\\\\\\\\\\\\\\\\\\\\\\\</span>
<b>\chapter*</b>{<b>Conclusiones</b>}
<span style='color:#644a9b;'>\addcontentsline</span>{toc}{chapter}{Conclusiones}
<span style='color:#898887;'>%///////////////////////////////////////////////////////////////////////////////////////////////////////////////////</span>
<span style='color:#898887;'>%\\\\\\\\\\\\\\\\\\\\\\\\\\\\\\\\\\\\\\\\\\\\\\\\\\\\\\\\\\\\\\\\\\\\\\\\\\\\\\\\\\\\\\\\\\\\\\\\\\\\\\\\\\\\\\\\\\\</span>


<span style='color:#644a9b;'>\backmatter</span>
<span style='color:#644a9b;'>\appendix</span>

<span style='color:#898887;'>%///////////////////////////////////////////////////////////////////////////////////////////////////////////////////////////////////////</span>
<span style='color:#898887;'>%///////////////////////////////////////////////////////////////////////////////////////////////////////////////////////////////////////</span>
<b>\chapter</b>{<b>Teoría de la Realización</b>}<b>\label</b>{<b><span style='color:#0095ff;'>Apendice A</span></b>}
<span style='color:#898887;'>%///////////////////////////////////////////////////////////////////////////////////////////////////////////////////////////////////////</span>
<span style='color:#898887;'>%///////////////////////////////////////////////////////////////////////////////////////////////////////////////////////////////////////</span>


<b>\begin</b>{<b><span style='color:#0095ff;'>longtable</span></b>}{ll}
<span style='color:#ff5500;'>$</span><span style='color:#3daee9;'>\H</span><span style='color:#ff5500;'>_{r,s}$</span><b>&amp;</b> MH de los elementos <span style='color:#ff5500;'>$1,</span><span style='color:#3daee9;'>\cdots</span><span style='color:#ff5500;'>, s+r-1$</span> de una sucesión dada.
<span style='color:#644a9b;'>\hspace</span>{20cm} <span style='color:#644a9b;'>\ \\</span>
<span style='color:#ff5500;'>$</span><span style='color:#3daee9;'>\H</span><span style='color:#ff5500;'>_{r,s_1:s_2}$</span><b>&amp;</b> MH de los elementos <span style='color:#ff5500;'>$s_1,</span><span style='color:#3daee9;'>\cdots</span><span style='color:#ff5500;'>, r+s_2-1$</span> de una sucesión dada.
<span style='color:#644a9b;'>\hspace</span>{20cm} <span style='color:#644a9b;'>\ \\</span>
<span style='color:#ff5500;'>$</span><span style='color:#3daee9;'>\H</span><span style='color:#ff5500;'>_{r_1:r_2,s}$</span><b>&amp;</b> MH de los elementos <span style='color:#ff5500;'>$r_1,</span><span style='color:#3daee9;'>\cdots</span><span style='color:#ff5500;'>, s+r_2-1$</span> de una sucesión dada.
<span style='color:#644a9b;'>\hspace</span>{20cm} <span style='color:#644a9b;'>\ \\</span>
<span style='color:#ff5500;'>$</span><span style='color:#3daee9;'>\H</span><span style='color:#ff5500;'>_{r_1:r_2,s_1:s_2}$</span><b>&amp;</b> MH de los elementos <span style='color:#ff5500;'>$r_1+s_1-1,</span><span style='color:#3daee9;'>\cdots</span><span style='color:#ff5500;'>,r_2+s_2-1$</span> de una sucesión dada.
<span style='color:#644a9b;'>\hspace</span>{20cm} <span style='color:#644a9b;'>\ \\</span>
<span style='color:#ff5500;'>$</span><span style='color:#3daee9;'>\F</span><span style='color:#ff5500;'>^{a</span><span style='color:#3daee9;'>\times</span><span style='color:#ff5500;'> b}$</span><b>&amp;</b> espacio de matrices de <span style='color:#ff5500;'>$a</span><span style='color:#3daee9;'>\times</span><span style='color:#ff5500;'> b$</span> sobre el cuerpo <span style='color:#ff5500;'>$</span><span style='color:#3daee9;'>\F</span><span style='color:#ff5500;'>$</span>.
<span style='color:#644a9b;'>\hspace</span>{20cm} <span style='color:#644a9b;'>\ \\</span>
<b>\end</b>{<b><span style='color:#0095ff;'>longtable</span></b>}

<b>\begin</b>{<b><span style='color:#0095ff;'>Teo</span></b>}<b>\label</b>{<b><span style='color:#0095ff;'>t1min</span></b>} Sea <span style='color:#ff5500;'>$</span><span style='color:#3daee9;'>\{</span><span style='color:#ff5500;'>J_t</span><span style='color:#3daee9;'>\}</span><span style='color:#ff5500;'>_{t</span><span style='color:#3daee9;'>\in\N</span><span style='color:#ff5500;'>_z}$</span> la sucesión parcial sobre <span style='color:#ff5500;'>$</span><span style='color:#3daee9;'>\F</span><span style='color:#ff5500;'>^{</span><span style='color:#3daee9;'>\a\times\b</span><span style='color:#ff5500;'>}$</span> y sean <span style='color:#ff5500;'>$p,q</span><span style='color:#3daee9;'>\in\N</span><span style='color:#ff5500;'>$</span> tales que <span style='color:#ff5500;'>$p+q+1</span><span style='color:#3daee9;'>\leq</span><span style='color:#ff5500;'> z$</span>. Si <span style='color:#ff5500;'>$</span><span style='color:#3daee9;'>\H</span><span style='color:#ff5500;'>_{p,q}</span><span style='color:#3daee9;'>\in\F</span><span style='color:#ff5500;'>^{pa</span><span style='color:#3daee9;'>\times</span><span style='color:#ff5500;'> qb}$</span> es la matriz de hankel formada por  <span style='color:#ff5500;'>$</span><span style='color:#3daee9;'>\{</span><span style='color:#ff5500;'>J_t</span><span style='color:#3daee9;'>\}</span><span style='color:#ff5500;'>_{t</span><span style='color:#3daee9;'>\in\N</span><span style='color:#ff5500;'>_n}$</span> tal que 
<span style='color:#644a9b;'>\ie\label</span>{tminreal}
rango(<span style='color:#644a9b;'>\H</span>_{p,q})=rango(<span style='color:#644a9b;'>\H</span>_{p,q+1})=
rango(<span style='color:#644a9b;'>\H</span>_{p+1,z-(p+1)})
 <span style='color:#644a9b;'>\fe</span>
 y si cualquier <span style='color:#ff5500;'>$</span><span style='color:#3daee9;'>\H</span><span style='color:#ff5500;'>_{r,s}$</span> con <span style='color:#ff5500;'>$r&lt;p$</span> y <span style='color:#ff5500;'>$s</span><span style='color:#3daee9;'>\leq</span><span style='color:#ff5500;'> q$</span> o <span style='color:#ff5500;'>$r= p$</span> y <span style='color:#ff5500;'>$s&lt;q$</span> no satisface <b>\eqref</b>{<b><span style='color:#0095ff;'>tminreal</span></b>}, entonces, toda <span style='color:#ff5500;'>$</span><span style='color:#3daee9;'>\H</span><span style='color:#ff5500;'>_{u,v}$</span> 
 con <span style='color:#ff5500;'>$u&lt;p$</span>, <span style='color:#ff5500;'>$v&gt;q$</span> o <span style='color:#ff5500;'>$u&gt;p$</span> y <span style='color:#ff5500;'>$v&lt;q$</span>,  y <span style='color:#ff5500;'>$u+v+1</span><span style='color:#3daee9;'>\leq</span><span style='color:#ff5500;'> z$</span>, no garantiza <b>\eqref</b>{<b><span style='color:#0095ff;'>tminreal</span></b>}.
 
<b>\end</b>{<b><span style='color:#0095ff;'>Teo</span></b>}


<b>\begin</b>{<b><span style='color:#0095ff;'>proof</span></b>}<span style='color:#644a9b;'>\ \-</span>

Sea <span style='color:#ff5500;'>$w$</span> el rango de <span style='color:#ff5500;'>$</span><span style='color:#3daee9;'>\H</span><span style='color:#ff5500;'>_{p,q}$</span>, sean <span style='color:#ff5500;'>$r_j$</span>, <span style='color:#ff5500;'>$j=1,</span><span style='color:#3daee9;'>\dots</span><span style='color:#ff5500;'>,w$</span> las posiciones de las columnas de <span style='color:#ff5500;'>$</span><span style='color:#3daee9;'>\H</span><span style='color:#ff5500;'>_{p,q}$</span> que forman la base de su espacio de columnas, y sea <span style='color:#ff5500;'>$</span><span style='color:#3daee9;'>\H</span><span style='color:#ff5500;'>_{u,v}$</span> con <span style='color:#ff5500;'>$u&lt;p$</span> y <span style='color:#ff5500;'>$v&gt;q$</span>, y  <span style='color:#ff5500;'>$u+v+1</span><span style='color:#3daee9;'>\leq</span><span style='color:#ff5500;'> z$</span>. Entonces, si suponemos que <span style='color:#ff5500;'>$</span><span style='color:#3daee9;'>\H</span><span style='color:#ff5500;'>_{u,v}$</span> satisface <b>\eqref</b>{<b><span style='color:#0095ff;'>tminreal</span></b>}, se demostrará que <span style='color:#ff5500;'>$</span><span style='color:#3daee9;'>\H</span><span style='color:#ff5500;'>_{u,q}$</span> satisface <b>\eqref</b>{<b><span style='color:#0095ff;'>tminreal</span></b>}, pero por hipótesis, 
<span style='color:#ff5500;'>$</span><span style='color:#3daee9;'>\H</span><span style='color:#ff5500;'>_{u,q}$</span> no puede satisfacer <b>\eqref</b>{<b><span style='color:#0095ff;'>tminreal</span></b>}. Por lo tanto, no existe <span style='color:#ff5500;'>$</span><span style='color:#3daee9;'>\H</span><span style='color:#ff5500;'>_{u,v}$</span>, con <span style='color:#ff5500;'>$u&lt;p$</span>, <span style='color:#ff5500;'>$v&gt;q$</span> y <span style='color:#ff5500;'>$u+v+1</span><span style='color:#3daee9;'>\leq</span><span style='color:#ff5500;'> z$</span>, que satisfaga <b>\eqref</b>{<b><span style='color:#0095ff;'>tminreal</span></b>}.<span style='color:#644a9b;'>\-</span>

Considerar unicamente la demostración del caso anterior no produce pérdida de generalidad. Para el segundo caso, si suponemos que existe una matriz <span style='color:#ff5500;'>$</span><span style='color:#3daee9;'>\H</span><span style='color:#ff5500;'>_{c,d}$</span> con <span style='color:#ff5500;'>$c&gt;p$</span>, <span style='color:#ff5500;'>$d&lt;q$</span> y <span style='color:#ff5500;'>$c+d+1</span><span style='color:#3daee9;'>\leq</span><span style='color:#ff5500;'> z$</span> que satisface <b>\eqref</b>{<b><span style='color:#0095ff;'>tminreal</span></b>} y dado que <span style='color:#ff5500;'>$</span><span style='color:#3daee9;'>\H</span><span style='color:#ff5500;'>_{p,q}$</span> con <span style='color:#ff5500;'>$p&lt;c$</span>, <span style='color:#ff5500;'>$q&gt;d$</span> y <span style='color:#ff5500;'>$p+q+1</span><span style='color:#3daee9;'>\leq</span><span style='color:#ff5500;'> z$</span> satisface <b>\eqref</b>{<b><span style='color:#0095ff;'>tminreal</span></b>} por hipótesis, entonces <span style='color:#ff5500;'>$</span><span style='color:#3daee9;'>\H</span><span style='color:#ff5500;'>_{c,d}$</span> y <span style='color:#ff5500;'>$</span><span style='color:#3daee9;'>\H</span><span style='color:#ff5500;'>_{p,q}$</span> ocupan el lugar de <span style='color:#ff5500;'>$</span><span style='color:#3daee9;'>\H</span><span style='color:#ff5500;'>_{p,q}$</span> y <span style='color:#ff5500;'>$</span><span style='color:#3daee9;'>\H</span><span style='color:#ff5500;'>_{u,v}$</span> del caso anterior, respectivamente. Esto nos lleva a que <span style='color:#ff5500;'>$</span><span style='color:#3daee9;'>\H</span><span style='color:#ff5500;'>_{p,d}$</span> satisface <b>\eqref</b>{<b><span style='color:#0095ff;'>tminreal</span></b>}, pero como <span style='color:#ff5500;'>$</span><span style='color:#3daee9;'>\H</span><span style='color:#ff5500;'>_{p,d}$</span> tiene que <span style='color:#ff5500;'>$p=p$</span> y <span style='color:#ff5500;'>$d&lt;q$</span>, entonces no puede satisfacer <b>\eqref</b>{<b><span style='color:#0095ff;'>tminreal</span></b>} por hipótesis. Por lo tanto, como esta contradicción surgió del supuesto inicial de este caso, no puede existir <span style='color:#ff5500;'>$</span><span style='color:#3daee9;'>\H</span><span style='color:#ff5500;'>_{c,d}$</span>, <span style='color:#ff5500;'>$c&gt;p$</span>, <span style='color:#ff5500;'>$d&lt;q$</span> y <span style='color:#ff5500;'>$c+d+1</span><span style='color:#3daee9;'>\leq</span><span style='color:#ff5500;'> z$</span>, que satisfaga <b>\eqref</b>{<b><span style='color:#0095ff;'>tminreal</span></b>}.<span style='color:#644a9b;'>\-</span>

Iniciamos la demost ración con unas observaciones que se usaran en la demostración:
<b>\begin</b>{<b><span style='color:#0095ff;'>enumerate</span></b>}[i)]
 <span style='color:#644a9b;'>\item</span> Las columnas de <span style='color:#ff5500;'>$</span><span style='color:#3daee9;'>\H</span><span style='color:#ff5500;'>_{p,q}$</span> son también columnas de <span style='color:#ff5500;'>$</span><span style='color:#3daee9;'>\H</span><span style='color:#ff5500;'>_{p,q+1}$</span>, por definición de Matriz de Hankel (MH).
 <span style='color:#644a9b;'>\item</span> Las columnas <span style='color:#ff5500;'>$r_j$</span> de <span style='color:#ff5500;'>$</span><span style='color:#3daee9;'>\H</span><span style='color:#ff5500;'>_{p,q}$</span> también son una base del espacio de columnas de <span style='color:#ff5500;'>$</span><span style='color:#3daee9;'>\H</span><span style='color:#ff5500;'>_{p,q+1}$</span>, por la condición <b>\eqref</b>{<b><span style='color:#0095ff;'>tminreal</span></b>}.
 <span style='color:#644a9b;'>\item\label</span>{inc3} <span style='color:#ff5500;'>$</span><span style='color:#3daee9;'>\H</span><span style='color:#ff5500;'>_{u,q+1}$</span> y <span style='color:#ff5500;'>$</span><span style='color:#3daee9;'>\H</span><span style='color:#ff5500;'>_{p,q+1}$</span> tienen el mismo número de columnas y la columna <span style='color:#ff5500;'>$k$</span> de <span style='color:#ff5500;'>$</span><span style='color:#3daee9;'>\H</span><span style='color:#ff5500;'>_{u,q+1}$</span> está compuesta por los elementos en las primeras <span style='color:#ff5500;'>$</span><span style='color:#3daee9;'>\alpha</span><span style='color:#ff5500;'> u$</span> coordenadas de la columna <span style='color:#ff5500;'>$k$</span> de <span style='color:#ff5500;'>$</span><span style='color:#3daee9;'>\H</span><span style='color:#ff5500;'>_{p,q+1}$</span>, por definición de MH.
 <span style='color:#644a9b;'>\item</span> Las columnas <span style='color:#ff5500;'>$r_j$</span> de <span style='color:#ff5500;'>$</span><span style='color:#3daee9;'>\H</span><span style='color:#ff5500;'>_{u,q+1}$</span> generan todas las columnas de <span style='color:#ff5500;'>$</span><span style='color:#3daee9;'>\H</span><span style='color:#ff5500;'>_{u,q+1}$</span>, por el inciso <b>\eqref</b>{<b><span style='color:#0095ff;'>inc3</span></b>}.
 <span style='color:#644a9b;'>\item</span> Las columnas <span style='color:#ff5500;'>$r_j$</span> de <span style='color:#ff5500;'>$</span><span style='color:#3daee9;'>\H</span><span style='color:#ff5500;'>_{u,q}$</span> generan las columnas de <span style='color:#ff5500;'>$</span><span style='color:#3daee9;'>\H</span><span style='color:#ff5500;'>_{u,q+1}$</span>, debido a que cada <span style='color:#ff5500;'>$r_j</span><span style='color:#3daee9;'>\leq</span><span style='color:#ff5500;'> bq$</span> y las primeras <span style='color:#ff5500;'>$bq$</span> columnas de <span style='color:#ff5500;'>$</span><span style='color:#3daee9;'>\H</span><span style='color:#ff5500;'>_{u,q+1}$</span> forman <span style='color:#ff5500;'>$</span><span style='color:#3daee9;'>\H</span><span style='color:#ff5500;'>_{u,q}$</span>, por definición de MH.
 <span style='color:#644a9b;'>\item</span> Hay al menos tantos bloques fila en <span style='color:#ff5500;'>$</span><span style='color:#3daee9;'>\H</span><span style='color:#ff5500;'>_{u+1:p,q+1}$</span> como bloques columna en <span style='color:#ff5500;'>$</span><span style='color:#3daee9;'>\H</span><span style='color:#ff5500;'>_{u,q+1:v}$</span>. Esto a causa de las restricciones
<span style='color:#644a9b;'>\ies</span>
v+u+1&amp;<span style='color:#644a9b;'>\leq</span>&amp; z<span style='color:#644a9b;'>\\</span>
p+q+1&amp;<span style='color:#644a9b;'>\leq</span>&amp; z
<span style='color:#644a9b;'>\fes</span>
 que implican lo que se afirma 
<span style='color:#644a9b;'>\ie\label</span>{exp1}
v-q<span style='color:#644a9b;'>\leq</span> z-(u+1)-(z-(p+1))=p-u
<span style='color:#644a9b;'>\fe</span>
lo cuál se puede ver en las siguientes representaciones matriciales
<span style='color:#644a9b;'>\ies</span>
<span style='color:#644a9b;'>\H</span>_{p,q+1}&amp;=&amp;<span style='color:#644a9b;'>\left</span>(<b>\begin</b>{<b><span style='color:#0095ff;'>array</span></b>}{cc}
<span style='color:#644a9b;'>\H</span>_{u,q+1}<span style='color:#644a9b;'>\\</span>
<span style='color:#644a9b;'>\H</span>_{u+1:p,q+1}
<b>\end</b>{<b><span style='color:#0095ff;'>array</span></b>}<span style='color:#644a9b;'>\right</span>)=<span style='color:#644a9b;'>\left</span>(<b>\begin</b>{<b><span style='color:#0095ff;'>array</span></b>}{c}<b>\begin</b>{<b><span style='color:#0095ff;'>matrix</span></b>}
J_{1}&amp;<span style='color:#644a9b;'>\dots</span>&amp;J_{q+1}<span style='color:#644a9b;'>\\</span>
<span style='color:#644a9b;'>\vdots</span>&amp;<span style='color:#644a9b;'>\ddots</span>&amp;<span style='color:#644a9b;'>\vdots\\</span>
J_{u}&amp;<span style='color:#644a9b;'>\dots</span>&amp;J_{u+q}
<b>\end</b>{<b><span style='color:#0095ff;'>matrix</span></b>}<span style='color:#644a9b;'>\\\begin</span>{matrix}
J_{u+1}&amp;<span style='color:#644a9b;'>\dots</span>&amp;J_{u+q+1}<span style='color:#644a9b;'>\\</span>
<span style='color:#644a9b;'>\vdots</span>&amp;<span style='color:#644a9b;'>\ddots</span>&amp;<span style='color:#644a9b;'>\vdots\\</span>
J_{p}&amp;<span style='color:#644a9b;'>\dots</span>&amp;J_{p+q}
<b>\end</b>{<b><span style='color:#0095ff;'>matrix</span></b>}<b>\end</b>{<b><span style='color:#0095ff;'>array</span></b>}<span style='color:#644a9b;'>\right</span>)<b>\begin</b>{<b><span style='color:#0095ff;'>matrix</span></b>}
<span style='color:#644a9b;'>\vspace</span>{3em}<span style='color:#644a9b;'>\\</span>
<span style='color:#644a9b;'>\left\}</span><b>\begin</b>{<b><span style='color:#0095ff;'>matrix</span></b>}<span style='color:#644a9b;'>\\</span>
p-u<span style='color:#644a9b;'>\\</span>
<span style='color:#644a9b;'>\\</span>
<b>\end</b>{<b><span style='color:#0095ff;'>matrix</span></b>}<span style='color:#644a9b;'>\right</span>.
<b>\end</b>{<b><span style='color:#0095ff;'>matrix</span></b>}
<span style='color:#644a9b;'>\\</span>
&amp;&amp;<span style='color:#644a9b;'>\hspace</span>{21.7em}<b>\begin</b>{<b><span style='color:#0095ff;'>matrix</span></b>}v-q
<b>\end</b>{<b><span style='color:#0095ff;'>matrix</span></b>}
<span style='color:#644a9b;'>\\</span>
<span style='color:#644a9b;'>\H</span>_{u,v}&amp;=&amp;<span style='color:#644a9b;'>\left</span>(<b>\begin</b>{<b><span style='color:#0095ff;'>array</span></b>}{cc}
<span style='color:#644a9b;'>\H</span>_{u,q}&amp;
<span style='color:#644a9b;'>\H</span>_{u,q+1:v}
<b>\end</b>{<b><span style='color:#0095ff;'>array</span></b>}<span style='color:#644a9b;'>\right</span>)=<span style='color:#644a9b;'>\left</span>(<b>\begin</b>{<b><span style='color:#0095ff;'>matrix</span></b>}<span style='color:#644a9b;'>\\</span>
<span style='color:#644a9b;'>\\</span>
<span style='color:#644a9b;'>\\</span>
<span style='color:#644a9b;'>\\\end</span>{matrix}<span style='color:#644a9b;'>\right</span>.<b>\begin</b>{<b><span style='color:#0095ff;'>array</span></b>}{cc}<b>\begin</b>{<b><span style='color:#0095ff;'>matrix</span></b>}
J_{1}&amp;<span style='color:#644a9b;'>\dots</span>&amp;J_{q}<span style='color:#644a9b;'>\\</span>
<span style='color:#644a9b;'>\vdots</span>&amp;<span style='color:#644a9b;'>\ddots</span>&amp;<span style='color:#644a9b;'>\vdots\\</span>
J_{u}&amp;<span style='color:#644a9b;'>\dots</span>&amp;J_{u+q-1}
<b>\end</b>{<b><span style='color:#0095ff;'>matrix</span></b>}&amp;<span style='color:#644a9b;'>\overbrace</span>{<b>\begin</b>{<b><span style='color:#0095ff;'>matrix</span></b>}
J_{q+1}&amp;<span style='color:#644a9b;'>\dots</span>&amp;J_{v}<span style='color:#644a9b;'>\\</span>
<span style='color:#644a9b;'>\vdots</span>&amp;<span style='color:#644a9b;'>\ddots</span>&amp;<span style='color:#644a9b;'>\vdots\\</span>
J_{u+q}&amp;<span style='color:#644a9b;'>\dots</span>&amp;J_{u+v-1}
<b>\end</b>{<b><span style='color:#0095ff;'>matrix</span></b>}}<b>\end</b>{<b><span style='color:#0095ff;'>array</span></b>}<span style='color:#644a9b;'>\left</span>.<b>\begin</b>{<b><span style='color:#0095ff;'>matrix</span></b>}<span style='color:#644a9b;'>\\</span>
<span style='color:#644a9b;'>\\</span>
<span style='color:#644a9b;'>\\</span>
<span style='color:#644a9b;'>\\\end</span>{matrix}<span style='color:#644a9b;'>\right</span>)
<span style='color:#644a9b;'>\fes</span>

<span style='color:#644a9b;'>\item</span> Para <span style='color:#ff5500;'>$i=1,</span><span style='color:#3daee9;'>\dots</span><span style='color:#ff5500;'>,v-(q+1)$</span> se sostiene la siguiente igualdad
<span style='color:#644a9b;'>\ies</span>
<span style='color:#644a9b;'>\H</span>_{1+i:u+i,q+1}=<span style='color:#644a9b;'>\H</span>_{u,1+i:q+1+i}
<span style='color:#644a9b;'>\fes</span> 
Igualdad que es valida ya que de <b>\eqref</b>{<b><span style='color:#0095ff;'>exp1</span></b>} tenemos <span style='color:#ff5500;'>$u+v-q</span><span style='color:#3daee9;'>\leq</span><span style='color:#ff5500;'> p$</span>, y por la estructura de la MH. La igualdad se puede ver en la siguiente representación matricial
<span style='color:#644a9b;'>\ies</span>
<span style='color:#644a9b;'>\H</span>_{p,q+1}&amp;=&amp;<span style='color:#644a9b;'>\left</span>(<b>\begin</b>{<b><span style='color:#0095ff;'>array</span></b>}{c}<b>\begin</b>{<b><span style='color:#0095ff;'>matrix</span></b>}
J_{1}&amp;<span style='color:#644a9b;'>\dots</span>&amp;J_{q+1}<span style='color:#644a9b;'>\\</span>
<span style='color:#644a9b;'>\vdots</span>&amp;<span style='color:#644a9b;'>\ddots</span>&amp;<span style='color:#644a9b;'>\vdots\\</span>
J_{i}&amp;<span style='color:#644a9b;'>\dots</span>&amp;J_{q+i}
<b>\end</b>{<b><span style='color:#0095ff;'>matrix</span></b>}<span style='color:#644a9b;'>\\</span>
<b>\begin</b>{<b><span style='color:#0095ff;'>pmatrix</span></b>}
<span style='color:#ff5500;'>J_{1+i}&amp;</span><span style='color:#3daee9;'>\dots</span><span style='color:#ff5500;'>&amp;J_{q+1+i}</span><span style='color:#3daee9;'>\\</span>
<span style='color:#3daee9;'>\vdots</span><span style='color:#ff5500;'>&amp;</span><span style='color:#3daee9;'>\ddots</span><span style='color:#ff5500;'>&amp;</span><span style='color:#3daee9;'>\vdots\\</span>
<span style='color:#ff5500;'>J_{u+i}&amp;</span><span style='color:#3daee9;'>\dots</span><span style='color:#ff5500;'>&amp;J_{q+u+i}</span>
<b>\end</b>{<b><span style='color:#0095ff;'>pmatrix</span></b>}<span style='color:#644a9b;'>\\</span>
<b>\begin</b>{<b><span style='color:#0095ff;'>matrix</span></b>}
J_{u+i+1}&amp;<span style='color:#644a9b;'>\dots</span>&amp;J_{q+u+1+i}<span style='color:#644a9b;'>\\</span>
<span style='color:#644a9b;'>\vdots</span>&amp;<span style='color:#644a9b;'>\ddots</span>&amp;<span style='color:#644a9b;'>\vdots\\</span>
J_{p}&amp;<span style='color:#644a9b;'>\dots</span>&amp;J_{p+q}
<b>\end</b>{<b><span style='color:#0095ff;'>matrix</span></b>}<b>\end</b>{<b><span style='color:#0095ff;'>array</span></b>}<span style='color:#644a9b;'>\right</span>)<span style='color:#644a9b;'>\left\}</span><b>\begin</b>{<b><span style='color:#0095ff;'>matrix</span></b>}
<span style='color:#644a9b;'>\vspace</span>{.3em}<span style='color:#644a9b;'>\\</span>
<span style='color:#644a9b;'>\H</span>_{1+i:u+i,q+1}<span style='color:#644a9b;'>\\</span>
<span style='color:#644a9b;'>\vspace</span>{.3em}<span style='color:#644a9b;'>\\</span>
<b>\end</b>{<b><span style='color:#0095ff;'>matrix</span></b>}<span style='color:#644a9b;'>\right</span>.
<span style='color:#644a9b;'>\\</span>
&amp;&amp;<span style='color:#644a9b;'>\hspace</span>{12em}<b>\begin</b>{<b><span style='color:#0095ff;'>matrix</span></b>}<span style='color:#644a9b;'>\H</span>_{u,1+i:q+1+i}
<b>\end</b>{<b><span style='color:#0095ff;'>matrix</span></b>}
<span style='color:#644a9b;'>\\</span>
<span style='color:#644a9b;'>\H</span>_{u,v}&amp;=&amp;<span style='color:#644a9b;'>\left</span>(<b>\begin</b>{<b><span style='color:#0095ff;'>array</span></b>}{ccc}<b>\begin</b>{<b><span style='color:#0095ff;'>matrix</span></b>}
J_{1}&amp;<span style='color:#644a9b;'>\dots</span>&amp;J_{i}<span style='color:#644a9b;'>\\</span>
<span style='color:#644a9b;'>\vdots</span>&amp;<span style='color:#644a9b;'>\ddots</span>&amp;<span style='color:#644a9b;'>\vdots\\</span>
J_{u}&amp;<span style='color:#644a9b;'>\dots</span>&amp;J_{u+i-1}
<b>\end</b>{<b><span style='color:#0095ff;'>matrix</span></b>}&amp;<span style='color:#644a9b;'>\overbrace</span>{<b>\begin</b>{<b><span style='color:#0095ff;'>pmatrix</span></b>}
<span style='color:#ff5500;'>J_{1+i}&amp;</span><span style='color:#3daee9;'>\dots</span><span style='color:#ff5500;'>&amp;J_{q+1+i}</span><span style='color:#3daee9;'>\\</span>
<span style='color:#3daee9;'>\vdots</span><span style='color:#ff5500;'>&amp;</span><span style='color:#3daee9;'>\ddots</span><span style='color:#ff5500;'>&amp;</span><span style='color:#3daee9;'>\vdots\\</span>
<span style='color:#ff5500;'>J_{u+i}&amp;</span><span style='color:#3daee9;'>\dots</span><span style='color:#ff5500;'>&amp;J_{q+u+i}</span>
<b>\end</b>{<b><span style='color:#0095ff;'>pmatrix</span></b>}}&amp;<b>\begin</b>{<b><span style='color:#0095ff;'>matrix</span></b>}
J_{q+2+i}&amp;<span style='color:#644a9b;'>\dots</span>&amp;J_{v}<span style='color:#644a9b;'>\\</span>
<span style='color:#644a9b;'>\vdots</span>&amp;<span style='color:#644a9b;'>\ddots</span>&amp;<span style='color:#644a9b;'>\vdots\\</span>
J_{q+u+i+1}&amp;<span style='color:#644a9b;'>\dots</span>&amp;J_{u+v-1}
<b>\end</b>{<b><span style='color:#0095ff;'>matrix</span></b>}<b>\end</b>{<b><span style='color:#0095ff;'>array</span></b>}<span style='color:#644a9b;'>\right</span>)
<span style='color:#644a9b;'>\fes</span>
<b>\end</b>{<b><span style='color:#0095ff;'>enumerate</span></b>}

Prosiguiendo con la demostración, si tomamos <span style='color:#ff5500;'>$i=1$</span>, entonces <span style='color:#ff5500;'>$</span><span style='color:#3daee9;'>\H</span><span style='color:#ff5500;'>_{u,2:q+2}=</span><span style='color:#3daee9;'>\H</span><span style='color:#ff5500;'>_{2:u+1,q+1}$</span>, y como <span style='color:#ff5500;'>$</span><span style='color:#3daee9;'>\H</span><span style='color:#ff5500;'>_{2:u+1,q+1}$</span> y 
<span style='color:#ff5500;'>$</span><span style='color:#3daee9;'>\H</span><span style='color:#ff5500;'>_{p,q+1}$</span> tienen el mismo número de columnas, y la columna <span style='color:#ff5500;'>$k$</span> de <span style='color:#ff5500;'>$</span><span style='color:#3daee9;'>\H</span><span style='color:#ff5500;'>_{2:u+1,q+1}$</span> es igual a los elementos en las coordenadas <span style='color:#ff5500;'>$</span><span style='color:#3daee9;'>\a</span><span style='color:#ff5500;'>+1$</span> hasta <span style='color:#ff5500;'>$(u+1)</span><span style='color:#3daee9;'>\a</span><span style='color:#ff5500;'>$</span> de la columna <span style='color:#ff5500;'>$k$</span> de <span style='color:#ff5500;'>$</span><span style='color:#3daee9;'>\H</span><span style='color:#ff5500;'>_{p,q+1}$</span>, entonces para <span style='color:#ff5500;'>$</span><span style='color:#3daee9;'>\H</span><span style='color:#ff5500;'>_{2:u+1,q+1}$</span>, sus columnas <span style='color:#ff5500;'>$r_j$</span> generan su espacio de columnas. Por lo cuaĺ, las columas <span style='color:#ff5500;'>$r_j$</span> de  <span style='color:#ff5500;'>$</span><span style='color:#3daee9;'>\H</span><span style='color:#ff5500;'>_{u,2:q+2}$</span> generan cualquiera de las columnas de <span style='color:#ff5500;'>$</span><span style='color:#3daee9;'>\H</span><span style='color:#ff5500;'>_{u,2:q+2}$</span>. Pero como la columna <span style='color:#ff5500;'>$k$</span> de <span style='color:#ff5500;'>$</span><span style='color:#3daee9;'>\H</span><span style='color:#ff5500;'>_{u,2:q+2}$</span> es la columna <span style='color:#ff5500;'>$k+</span><span style='color:#3daee9;'>\b</span><span style='color:#ff5500;'>$</span> de <span style='color:#ff5500;'>$</span><span style='color:#3daee9;'>\H</span><span style='color:#ff5500;'>_{u,q+2}$</span>, entonces las columnas <span style='color:#ff5500;'>$r_j+</span><span style='color:#3daee9;'>\b</span><span style='color:#ff5500;'>$</span> de <span style='color:#ff5500;'>$</span><span style='color:#3daee9;'>\H</span><span style='color:#ff5500;'>_{u,q+2}$</span> generan cualquier columna de <span style='color:#ff5500;'>$</span><span style='color:#3daee9;'>\H</span><span style='color:#ff5500;'>_{u,2:q+2}$</span>. Más aún, como <span style='color:#ff5500;'>$r_j+</span><span style='color:#3daee9;'>\b\leq</span><span style='color:#ff5500;'> </span><span style='color:#3daee9;'>\b</span><span style='color:#ff5500;'> q+</span><span style='color:#3daee9;'>\b</span><span style='color:#ff5500;'>=</span><span style='color:#3daee9;'>\b</span><span style='color:#ff5500;'>(q+1)$</span> y las primeras <span style='color:#ff5500;'>$</span><span style='color:#3daee9;'>\b</span><span style='color:#ff5500;'>(q+1)$</span> columnas de <span style='color:#ff5500;'>$</span><span style='color:#3daee9;'>\H</span><span style='color:#ff5500;'>_{u,q+2}$</span> forman a <span style='color:#ff5500;'>$</span><span style='color:#3daee9;'>\H</span><span style='color:#ff5500;'>_{u,q+1}$</span>, entonces las columas <span style='color:#ff5500;'>$r_j+</span><span style='color:#3daee9;'>\b</span><span style='color:#ff5500;'>$</span> son columnas de <span style='color:#ff5500;'>$</span><span style='color:#3daee9;'>\H</span><span style='color:#ff5500;'>_{u,q+1}$</span>.<span style='color:#644a9b;'>\-</span>

En virtud de que las columnas <span style='color:#ff5500;'>$r_j$</span> de <span style='color:#ff5500;'>$</span><span style='color:#3daee9;'>\H</span><span style='color:#ff5500;'>_{u,q}$</span> generan las columnas de <span style='color:#ff5500;'>$</span><span style='color:#3daee9;'>\H</span><span style='color:#ff5500;'>_{u,q+1}$</span>, entonces las columnas <span style='color:#ff5500;'>$r_j$</span> de <span style='color:#ff5500;'>$</span><span style='color:#3daee9;'>\H</span><span style='color:#ff5500;'>_{u,q}$</span> generan las columnas <span style='color:#ff5500;'>$r_j+</span><span style='color:#3daee9;'>\b</span><span style='color:#ff5500;'>$</span> de <span style='color:#ff5500;'>$</span><span style='color:#3daee9;'>\H</span><span style='color:#ff5500;'>_{u,q+1}$</span>. Y como cualquier columna de <span style='color:#ff5500;'>$</span><span style='color:#3daee9;'>\H</span><span style='color:#ff5500;'>_{u,q+2}$</span> es columna de <span style='color:#ff5500;'>$</span><span style='color:#3daee9;'>\H</span><span style='color:#ff5500;'>_{u,q+1}$</span> o <span style='color:#ff5500;'>$</span><span style='color:#3daee9;'>\H</span><span style='color:#ff5500;'>_{u,2:q+2}$</span>, entonces las columnas <span style='color:#ff5500;'>$r_j$</span> de  
<span style='color:#ff5500;'>$</span><span style='color:#3daee9;'>\H</span><span style='color:#ff5500;'>_{u,q}$</span> generan cualquier columna de <span style='color:#ff5500;'>$</span><span style='color:#3daee9;'>\H</span><span style='color:#ff5500;'>_{u,q+2}$</span>.<span style='color:#644a9b;'>\-</span>

Ahora, si para <span style='color:#ff5500;'>$n&lt;v-(q+1)$</span>, las columnas <span style='color:#ff5500;'>$r_j$</span> de <span style='color:#ff5500;'>$</span><span style='color:#3daee9;'>\H</span><span style='color:#ff5500;'>_{u,q}$</span> geneneran las columnas de <span style='color:#ff5500;'>$</span><span style='color:#3daee9;'>\H</span><span style='color:#ff5500;'>_{u,1+n:q+n+2}$</span>. Entonces, como <span style='color:#ff5500;'>$</span><span style='color:#3daee9;'>\H</span><span style='color:#ff5500;'>_{u,2+n:q+n+2}=</span><span style='color:#3daee9;'>\H</span><span style='color:#ff5500;'>_{2+n:u+n+1,q+1}$</span>, 
y ya que cada columna <span style='color:#ff5500;'>$k$</span> de <span style='color:#ff5500;'>$</span><span style='color:#3daee9;'>\H</span><span style='color:#ff5500;'>_{2+n:u+n+1,q+1}$</span> está formada por los elementos entre las coordenadas 
<span style='color:#ff5500;'>$</span><span style='color:#3daee9;'>\a</span><span style='color:#ff5500;'>(n+1)+1$</span> y <span style='color:#ff5500;'>$</span><span style='color:#3daee9;'>\a</span><span style='color:#ff5500;'>(u+n+1)$</span> de la columna <span style='color:#ff5500;'>$k$</span> de <span style='color:#ff5500;'>$</span><span style='color:#3daee9;'>\H</span><span style='color:#ff5500;'>_{p,q+1}$</span>, entonces las columnas <span style='color:#ff5500;'>$r_j$</span> de <span style='color:#ff5500;'>$</span><span style='color:#3daee9;'>\H</span><span style='color:#ff5500;'>_{2+n:u+n+1,q+1}$</span> generan cualquier columna de <span style='color:#ff5500;'>$</span><span style='color:#3daee9;'>\H</span><span style='color:#ff5500;'>_{2+n:u+n+1,q}$</span>. Así, las columnas <span style='color:#ff5500;'>$r_j$</span> de <span style='color:#ff5500;'>$</span><span style='color:#3daee9;'>\H</span><span style='color:#ff5500;'>_{u,2+n:q+n+2}$</span> generan cada columna de <span style='color:#ff5500;'>$</span><span style='color:#3daee9;'>\H</span><span style='color:#ff5500;'>_{u,2+n:q+n+1}$</span>. Pero como cada columna <span style='color:#ff5500;'>$k$</span> de <span style='color:#ff5500;'>$</span><span style='color:#3daee9;'>\H</span><span style='color:#ff5500;'>_{u,2+n:q+n+2}$</span> es la columna <span style='color:#ff5500;'>$k+</span><span style='color:#3daee9;'>\b</span><span style='color:#ff5500;'>(n+1)$</span> de <span style='color:#ff5500;'>$</span><span style='color:#3daee9;'>\H</span><span style='color:#ff5500;'>_{u,q+n+2}$</span>, entonces las columnas <span style='color:#ff5500;'>$r_j+</span><span style='color:#3daee9;'>\b</span><span style='color:#ff5500;'>(n+1)$</span> de 
<span style='color:#ff5500;'>$</span><span style='color:#3daee9;'>\H</span><span style='color:#ff5500;'>_{u,q+n+2}$</span> generan cualquier columna de 
<span style='color:#ff5500;'>$</span><span style='color:#3daee9;'>\H</span><span style='color:#ff5500;'>_{u,2+n:q+n+2}$</span>. Incluso, como <span style='color:#ff5500;'>$r_j+</span><span style='color:#3daee9;'>\b</span><span style='color:#ff5500;'>(n+1)</span><span style='color:#3daee9;'>\leq</span><span style='color:#ff5500;'> </span><span style='color:#3daee9;'>\b</span><span style='color:#ff5500;'> q+</span><span style='color:#3daee9;'>\b</span><span style='color:#ff5500;'>(n+1)=</span><span style='color:#3daee9;'>\b</span><span style='color:#ff5500;'>(q+n+1)$</span> y las primeras <span style='color:#ff5500;'>$</span><span style='color:#3daee9;'>\b</span><span style='color:#ff5500;'>(q+n+1)$</span> columnas de <span style='color:#ff5500;'>$</span><span style='color:#3daee9;'>\H</span><span style='color:#ff5500;'>_{u,q+n+2}$</span> forman a <span style='color:#ff5500;'>$</span><span style='color:#3daee9;'>\H</span><span style='color:#ff5500;'>_{u,q+n+1}$</span>, entonces las columas <span style='color:#ff5500;'>$r_j+</span><span style='color:#3daee9;'>\b</span><span style='color:#ff5500;'>(n+1)$</span> de <span style='color:#ff5500;'>$</span><span style='color:#3daee9;'>\H</span><span style='color:#ff5500;'>_{u,q+n+2}$</span> son columnas de <span style='color:#ff5500;'>$</span><span style='color:#3daee9;'>\H</span><span style='color:#ff5500;'>_{u,q+n+1}$</span>.<span style='color:#644a9b;'>\-</span>

Luego, ya que por hipótesis cada columna de <span style='color:#ff5500;'>$</span><span style='color:#3daee9;'>\H</span><span style='color:#ff5500;'>_{u,q+n+1}$</span> es generada por las <span style='color:#ff5500;'>$r_j$</span> columnas de <span style='color:#ff5500;'>$</span><span style='color:#3daee9;'>\H</span><span style='color:#ff5500;'>_{u,q}$</span>, entonces las columnas <span style='color:#ff5500;'>$r_j+</span><span style='color:#3daee9;'>\b</span><span style='color:#ff5500;'>(n+1)$</span> de <span style='color:#ff5500;'>$</span><span style='color:#3daee9;'>\H</span><span style='color:#ff5500;'>_{u,q+n+1}$</span> son generadas por las <span style='color:#ff5500;'>$r_j$</span> columnas de <span style='color:#ff5500;'>$</span><span style='color:#3daee9;'>\H</span><span style='color:#ff5500;'>_{u,q}$</span>. A partir de esto, tenemos que las columnas <span style='color:#ff5500;'>$r_j$</span> de <span style='color:#ff5500;'>$</span><span style='color:#3daee9;'>\H</span><span style='color:#ff5500;'>_{u,q}$</span> generan las columnas de <span style='color:#ff5500;'>$</span><span style='color:#3daee9;'>\H</span><span style='color:#ff5500;'>_{u,2+n:q+n+2}$</span>. Pero como toda columna de <span style='color:#ff5500;'>$</span><span style='color:#3daee9;'>\H</span><span style='color:#ff5500;'>_{u,q+n+2}$</span> es columna de <span style='color:#ff5500;'>$</span><span style='color:#3daee9;'>\H</span><span style='color:#ff5500;'>_{u,q+n+1}$</span> o <span style='color:#ff5500;'>$</span><span style='color:#3daee9;'>\H</span><span style='color:#ff5500;'>_{u,2+n:q+n+2}$</span>, entonces las columnas <span style='color:#ff5500;'>$r_j$</span> de <span style='color:#ff5500;'>$</span><span style='color:#3daee9;'>\H</span><span style='color:#ff5500;'>_{u,q}$</span> generan las columnas de <span style='color:#ff5500;'>$</span><span style='color:#3daee9;'>\H</span><span style='color:#ff5500;'>_{u,q+n+2}$</span>.<span style='color:#644a9b;'>\-</span> 

Por consiguiente, hemos demostrado que las columnas de <span style='color:#ff5500;'>$r_j$</span> de <span style='color:#ff5500;'>$</span><span style='color:#3daee9;'>\H</span><span style='color:#ff5500;'>_{u,q}$</span> generan las columnas de <span style='color:#ff5500;'>$</span><span style='color:#3daee9;'>\H</span><span style='color:#ff5500;'>_{u,v}$</span>. Así mismo, como el rango de <span style='color:#ff5500;'>$</span><span style='color:#3daee9;'>\H</span><span style='color:#ff5500;'>_{u,v}$</span> es <span style='color:#ff5500;'>$w$</span>, entonces las columnas <span style='color:#ff5500;'>$r_j$</span> de <span style='color:#ff5500;'>$</span><span style='color:#3daee9;'>\H</span><span style='color:#ff5500;'>_{u,q}$</span> son una base del espacio de columnas de <span style='color:#ff5500;'>$</span><span style='color:#3daee9;'>\H</span><span style='color:#ff5500;'>_{u,v}$</span>. En efecto, ellas generan las columnas de <span style='color:#ff5500;'>$</span><span style='color:#3daee9;'>\H</span><span style='color:#ff5500;'>_{u,v}$</span>, y si no son linealmente independientes, un subconjunto de ellas con menos de <span style='color:#ff5500;'>$w$</span> elementos sería una base de <span style='color:#ff5500;'>$</span><span style='color:#3daee9;'>\H</span><span style='color:#ff5500;'>_{u,v}$</span>, sin embargo, esto contradice que el rango de <span style='color:#ff5500;'>$</span><span style='color:#3daee9;'>\H</span><span style='color:#ff5500;'>_{u,v}$</span> es <span style='color:#ff5500;'>$w$</span>.<span style='color:#644a9b;'>\-</span>

Luego, debido a que <span style='color:#ff5500;'>$</span><span style='color:#3daee9;'>\H</span><span style='color:#ff5500;'>_{u,q}$</span> tiene <span style='color:#ff5500;'>$w$</span> elementos linealmente independientes que generan cualquiera de sus columnas, entonces su rango es de <span style='color:#ff5500;'>$w$</span>. Pero como  los rangos de <span style='color:#ff5500;'>$</span><span style='color:#3daee9;'>\H</span><span style='color:#ff5500;'>_{u,q+1}$</span> y <span style='color:#ff5500;'>$</span><span style='color:#3daee9;'>\H</span><span style='color:#ff5500;'>_{u+1,v}$</span> también son  <span style='color:#ff5500;'>$w$</span>, tenemos que <span style='color:#ff5500;'>$</span><span style='color:#3daee9;'>\H</span><span style='color:#ff5500;'>_{u,q}$</span> satisface <b>\eqref</b>{<b><span style='color:#0095ff;'>tminreal</span></b>}, lo cual es una contradicción.<span style='color:#644a9b;'>\-</span> 


<span style='color:#644a9b;'>\t</span> Toda <span style='color:#ff5500;'>$</span><span style='color:#3daee9;'>\H</span><span style='color:#ff5500;'>_{u,v}$</span> con <span style='color:#ff5500;'>$u&lt;p$</span>, <span style='color:#ff5500;'>$v&gt;q$</span> y <span style='color:#ff5500;'>$u+v+1</span><span style='color:#3daee9;'>\leq</span><span style='color:#ff5500;'> n$</span> no puede garantizar <b>\eqref</b>{<b><span style='color:#0095ff;'>tminreal</span></b>}, como se quería demostrar.<span style='color:#644a9b;'>\-</span>

<b>\end</b>{<b><span style='color:#0095ff;'>proof</span></b>}


<b>\begin</b>{<b><span style='color:#0095ff;'>Def</span></b>}<b>\label</b>{<b><span style='color:#0095ff;'>dt1min</span></b>}
 Sea la sucesión <span style='color:#ff5500;'>$</span><span style='color:#3daee9;'>\{</span><span style='color:#ff5500;'>J_t</span><span style='color:#3daee9;'>\}</span><span style='color:#ff5500;'>_{t</span><span style='color:#3daee9;'>\in\N</span><span style='color:#ff5500;'>_z}$</span>, con elementos en <span style='color:#ff5500;'>$</span><span style='color:#3daee9;'>\F</span><span style='color:#ff5500;'>^{</span><span style='color:#3daee9;'>\a\times</span><span style='color:#ff5500;'> </span><span style='color:#3daee9;'>\b</span><span style='color:#ff5500;'>}$</span> y sean <span style='color:#ff5500;'>$p,q</span><span style='color:#3daee9;'>\in\N</span><span style='color:#ff5500;'>$</span> tales que <span style='color:#ff5500;'>$p+q+1</span><span style='color:#3daee9;'>\leq</span><span style='color:#ff5500;'> z$</span>. Se dice que la matriz de hankel <span style='color:#ff5500;'>$</span><span style='color:#3daee9;'>\H</span><span style='color:#ff5500;'>_{p,q}$</span> formada por <span style='color:#ff5500;'>$</span><span style='color:#3daee9;'>\{</span><span style='color:#ff5500;'>J_t</span><span style='color:#3daee9;'>\}</span><span style='color:#ff5500;'>_{t</span><span style='color:#3daee9;'>\in\N</span><span style='color:#ff5500;'>_z}$</span> es mínima siempre que <span style='color:#ff5500;'>$p,q$</span> son los mínimos enteros para los cuales
 <span style='color:#644a9b;'>\ies</span>
rango<span style='color:#644a9b;'>\ \mathscr</span>{H}_{p,q}=rango<span style='color:#644a9b;'>\ \mathscr</span>{H}_{p,q+1}=
rango<span style='color:#644a9b;'>\ \mathscr</span>{H}_{p+1,z-(p+1)}
 <span style='color:#644a9b;'>\fes</span>
<b>\end</b>{<b><span style='color:#0095ff;'>Def</span></b>}

<span style='color:#898887;'>%///////////////////////////////////////////////////////////////////////////////////////////////////////////////////////////////////////</span>
<span style='color:#898887;'>%///////////////////////////////////////////////////////////////////////////////////////////////////////////////////////////////////////</span>
<b>\chapter</b>{<b>Álgebra lineal</b>}<b>\label</b>{<b><span style='color:#0095ff;'>Apendice B</span></b>}
<span style='color:#898887;'>%///////////////////////////////////////////////////////////////////////////////////////////////////////////////////////////////////////</span>
<span style='color:#898887;'>%///////////////////////////////////////////////////////////////////////////////////////////////////////////////////////////////////////</span>



<b>\begin</b>{<b><span style='color:#0095ff;'>Def</span></b>}[Espacio Vectorial]<b>\label</b>{<b><span style='color:#0095ff;'>espvect</span></b>}
Sea <span style='color:#ff5500;'>$</span><span style='color:#3daee9;'>\F</span><span style='color:#ff5500;'>$</span> un cuerpo. Un conjunto <span style='color:#ff5500;'>$V</span><span style='color:#3daee9;'>\neq\emptyset</span><span style='color:#ff5500;'>$</span> con las operaciones <span style='color:#ff5500;'>$+:V</span><span style='color:#3daee9;'>\times</span><span style='color:#ff5500;'> V</span><span style='color:#3daee9;'>\to</span><span style='color:#ff5500;'> V$</span> y <span style='color:#ff5500;'>$</span><span style='color:#3daee9;'>\cdot</span><span style='color:#ff5500;'>:</span><span style='color:#3daee9;'>\F\times</span><span style='color:#ff5500;'> V</span><span style='color:#3daee9;'>\to</span><span style='color:#ff5500;'> V$</span>,
llamadas adición y multiplicación escalar,
respectivamente, se dice que es un <span style='color:#644a9b;'>\textbf</span>{espacio vectorial} 
sobre un cuerpo <span style='color:#ff5500;'>$</span><span style='color:#3daee9;'>\F</span><span style='color:#ff5500;'>$</span> si
 <b>\begin</b>{<b><span style='color:#0095ff;'>enumerate</span></b>}[(a)]
  <span style='color:#644a9b;'>\item</span> <span style='color:#ff5500;'>$</span><span style='color:#3daee9;'>\mathbf</span><span style='color:#ff5500;'>{a}+</span><span style='color:#3daee9;'>\mathbf</span><span style='color:#ff5500;'>{b}=</span><span style='color:#3daee9;'>\mathbf</span><span style='color:#ff5500;'>{b}+</span><span style='color:#3daee9;'>\mathbf</span><span style='color:#ff5500;'>{a}$</span>;
  <span style='color:#644a9b;'>\item</span> <span style='color:#ff5500;'>$</span><span style='color:#3daee9;'>\mathbf</span><span style='color:#ff5500;'>{a}+(</span><span style='color:#3daee9;'>\mathbf</span><span style='color:#ff5500;'>{b}+</span><span style='color:#3daee9;'>\mathbf</span><span style='color:#ff5500;'>{g})=(</span><span style='color:#3daee9;'>\mathbf</span><span style='color:#ff5500;'>{a}+</span><span style='color:#3daee9;'>\mathbf</span><span style='color:#ff5500;'>{b})+</span><span style='color:#3daee9;'>\mathbf</span><span style='color:#ff5500;'>{g}$</span>;
  <span style='color:#644a9b;'>\item</span> <span style='color:#ff5500;'>$</span><span style='color:#3daee9;'>\exists</span><span style='color:#ff5500;'>!</span><span style='color:#3daee9;'>\ \mathbf</span><span style='color:#ff5500;'>{0}</span><span style='color:#3daee9;'>\in</span><span style='color:#ff5500;'> V$</span> tal que <span style='color:#ff5500;'>$</span><span style='color:#3daee9;'>\mathbf</span><span style='color:#ff5500;'>{a}+</span><span style='color:#3daee9;'>\mathbf</span><span style='color:#ff5500;'>{0}=</span><span style='color:#3daee9;'>\mathbf</span><span style='color:#ff5500;'>{a}$</span>, <span style='color:#ff5500;'>$</span><span style='color:#3daee9;'>\f</span><span style='color:#ff5500;'> </span><span style='color:#3daee9;'>\mathbf</span><span style='color:#ff5500;'>{a}</span><span style='color:#3daee9;'>\in</span><span style='color:#ff5500;'> V$</span>;
  <span style='color:#644a9b;'>\item</span> <span style='color:#ff5500;'>$</span><span style='color:#3daee9;'>\f</span><span style='color:#ff5500;'> </span><span style='color:#3daee9;'>\mathbf</span><span style='color:#ff5500;'>{a}</span><span style='color:#3daee9;'>\in</span><span style='color:#ff5500;'> V$</span>, <span style='color:#ff5500;'>$</span><span style='color:#3daee9;'>\exists</span><span style='color:#ff5500;'>!</span><span style='color:#3daee9;'>\ </span><span style='color:#ff5500;'>-</span><span style='color:#3daee9;'>\mathbf</span><span style='color:#ff5500;'>{a}</span><span style='color:#3daee9;'>\in</span><span style='color:#ff5500;'> V$</span> tal que <span style='color:#ff5500;'>$</span><span style='color:#3daee9;'>\mathbf</span><span style='color:#ff5500;'>{a}+(-</span><span style='color:#3daee9;'>\mathbf</span><span style='color:#ff5500;'>{a})=</span><span style='color:#3daee9;'>\mathbf</span><span style='color:#ff5500;'>{0}$</span>.
  <span style='color:#644a9b;'>\item</span> <span style='color:#ff5500;'>$1</span><span style='color:#3daee9;'>\mathbf</span><span style='color:#ff5500;'>{a}=</span><span style='color:#3daee9;'>\mathbf</span><span style='color:#ff5500;'>{a}$</span>, <span style='color:#ff5500;'>$</span><span style='color:#3daee9;'>\f</span><span style='color:#ff5500;'> </span><span style='color:#3daee9;'>\mathbf</span><span style='color:#ff5500;'>{a}</span><span style='color:#3daee9;'>\in</span><span style='color:#ff5500;'> V$</span>;
  <span style='color:#644a9b;'>\item</span> <span style='color:#ff5500;'>$(c_1c_2)</span><span style='color:#3daee9;'>\mathbf</span><span style='color:#ff5500;'>{a}=c_1(c_2</span><span style='color:#3daee9;'>\mathbf</span><span style='color:#ff5500;'>{a})$</span>;
  <span style='color:#644a9b;'>\item</span> <span style='color:#ff5500;'>$c(</span><span style='color:#3daee9;'>\mathbf</span><span style='color:#ff5500;'>{a}+</span><span style='color:#3daee9;'>\mathbf</span><span style='color:#ff5500;'>{b})=c</span><span style='color:#3daee9;'>\mathbf</span><span style='color:#ff5500;'>{a}+c</span><span style='color:#3daee9;'>\mathbf</span><span style='color:#ff5500;'>{b}$</span>;
  <span style='color:#644a9b;'>\item</span> <span style='color:#ff5500;'>$(c_1+c_2)</span><span style='color:#3daee9;'>\mathbf</span><span style='color:#ff5500;'>{a}=c_1</span><span style='color:#3daee9;'>\mathbf</span><span style='color:#ff5500;'>{a}+c_2</span><span style='color:#3daee9;'>\mathbf</span><span style='color:#ff5500;'>{a}$</span>.
 <b>\end</b>{<b><span style='color:#0095ff;'>enumerate</span></b>}
A los elementos de <span style='color:#ff5500;'>$V$</span> les llamaremos vectores y a los de <span style='color:#ff5500;'>$</span><span style='color:#3daee9;'>\F</span><span style='color:#ff5500;'>$</span> escalares. 
El vector <span style='color:#ff5500;'>$</span><span style='color:#3daee9;'>\mathbf</span><span style='color:#ff5500;'>{0}</span><span style='color:#3daee9;'>\in</span><span style='color:#ff5500;'> V$</span> que satisface el inciso (c) lo llamamos <span style='color:#644a9b;'>\textbf</span>{vector nulo} o <span style='color:#644a9b;'>\textbf</span>{vector cero}, 
y al vector <span style='color:#ff5500;'>$-</span><span style='color:#3daee9;'>\mathbf</span><span style='color:#ff5500;'>{a}</span><span style='color:#3daee9;'>\in</span><span style='color:#ff5500;'> V$</span> que satisface (d) se le denomina <span style='color:#644a9b;'>\textbf</span>{vector inverso aditivo} de <span style='color:#ff5500;'>$</span><span style='color:#3daee9;'>\mathbf</span><span style='color:#ff5500;'>{a}</span><span style='color:#3daee9;'>\in</span><span style='color:#ff5500;'> V$</span>.
 <b>\end</b>{<b><span style='color:#0095ff;'>Def</span></b>}

 
<b>\begin</b>{<b><span style='color:#0095ff;'>Def</span></b>}[Combinación Lineal]<b>\label</b>{<b><span style='color:#0095ff;'>comblin</span></b>}
Sea <span style='color:#ff5500;'>$V$</span> un espacio vectorial sobre un cuerpo <span style='color:#ff5500;'>$</span><span style='color:#3daee9;'>\F</span><span style='color:#ff5500;'>$</span>. Un vector <span style='color:#ff5500;'>$</span><span style='color:#3daee9;'>\mathbf</span><span style='color:#ff5500;'>{b}</span><span style='color:#3daee9;'>\in</span><span style='color:#ff5500;'> V$</span> se dice que es una 
<span style='color:#644a9b;'>\textbf</span>{combinación lineal} de los vectores 
<span style='color:#ff5500;'>$</span><span style='color:#3daee9;'>\mathbf</span><span style='color:#ff5500;'>{a_1},</span><span style='color:#3daee9;'>\dots</span><span style='color:#ff5500;'>,</span><span style='color:#3daee9;'>\mathbf</span><span style='color:#ff5500;'>{a_n}</span><span style='color:#3daee9;'>\in</span><span style='color:#ff5500;'> V$</span> siempre que existan 
<span style='color:#ff5500;'>$c_1,</span><span style='color:#3daee9;'>\dots</span><span style='color:#ff5500;'>,c_n</span><span style='color:#3daee9;'>\in</span><span style='color:#ff5500;'> F$</span> tales que 
<span style='color:#644a9b;'>\ies</span>
<span style='color:#644a9b;'>\mathbf</span>{b}=<span style='color:#644a9b;'>\sum</span>_{i=1}^{n}c_i<span style='color:#644a9b;'>\mathbf</span>{a}_i
<span style='color:#644a9b;'>\fes</span>
<b>\end</b>{<b><span style='color:#0095ff;'>Def</span></b>} 

 
<b>\begin</b>{<b><span style='color:#0095ff;'>Def</span></b>}[Subespacio Generado]<b>\label</b>{<b><span style='color:#0095ff;'>subgen</span></b>}
Sea <span style='color:#ff5500;'>$V$</span> un espacio vectorial sobre un cuerpo <span style='color:#ff5500;'>$</span><span style='color:#3daee9;'>\F</span><span style='color:#ff5500;'>$</span>, sea <span style='color:#ff5500;'>$S</span><span style='color:#3daee9;'>\subseteq</span><span style='color:#ff5500;'> V$</span> y sea <span style='color:#ff5500;'>$A_s$</span> el conjunto de subespacios de <span style='color:#ff5500;'>$V$</span> de contienen a <span style='color:#ff5500;'>$S$</span>. 
Se dice que el <span style='color:#644a9b;'>\textbf</span>{subespacio generado} por <span style='color:#ff5500;'>$S$</span> es el conjunto <span style='color:#ff5500;'>$W_s$</span> definido como 
<span style='color:#644a9b;'>\ies</span>
W_s=<span style='color:#644a9b;'>\bigcap\limits</span>_{_{N<span style='color:#644a9b;'>\in</span> A_s}} N
<span style='color:#644a9b;'>\fes</span>
<b>\end</b>{<b><span style='color:#0095ff;'>Def</span></b>} 

<b>\begin</b>{<b><span style='color:#0095ff;'>Teo</span></b>}
 Sea <span style='color:#ff5500;'>$V$</span> un espacio vectorial sobre un cuerpo <span style='color:#ff5500;'>$</span><span style='color:#3daee9;'>\F</span><span style='color:#ff5500;'>$</span>,
 sea <span style='color:#ff5500;'>$S</span><span style='color:#3daee9;'>\subseteq</span><span style='color:#ff5500;'> V$</span> no vacio. Si <span style='color:#ff5500;'>$L$</span> es el conjunto de combinaciones lineales de <span style='color:#ff5500;'>$S$</span>, entonces, el 
 espacio generado por <span style='color:#ff5500;'>$S$</span> es <span style='color:#ff5500;'>$L$</span>.
<b>\end</b>{<b><span style='color:#0095ff;'>Teo</span></b>}
 <b>\begin</b>{<b><span style='color:#0095ff;'>proof</span></b>} Sea <span style='color:#ff5500;'>$W_s$</span> el espacio generado por <span style='color:#ff5500;'>$S$</span>. 
 Tomando <span style='color:#ff5500;'>$</span><span style='color:#3daee9;'>\mathbf</span><span style='color:#ff5500;'>{a}</span><span style='color:#3daee9;'>\in</span><span style='color:#ff5500;'> L$</span> arbitrario, entonces <span style='color:#ff5500;'>$</span><span style='color:#3daee9;'>\mathbf</span><span style='color:#ff5500;'>{a}=</span><span style='color:#3daee9;'>\sum</span><span style='color:#ff5500;'>_{_{</span><span style='color:#3daee9;'>\mathbf</span><span style='color:#ff5500;'>{b}</span><span style='color:#3daee9;'>\in</span><span style='color:#ff5500;'> S}}c_b</span><span style='color:#3daee9;'>\mathbf</span><span style='color:#ff5500;'>{b}$</span>, con <span style='color:#ff5500;'>$c_b</span><span style='color:#3daee9;'>\in\F</span><span style='color:#ff5500;'>$</span>. 
 Como cada <span style='color:#ff5500;'>$</span><span style='color:#3daee9;'>\mathbf</span><span style='color:#ff5500;'>{b}</span><span style='color:#3daee9;'>\in</span><span style='color:#ff5500;'> S</span><span style='color:#3daee9;'>\subseteq</span><span style='color:#ff5500;'> W_s$</span> y <span style='color:#ff5500;'>$W_s$</span> es un espacio vectorial, entonces cada <span style='color:#ff5500;'>$c_b</span><span style='color:#3daee9;'>\mathbf</span><span style='color:#ff5500;'>{b}$</span> 
 pertenece a <span style='color:#ff5500;'>$W_s$</span>, es más, la suma de todos ellos pertenece a <span style='color:#ff5500;'>$W_s$</span>. 
 Por consiguiente <span style='color:#ff5500;'>$</span><span style='color:#3daee9;'>\mathbf</span><span style='color:#ff5500;'>{a}</span><span style='color:#3daee9;'>\in</span><span style='color:#ff5500;'> W_s$</span>, y así tenemos que <span style='color:#ff5500;'>$L</span><span style='color:#3daee9;'>\subseteq</span><span style='color:#ff5500;'> W$</span>.<span style='color:#644a9b;'>\-</span>
 
 Por otro lado, sean <span style='color:#ff5500;'>$</span><span style='color:#3daee9;'>\mathbf</span><span style='color:#ff5500;'>{a_1},</span><span style='color:#3daee9;'>\mathbf</span><span style='color:#ff5500;'>{a_2}</span><span style='color:#3daee9;'>\in</span><span style='color:#ff5500;'> L$</span> y <span style='color:#ff5500;'>$c_{a_1}</span><span style='color:#3daee9;'>\in</span><span style='color:#ff5500;'> </span><span style='color:#3daee9;'>\F</span><span style='color:#ff5500;'>$</span> arbitrarios. 
 Entonces <span style='color:#ff5500;'>$</span><span style='color:#3daee9;'>\mathbf</span><span style='color:#ff5500;'>{a_1}=</span><span style='color:#3daee9;'>\sum</span><span style='color:#ff5500;'>_{_{</span><span style='color:#3daee9;'>\mathbf</span><span style='color:#ff5500;'>{b}</span><span style='color:#3daee9;'>\in</span><span style='color:#ff5500;'> S}}c_b</span><span style='color:#3daee9;'>\mathbf</span><span style='color:#ff5500;'>{b}$</span> y <span style='color:#ff5500;'>$</span><span style='color:#3daee9;'>\mathbf</span><span style='color:#ff5500;'>{a_2}=</span><span style='color:#3daee9;'>\sum</span><span style='color:#ff5500;'>_{_{</span><span style='color:#3daee9;'>\mathbf</span><span style='color:#ff5500;'>{b}</span><span style='color:#3daee9;'>\in</span><span style='color:#ff5500;'> S}}d_b</span><span style='color:#3daee9;'>\mathbf</span><span style='color:#ff5500;'>{b}$</span>, y por consiguiente
 <span style='color:#644a9b;'>\ies</span>
 c_{a_1}<span style='color:#644a9b;'>\mathbf</span>{a_1}+<span style='color:#644a9b;'>\mathbf</span>{a_2}&amp;=&amp;c_{a_1}<span style='color:#644a9b;'>\sum</span>_{_{<span style='color:#644a9b;'>\mathbf</span>{b}<span style='color:#644a9b;'>\in</span> S}}c_b<span style='color:#644a9b;'>\mathbf</span>{b}+<span style='color:#644a9b;'>\sum</span>_{_{<span style='color:#644a9b;'>\mathbf</span>{b}<span style='color:#644a9b;'>\in</span> S}}d_b<span style='color:#644a9b;'>\mathbf</span>{b}<span style='color:#644a9b;'>\\</span>
 &amp;=&amp;<span style='color:#644a9b;'>\sum</span>_{_{<span style='color:#644a9b;'>\mathbf</span>{b}<span style='color:#644a9b;'>\in</span> S}}c_{a_1}c_b<span style='color:#644a9b;'>\mathbf</span>{b}+<span style='color:#644a9b;'>\sum</span>_{_{<span style='color:#644a9b;'>\mathbf</span>{b}<span style='color:#644a9b;'>\in</span> S}}d_b<span style='color:#644a9b;'>\mathbf</span>{b}<span style='color:#644a9b;'>\\</span>
 &amp;=&amp;<span style='color:#644a9b;'>\sum</span>_{_{<span style='color:#644a9b;'>\mathbf</span>{b}<span style='color:#644a9b;'>\in</span> S}}(c_{a_1}c_b<span style='color:#644a9b;'>\mathbf</span>{b}+d_b<span style='color:#644a9b;'>\mathbf</span>{b})<span style='color:#644a9b;'>\\</span>
 &amp;=&amp;<span style='color:#644a9b;'>\sum</span>_{_{<span style='color:#644a9b;'>\mathbf</span>{b}<span style='color:#644a9b;'>\in</span> S}}(c_{a_1}c_b+d_b)<span style='color:#644a9b;'>\mathbf</span>{b}
 <span style='color:#644a9b;'>\fes</span>
 Por lo que <span style='color:#ff5500;'>$c_{a_1}</span><span style='color:#3daee9;'>\mathbf</span><span style='color:#ff5500;'>{a_1}+</span><span style='color:#3daee9;'>\mathbf</span><span style='color:#ff5500;'>{a_2}</span><span style='color:#3daee9;'>\in</span><span style='color:#ff5500;'> L$</span>. Por tanto <span style='color:#ff5500;'>$L$</span> es subespacio vectorial y como contiene a <span style='color:#ff5500;'>$S$</span>, entonces <span style='color:#ff5500;'>$W</span><span style='color:#3daee9;'>\subseteq</span><span style='color:#ff5500;'> L$</span>.<span style='color:#644a9b;'>\-</span>
 
 <span style='color:#644a9b;'>\t</span> <span style='color:#ff5500;'>$W=L$</span>.<span style='color:#644a9b;'>\-</span>
  
 <b>\end</b>{<b><span style='color:#0095ff;'>proof</span></b>}

 <b>\begin</b>{<b><span style='color:#0095ff;'>Def</span></b>}[Dependencia Lineal]<b>\label</b>{<b><span style='color:#0095ff;'>deplin</span></b>}
Sea <span style='color:#ff5500;'>$V$</span> un espacio vectorial sobre el cuerpo <span style='color:#ff5500;'>$</span><span style='color:#3daee9;'>\F</span><span style='color:#ff5500;'>$</span> y sea <span style='color:#ff5500;'>$S</span><span style='color:#3daee9;'>\subseteq</span><span style='color:#ff5500;'> V$</span>. Se dice que <span style='color:#ff5500;'>$S$</span> es <span style='color:#644a9b;'>\textbf</span>{linealmente dependiente} 
si existe una combinación lineal de los vectores <span style='color:#ff5500;'>$</span><span style='color:#3daee9;'>\mathbf</span><span style='color:#ff5500;'>{a_1},</span><span style='color:#3daee9;'>\mathbf</span><span style='color:#ff5500;'>{a_2},</span><span style='color:#3daee9;'>\dots</span><span style='color:#ff5500;'>,</span><span style='color:#3daee9;'>\mathbf</span><span style='color:#ff5500;'>{a_n}</span><span style='color:#3daee9;'>\in</span><span style='color:#ff5500;'> S$</span> tal que  
<span style='color:#644a9b;'>\ies</span>
c_1<span style='color:#644a9b;'>\mathbf</span>{a_1}+c_2<span style='color:#644a9b;'>\mathbf</span>{a_2}+<span style='color:#644a9b;'>\cdots+</span>c_n<span style='color:#644a9b;'>\mathbf</span>{a_n}=<span style='color:#644a9b;'>\bar</span>{0}.
<span style='color:#644a9b;'>\fes</span>
donde algún escalar <span style='color:#ff5500;'>$c_1,c_2,</span><span style='color:#3daee9;'>\dots</span><span style='color:#ff5500;'>,c_n</span><span style='color:#3daee9;'>\in</span><span style='color:#ff5500;'> </span><span style='color:#3daee9;'>\F</span><span style='color:#ff5500;'>$</span> no sea nulo. Si un conjunto no es linealmente dependiente diremos que es
<span style='color:#644a9b;'>\textbf</span>{linealmente independiente}. Decir que un conjunto es linealmente dependiente o independiente será lo mismo que 
decir que sus vectores son linealmente dependientes o independientes.
<b>\end</b>{<b><span style='color:#0095ff;'>Def</span></b>}
 
 <b>\begin</b>{<b><span style='color:#0095ff;'>Def</span></b>}[Base de un espacio vectorial]<b>\label</b>{<b><span style='color:#0095ff;'>baseespvect</span></b>}
Sea <span style='color:#ff5500;'>$V$</span> un espacio vectorial sobre un cuerpo <span style='color:#ff5500;'>$</span><span style='color:#3daee9;'>\F</span><span style='color:#ff5500;'>$</span>  y sea <span style='color:#ff5500;'>$S</span><span style='color:#3daee9;'>\subseteq</span><span style='color:#ff5500;'> V$</span>. Se dice que <span style='color:#ff5500;'>$S$</span> es una <span style='color:#644a9b;'>\textbf</span>{base} de <span style='color:#ff5500;'>$V$</span> si es linealmente 
independiente y genera a <span style='color:#ff5500;'>$V$</span>. Definimos la <span style='color:#644a9b;'>\textbf</span>{dimensión} de <span style='color:#ff5500;'>$V$</span> como la cardinalidad de <span style='color:#ff5500;'>$S$</span> y la representamos como <span style='color:#ff5500;'>$dim(V)$</span>. 
Si la cardinalidad de <span style='color:#ff5500;'>$S$</span> es finita, decimos que <span style='color:#ff5500;'>$V$</span> es finito.
<b>\end</b>{<b><span style='color:#0095ff;'>Def</span></b>}

 
  <b>\begin</b>{<b><span style='color:#0095ff;'>Teo</span></b>}
Sea <span style='color:#ff5500;'>$V$</span> un espacio sobre el cuerpo <span style='color:#ff5500;'>$</span><span style='color:#3daee9;'>\F</span><span style='color:#ff5500;'>$</span> que es generado por un conjunto finito de vectores 
<span style='color:#ff5500;'>$</span><span style='color:#3daee9;'>\mathbf</span><span style='color:#ff5500;'>{b_1},</span><span style='color:#3daee9;'>\mathbf</span><span style='color:#ff5500;'>{b_2},</span><span style='color:#3daee9;'>\dots</span><span style='color:#ff5500;'>,</span><span style='color:#3daee9;'>\mathbf</span><span style='color:#ff5500;'>{b_m}$</span>. 
Entonces cualquier conjunto 
independiente de vectores en <span style='color:#ff5500;'>$V$</span> es finito y contiene no más de <span style='color:#ff5500;'>$m$</span> vectores.
<b>\end</b>{<b><span style='color:#0095ff;'>Teo</span></b>}
<b>\begin</b>{<b><span style='color:#0095ff;'>proof</span></b>}
 Vease <b>\citet</b>[p.44]{<b><span style='color:#0095ff;'>hoffman</span></b>}.<span style='color:#644a9b;'>\-</span>
 
<b>\end</b>{<b><span style='color:#0095ff;'>proof</span></b>}

 
 <b>\begin</b>{<b><span style='color:#0095ff;'>Cor</span></b>}<b>\label</b>{<b><span style='color:#0095ff;'>cor2abdim</span></b>}
 Sea <span style='color:#ff5500;'>$V$</span> un espacio vectorial de dimensión finita sobre un cuerpo <span style='color:#ff5500;'>$</span><span style='color:#3daee9;'>\F</span><span style='color:#ff5500;'>$</span> y sea <span style='color:#ff5500;'>$n=dim</span><span style='color:#3daee9;'>\ </span><span style='color:#ff5500;'>V$</span>. Entonces
<b>\begin</b>{<b><span style='color:#0095ff;'>enumerate</span></b>}[(a)]
 <span style='color:#644a9b;'>\item</span> cualquier subconjunto de <span style='color:#ff5500;'>$V$</span> que contenga más de <span style='color:#ff5500;'>$n$</span> vectores es linealmente dependiente;
 <span style='color:#644a9b;'>\item</span> ningún subconjunto de <span style='color:#ff5500;'>$V$</span> que contenga menos de <span style='color:#ff5500;'>$n$</span> vectores puede generar <span style='color:#ff5500;'>$V$</span>. 
<b>\end</b>{<b><span style='color:#0095ff;'>enumerate</span></b>}
<b>\end</b>{<b><span style='color:#0095ff;'>Cor</span></b>}
<b>\begin</b>{<b><span style='color:#0095ff;'>proof</span></b>}
 Vease <b>\citet</b>[p.44-45]{<b><span style='color:#0095ff;'>hoffman</span></b>}.<span style='color:#644a9b;'>\-</span>
 
<b>\end</b>{<b><span style='color:#0095ff;'>proof</span></b>}

<b>\begin</b>{<b><span style='color:#0095ff;'>Def</span></b>}<b>\label</b>{<b><span style='color:#0095ff;'>rangfilrangcoldef</span></b>}
Sea <span style='color:#ff5500;'>$A</span><span style='color:#3daee9;'>\in\R</span><span style='color:#ff5500;'>^{n</span><span style='color:#3daee9;'>\times</span><span style='color:#ff5500;'> m}(</span><span style='color:#3daee9;'>\F</span><span style='color:#ff5500;'>)$</span>. Se dice que el <span style='color:#644a9b;'>\textbf</span>{rango fila} de <span style='color:#ff5500;'>$A$</span> es la dimensión del espacio generado por las filas de <span style='color:#ff5500;'>$A$</span>. El <span style='color:#644a9b;'>\textbf</span>{rango  
columna} de <span style='color:#ff5500;'>$A$</span> se define como la <span style='color:#644a9b;'>\textbf</span>{dimensión} del espacio generado por las columnas de <span style='color:#ff5500;'>$A$</span>.
<b>\end</b>{<b><span style='color:#0095ff;'>Def</span></b>}

 <b>\begin</b>{<b><span style='color:#0095ff;'>Teo</span></b>}<b>\label</b>{<b><span style='color:#0095ff;'>rangfilrangcol</span></b>}
Si <span style='color:#ff5500;'>$A</span><span style='color:#3daee9;'>\in\R</span><span style='color:#ff5500;'>^{n</span><span style='color:#3daee9;'>\times</span><span style='color:#ff5500;'> m}(</span><span style='color:#3daee9;'>\F</span><span style='color:#ff5500;'>)$</span>, entonces <span style='color:#ff5500;'>$rango</span><span style='color:#3daee9;'>\ </span><span style='color:#ff5500;'>fila(A) = rango</span><span style='color:#3daee9;'>\ </span><span style='color:#ff5500;'>columna(A)$</span>.
<b>\end</b>{<b><span style='color:#0095ff;'>Teo</span></b>}
<b>\begin</b>{<b><span style='color:#0095ff;'>proof</span></b>}
 Vease <b>\citet</b>[p.72]{<b><span style='color:#0095ff;'>hoffman</span></b>}.<span style='color:#644a9b;'>\-</span>
 
<b>\end</b>{<b><span style='color:#0095ff;'>proof</span></b>}

<b>\begin</b>{<b><span style='color:#0095ff;'>Def</span></b>}<b>\label</b>{<b><span style='color:#0095ff;'>rangdef</span></b>}
Sea <span style='color:#ff5500;'>$A</span><span style='color:#3daee9;'>\in\R</span><span style='color:#ff5500;'>^{n</span><span style='color:#3daee9;'>\times</span><span style='color:#ff5500;'> m}(</span><span style='color:#3daee9;'>\F</span><span style='color:#ff5500;'>)$</span>. Se dice que el <span style='color:#644a9b;'>\textbf</span>{rango} de <span style='color:#ff5500;'>$A$</span> es la dimensión del espacio generado por las filas o las columnas de <span style='color:#ff5500;'>$A$</span>.
<b>\end</b>{<b><span style='color:#0095ff;'>Def</span></b>}

<b>\begin</b>{<b><span style='color:#0095ff;'>Teo</span></b>}<b>\label</b>{<b><span style='color:#0095ff;'>detA</span></b>}
 Sea <span style='color:#ff5500;'>$K$</span> un anillo conmutativo con unidad y sea <span style='color:#ff5500;'>$n$</span> un entero positivo. Existe exactamente una función determinante sobre el conjunto de las matrices 
 <span style='color:#ff5500;'>$n</span><span style='color:#3daee9;'>\times</span><span style='color:#ff5500;'> n$</span> sobre <span style='color:#ff5500;'>$</span><span style='color:#3daee9;'>\F</span><span style='color:#ff5500;'>$</span>, y es la función <span style='color:#ff5500;'>$det$</span> definida por <span style='color:#ff5500;'>$det(A)=</span><span style='color:#3daee9;'>\sum</span><span style='color:#ff5500;'>_{</span><span style='color:#3daee9;'>\mathbf</span><span style='color:#ff5500;'>{</span><span style='color:#3daee9;'>\sigma</span><span style='color:#ff5500;'>}}(sgn</span><span style='color:#3daee9;'>\ \sigma</span><span style='color:#ff5500;'>)A(1,</span><span style='color:#3daee9;'>\sigma</span><span style='color:#ff5500;'>_1)</span><span style='color:#3daee9;'>\cdots</span><span style='color:#ff5500;'> A(n,</span><span style='color:#3daee9;'>\sigma</span><span style='color:#ff5500;'>_n)$</span>.<span style='color:#644a9b;'>\footnote</span>{La función 
 <span style='color:#ff5500;'>$sgn</span><span style='color:#3daee9;'>\ \sigma</span><span style='color:#ff5500;'>$</span> es <span style='color:#ff5500;'>$1$</span> si <span style='color:#ff5500;'>$</span><span style='color:#3daee9;'>\sigma</span><span style='color:#ff5500;'>$</span> es par y <span style='color:#ff5500;'>$-1$</span> si <span style='color:#ff5500;'>$</span><span style='color:#3daee9;'>\sigma</span><span style='color:#ff5500;'>$</span> es impar. Aquí <span style='color:#ff5500;'>$</span><span style='color:#3daee9;'>\sigma</span><span style='color:#ff5500;'>$</span> es una permutación de  }  
<b>\end</b>{<b><span style='color:#0095ff;'>Teo</span></b>}
<b>\begin</b>{<b><span style='color:#0095ff;'>proof</span></b>}
 Vease <b>\citet</b>[p.150-152]{<b><span style='color:#0095ff;'>hoffman</span></b>}.<span style='color:#644a9b;'>\-</span>
 
<b>\end</b>{<b><span style='color:#0095ff;'>proof</span></b>}


<b>\begin</b>{<b><span style='color:#0095ff;'>Teo</span></b>}<b>\label</b>{<b><span style='color:#0095ff;'>detA</span></b>}
 Sea <span style='color:#ff5500;'>$A</span><span style='color:#3daee9;'>\in\textnormal</span><span style='color:#ff5500;'>{It}_{n}(</span><span style='color:#3daee9;'>\F</span><span style='color:#ff5500;'>)$</span> con <span style='color:#ff5500;'>$A_{i,i}</span><span style='color:#3daee9;'>\neq</span><span style='color:#ff5500;'>0$</span>, <span style='color:#ff5500;'>$</span><span style='color:#3daee9;'>\f</span><span style='color:#ff5500;'> i</span><span style='color:#3daee9;'>\in\N</span><span style='color:#ff5500;'>_n$</span>. Entonces existe su inversa, <span style='color:#ff5500;'>$A^{-1}$</span>. 
<b>\end</b>{<b><span style='color:#0095ff;'>Teo</span></b>}
<b>\begin</b>{<b><span style='color:#0095ff;'>proof</span></b>}
Solamente hay que probar que las columnas de la <span style='color:#ff5500;'>$A$</span> son linealmente independientes. En efecto, si lo son, generan el espacio <span style='color:#ff5500;'>$</span><span style='color:#3daee9;'>\F</span><span style='color:#ff5500;'>^n$</span>, en particular, los vectores canónicos de este espacio. Así, existen coeficientes asociados a cada combinación lineal. Todos ellos forman la matriz inverza que buscamos.<span style='color:#644a9b;'>\-</span>

Si existen coeficientes no nulos <span style='color:#ff5500;'>$b_{j_1}, b_{j_2}, </span><span style='color:#3daee9;'>\dots</span><span style='color:#ff5500;'>, b_{j_k}$</span> tales que la combinación lineal de las columnas de <span style='color:#ff5500;'>$A$</span> es igual al vector nulo, entonces tomando el menor de ellos
<span style='color:#644a9b;'>\ies</span>
<span style='color:#644a9b;'>\sum</span>_{i=1}^k b_{j_i}A_{j_rj_i}=b_{j_i}A_{j_ij_i}=0
<span style='color:#644a9b;'>\fes</span>
Sin embargo, como  <span style='color:#ff5500;'>$A_{j_ij_i}, b_{j_i}</span><span style='color:#3daee9;'>\neq</span><span style='color:#ff5500;'> 0$</span>, llegamos a una contradicción.<span style='color:#644a9b;'>\-</span>

<span style='color:#644a9b;'>\t</span> Las columnas de <span style='color:#ff5500;'>$A$</span> son linealmente independientes.<span style='color:#644a9b;'>\-</span>

<b>\end</b>{<b><span style='color:#0095ff;'>proof</span></b>}


<b>\begin</b>{<b><span style='color:#0095ff;'>Teo</span></b>}<b>\label</b>{<b><span style='color:#0095ff;'>IMTI</span></b>}
  Sea <span style='color:#ff5500;'>$A</span><span style='color:#3daee9;'>\in\textnormal</span><span style='color:#ff5500;'>{It}_{n}(</span><span style='color:#3daee9;'>\F</span><span style='color:#ff5500;'>)$</span> tal que su diagonal principal es unitaria, y sea <span style='color:#ff5500;'>$n_1</span><span style='color:#3daee9;'>\in\N</span><span style='color:#ff5500;'>_{n-1}$</span>. Si <span style='color:#ff5500;'>$A_1=A_{1:n_1,1:n_1}$</span> es la submatriz de las primeras <span style='color:#ff5500;'>$n_1$</span> filas y columnas de <span style='color:#ff5500;'>$A$</span>. Entonces, 
  <span style='color:#644a9b;'>\ies</span>
  A_1^{-1}=A_2
  <span style='color:#644a9b;'>\fes</span>
  donde <span style='color:#ff5500;'>$A_2=(A^{-1})_{1:n_1,1:n_1}$</span> es la submatriz de las primeras <span style='color:#ff5500;'>$n_1$</span> filas y columnas de <span style='color:#ff5500;'>$A^{-1}$</span> 
 <b>\end</b>{<b><span style='color:#0095ff;'>Teo</span></b>}
 <b>\begin</b>{<b><span style='color:#0095ff;'>proof</span></b>}
 Por el Teorema <b>\ref</b>{<b><span style='color:#0095ff;'>detA</span></b>}, entonces existen <span style='color:#ff5500;'>$A_1^{-1}$</span> y <span style='color:#ff5500;'>$A^{-1}$</span>, y por ende existe <span style='color:#ff5500;'>$A_2$</span>. Luego, para cada <span style='color:#ff5500;'>$i,j</span><span style='color:#3daee9;'>\in\N</span><span style='color:#ff5500;'>_{n_1}$</span>
 <span style='color:#644a9b;'>\ies</span>
 (A_1A_2)_{i,j}=<span style='color:#644a9b;'>\sum</span>_{r=1}^{n_1}(A_1)_{i,r}(A_2)_{r,j}=<span style='color:#644a9b;'>\sum</span>_{r=1}^{n_1}(A)_{i,r}(A^{-1})_{r,j}
 <span style='color:#644a9b;'>\fes</span>
 Como <span style='color:#ff5500;'>$(A)_{i,r}=0$</span>, si <span style='color:#ff5500;'>$i&lt;r$</span>, entonces
 <span style='color:#644a9b;'>\ies</span>
 <span style='color:#644a9b;'>\sum</span>_{r=1}^{n_1}(A)_{i,r}(A^{-1})_{r,j}&amp;=&amp;<span style='color:#644a9b;'>\sum</span>_{r=1}^{n_1}(A)_{i,r}(A^{-1})_{r,j}+0<span style='color:#644a9b;'>\hspace</span>{5em}
 <span style='color:#644a9b;'>\fes</span>
 <span style='color:#644a9b;'>\ies</span>
 <span style='color:#644a9b;'>\hspace</span>{10em}&amp;=&amp;<span style='color:#644a9b;'>\sum</span>_{r=1}^{n_1}(A)_{i,r}(A^{-1})_{r,j}+<span style='color:#644a9b;'>\sum</span>_{r=n_1+1}^{n}(A)_{i,r}(A^{-1})_{r,j}<span style='color:#644a9b;'>\\</span>
 &amp;=&amp;<span style='color:#644a9b;'>\sum</span>_{r=1}^{n}(A)_{i,r}(A^{-1})_{r,j}<span style='color:#644a9b;'>\\</span>
 &amp;=&amp;<span style='color:#644a9b;'>\delta</span>(i-j)
 <span style='color:#644a9b;'>\fes</span>
  <span style='color:#898887;'>%Lema pagina 21 Hoffman &amp; Kuzne</span>
 Es decir, <span style='color:#ff5500;'>$A_1A_2=I_{n_1}$</span>, lo cuál implica que 
 <span style='color:#644a9b;'>\ies</span> 
 A_2=I_{n_1}A_2=(A_1^{-1}A_1)A_2=A_1^{-1}(A_1A_2)=A_1^{-1}I_{n_1}=A_1^{-1}<span style='color:#644a9b;'>\fes</span>
 
 <span style='color:#644a9b;'>\t</span> <span style='color:#ff5500;'>$A_2=A^{-1}$</span>. 
 
 <b>\end</b>{<b><span style='color:#0095ff;'>proof</span></b>}<span style='color:#644a9b;'>\-</span>
 
 <b>\begin</b>{<b><span style='color:#0095ff;'>Def</span></b>}
  Sea <span style='color:#ff5500;'>$z=a+bi</span><span style='color:#3daee9;'>\in\C</span><span style='color:#ff5500;'>$</span>, con <span style='color:#ff5500;'>$a,b</span><span style='color:#3daee9;'>\in\R</span><span style='color:#ff5500;'>$</span>. Decimos que <span style='color:#ff5500;'>$a-bi</span><span style='color:#3daee9;'>\in\C</span><span style='color:#ff5500;'>$</span> es el <span style='color:#644a9b;'>\textbf</span>{conjugado} de <span style='color:#ff5500;'>$z$</span>, y lo denotaremos como <span style='color:#ff5500;'>$</span><span style='color:#3daee9;'>\overline</span><span style='color:#ff5500;'>{z}$</span>.
  <b>\end</b>{<b><span style='color:#0095ff;'>Def</span></b>}<span style='color:#644a9b;'>\-</span>

 <b>\begin</b>{<b><span style='color:#0095ff;'>Prop</span></b>}<b>\label</b>{<b><span style='color:#0095ff;'>sumultcomp</span></b>}
 Sean <span style='color:#ff5500;'>$w,z</span><span style='color:#3daee9;'>\in\C</span><span style='color:#ff5500;'>$</span>. Entonces, <span style='color:#ff5500;'>$</span><span style='color:#3daee9;'>\overline</span><span style='color:#ff5500;'>{w+z}=</span><span style='color:#3daee9;'>\overline</span><span style='color:#ff5500;'>{w}+</span><span style='color:#3daee9;'>\overline</span><span style='color:#ff5500;'>{z}$</span> y <span style='color:#ff5500;'>$</span><span style='color:#3daee9;'>\overline</span><span style='color:#ff5500;'>{wz}=(</span><span style='color:#3daee9;'>\overline</span><span style='color:#ff5500;'>{w})(</span><span style='color:#3daee9;'>\overline</span><span style='color:#ff5500;'>{z})$</span>.
 <b>\end</b>{<b><span style='color:#0095ff;'>Prop</span></b>}
 <b>\begin</b>{<b><span style='color:#0095ff;'>proof</span></b>} Sea <span style='color:#ff5500;'>$w=a+bi$</span> y <span style='color:#ff5500;'>$z=c+di$</span>, entonces por las propiedades de cuerpo
<span style='color:#644a9b;'>\ies</span>
<span style='color:#644a9b;'>\overline</span>{(a+bi)+(c+di)}&amp;=&amp;
  <span style='color:#644a9b;'>\overline</span>{a+c+(b+d)i}<span style='color:#644a9b;'>\\</span>
  &amp;=&amp;a+c-(b+d)i<span style='color:#644a9b;'>\\</span>
  &amp;=&amp;(a-bi)+(c-di)<span style='color:#644a9b;'>\\</span>
  &amp;=&amp;(<span style='color:#644a9b;'>\overline</span>{a+bi})+(<span style='color:#644a9b;'>\overline</span>{c+di})
<span style='color:#644a9b;'>\fes</span>
Y por otro lado tenemos que
<span style='color:#644a9b;'>\ies</span>
<span style='color:#644a9b;'>\overline</span>{(a+bi)(c+di)}&amp;=&amp;<span style='color:#644a9b;'>\overline</span>{ac+adi+bic+bidi}<span style='color:#644a9b;'>\\</span>
&amp;=&amp;<span style='color:#644a9b;'>\overline</span>{ac+adi+bci-bd}<span style='color:#644a9b;'>\\</span>
&amp;=&amp;ac-bd-(ad+cb)i<span style='color:#644a9b;'>\\</span>
&amp;=&amp;ac+bdii-adi-cbi<span style='color:#644a9b;'>\\</span>
&amp;=&amp;(a-bi)(c-di)<span style='color:#644a9b;'>\\</span>
&amp;=&amp;(<span style='color:#644a9b;'>\overline</span>{a+bi})(<span style='color:#644a9b;'>\overline</span>{c+di})
<span style='color:#644a9b;'>\fes</span>
 <b>\end</b>{<b><span style='color:#0095ff;'>proof</span></b>}<span style='color:#644a9b;'>\-</span>

 <b>\begin</b>{<b><span style='color:#0095ff;'>Def</span></b>}<b>\label</b>{<b><span style='color:#0095ff;'>trconjadj</span></b>}
Sean <span style='color:#ff5500;'>$</span><span style='color:#3daee9;'>\R</span><span style='color:#ff5500;'>^{n</span><span style='color:#3daee9;'>\times</span><span style='color:#ff5500;'> m}(</span><span style='color:#3daee9;'>\mathbb</span><span style='color:#ff5500;'>{C})$</span> y <span style='color:#ff5500;'>$</span><span style='color:#3daee9;'>\R</span><span style='color:#ff5500;'>^{m</span><span style='color:#3daee9;'>\times</span><span style='color:#ff5500;'> n}(</span><span style='color:#3daee9;'>\mathbb</span><span style='color:#ff5500;'>{C})$</span>. Entonces si <span style='color:#ff5500;'>$A</span><span style='color:#3daee9;'>\in\R</span><span style='color:#ff5500;'>^{n</span><span style='color:#3daee9;'>\times</span><span style='color:#ff5500;'> m}(</span><span style='color:#3daee9;'>\mathbb</span><span style='color:#ff5500;'>{C})$</span> decimos que  
<b>\begin</b>{<b><span style='color:#0095ff;'>enumerate</span></b>}
 <span style='color:#644a9b;'>\item</span> <span style='color:#ff5500;'>$A'</span><span style='color:#3daee9;'>\in\R</span><span style='color:#ff5500;'>^{m</span><span style='color:#3daee9;'>\times</span><span style='color:#ff5500;'> n}(</span><span style='color:#3daee9;'>\mathbb</span><span style='color:#ff5500;'>{C})$</span> es la <span style='color:#644a9b;'>\textbf</span>{matriz transpuesta} de <span style='color:#ff5500;'>$A$</span> siempre y cuando 
 <span style='color:#ff5500;'>$A'_{i,j}=A_{j,i}$</span>.
 <span style='color:#644a9b;'>\item</span> <span style='color:#ff5500;'>$</span><span style='color:#3daee9;'>\overline</span><span style='color:#ff5500;'>{A}</span><span style='color:#3daee9;'>\in\R</span><span style='color:#ff5500;'>^{n</span><span style='color:#3daee9;'>\times</span><span style='color:#ff5500;'> m}(</span><span style='color:#3daee9;'>\mathbb</span><span style='color:#ff5500;'>{C})$</span> es la <span style='color:#644a9b;'>\textbf</span>{matriz conjugada} de 
 <span style='color:#ff5500;'>$A$</span> siempre y cuando <span style='color:#ff5500;'>$(</span><span style='color:#3daee9;'>\overline</span><span style='color:#ff5500;'>{A})_{i,j}=</span><span style='color:#3daee9;'>\overline</span><span style='color:#ff5500;'>{A_{i,j}}$</span>.
 <span style='color:#644a9b;'>\item</span> <span style='color:#ff5500;'>$A^*</span><span style='color:#3daee9;'>\in\R</span><span style='color:#ff5500;'>^{m</span><span style='color:#3daee9;'>\times</span><span style='color:#ff5500;'> n}(</span><span style='color:#3daee9;'>\mathbb</span><span style='color:#ff5500;'>{C})$</span> es la <span style='color:#644a9b;'>\textbf</span>{matriz adjunta} de <span style='color:#ff5500;'>$A$</span> siempre y cuando <span style='color:#ff5500;'>$A^*=</span><span style='color:#3daee9;'>\overline</span><span style='color:#ff5500;'>{A'}$</span>.
<b>\end</b>{<b><span style='color:#0095ff;'>enumerate</span></b>}
<b>\end</b>{<b><span style='color:#0095ff;'>Def</span></b>}<span style='color:#644a9b;'>\-</span>

<b>\begin</b>{<b><span style='color:#0095ff;'>Prop</span></b>}<b>\label</b>{<b><span style='color:#0095ff;'>propmatt</span></b>}
Sea <span style='color:#ff5500;'>$A,B</span><span style='color:#3daee9;'>\in\R</span><span style='color:#ff5500;'>^{n</span><span style='color:#3daee9;'>\times</span><span style='color:#ff5500;'> m}(</span><span style='color:#3daee9;'>\mathbb</span><span style='color:#ff5500;'>{C})$</span>, <span style='color:#ff5500;'>$C</span><span style='color:#3daee9;'>\in\R</span><span style='color:#ff5500;'>^{m</span><span style='color:#3daee9;'>\times</span><span style='color:#ff5500;'> n}(</span><span style='color:#3daee9;'>\mathbb</span><span style='color:#ff5500;'>{C})$</span> y <span style='color:#ff5500;'>$z</span><span style='color:#3daee9;'>\in\C</span><span style='color:#ff5500;'>$</span>. Entonces 
<b>\begin</b>{<b><span style='color:#0095ff;'>enumerate</span></b>}
<span style='color:#644a9b;'>\item</span> <span style='color:#ff5500;'>$(A')'=A$</span>;
 <span style='color:#644a9b;'>\item</span> <span style='color:#ff5500;'>$(AC)'=C'A'$</span>;
 <span style='color:#644a9b;'>\item</span> <span style='color:#ff5500;'>$(A+B)'=A'+B'$</span>; y
 <span style='color:#644a9b;'>\item</span> <span style='color:#ff5500;'>$(zA)'=zA'$</span>
<b>\end</b>{<b><span style='color:#0095ff;'>enumerate</span></b>}
<b>\end</b>{<b><span style='color:#0095ff;'>Prop</span></b>}
<b>\begin</b>{<b><span style='color:#0095ff;'>proof</span></b>} Por la Definición <b>\ref</b>{<b><span style='color:#0095ff;'>trconjadj</span></b>} (1) tenemos que
<b>\begin</b>{<b><span style='color:#0095ff;'>enumerate</span></b>} 
 <span style='color:#644a9b;'>\item</span> <span style='color:#ff5500;'>$(A')'_{i,j}=A'_{j,i}=A_{i,j}$</span>. Por lo tanto, 
 <span style='color:#ff5500;'>$(A')'=A$</span>.
 <span style='color:#644a9b;'>\item</span> <span style='color:#ff5500;'>$(AC)'_{i,j}=(AC)_{j,i}=</span><span style='color:#3daee9;'>\sum</span><span style='color:#ff5500;'>_{r=1}^{m}A_{j,r}C_{r,i}=</span><span style='color:#3daee9;'>\sum</span><span style='color:#ff5500;'>_{r=1}^{m}C'_{i,r}A'_{r,j}=(C'A')_{i,j}$</span>, <span style='color:#ff5500;'>$</span><span style='color:#3daee9;'>\f</span><span style='color:#ff5500;'> (i,j)</span><span style='color:#3daee9;'>\in\N</span><span style='color:#ff5500;'>_{n}</span><span style='color:#3daee9;'>\times\N</span><span style='color:#ff5500;'>_{m}$</span>. 
 Por lo tanto, <span style='color:#ff5500;'>$(AC)'=C'A'$</span>.
 <span style='color:#644a9b;'>\item</span> <span style='color:#ff5500;'>$(A+B)'_{i,j}=(A+B)_{j,i}=A_{j,i}+B_{j,i}=A'_{i,j}+B'_{i,j}$</span>. Por lo tanto, <span style='color:#ff5500;'>$(A+B)'=A'+B'$</span>.
 <span style='color:#644a9b;'>\item</span> <span style='color:#ff5500;'>$(zA)'_{i,j}=(zA)_{j,i}=z A_{j,i}= z A'_{i,j}$</span>. Por lo tanto, <span style='color:#ff5500;'>$(zA)'=z A'$</span>.
<b>\end</b>{<b><span style='color:#0095ff;'>enumerate</span></b>}

<b>\end</b>{<b><span style='color:#0095ff;'>proof</span></b>}


<b>\begin</b>{<b><span style='color:#0095ff;'>Prop</span></b>}<b>\label</b>{<b><span style='color:#0095ff;'>sumconj</span></b>}
 Sea <span style='color:#ff5500;'>$A,B</span><span style='color:#3daee9;'>\in\R</span><span style='color:#ff5500;'>^{n</span><span style='color:#3daee9;'>\times</span><span style='color:#ff5500;'> m}(</span><span style='color:#3daee9;'>\mathbb</span><span style='color:#ff5500;'>{C})$</span> y <span style='color:#ff5500;'>$z</span><span style='color:#3daee9;'>\in\C</span><span style='color:#ff5500;'>$</span>. Entonces, <span style='color:#ff5500;'>$(A+B)^*=A^*+B^*$</span> y <span style='color:#ff5500;'>$(zA)^*=</span><span style='color:#3daee9;'>\overline</span><span style='color:#ff5500;'>{z}A^*$</span>.
<b>\end</b>{<b><span style='color:#0095ff;'>Prop</span></b>}
<b>\begin</b>{<b><span style='color:#0095ff;'>proof</span></b>} Por la Proposición <b>\ref</b>{<b><span style='color:#0095ff;'>propmatt</span></b>} tenemos que <span style='color:#ff5500;'>$(A+B)'=A'+B'$</span>. Además, por la Proposición <b>\ref</b>{<b><span style='color:#0095ff;'>sumultcomp</span></b>}
<span style='color:#644a9b;'>\ies</span>
<span style='color:#644a9b;'>\overline</span>{(A+B)'_{i,j}}&amp;=&amp;<span style='color:#644a9b;'>\overline</span>{A'_{i,j}+B'_{i,j}}<span style='color:#644a9b;'>\\</span>
&amp;=&amp;<span style='color:#644a9b;'>\overline</span>{A'_{i,j}}+<span style='color:#644a9b;'>\overline</span>{B'_{i,j}}
<span style='color:#644a9b;'>\fes</span>
Por otra parte, por la Proposición <b>\ref</b>{<b><span style='color:#0095ff;'>sumultcomp</span></b>} y la Definición <b>\ref</b>{<b><span style='color:#0095ff;'>trconjadj</span></b>},
<span style='color:#ff5500;'>$(zA)^*_{i,j}=</span><span style='color:#3daee9;'>\overline</span><span style='color:#ff5500;'>{(zA)'_{i,j}}=</span><span style='color:#3daee9;'>\overline</span><span style='color:#ff5500;'>{(zA)_{j,i}}</span>
<span style='color:#ff5500;'>=</span><span style='color:#3daee9;'>\overline</span><span style='color:#ff5500;'>{zA_{j,i}}=</span><span style='color:#3daee9;'>\overline</span><span style='color:#ff5500;'>{z}</span><span style='color:#3daee9;'>\overline</span><span style='color:#ff5500;'>{A_{j,i}}=</span><span style='color:#3daee9;'>\overline</span><span style='color:#ff5500;'>{z}</span><span style='color:#3daee9;'>\overline</span><span style='color:#ff5500;'>{A'_{i,j}}=</span><span style='color:#3daee9;'>\overline</span><span style='color:#ff5500;'>{z}A^*_{i,j}$</span>.

<b>\end</b>{<b><span style='color:#0095ff;'>proof</span></b>}<span style='color:#644a9b;'>\-</span>


 <b>\begin</b>{<b><span style='color:#0095ff;'>Def</span></b>}[Producto interno]<b>\label</b>{<b><span style='color:#0095ff;'>prod_interno</span></b>}
 Sea <span style='color:#ff5500;'>$V$</span> un espacio vectorial sobre <span style='color:#ff5500;'>$</span><span style='color:#3daee9;'>\C</span><span style='color:#ff5500;'>$</span>. Decimos que
 la fución <span style='color:#ff5500;'>$</span><span style='color:#3daee9;'>\langle\cdot</span><span style='color:#ff5500;'>,</span><span style='color:#3daee9;'>\cdot\rangle</span><span style='color:#ff5500;'>:V</span><span style='color:#3daee9;'>\times</span><span style='color:#ff5500;'> V</span><span style='color:#3daee9;'>\to</span><span style='color:#ff5500;'> </span><span style='color:#3daee9;'>\C</span><span style='color:#ff5500;'>$</span> es el <span style='color:#644a9b;'>\textbf</span>{producto interior} de <span style='color:#ff5500;'>$V$</span> siempre que  
 <span style='color:#ff5500;'>$</span><span style='color:#3daee9;'>\f</span><span style='color:#ff5500;'> </span><span style='color:#3daee9;'>\mathbf</span><span style='color:#ff5500;'>{u},</span><span style='color:#3daee9;'>\mathbf</span><span style='color:#ff5500;'>{v},</span><span style='color:#3daee9;'>\mathbf</span><span style='color:#ff5500;'>{w}</span><span style='color:#3daee9;'>\in</span><span style='color:#ff5500;'> V$</span> y 
 <span style='color:#ff5500;'>$c</span><span style='color:#3daee9;'>\in\C</span><span style='color:#ff5500;'>$</span> se cumple que
 <b>\begin</b>{<b><span style='color:#0095ff;'>enumerate</span></b>}
  <span style='color:#644a9b;'>\item</span> <span style='color:#ff5500;'>$</span><span style='color:#3daee9;'>\langle</span><span style='color:#ff5500;'> </span><span style='color:#3daee9;'>\mathbf</span><span style='color:#ff5500;'>{u}+</span><span style='color:#3daee9;'>\mathbf</span><span style='color:#ff5500;'>{v},</span><span style='color:#3daee9;'>\mathbf</span><span style='color:#ff5500;'>{w}</span><span style='color:#3daee9;'>\rangle</span><span style='color:#ff5500;'>=</span><span style='color:#3daee9;'>\langle</span><span style='color:#ff5500;'> </span><span style='color:#3daee9;'>\mathbf</span><span style='color:#ff5500;'>{u},</span><span style='color:#3daee9;'>\mathbf</span><span style='color:#ff5500;'>{w}</span><span style='color:#3daee9;'>\rangle</span><span style='color:#ff5500;'>+</span><span style='color:#3daee9;'>\langle</span><span style='color:#ff5500;'> </span><span style='color:#3daee9;'>\mathbf</span><span style='color:#ff5500;'>{v},</span><span style='color:#3daee9;'>\mathbf</span><span style='color:#ff5500;'>{w}</span><span style='color:#3daee9;'>\rangle</span><span style='color:#ff5500;'>$</span>;
  <span style='color:#644a9b;'>\item</span> <span style='color:#ff5500;'>$</span><span style='color:#3daee9;'>\langle</span><span style='color:#ff5500;'> c</span><span style='color:#3daee9;'>\mathbf</span><span style='color:#ff5500;'>{u},</span><span style='color:#3daee9;'>\mathbf</span><span style='color:#ff5500;'>{v}</span><span style='color:#3daee9;'>\rangle</span><span style='color:#ff5500;'>=c</span><span style='color:#3daee9;'>\langle</span><span style='color:#ff5500;'> </span><span style='color:#3daee9;'>\mathbf</span><span style='color:#ff5500;'>{u},</span><span style='color:#3daee9;'>\mathbf</span><span style='color:#ff5500;'>{v}</span><span style='color:#3daee9;'>\rangle</span><span style='color:#ff5500;'>$</span>;
  <span style='color:#644a9b;'>\item</span> <span style='color:#ff5500;'>$</span><span style='color:#3daee9;'>\langle</span><span style='color:#ff5500;'> </span><span style='color:#3daee9;'>\mathbf</span><span style='color:#ff5500;'>{v},</span><span style='color:#3daee9;'>\mathbf</span><span style='color:#ff5500;'>{u}</span><span style='color:#3daee9;'>\rangle</span><span style='color:#ff5500;'>=</span><span style='color:#3daee9;'>\overline</span><span style='color:#ff5500;'>{</span><span style='color:#3daee9;'>\langle</span><span style='color:#ff5500;'> </span><span style='color:#3daee9;'>\mathbf</span><span style='color:#ff5500;'>{u},</span><span style='color:#3daee9;'>\mathbf</span><span style='color:#ff5500;'>{v}</span><span style='color:#3daee9;'>\rangle</span><span style='color:#ff5500;'>}$</span>; y
  <span style='color:#644a9b;'>\item</span> <span style='color:#ff5500;'>$</span><span style='color:#3daee9;'>\langle</span><span style='color:#ff5500;'> </span><span style='color:#3daee9;'>\mathbf</span><span style='color:#ff5500;'>{u},</span><span style='color:#3daee9;'>\mathbf</span><span style='color:#ff5500;'>{u}</span><span style='color:#3daee9;'>\rangle</span><span style='color:#ff5500;'>&gt;0$</span> si <span style='color:#ff5500;'>$</span><span style='color:#3daee9;'>\mathbf</span><span style='color:#ff5500;'>{u}</span><span style='color:#3daee9;'>\neq\mathbf</span><span style='color:#ff5500;'>{0}$</span> 
 <b>\end</b>{<b><span style='color:#0095ff;'>enumerate</span></b>}
 Si <span style='color:#ff5500;'>$</span><span style='color:#3daee9;'>\langle\cdot</span><span style='color:#ff5500;'>,</span><span style='color:#3daee9;'>\cdot\rangle</span><span style='color:#ff5500;'>$</span> es un producto interno en <span style='color:#ff5500;'>$V$</span>, entonces diremos 
 que <span style='color:#ff5500;'>$(</span><span style='color:#3daee9;'>\langle\cdot</span><span style='color:#ff5500;'>,</span><span style='color:#3daee9;'>\cdot\rangle</span><span style='color:#ff5500;'>,V)$</span> es un <span style='color:#644a9b;'>\textbf</span>{espacio producto interno}.
<b>\end</b>{<b><span style='color:#0095ff;'>Def</span></b>}<span style='color:#644a9b;'>\-</span>

<b>\begin</b>{<b><span style='color:#0095ff;'>Lem</span></b>}<b>\label</b>{<b><span style='color:#0095ff;'>lema_anterior</span></b>}
 Sea <span style='color:#ff5500;'>$(</span><span style='color:#3daee9;'>\langle\cdot</span><span style='color:#ff5500;'>,</span><span style='color:#3daee9;'>\cdot\rangle</span><span style='color:#ff5500;'>,V)$</span> un espacio producto interno sobre <span style='color:#ff5500;'>$</span><span style='color:#3daee9;'>\C</span><span style='color:#ff5500;'>$</span>. Entonces <span style='color:#ff5500;'>$</span><span style='color:#3daee9;'>\langle</span><span style='color:#ff5500;'> </span><span style='color:#3daee9;'>\mathbf</span><span style='color:#ff5500;'>{0},</span><span style='color:#3daee9;'>\mathbf</span><span style='color:#ff5500;'>{v}</span><span style='color:#3daee9;'>\rangle</span><span style='color:#ff5500;'>=0$</span> para todo <span style='color:#ff5500;'>$</span><span style='color:#3daee9;'>\mathbf</span><span style='color:#ff5500;'>{v}</span><span style='color:#3daee9;'>\in</span><span style='color:#ff5500;'> V$</span>.
<b>\end</b>{<b><span style='color:#0095ff;'>Lem</span></b>}
<b>\begin</b>{<b><span style='color:#0095ff;'>proof</span></b>}
 De la propiedad 1 de la Definición <b>\ref</b>{<b><span style='color:#0095ff;'>prod_interno</span></b>} tenemos
 <span style='color:#644a9b;'>\ies</span>
 <span style='color:#644a9b;'>\langle</span> <span style='color:#644a9b;'>\mathbf</span>{0},<span style='color:#644a9b;'>\mathbf</span>{v}<span style='color:#644a9b;'>\rangle</span>=<span style='color:#644a9b;'>\langle</span> <span style='color:#644a9b;'>\mathbf</span>{0}+<span style='color:#644a9b;'>\mathbf</span>{0},<span style='color:#644a9b;'>\mathbf</span>{v}<span style='color:#644a9b;'>\rangle</span>=<span style='color:#644a9b;'>\langle</span> <span style='color:#644a9b;'>\mathbf</span>{0},<span style='color:#644a9b;'>\mathbf</span>{v}<span style='color:#644a9b;'>\rangle+\langle</span> <span style='color:#644a9b;'>\mathbf</span>{0},<span style='color:#644a9b;'>\mathbf</span>{v}<span style='color:#644a9b;'>\rangle</span>
 <span style='color:#644a9b;'>\fes</span>
 siempre y cuando, <span style='color:#ff5500;'>$</span><span style='color:#3daee9;'>\langle</span><span style='color:#ff5500;'> </span><span style='color:#3daee9;'>\mathbf</span><span style='color:#ff5500;'>{0},</span><span style='color:#3daee9;'>\mathbf</span><span style='color:#ff5500;'>{v}</span><span style='color:#3daee9;'>\rangle</span><span style='color:#ff5500;'>=0$</span><span style='color:#644a9b;'>\-</span>
 
<b>\end</b>{<b><span style='color:#0095ff;'>proof</span></b>}

<b>\begin</b>{<b><span style='color:#0095ff;'>Prop</span></b>}<b>\label</b>{<b><span style='color:#0095ff;'>propequiv0</span></b>}
 Sea <span style='color:#ff5500;'>$(</span><span style='color:#3daee9;'>\langle\cdot</span><span style='color:#ff5500;'>,</span><span style='color:#3daee9;'>\cdot\rangle</span><span style='color:#ff5500;'>,V)$</span> un espacio producto interno. 
 Si <span style='color:#ff5500;'>$</span><span style='color:#3daee9;'>\langle</span><span style='color:#ff5500;'> </span><span style='color:#3daee9;'>\mathbf</span><span style='color:#ff5500;'>{v},</span><span style='color:#3daee9;'>\mathbf</span><span style='color:#ff5500;'>{v}</span><span style='color:#3daee9;'>\rangle</span><span style='color:#ff5500;'>=0$</span> entonces <span style='color:#ff5500;'>$</span><span style='color:#3daee9;'>\mathbf</span><span style='color:#ff5500;'>{v}=</span><span style='color:#3daee9;'>\mathbf</span><span style='color:#ff5500;'>{0}$</span>. 
<b>\end</b>{<b><span style='color:#0095ff;'>Prop</span></b>}
<b>\begin</b>{<b><span style='color:#0095ff;'>proof</span></b>}
Por el Lema <b>\ref</b>{<b><span style='color:#0095ff;'>lema_anterior</span></b>} sabemos que <span style='color:#ff5500;'>$</span><span style='color:#3daee9;'>\langle</span><span style='color:#ff5500;'> </span><span style='color:#3daee9;'>\mathbf</span><span style='color:#ff5500;'>{v},</span><span style='color:#3daee9;'>\mathbf</span><span style='color:#ff5500;'>{v}</span><span style='color:#3daee9;'>\rangle</span><span style='color:#ff5500;'>=0$</span> está bien definida. Ahora, si suponemos que <span style='color:#ff5500;'>$</span><span style='color:#3daee9;'>\mathbf</span><span style='color:#ff5500;'>{v}</span><span style='color:#3daee9;'>\neq\mathbf</span><span style='color:#ff5500;'>{0}$</span> entonces por la 
propiedad 4 de la Definición <b>\ref</b>{<b><span style='color:#0095ff;'>prod_interno</span></b>} <span style='color:#ff5500;'>$</span><span style='color:#3daee9;'>\langle</span><span style='color:#ff5500;'> </span><span style='color:#3daee9;'>\mathbf</span><span style='color:#ff5500;'>{v},</span><span style='color:#3daee9;'>\mathbf</span><span style='color:#ff5500;'>{v}</span><span style='color:#3daee9;'>\rangle</span><span style='color:#ff5500;'>&gt;0$</span>, contradiciendo la hipótesis.
 
 <b>\end</b>{<b><span style='color:#0095ff;'>proof</span></b>}<span style='color:#644a9b;'>\-</span>

 <b>\begin</b>{<b><span style='color:#0095ff;'>Teo</span></b>}<b>\label</b>{<b><span style='color:#0095ff;'>prodintanorma</span></b>}
  Sea <span style='color:#ff5500;'>$V$</span> es un espacio producto interno sobre <span style='color:#ff5500;'>$</span><span style='color:#3daee9;'>\C</span><span style='color:#ff5500;'>$</span>. Si <span style='color:#ff5500;'>$M(</span><span style='color:#3daee9;'>\cdot</span><span style='color:#ff5500;'>):V</span><span style='color:#3daee9;'>\to\R</span><span style='color:#ff5500;'>$</span> está definida como <span style='color:#ff5500;'>$M(</span><span style='color:#3daee9;'>\mathbf</span><span style='color:#ff5500;'>{a})=</span><span style='color:#3daee9;'>\sqrt</span><span style='color:#ff5500;'>{</span><span style='color:#3daee9;'>\langle\mathbf</span><span style='color:#ff5500;'>{a},</span><span style='color:#3daee9;'>\mathbf</span><span style='color:#ff5500;'>{a}</span><span style='color:#3daee9;'>\rangle</span><span style='color:#ff5500;'>}$</span>,
  entonces para todo <span style='color:#ff5500;'>$</span><span style='color:#3daee9;'>\mathbf</span><span style='color:#ff5500;'>{a},</span><span style='color:#3daee9;'>\mathbf</span><span style='color:#ff5500;'>{b}</span><span style='color:#3daee9;'>\in</span><span style='color:#ff5500;'> V$</span> y <span style='color:#ff5500;'>$c</span><span style='color:#3daee9;'>\in\C</span><span style='color:#ff5500;'>$</span>,
  <b>\begin</b>{<b><span style='color:#0095ff;'>enumerate</span></b>}
   <span style='color:#644a9b;'>\item</span> <span style='color:#ff5500;'>$M(c</span><span style='color:#3daee9;'>\mathbf</span><span style='color:#ff5500;'>{a})=</span><span style='color:#3daee9;'>\abs</span><span style='color:#ff5500;'>{c}M(</span><span style='color:#3daee9;'>\mathbf</span><span style='color:#ff5500;'>{a})$</span>;
   <span style='color:#644a9b;'>\item</span> <span style='color:#ff5500;'>$M(</span><span style='color:#3daee9;'>\mathbf</span><span style='color:#ff5500;'>{a})&gt;0$</span> cuando <span style='color:#ff5500;'>$</span><span style='color:#3daee9;'>\mathbf</span><span style='color:#ff5500;'>{a}</span><span style='color:#3daee9;'>\neq\mathbf</span><span style='color:#ff5500;'>{0}$</span>;
   <span style='color:#644a9b;'>\item</span> <span style='color:#ff5500;'>$</span><span style='color:#3daee9;'>\abs</span><span style='color:#ff5500;'>{</span><span style='color:#3daee9;'>\langle\mathbf</span><span style='color:#ff5500;'>{a},</span><span style='color:#3daee9;'>\mathbf</span><span style='color:#ff5500;'>{b}</span><span style='color:#3daee9;'>\rangle</span><span style='color:#ff5500;'>}</span><span style='color:#3daee9;'>\leq</span><span style='color:#ff5500;'> M(</span><span style='color:#3daee9;'>\mathbf</span><span style='color:#ff5500;'>{a})M(</span><span style='color:#3daee9;'>\mathbf</span><span style='color:#ff5500;'>{b})$</span>
   <span style='color:#644a9b;'>\item</span> <span style='color:#ff5500;'>$M(</span><span style='color:#3daee9;'>\mathbf</span><span style='color:#ff5500;'>{a}+</span><span style='color:#3daee9;'>\mathbf</span><span style='color:#ff5500;'>{b})</span><span style='color:#3daee9;'>\leq</span><span style='color:#ff5500;'> M(</span><span style='color:#3daee9;'>\mathbf</span><span style='color:#ff5500;'>{a})+M(</span><span style='color:#3daee9;'>\mathbf</span><span style='color:#ff5500;'>{b})$</span>
  <b>\end</b>{<b><span style='color:#0095ff;'>enumerate</span></b>}

 <b>\end</b>{<b><span style='color:#0095ff;'>Teo</span></b>}
 <b>\begin</b>{<b><span style='color:#0095ff;'>proof</span></b>}
 Véase <b>\citet</b>[pp.273 y 277]{<b><span style='color:#0095ff;'>hoffman</span></b>}
 
 <b>\end</b>{<b><span style='color:#0095ff;'>proof</span></b>}<span style='color:#644a9b;'>\-</span>
 <b>\begin</b>{<b><span style='color:#0095ff;'>Def</span></b>}[Norma]<b>\label</b>{<b><span style='color:#0095ff;'>normavect</span></b>}
 Sea <span style='color:#ff5500;'>$V$</span> un espacio vectorial sobre <span style='color:#ff5500;'>$</span><span style='color:#3daee9;'>\C</span><span style='color:#ff5500;'>$</span>. Decimos que la fución <span style='color:#ff5500;'>$</span><span style='color:#3daee9;'>\norm</span><span style='color:#ff5500;'>{</span><span style='color:#3daee9;'>\cdot</span><span style='color:#ff5500;'>}:V</span><span style='color:#3daee9;'>\to</span><span style='color:#ff5500;'> </span><span style='color:#3daee9;'>\R</span><span style='color:#ff5500;'>$</span> es una <span style='color:#644a9b;'>\textbf</span>{norma} en <span style='color:#ff5500;'>$V$</span> 
 siempre que
 <b>\begin</b>{<b><span style='color:#0095ff;'>enumerate</span></b>}
  <span style='color:#644a9b;'>\item</span> <span style='color:#ff5500;'>$</span><span style='color:#3daee9;'>\norm</span><span style='color:#ff5500;'>{</span><span style='color:#3daee9;'>\mathbf</span><span style='color:#ff5500;'>{v}}&gt;0$</span>, <span style='color:#ff5500;'>$</span><span style='color:#3daee9;'>\f</span><span style='color:#ff5500;'> </span><span style='color:#3daee9;'>\mathbf</span><span style='color:#ff5500;'>{v}</span><span style='color:#3daee9;'>\in</span><span style='color:#ff5500;'> V/</span><span style='color:#3daee9;'>\{</span><span style='color:#ff5500;'>0</span><span style='color:#3daee9;'>\}</span><span style='color:#ff5500;'>$</span>;
  <span style='color:#644a9b;'>\item</span> <span style='color:#ff5500;'>$</span><span style='color:#3daee9;'>\norm</span><span style='color:#ff5500;'>{</span><span style='color:#3daee9;'>\mathbf</span><span style='color:#ff5500;'>{v}}=0 </span><span style='color:#3daee9;'>\Leftrightarrow</span><span style='color:#ff5500;'> </span><span style='color:#3daee9;'>\mathbf</span><span style='color:#ff5500;'>{v}=</span><span style='color:#3daee9;'>\mathbf</span><span style='color:#ff5500;'>{0}$</span>;
  <span style='color:#644a9b;'>\item</span> <span style='color:#ff5500;'>$</span><span style='color:#3daee9;'>\norm</span><span style='color:#ff5500;'>{</span><span style='color:#3daee9;'>\a</span><span style='color:#ff5500;'> </span><span style='color:#3daee9;'>\mathbf</span><span style='color:#ff5500;'>{v}}=</span><span style='color:#3daee9;'>\abs</span><span style='color:#ff5500;'>{</span><span style='color:#3daee9;'>\a</span><span style='color:#ff5500;'>}</span><span style='color:#3daee9;'>\norm</span><span style='color:#ff5500;'>{</span><span style='color:#3daee9;'>\mathbf</span><span style='color:#ff5500;'>{v}}$</span>, <span style='color:#ff5500;'>$</span><span style='color:#3daee9;'>\f</span><span style='color:#ff5500;'> </span><span style='color:#3daee9;'>\mathbf</span><span style='color:#ff5500;'>{v}</span><span style='color:#3daee9;'>\in</span><span style='color:#ff5500;'> V$</span> y <span style='color:#ff5500;'>$</span><span style='color:#3daee9;'>\f</span><span style='color:#ff5500;'> </span><span style='color:#3daee9;'>\a\in\C</span><span style='color:#ff5500;'>$</span>; y
  <span style='color:#644a9b;'>\item</span> <span style='color:#ff5500;'>$</span><span style='color:#3daee9;'>\norm</span><span style='color:#ff5500;'>{</span><span style='color:#3daee9;'>\mathbf</span><span style='color:#ff5500;'>{v}+</span><span style='color:#3daee9;'>\mathbf</span><span style='color:#ff5500;'>{u}}</span><span style='color:#3daee9;'>\leq</span><span style='color:#ff5500;'> </span><span style='color:#3daee9;'>\norm</span><span style='color:#ff5500;'>{</span><span style='color:#3daee9;'>\mathbf</span><span style='color:#ff5500;'>{v}}+</span><span style='color:#3daee9;'>\norm</span><span style='color:#ff5500;'>{</span><span style='color:#3daee9;'>\mathbf</span><span style='color:#ff5500;'>{u}}$</span>, <span style='color:#ff5500;'>$</span><span style='color:#3daee9;'>\f</span><span style='color:#ff5500;'> </span><span style='color:#3daee9;'>\mathbf</span><span style='color:#ff5500;'>{v}, </span><span style='color:#3daee9;'>\mathbf</span><span style='color:#ff5500;'>{u}</span><span style='color:#3daee9;'>\in</span><span style='color:#ff5500;'> V$</span>.
 <b>\end</b>{<b><span style='color:#0095ff;'>enumerate</span></b>} 
  Si <span style='color:#ff5500;'>$</span><span style='color:#3daee9;'>\norm</span><span style='color:#ff5500;'>{</span><span style='color:#3daee9;'>\cdot</span><span style='color:#ff5500;'>}$</span> es una norma en <span style='color:#ff5500;'>$V$</span>, entonces diremos que <span style='color:#ff5500;'>$(</span><span style='color:#3daee9;'>\norm</span><span style='color:#ff5500;'>{</span><span style='color:#3daee9;'>\cdot</span><span style='color:#ff5500;'>},V)$</span> es un <span style='color:#644a9b;'>\textbf</span>{espacio normado}.
<b>\end</b>{<b><span style='color:#0095ff;'>Def</span></b>}<span style='color:#644a9b;'>\-</span>

<b>\begin</b>{<b><span style='color:#0095ff;'>Teo</span></b>}
Sea <span style='color:#ff5500;'>$(</span><span style='color:#3daee9;'>\langle\cdot</span><span style='color:#ff5500;'>,</span><span style='color:#3daee9;'>\cdot\rangle</span><span style='color:#ff5500;'>,V)$</span> un espacio producto interno. Entonces <span style='color:#ff5500;'>$M(</span><span style='color:#3daee9;'>\mathbf</span><span style='color:#ff5500;'>{v})=</span><span style='color:#3daee9;'>\sqrt</span><span style='color:#ff5500;'>{</span><span style='color:#3daee9;'>\langle\mathbf</span><span style='color:#ff5500;'>{v},</span><span style='color:#3daee9;'>\mathbf</span><span style='color:#ff5500;'>{v}</span><span style='color:#3daee9;'>\rangle</span><span style='color:#ff5500;'>}$</span>
es una norma sobre <span style='color:#ff5500;'>$V$</span>.
<b>\end</b>{<b><span style='color:#0095ff;'>Teo</span></b>}
<b>\begin</b>{<b><span style='color:#0095ff;'>proof</span></b>}
 Las condiciones 1, 3 y 4 de la Definición <b>\ref</b>{<b><span style='color:#0095ff;'>normavect</span></b>} se siguen del Teorema <b>\ref</b>{<b><span style='color:#0095ff;'>prodintanorma</span></b>}, mientras que 
 la condición 2 de la misma definición del Lema <b>\ref</b>{<b><span style='color:#0095ff;'>lema_anterior</span></b>} y la Proposición <b>\ref</b>{<b><span style='color:#0095ff;'>propequiv0</span></b>}.<span style='color:#644a9b;'>\-</span>
 
<b>\end</b>{<b><span style='color:#0095ff;'>proof</span></b>}

<b>\begin</b>{<b><span style='color:#0095ff;'>Def</span></b>}[Norma inducida por producto interno]
 Sea <span style='color:#ff5500;'>$(</span><span style='color:#3daee9;'>\langle\cdot</span><span style='color:#ff5500;'>,</span><span style='color:#3daee9;'>\cdot\rangle</span><span style='color:#ff5500;'>,V)$</span> un espacio producto interno. Decimos que la  
 <span style='color:#644a9b;'>\textbf</span>{norma inducida por el producto interno} en <span style='color:#ff5500;'>$V$</span> es la función <span style='color:#ff5500;'>$</span><span style='color:#3daee9;'>\norm</span><span style='color:#ff5500;'>{</span><span style='color:#3daee9;'>\cdot</span><span style='color:#ff5500;'>}:V</span><span style='color:#3daee9;'>\to\R</span><span style='color:#ff5500;'>$</span> definida como
 <span style='color:#ff5500;'>$</span><span style='color:#3daee9;'>\norm</span><span style='color:#ff5500;'>{</span><span style='color:#3daee9;'>\mathbf</span><span style='color:#ff5500;'>{a}}=</span><span style='color:#3daee9;'>\sqrt</span><span style='color:#ff5500;'>{</span><span style='color:#3daee9;'>\langle\mathbf</span><span style='color:#ff5500;'>{a},</span><span style='color:#3daee9;'>\mathbf</span><span style='color:#ff5500;'>{a}</span><span style='color:#3daee9;'>\rangle</span><span style='color:#ff5500;'>}$</span>.
<b>\end</b>{<b><span style='color:#0095ff;'>Def</span></b>}<span style='color:#644a9b;'>\-</span>

 <b>\begin</b>{<b><span style='color:#0095ff;'>Def</span></b>}[Distancia]
 Sea <span style='color:#ff5500;'>$(</span><span style='color:#3daee9;'>\norm</span><span style='color:#ff5500;'>{</span><span style='color:#3daee9;'>\cdot</span><span style='color:#ff5500;'>},V)$</span> un espacio normado. Decimos que la  
 <span style='color:#644a9b;'>\textbf</span>{distancia inducida por una norma} en <span style='color:#ff5500;'>$V$</span> es la función <span style='color:#ff5500;'>$d(</span><span style='color:#3daee9;'>\cdot</span><span style='color:#ff5500;'>,</span><span style='color:#3daee9;'>\cdot</span><span style='color:#ff5500;'>):V</span><span style='color:#3daee9;'>\times</span><span style='color:#ff5500;'> V</span><span style='color:#3daee9;'>\to\R</span><span style='color:#ff5500;'>$</span> definida como
 <span style='color:#ff5500;'>$d(</span><span style='color:#3daee9;'>\mathbf</span><span style='color:#ff5500;'>{a},</span><span style='color:#3daee9;'>\mathbf</span><span style='color:#ff5500;'>{b})=</span><span style='color:#3daee9;'>\norm</span><span style='color:#ff5500;'>{</span><span style='color:#3daee9;'>\mathbf</span><span style='color:#ff5500;'>{a}-</span><span style='color:#3daee9;'>\mathbf</span><span style='color:#ff5500;'>{b}}$</span>.
<b>\end</b>{<b><span style='color:#0095ff;'>Def</span></b>}<span style='color:#644a9b;'>\-</span>

 <b>\begin</b>{<b><span style='color:#0095ff;'>Def</span></b>}
 Sea <span style='color:#ff5500;'>$A</span><span style='color:#3daee9;'>\in\R</span><span style='color:#ff5500;'>^{n}(</span><span style='color:#3daee9;'>\C</span><span style='color:#ff5500;'>)$</span>. Se dice que la función <span style='color:#ff5500;'>$tr(</span><span style='color:#3daee9;'>\cdot</span><span style='color:#ff5500;'>):</span><span style='color:#3daee9;'>\R</span><span style='color:#ff5500;'>^{n}(</span><span style='color:#3daee9;'>\C</span><span style='color:#ff5500;'>)</span><span style='color:#3daee9;'>\to\C</span><span style='color:#ff5500;'>$</span>  
 dada por <span style='color:#ff5500;'>$tr(A)=</span><span style='color:#3daee9;'>\sum</span><span style='color:#ff5500;'>_{i=1}^{n}A_{i,i}$</span> es la <span style='color:#644a9b;'>\textbf</span>{traza} de <span style='color:#ff5500;'>$A$</span>. 
<b>\end</b>{<b><span style='color:#0095ff;'>Def</span></b>}<span style='color:#644a9b;'>\-</span>

<b>\begin</b>{<b><span style='color:#0095ff;'>Prop</span></b>}<b>\label</b>{<b><span style='color:#0095ff;'>sumtraz</span></b>}
 Sean <span style='color:#ff5500;'>$A,B</span><span style='color:#3daee9;'>\in\R</span><span style='color:#ff5500;'>^{n}(</span><span style='color:#3daee9;'>\C</span><span style='color:#ff5500;'>)$</span>, <span style='color:#ff5500;'>$C</span><span style='color:#3daee9;'>\in\R</span><span style='color:#ff5500;'>^{n</span><span style='color:#3daee9;'>\times</span><span style='color:#ff5500;'> m}$</span>, <span style='color:#ff5500;'>$D</span><span style='color:#3daee9;'>\in\R</span><span style='color:#ff5500;'>^{m</span><span style='color:#3daee9;'>\times</span><span style='color:#ff5500;'> n}$</span> y <span style='color:#ff5500;'>$z</span><span style='color:#3daee9;'>\in\C</span><span style='color:#ff5500;'>$</span>. Entonces se cumplen las siguientes igualdades
 <span style='color:#ff5500;'>$tr(zA+B)=ztr(A)+tr(B)$</span> y <span style='color:#ff5500;'>$tr(CD)=tr(DC)$</span>
<b>\end</b>{<b><span style='color:#0095ff;'>Prop</span></b>}
<b>\begin</b>{<b><span style='color:#0095ff;'>proof</span></b>}
 <span style='color:#644a9b;'>\ies</span>
 tr(zA+B)&amp;=&amp;<span style='color:#644a9b;'>\sum</span>_{i=1}^{n}(zA+B)_{i,i}
 =<span style='color:#644a9b;'>\sum</span>_{i=1}^{n}[(zA)_{i,i}+B_{i,i}]
 =<span style='color:#644a9b;'>\sum</span>_{i=1}^{n}(zA)_{i,i}+<span style='color:#644a9b;'>\sum</span>_{i=1}^{n}B_{i,i}
 =<span style='color:#644a9b;'>\sum</span>_{i=1}^{n}zA_{i,i}+<span style='color:#644a9b;'>\sum</span>_{i=1}^{n}B_{i,i}<span style='color:#644a9b;'>\\</span>
 &amp;=&amp;z<span style='color:#644a9b;'>\sum</span>_{i=1}^{n}A_{i,i}+<span style='color:#644a9b;'>\sum</span>_{i=1}^{n}B_{i,i}
 =ztr(A)+tr(B)
<span style='color:#644a9b;'>\\</span>
 tr(CD)&amp;=&amp;<span style='color:#644a9b;'>\sum</span>_{i=1}^{n}(CD)_{i,i}=<span style='color:#644a9b;'>\sum</span>_{i=1}^{n}<span style='color:#644a9b;'>\sum</span>_{r=1}^{m}C_{i,r}D_{r,i}
 =<span style='color:#644a9b;'>\sum</span>_{r=1}^{m}<span style='color:#644a9b;'>\sum</span>_{i=1}^{n}D_{r,i}C_{i,r}= <span style='color:#644a9b;'>\sum</span>_{r=1}^{m}(DC)_{r,r}
=tr(DC)
 <span style='color:#644a9b;'>\fes</span>
 
<b>\end</b>{<b><span style='color:#0095ff;'>proof</span></b>}<span style='color:#644a9b;'>\-</span>
  
 <b>\begin</b>{<b><span style='color:#0095ff;'>Teo</span></b>}
 Sea <span style='color:#ff5500;'>$</span><span style='color:#3daee9;'>\R</span><span style='color:#ff5500;'>^{n</span><span style='color:#3daee9;'>\times</span><span style='color:#ff5500;'> m}(</span><span style='color:#3daee9;'>\C</span><span style='color:#ff5500;'>)$</span>. Entonces <span style='color:#ff5500;'>$</span><span style='color:#3daee9;'>\langle</span><span style='color:#ff5500;'> A,B</span><span style='color:#3daee9;'>\rangle</span><span style='color:#ff5500;'>=tr(AB^*)$</span>
 es un producto interno en <span style='color:#ff5500;'>$</span><span style='color:#3daee9;'>\R</span><span style='color:#ff5500;'>^{n</span><span style='color:#3daee9;'>\times</span><span style='color:#ff5500;'> m}(</span><span style='color:#3daee9;'>\mathbb</span><span style='color:#ff5500;'>{C})$</span>.
<b>\end</b>{<b><span style='color:#0095ff;'>Teo</span></b>}
 <b>\begin</b>{<b><span style='color:#0095ff;'>proof</span></b>}
  <b>\begin</b>{<b><span style='color:#0095ff;'>enumerate</span></b>}
   <span style='color:#644a9b;'>\item</span> Sea <span style='color:#ff5500;'>$A,B</span><span style='color:#3daee9;'>\in\R</span><span style='color:#ff5500;'>^{n</span><span style='color:#3daee9;'>\times</span><span style='color:#ff5500;'> m}(</span><span style='color:#3daee9;'>\C</span><span style='color:#ff5500;'>)$</span> y <span style='color:#ff5500;'>$z</span><span style='color:#3daee9;'>\in\C</span><span style='color:#ff5500;'>$</span>. Entonces por la Proposición <b>\ref</b>{<b><span style='color:#0095ff;'>sumtraz</span></b>}
   <span style='color:#644a9b;'>\ies</span>
   <span style='color:#644a9b;'>\langle</span> z (A+B),C<span style='color:#644a9b;'>\rangle</span>&amp;=&amp;tr((z(A+B))C^*)<span style='color:#644a9b;'>\\</span>
   &amp;=&amp;tr((z A)C^*+(z B)C^*)<span style='color:#644a9b;'>\\</span>
   &amp;=&amp;ztr(AC^*)+ztr(BC^*)<span style='color:#644a9b;'>\\</span>
   &amp;=&amp;z<span style='color:#644a9b;'>\langle</span> A,C<span style='color:#644a9b;'>\rangle</span> +z<span style='color:#644a9b;'>\langle</span> B,C<span style='color:#644a9b;'>\rangle</span>
   <span style='color:#644a9b;'>\fes</span>   
   Como se cumple para un <span style='color:#ff5500;'>$z</span><span style='color:#3daee9;'>\in\C</span><span style='color:#ff5500;'>$</span> arbitrario, tomando <span style='color:#ff5500;'>$z=1+0i=1$</span> se cumple la condición 1 de la 
   Definición <b>\ref</b>{<b><span style='color:#0095ff;'>prod_interno</span></b>}. Inclusive,  
   como <span style='color:#ff5500;'>$B$</span> es arbitrario, tomando <span style='color:#ff5500;'>$B=</span><span style='color:#3daee9;'>\mathbf</span><span style='color:#ff5500;'>{0}$</span> se cumple la condición 2 de la misma definición.<span style='color:#644a9b;'>\-</span>
   <span style='color:#644a9b;'>\item</span> Sean <span style='color:#ff5500;'>$A,B</span><span style='color:#3daee9;'>\in\R</span><span style='color:#ff5500;'>^{n</span><span style='color:#3daee9;'>\times</span><span style='color:#ff5500;'> m}(</span><span style='color:#3daee9;'>\C</span><span style='color:#ff5500;'>)$</span>, entonces por la Proposición <b>\ref</b>{<b><span style='color:#0095ff;'>sumultcomp</span></b>} y la Definición <b>\ref</b>{<b><span style='color:#0095ff;'>trconjadj</span></b>}
   <span style='color:#644a9b;'>\ies</span>
   <span style='color:#644a9b;'>\langle</span> A,B<span style='color:#644a9b;'>\rangle</span>&amp;=&amp;tr(AB^*)=<span style='color:#644a9b;'>\sum</span>_{i=1}^{n}(AB^*)_{i,i}=<span style='color:#644a9b;'>\sum</span>_{i=1}^{n}<span style='color:#644a9b;'>\sum</span>_{r=1}^{m}A_{i,r}<span style='color:#644a9b;'>\overline</span>{B_{i,r}}
   =<span style='color:#644a9b;'>\sum</span>_{i=1}^{n}<span style='color:#644a9b;'>\sum</span>_{r=1}^{m}<span style='color:#644a9b;'>\overline</span>{B_{i,r}}(<span style='color:#644a9b;'>\overline</span>{A^*_{r,i}})<span style='color:#644a9b;'>\\</span>
   &amp;=&amp;<span style='color:#644a9b;'>\sum</span>_{i=1}^{n}<span style='color:#644a9b;'>\sum</span>_{r=1}^{m}<span style='color:#644a9b;'>\overline</span>{B_{i,r}A^*_{r,i}}
   =<span style='color:#644a9b;'>\sum</span>_{i=1}^{n}<span style='color:#644a9b;'>\overline</span>{(BA^*)_{i,i}}
   =<span style='color:#644a9b;'>\overline</span>{<span style='color:#644a9b;'>\sum</span>_{i=1}^{n}(BA^*)_{i,i}}<span style='color:#644a9b;'>\\</span>
   &amp;=&amp;<span style='color:#644a9b;'>\overline</span>{<span style='color:#644a9b;'>\langle</span> B,A<span style='color:#644a9b;'>\rangle</span>}
   <span style='color:#644a9b;'>\fes</span>
   Lo que comprueba la condición 3 de la Definición <b>\ref</b>{<b><span style='color:#0095ff;'>prod_interno</span></b>}.
   <span style='color:#644a9b;'>\item</span> Sea <span style='color:#ff5500;'>$A</span><span style='color:#3daee9;'>\in\R</span><span style='color:#ff5500;'>^{n</span><span style='color:#3daee9;'>\times</span><span style='color:#ff5500;'> m}$</span> no nula. Entonces <span style='color:#ff5500;'>$A_{p,q}</span><span style='color:#3daee9;'>\neq</span><span style='color:#ff5500;'> 0$</span> para algún <span style='color:#ff5500;'>$(p,q)</span><span style='color:#3daee9;'>\in\N</span><span style='color:#ff5500;'>_n</span><span style='color:#3daee9;'>\times\N</span><span style='color:#ff5500;'>_m$</span>, que implica
   <span style='color:#644a9b;'>\ies</span>
   <span style='color:#644a9b;'>\langle</span> A,A<span style='color:#644a9b;'>\rangle</span>&amp;=&amp;tr(AA^*)=<span style='color:#644a9b;'>\sum</span>_{i=1}^{n}(AA^*)_{i,i}=<span style='color:#644a9b;'>\sum</span>_{i=1}^{n}<span style='color:#644a9b;'>\sum</span>_{r=1}^mA_{i,r}<span style='color:#644a9b;'>\overline</span>{A_{i,r}}  
   =<span style='color:#644a9b;'>\sum</span>_{i=1}^{n}<span style='color:#644a9b;'>\sum</span>_{r=1}^mRe(A_{i,r})^2+Im(A_{i,r})^2<span style='color:#644a9b;'>\\</span>
   &amp;<span style='color:#644a9b;'>\geq</span>&amp; Re(A_{p,q})^2+Im(A_{p,q})^2&gt;0
   <span style='color:#644a9b;'>\fes</span>
 <b>\end</b>{<b><span style='color:#0095ff;'>enumerate</span></b>}

 <b>\end</b>{<b><span style='color:#0095ff;'>proof</span></b>}<span style='color:#644a9b;'>\-</span>
 
  <b>\begin</b>{<b><span style='color:#0095ff;'>Def</span></b>}[Norma de Frobenius]<b>\label</b>{<b><span style='color:#0095ff;'>normafrob</span></b>}
 Decimos que la función <span style='color:#ff5500;'>$</span><span style='color:#3daee9;'>\norm</span><span style='color:#ff5500;'>{</span><span style='color:#3daee9;'>\cdot</span><span style='color:#ff5500;'>}_F:</span><span style='color:#3daee9;'>\R</span><span style='color:#ff5500;'>^{n</span><span style='color:#3daee9;'>\times</span><span style='color:#ff5500;'> m}(</span><span style='color:#3daee9;'>\C</span><span style='color:#ff5500;'>)</span><span style='color:#3daee9;'>\to\R</span><span style='color:#ff5500;'>$</span> dada por 
 <span style='color:#644a9b;'>\ies</span>
 <span style='color:#644a9b;'>\norm</span>{A}_F=<span style='color:#644a9b;'>\sqrt</span>{<span style='color:#644a9b;'>\langle</span> A,A<span style='color:#644a9b;'>\rangle</span>}=<span style='color:#644a9b;'>\sqrt</span>{tr(AA^*)}=<span style='color:#644a9b;'>\footnotemark\sqrt</span>{tr(A^*A)}
 <span style='color:#644a9b;'>\fes</span>
 <span style='color:#644a9b;'>\footnotetext</span>{Esta igual es dada por la Proposición <b>\ref</b>{<b><span style='color:#0095ff;'>sumtraz</span></b>}}
 es la <span style='color:#644a9b;'>\textbf</span>{norma de Frobenius} en <span style='color:#ff5500;'>$</span><span style='color:#3daee9;'>\R</span><span style='color:#ff5500;'>^{n</span><span style='color:#3daee9;'>\times</span><span style='color:#ff5500;'> m}$</span>.
<b>\end</b>{<b><span style='color:#0095ff;'>Def</span></b>}<span style='color:#644a9b;'>\-</span>
 
<b>\begin</b>{<b><span style='color:#0095ff;'>Def</span></b>}<b>\label</b>{<b><span style='color:#0095ff;'>defmatrices</span></b>}
Sea <span style='color:#ff5500;'>$A</span><span style='color:#3daee9;'>\in\R</span><span style='color:#ff5500;'>^{n}(</span><span style='color:#3daee9;'>\mathbb</span><span style='color:#ff5500;'>{C})$</span>. Entonces decimos que <span style='color:#ff5500;'>$A$</span> es  
<b>\begin</b>{<b><span style='color:#0095ff;'>enumerate</span></b>}
 <span style='color:#644a9b;'>\item</span> <span style='color:#644a9b;'>\textbf</span>{Simétrica} si <span style='color:#ff5500;'>$A_{i,j}</span><span style='color:#3daee9;'>\in\R</span><span style='color:#ff5500;'>$</span>, <span style='color:#ff5500;'>$</span><span style='color:#3daee9;'>\f</span><span style='color:#ff5500;'> (i,j)</span><span style='color:#3daee9;'>\in\N</span><span style='color:#ff5500;'>_{n}</span><span style='color:#3daee9;'>\times\N</span><span style='color:#ff5500;'>_{m}$</span>, y <span style='color:#ff5500;'>$A=A'$</span>.
 <span style='color:#644a9b;'>\item</span> <span style='color:#644a9b;'>\textbf</span>{Hermitiana} si <span style='color:#ff5500;'>$A=A^*$</span>.
 <span style='color:#644a9b;'>\item</span> <span style='color:#644a9b;'>\textbf</span>{Ortogonal} si <span style='color:#ff5500;'>$A_{i,j}</span><span style='color:#3daee9;'>\in\R</span><span style='color:#ff5500;'>$</span>, <span style='color:#ff5500;'>$</span><span style='color:#3daee9;'>\f</span><span style='color:#ff5500;'> (i,j)</span><span style='color:#3daee9;'>\in\N</span><span style='color:#ff5500;'>_{n}</span><span style='color:#3daee9;'>\times\N</span><span style='color:#ff5500;'>_{m}$</span>, y <span style='color:#ff5500;'>$AA'=A'A=I_n$</span>.
 <span style='color:#644a9b;'>\item</span> <span style='color:#644a9b;'>\textbf</span>{Normal} si <span style='color:#ff5500;'>$A^*A=AA^*$</span>.
 <span style='color:#644a9b;'>\item</span> <span style='color:#644a9b;'>\textbf</span>{Unitaria} si <span style='color:#ff5500;'>$A^*A=AA^*=I_n$</span>.
 <span style='color:#644a9b;'>\item</span> <span style='color:#644a9b;'>\textbf</span>{Regular} si <span style='color:#ff5500;'>$</span><span style='color:#3daee9;'>\exists</span><span style='color:#ff5500;'> B</span><span style='color:#3daee9;'>\in\R</span><span style='color:#ff5500;'>^{n}(</span><span style='color:#3daee9;'>\mathbb</span><span style='color:#ff5500;'>{C})$</span> tal que <span style='color:#ff5500;'>$AB=BA=I_n$</span>, donde a <span style='color:#ff5500;'>$B$</span> se le denomina <span style='color:#644a9b;'>\textbf</span>{inversa} de <span style='color:#ff5500;'>$A$</span>.
 <span style='color:#644a9b;'>\item</span> <span style='color:#644a9b;'>\textbf</span>{Semidefinida positiva} si <span style='color:#ff5500;'>$</span><span style='color:#3daee9;'>\mathbf</span><span style='color:#ff5500;'>{x}^*A</span><span style='color:#3daee9;'>\mathbf</span><span style='color:#ff5500;'>{x}</span><span style='color:#3daee9;'>\geq</span><span style='color:#ff5500;'>0$</span>, <span style='color:#ff5500;'>$</span><span style='color:#3daee9;'>\f</span><span style='color:#ff5500;'> </span><span style='color:#3daee9;'>\mathbf</span><span style='color:#ff5500;'>{x}</span><span style='color:#3daee9;'>\in\C</span><span style='color:#ff5500;'>^n$</span>.
 <b>\end</b>{<b><span style='color:#0095ff;'>enumerate</span></b>}
<b>\end</b>{<b><span style='color:#0095ff;'>Def</span></b>}<span style='color:#644a9b;'>\-</span>

<b>\begin</b>{<b><span style='color:#0095ff;'>Def</span></b>}[Valor propio]
Sea <span style='color:#ff5500;'>$V$</span> un espacio vectorial sobre el campo <span style='color:#ff5500;'>$</span><span style='color:#3daee9;'>\mathbb</span><span style='color:#ff5500;'>{C}$</span>. Se dice que <span style='color:#ff5500;'>$</span><span style='color:#3daee9;'>\lambda\in\mathbb</span><span style='color:#ff5500;'>{C}$</span> es un <span style='color:#644a9b;'>\textbf</span>{valor propio} de 
<span style='color:#ff5500;'>$A</span><span style='color:#3daee9;'>\in\R</span><span style='color:#ff5500;'>^{n}(</span><span style='color:#3daee9;'>\mathbb</span><span style='color:#ff5500;'>{C})$</span> si <span style='color:#ff5500;'>$</span><span style='color:#3daee9;'>\exists</span><span style='color:#ff5500;'> </span><span style='color:#3daee9;'>\bar</span><span style='color:#ff5500;'>{v}</span><span style='color:#3daee9;'>\in</span><span style='color:#ff5500;'> V$</span>, <span style='color:#ff5500;'>$</span><span style='color:#3daee9;'>\bar</span><span style='color:#ff5500;'>{v}</span><span style='color:#3daee9;'>\neq\bar</span><span style='color:#ff5500;'>{0}$</span>, tal que <span style='color:#ff5500;'>$A</span><span style='color:#3daee9;'>\bar</span><span style='color:#ff5500;'>{v}=</span><span style='color:#3daee9;'>\lambda\bar</span><span style='color:#ff5500;'>{v}$</span>. En este caso, 
el vector <span style='color:#ff5500;'>$</span><span style='color:#3daee9;'>\bar</span><span style='color:#ff5500;'>{v}$</span> asociado a <span style='color:#ff5500;'>$</span><span style='color:#3daee9;'>\lambda</span><span style='color:#ff5500;'>$</span> se conoce como <span style='color:#644a9b;'>\textbf</span>{vector propio}.<span style='color:#644a9b;'>\-</span>

Además, definimos a <span style='color:#ff5500;'>$p(</span><span style='color:#3daee9;'>\lambda</span><span style='color:#ff5500;'>)=det(A-</span><span style='color:#3daee9;'>\lambda</span><span style='color:#ff5500;'> I_n)$</span> como el <span style='color:#644a9b;'>\textbf</span>{polinomio característico} de <span style='color:#ff5500;'>$A$</span>. Así mismo, decimos que la 
<span style='color:#644a9b;'>\textbf</span>{multiplicidad algebraica} de <span style='color:#ff5500;'>$</span><span style='color:#3daee9;'>\lambda</span><span style='color:#ff5500;'>$</span>, denotada como <span style='color:#ff5500;'>$m_a(</span><span style='color:#3daee9;'>\lambda</span><span style='color:#ff5500;'>)$</span>, es la multiplicidad de <span style='color:#ff5500;'>$</span><span style='color:#3daee9;'>\lambda</span><span style='color:#ff5500;'>$</span> como raíz de 
<span style='color:#ff5500;'>$p(</span><span style='color:#3daee9;'>\lambda</span><span style='color:#ff5500;'>)$</span>.
<b>\end</b>{<b><span style='color:#0095ff;'>Def</span></b>}<span style='color:#644a9b;'>\-</span>
 
<b>\begin</b>{<b><span style='color:#0095ff;'>Teo</span></b>}[Shur]<b>\label</b>{<b><span style='color:#0095ff;'>shur</span></b>}
 Sea <span style='color:#ff5500;'>$A</span><span style='color:#3daee9;'>\in\R</span><span style='color:#ff5500;'>^{n}(</span><span style='color:#3daee9;'>\C</span><span style='color:#ff5500;'>)$</span>. Si <span style='color:#ff5500;'>$</span><span style='color:#3daee9;'>\lambda</span><span style='color:#ff5500;'>_1,</span><span style='color:#3daee9;'>\dots</span><span style='color:#ff5500;'>,</span><span style='color:#3daee9;'>\lambda</span><span style='color:#ff5500;'>_n$</span> son los valores propios de <span style='color:#ff5500;'>$A$</span> ordenados de forma arbitraria y considerando 
 su multiplicidad algebraica, entonces 
 existen <span style='color:#ff5500;'>$U</span><span style='color:#3daee9;'>\in\R</span><span style='color:#ff5500;'>^{n}$</span> unitaria y <span style='color:#ff5500;'>$</span><span style='color:#3daee9;'>\S\in\R</span><span style='color:#ff5500;'>^{n}$</span> triangular superior tales que <span style='color:#ff5500;'>$A=U</span><span style='color:#3daee9;'>\S</span><span style='color:#ff5500;'> U^*$</span> y <span style='color:#ff5500;'>$</span><span style='color:#3daee9;'>\S</span><span style='color:#ff5500;'>_{i,i}=</span><span style='color:#3daee9;'>\lambda</span><span style='color:#ff5500;'>_i$</span>, <span style='color:#ff5500;'>$</span><span style='color:#3daee9;'>\f</span><span style='color:#ff5500;'> i</span><span style='color:#3daee9;'>\in\N</span><span style='color:#ff5500;'>_n$</span>.
<b>\end</b>{<b><span style='color:#0095ff;'>Teo</span></b>}
<b>\begin</b>{<b><span style='color:#0095ff;'>proof</span></b>}
 Véase <b>\citet</b>[p.15]{<b><span style='color:#0095ff;'>stoj</span></b>}
<b>\end</b>{<b><span style='color:#0095ff;'>proof</span></b>}

<b>\begin</b>{<b><span style='color:#0095ff;'>Cor</span></b>}<b>\label</b>{<b><span style='color:#0095ff;'>eqfrob</span></b>}
 Sea <span style='color:#ff5500;'>$A</span><span style='color:#3daee9;'>\in\R</span><span style='color:#ff5500;'>^{n}(</span><span style='color:#3daee9;'>\C</span><span style='color:#ff5500;'>)$</span> con valores propios <span style='color:#ff5500;'>$</span><span style='color:#3daee9;'>\lambda</span><span style='color:#ff5500;'>_1,</span><span style='color:#3daee9;'>\dots</span><span style='color:#ff5500;'>,</span><span style='color:#3daee9;'>\lambda</span><span style='color:#ff5500;'>_n$</span> en un orden arbitrario y considerando su multiplicidad 
 algebraica. Entonces 
 <span style='color:#644a9b;'>\ies</span>
 tr(A)=<span style='color:#644a9b;'>\sum</span>_{i=1}^{n}<span style='color:#644a9b;'>\lambda</span>_i
 <span style='color:#644a9b;'>\fes</span>
<b>\end</b>{<b><span style='color:#0095ff;'>Cor</span></b>}
<b>\begin</b>{<b><span style='color:#0095ff;'>proof</span></b>} Por el Teorema <b>\ref</b>{<b><span style='color:#0095ff;'>shur</span></b>}, existen <span style='color:#ff5500;'>$U</span><span style='color:#3daee9;'>\in\R</span><span style='color:#ff5500;'>^{n}$</span> unitaria y <span style='color:#ff5500;'>$</span><span style='color:#3daee9;'>\S\in\R</span><span style='color:#ff5500;'>^{n}$</span> 
triangular superior tales que <span style='color:#ff5500;'>$A=U</span><span style='color:#3daee9;'>\S</span><span style='color:#ff5500;'> U^*$</span> y <span style='color:#ff5500;'>$</span><span style='color:#3daee9;'>\S</span><span style='color:#ff5500;'>_{i,i}=</span><span style='color:#3daee9;'>\lambda</span><span style='color:#ff5500;'>_i$</span>, <span style='color:#ff5500;'>$</span><span style='color:#3daee9;'>\f</span><span style='color:#ff5500;'> i</span><span style='color:#3daee9;'>\in\N</span><span style='color:#ff5500;'>_n$</span>. Luego, por la Proposición <b>\ref</b>{<b><span style='color:#0095ff;'>sumtraz</span></b>} y la Definición 
<b>\ref</b>{<b><span style='color:#0095ff;'>defmatrices</span></b>} en su inciso 5 sobre la matriz unitaria,
<span style='color:#644a9b;'>\ies</span>
tr(A)=tr(U<span style='color:#644a9b;'>\S</span> U^*)=tr(<span style='color:#644a9b;'>\S</span> U^*U)=tr(<span style='color:#644a9b;'>\S</span> I_n)=tr(<span style='color:#644a9b;'>\S</span>)=<span style='color:#644a9b;'>\sum</span>_{i=1}^{n}<span style='color:#644a9b;'>\S</span>_{i,i}=<span style='color:#644a9b;'>\sum</span>_{i=1}^{n}<span style='color:#644a9b;'>\lambda</span>_i
<span style='color:#644a9b;'>\fes</span>

<b>\end</b>{<b><span style='color:#0095ff;'>proof</span></b>}<span style='color:#644a9b;'>\-</span>
 
<b>\begin</b>{<b><span style='color:#0095ff;'>Lem</span></b>}<b>\label</b>{<b><span style='color:#0095ff;'>madjmtra</span></b>}
Sea <span style='color:#ff5500;'>$A</span><span style='color:#3daee9;'>\in\R</span><span style='color:#ff5500;'>^{n}(</span><span style='color:#3daee9;'>\mathbb</span><span style='color:#ff5500;'>{C})$</span>. Si <span style='color:#ff5500;'>$A_{i,j}</span><span style='color:#3daee9;'>\in\R</span><span style='color:#ff5500;'>$</span>, <span style='color:#ff5500;'>$</span><span style='color:#3daee9;'>\f</span><span style='color:#ff5500;'> i,j</span><span style='color:#3daee9;'>\in\N</span><span style='color:#ff5500;'>_{n}$</span>, entonces <span style='color:#ff5500;'>$A^*=A'$</span>. 
<b>\end</b>{<b><span style='color:#0095ff;'>Lem</span></b>}
<b>\begin</b>{<b><span style='color:#0095ff;'>proof</span></b>}
 Debido a que <span style='color:#ff5500;'>$A_{i,j}</span><span style='color:#3daee9;'>\in\R</span><span style='color:#ff5500;'>$</span>, entonces <span style='color:#ff5500;'>$A_{i,j}=Re(A_{i,j})=Re(A_{i,j})+0*Im(A_{i,j})$</span>.
Por consiguiente, <span style='color:#ff5500;'>$</span><span style='color:#3daee9;'>\overline</span><span style='color:#ff5500;'>{A_{i,j}}=Re(A_{i,j})-0*Im(A_{i,j})=Re(A_{i,j})=A_{i,j}$</span>. 
De lo cual se sigue que <span style='color:#ff5500;'>$A^*_{i,j}=</span><span style='color:#3daee9;'>\overline</span><span style='color:#ff5500;'>{A'_{i,j}}=</span><span style='color:#3daee9;'>\overline</span><span style='color:#ff5500;'>{A_{j,i}}=A_{j,i}$</span>, <span style='color:#ff5500;'>$</span><span style='color:#3daee9;'>\f</span><span style='color:#ff5500;'> i,j</span><span style='color:#3daee9;'>\in\N</span><span style='color:#ff5500;'>_{n}$</span>.<span style='color:#644a9b;'>\-</span>

<span style='color:#644a9b;'>\t</span> Como lo anterior se cumple para todo <span style='color:#ff5500;'>$i,j</span><span style='color:#3daee9;'>\in\N</span><span style='color:#ff5500;'>_{n}$</span>, <span style='color:#ff5500;'>$A^*=A'$</span>.<span style='color:#644a9b;'>\-</span>
 
<b>\end</b>{<b><span style='color:#0095ff;'>proof</span></b>}


<b>\begin</b>{<b><span style='color:#0095ff;'>Lem</span></b>}<b>\label</b>{<b><span style='color:#0095ff;'>matnormal</span></b>}
Sea <span style='color:#ff5500;'>$A</span><span style='color:#3daee9;'>\in\R</span><span style='color:#ff5500;'>^{n}(</span><span style='color:#3daee9;'>\mathbb</span><span style='color:#ff5500;'>{C})$</span>. Si <span style='color:#ff5500;'>$A_{i,j}</span><span style='color:#3daee9;'>\in\R</span><span style='color:#ff5500;'>$</span>, <span style='color:#ff5500;'>$</span><span style='color:#3daee9;'>\f</span><span style='color:#ff5500;'> i,j</span><span style='color:#3daee9;'>\in\N</span><span style='color:#ff5500;'>_{n}$</span> y <span style='color:#ff5500;'>$A'A=AA'$</span>, entonces <span style='color:#ff5500;'>$A$</span> es normal. 
<b>\end</b>{<b><span style='color:#0095ff;'>Lem</span></b>}
<b>\begin</b>{<b><span style='color:#0095ff;'>proof</span></b>} Como <span style='color:#ff5500;'>$A_{i,j}</span><span style='color:#3daee9;'>\in\R</span><span style='color:#ff5500;'>$</span>, <span style='color:#ff5500;'>$</span><span style='color:#3daee9;'>\f</span><span style='color:#ff5500;'> (i,j)</span><span style='color:#3daee9;'>\in\N</span><span style='color:#ff5500;'>_{n}</span><span style='color:#3daee9;'>\times\N</span><span style='color:#ff5500;'>_{m}$</span>, por el Lema <b>\ref</b>{<b><span style='color:#0095ff;'>madjmtra</span></b>}, <span style='color:#ff5500;'>$A^*=A'$</span>. Luego, como 
 <span style='color:#ff5500;'>$A'A=AA'$</span>, entonces <span style='color:#ff5500;'>$A^*A=AA^*$</span>. <span style='color:#644a9b;'>\-</span>
 
 <span style='color:#644a9b;'>\t</span> Por la Definición <b>\ref</b>{<b><span style='color:#0095ff;'>defmatrices</span></b>} (4), <span style='color:#ff5500;'>$A$</span> es una matriz normal.
 
<b>\end</b>{<b><span style='color:#0095ff;'>proof</span></b>}

<b>\begin</b>{<b><span style='color:#0095ff;'>Teo</span></b>}<b>\label</b>{<b><span style='color:#0095ff;'>atahermysempos</span></b>}
Sea <span style='color:#ff5500;'>$A</span><span style='color:#3daee9;'>\in\R</span><span style='color:#ff5500;'>^{n</span><span style='color:#3daee9;'>\times</span><span style='color:#ff5500;'> m}(</span><span style='color:#3daee9;'>\mathbb</span><span style='color:#ff5500;'>{C})$</span> no nula. Si <span style='color:#ff5500;'>$A_{i,j}</span><span style='color:#3daee9;'>\in\R</span><span style='color:#ff5500;'>$</span>, <span style='color:#ff5500;'>$</span><span style='color:#3daee9;'>\f</span><span style='color:#ff5500;'> (i,j)</span><span style='color:#3daee9;'>\in\N</span><span style='color:#ff5500;'>_n</span><span style='color:#3daee9;'>\times\N</span><span style='color:#ff5500;'>_m$</span>, entonces <span style='color:#ff5500;'>$A'A$</span> es una matriz simétrica, hermitiana, normal y 
semidefinida positiva.
<b>\end</b>{<b><span style='color:#0095ff;'>Teo</span></b>}
<b>\begin</b>{<b><span style='color:#0095ff;'>proof</span></b>} A causa de la Proposición <b>\ref</b>{<b><span style='color:#0095ff;'>propmatt</span></b>}, <span style='color:#ff5500;'>$(A'A)'=A'(A')'=A'A$</span>, por lo cual <span style='color:#ff5500;'>$A'A$</span> es simétrica. Además, como
<span style='color:#ff5500;'>$(A'A)_{i,j}=</span><span style='color:#3daee9;'>\sum</span><span style='color:#ff5500;'>_{r=1}^{n}A'_{i,r}A_{r,j}=</span><span style='color:#3daee9;'>\sum</span><span style='color:#ff5500;'>_{r=1}^{n}A_{r,i}A_{r,j}</span><span style='color:#3daee9;'>\in\R</span><span style='color:#ff5500;'>$</span>, <span style='color:#ff5500;'>$</span><span style='color:#3daee9;'>\f</span><span style='color:#ff5500;'> (i,j)</span><span style='color:#3daee9;'>\in\N</span><span style='color:#ff5500;'>_{n}</span><span style='color:#3daee9;'>\times\N</span><span style='color:#ff5500;'>_{m}$</span> y del 
Lema <b>\ref</b>{<b><span style='color:#0095ff;'>madjmtra</span></b>}, entonces  <span style='color:#ff5500;'>$(A'A)^*=(A'A)'$</span>, por consiguiente <span style='color:#ff5500;'>$A'A$</span> es hermitiana.<span style='color:#644a9b;'>\-</span>

Incluso, por el Lema <b>\ref</b>{<b><span style='color:#0095ff;'>madjmtra</span></b>} y la Proposición <b>\ref</b>{<b><span style='color:#0095ff;'>propmatt</span></b>} 
<span style='color:#644a9b;'>\ies</span>
(A'A)'(A'A)&amp;=&amp;(A'(A')')(A'A)<span style='color:#644a9b;'>\\</span>
&amp;=&amp;(A'A)(A'A)<span style='color:#644a9b;'>\\</span>
&amp;=&amp;(A'A)(A'(A')')<span style='color:#644a9b;'>\\</span>
&amp;=&amp;(A'A)(A'A)'
<span style='color:#644a9b;'>\fes</span>
y debido al Lema <b>\ref</b>{<b><span style='color:#0095ff;'>matnormal</span></b>}, <span style='color:#ff5500;'>$A'A$</span> es una matriz normal. Luego, por la Propiedad <b>\ref</b>{<b><span style='color:#0095ff;'>sumconj</span></b>},
que <span style='color:#ff5500;'>$A$</span> es no nula con <span style='color:#ff5500;'>$A_{u,v}=Re(A_{u,v})+0i$</span>, y que para <span style='color:#ff5500;'>$z</span><span style='color:#3daee9;'>\in\C</span><span style='color:#ff5500;'>$</span> no nulo, <span style='color:#ff5500;'>$</span><span style='color:#3daee9;'>\overline</span><span style='color:#ff5500;'>{z}z&gt;0$</span><span style='color:#644a9b;'>\footnote</span>{Sea <span style='color:#ff5500;'>$z</span><span style='color:#3daee9;'>\in\C</span><span style='color:#ff5500;'>$</span> no nulo. Entonces 
<span style='color:#ff5500;'>$</span><span style='color:#3daee9;'>\overline</span><span style='color:#ff5500;'>{z}z=(a+bi)(a-bi)=a^2-abi+abi+b^2=a^2+b^2&gt;0$</span>}, entonces 
<span style='color:#644a9b;'>\ies</span>
<span style='color:#644a9b;'>\mathbf</span>{x}^*(A'A)<span style='color:#644a9b;'>\mathbf</span>{x}
&amp;=&amp;(A<span style='color:#644a9b;'>\overline</span>{<span style='color:#644a9b;'>\mathbf</span>{x}})'A<span style='color:#644a9b;'>\mathbf</span>{x}<span style='color:#644a9b;'>\\</span>
&amp;=&amp;(<span style='color:#644a9b;'>\overline</span>{A}<span style='color:#644a9b;'>\overline</span>{<span style='color:#644a9b;'>\mathbf</span>{x}})'A<span style='color:#644a9b;'>\mathbf</span>{x}<span style='color:#644a9b;'>\\</span>
&amp;=&amp;<span style='color:#644a9b;'>\sum</span>_{i=1}^n<span style='color:#644a9b;'>\left</span>(<span style='color:#644a9b;'>\sum</span>_{j=1}^m<span style='color:#644a9b;'>\overline</span>{A_{i,j}}<span style='color:#644a9b;'>\overline</span>{x_j}<span style='color:#644a9b;'>\right</span>)<span style='color:#644a9b;'>\left</span>(<span style='color:#644a9b;'>\sum</span>_{j=1}^mA_{i,j}x_j<span style='color:#644a9b;'>\right</span>) <span style='color:#644a9b;'>\\</span>
&amp;=&amp;<span style='color:#644a9b;'>\sum</span>_{i=1}^n<span style='color:#644a9b;'>\left</span>(<span style='color:#644a9b;'>\sum</span>_{j=1}^m<span style='color:#644a9b;'>\overline</span>{A_{i,j}x_j}<span style='color:#644a9b;'>\right</span>)<span style='color:#644a9b;'>\left</span>(<span style='color:#644a9b;'>\sum</span>_{j=1}^mA_{i,j}x_j<span style='color:#644a9b;'>\right</span>) <span style='color:#644a9b;'>\\</span>
&amp;=&amp;<span style='color:#644a9b;'>\sum</span>_{i=1}^n<span style='color:#644a9b;'>\left</span>(<span style='color:#644a9b;'>\overline</span>{<span style='color:#644a9b;'>\sum</span>_{j=1}^mA_{i,j}x_j}<span style='color:#644a9b;'>\right</span>)<span style='color:#644a9b;'>\left</span>(<span style='color:#644a9b;'>\sum</span>_{j=1}^mA_{i,j}x_j<span style='color:#644a9b;'>\right</span>) <span style='color:#644a9b;'>\\</span>
&amp;<span style='color:#644a9b;'>\geq</span>&amp; 0
<span style='color:#644a9b;'>\fes</span>
<span style='color:#644a9b;'>\t</span> <span style='color:#ff5500;'>$A'A$</span> es semipositiva definida.

<b>\end</b>{<b><span style='color:#0095ff;'>proof</span></b>}<span style='color:#644a9b;'>\-</span>

 <b>\begin</b>{<b><span style='color:#0095ff;'>Teo</span></b>}<b>\label</b>{<b><span style='color:#0095ff;'>hermysemidefpos</span></b>}
 Sea <span style='color:#ff5500;'>$A</span><span style='color:#3daee9;'>\in\R</span><span style='color:#ff5500;'>^{n}(</span><span style='color:#3daee9;'>\C</span><span style='color:#ff5500;'>)$</span>. Si <span style='color:#ff5500;'>$A$</span> es hermitiana y semidefinida positiva, entonces sus valores propios son todos reales no negativos.
 <b>\end</b>{<b><span style='color:#0095ff;'>Teo</span></b>}
 <b>\begin</b>{<b><span style='color:#0095ff;'>proof</span></b>}
  Véase <b>\citet</b>[pp.29 y 39]{<b><span style='color:#0095ff;'>stoj</span></b>}.<span style='color:#644a9b;'>\-</span>
  
 <b>\end</b>{<b><span style='color:#0095ff;'>proof</span></b>}<span style='color:#644a9b;'>\-</span>

 <b>\begin</b>{<b><span style='color:#0095ff;'>Def</span></b>}[Valor singular]<b>\label</b>{<b><span style='color:#0095ff;'>valorsingular</span></b>}
Sea <span style='color:#ff5500;'>$A</span><span style='color:#3daee9;'>\in\R</span><span style='color:#ff5500;'>^{n</span><span style='color:#3daee9;'>\times</span><span style='color:#ff5500;'> m}(</span><span style='color:#3daee9;'>\R</span><span style='color:#ff5500;'>)$</span>. Decimos que los <span style='color:#644a9b;'>\textbf</span>{valores singulares} de <span style='color:#ff5500;'>$A$</span> son las raíces cuadradas de los valores 
propios de <span style='color:#ff5500;'>$A'A$</span> y los designamos mediante <span style='color:#ff5500;'>$</span><span style='color:#3daee9;'>\sigma</span><span style='color:#ff5500;'>_1,</span><span style='color:#3daee9;'>\cdots</span><span style='color:#ff5500;'>,</span><span style='color:#3daee9;'>\sigma</span><span style='color:#ff5500;'>_m$</span>, donde el orden es tal que 
<span style='color:#ff5500;'>$</span><span style='color:#3daee9;'>\sigma</span><span style='color:#ff5500;'>_1</span><span style='color:#3daee9;'>\geq\cdots\geq\sigma</span><span style='color:#ff5500;'>_m$</span>.
<b>\end</b>{<b><span style='color:#0095ff;'>Def</span></b>}<span style='color:#644a9b;'>\-</span>

<b>\begin</b>{<b><span style='color:#0095ff;'>Teo</span></b>}<b>\label</b>{<b><span style='color:#0095ff;'>vsfrob</span></b>}
 Sea <span style='color:#ff5500;'>$A</span><span style='color:#3daee9;'>\in\R</span><span style='color:#ff5500;'>^{n</span><span style='color:#3daee9;'>\times</span><span style='color:#ff5500;'> m}(</span><span style='color:#3daee9;'>\R</span><span style='color:#ff5500;'>)$</span>. Si <span style='color:#ff5500;'>$</span><span style='color:#3daee9;'>\sigma</span><span style='color:#ff5500;'>_1,</span><span style='color:#3daee9;'>\dots</span><span style='color:#ff5500;'>,</span><span style='color:#3daee9;'>\sigma</span><span style='color:#ff5500;'>_m$</span> son los 
 valores singulares de <span style='color:#ff5500;'>$A$</span>, entonces  
 <span style='color:#644a9b;'>\ies</span>
 <span style='color:#644a9b;'>\norm</span>{A}_F=<span style='color:#644a9b;'>\sqrt</span>{<span style='color:#644a9b;'>\sum</span>_{i=1}^m<span style='color:#644a9b;'>\sigma</span>^2_i}
 <span style='color:#644a9b;'>\fes</span>
 <b>\end</b>{<b><span style='color:#0095ff;'>Teo</span></b>}
 <b>\begin</b>{<b><span style='color:#0095ff;'>proof</span></b>}
  La función <span style='color:#ff5500;'>$</span><span style='color:#3daee9;'>\sqrt</span><span style='color:#ff5500;'>{</span><span style='color:#3daee9;'>\sum</span><span style='color:#ff5500;'>_{i=1}^m</span><span style='color:#3daee9;'>\sigma</span><span style='color:#ff5500;'>^2_i}$</span> está bien definida por los teoremas <b>\ref</b>{<b><span style='color:#0095ff;'>atahermysempos</span></b>} y <b>\ref</b>{<b><span style='color:#0095ff;'>hermysemidefpos</span></b>}, y la 
  Definición <b>\ref</b>{<b><span style='color:#0095ff;'>valorsingular</span></b>}. 
  El resto de la demostración es inmediata a partir del Corolario <b>\ref</b>{<b><span style='color:#0095ff;'>eqfrob</span></b>} y la Definición <b>\ref</b>{<b><span style='color:#0095ff;'>normafrob</span></b>}.<span style='color:#644a9b;'>\-</span>
  
 <b>\end</b>{<b><span style='color:#0095ff;'>proof</span></b>}<span style='color:#644a9b;'>\-</span>

  <b>\begin</b>{<b><span style='color:#0095ff;'>Def</span></b>}[Distancia de Frobenius]<b>\label</b>{<b><span style='color:#0095ff;'>disfrob</span></b>}
Sean <span style='color:#ff5500;'>$A, B</span><span style='color:#3daee9;'>\in\R</span><span style='color:#ff5500;'>^{n</span><span style='color:#3daee9;'>\times</span><span style='color:#ff5500;'> m}(</span><span style='color:#3daee9;'>\R</span><span style='color:#ff5500;'>)$</span>. Decimos que la función <span style='color:#ff5500;'>$d_F(</span><span style='color:#3daee9;'>\cdot</span><span style='color:#ff5500;'>,</span><span style='color:#3daee9;'>\cdot</span><span style='color:#ff5500;'>):</span><span style='color:#3daee9;'>\R</span><span style='color:#ff5500;'>^{n</span><span style='color:#3daee9;'>\times</span><span style='color:#ff5500;'> m}(</span><span style='color:#3daee9;'>\R</span><span style='color:#ff5500;'>)</span><span style='color:#3daee9;'>\times\R</span><span style='color:#ff5500;'>^{n</span><span style='color:#3daee9;'>\times</span><span style='color:#ff5500;'> m}(</span><span style='color:#3daee9;'>\R</span><span style='color:#ff5500;'>)</span><span style='color:#3daee9;'>\to\R</span><span style='color:#ff5500;'>$</span> 
dada por <span style='color:#ff5500;'>$d_F(A,B)=</span><span style='color:#3daee9;'>\norm</span><span style='color:#ff5500;'>{A-B}_F$</span> es la <span style='color:#644a9b;'>\textbf</span>{distancia} en el espacio <span style='color:#ff5500;'>$</span><span style='color:#3daee9;'>\R</span><span style='color:#ff5500;'>^{n</span><span style='color:#3daee9;'>\times</span><span style='color:#ff5500;'> m}(</span><span style='color:#3daee9;'>\R</span><span style='color:#ff5500;'>)$</span>.
<b>\end</b>{<b><span style='color:#0095ff;'>Def</span></b>}<span style='color:#644a9b;'>\-</span>


 <b>\begin</b>{<b><span style='color:#0095ff;'>Teo</span></b>}[Descomposición en Valores Singulares]<b>\label</b>{<b><span style='color:#0095ff;'>svdt</span></b>}
 Sea <span style='color:#ff5500;'>$A</span><span style='color:#3daee9;'>\in\R</span><span style='color:#ff5500;'>^{n</span><span style='color:#3daee9;'>\times</span><span style='color:#ff5500;'> m}(</span><span style='color:#3daee9;'>\R</span><span style='color:#ff5500;'>)$</span> arbitraria con rango <span style='color:#ff5500;'>$r$</span> y sean <span style='color:#ff5500;'>$</span><span style='color:#3daee9;'>\sigma</span><span style='color:#ff5500;'>_1, </span><span style='color:#3daee9;'>\dots</span><span style='color:#ff5500;'>, </span><span style='color:#3daee9;'>\sigma</span><span style='color:#ff5500;'>_r$</span> sus valores singulares ordenados de mayor a menor. Entonces existen las matrices ortogonales <span style='color:#ff5500;'>$U</span><span style='color:#3daee9;'>\in\R</span><span style='color:#ff5500;'>^{n}$</span> y <span style='color:#ff5500;'>$V</span><span style='color:#3daee9;'>\in\R</span><span style='color:#ff5500;'>^{m}$</span>, y una matriz diagonal <span style='color:#ff5500;'>$</span><span style='color:#3daee9;'>\S</span><span style='color:#ff5500;'>$</span> tales que
 <b>\begin</b>{<b><span style='color:#0095ff;'>enumerate</span></b>}[i.]
  <span style='color:#644a9b;'>\item</span> <span style='color:#ff5500;'>$(</span><span style='color:#3daee9;'>\S</span><span style='color:#ff5500;'>)_{ii}=</span><span style='color:#3daee9;'>\sigma</span><span style='color:#ff5500;'>_i$</span>, si <span style='color:#ff5500;'>$i</span><span style='color:#3daee9;'>\leq</span><span style='color:#ff5500;'> r$</span> y <span style='color:#ff5500;'>$0$</span> si <span style='color:#ff5500;'>$i&gt;r$</span>;
  <span style='color:#644a9b;'>\item</span> <span style='color:#ff5500;'>$A=U</span><span style='color:#3daee9;'>\S</span><span style='color:#ff5500;'> V$</span>
 <b>\end</b>{<b><span style='color:#0095ff;'>enumerate</span></b>}
 <b>\end</b>{<b><span style='color:#0095ff;'>Teo</span></b>}
 <b>\begin</b>{<b><span style='color:#0095ff;'>proof</span></b>}
 Véase <b>\citet</b>[pp.419-420]{<b><span style='color:#0095ff;'>lay</span></b>}.<span style='color:#644a9b;'>\-</span> 
  
 <b>\end</b>{<b><span style='color:#0095ff;'>proof</span></b>}<span style='color:#644a9b;'>\-</span>

  <b>\begin</b>{<b><span style='color:#0095ff;'>Cor</span></b>}<b>\label</b>{<b><span style='color:#0095ff;'>frobentent</span></b>}[Equivalencia distancia de Frobenius]
 Sean <span style='color:#ff5500;'>$A,B</span><span style='color:#3daee9;'>\in\R</span><span style='color:#ff5500;'>^{n</span><span style='color:#3daee9;'>\times</span><span style='color:#ff5500;'> m}(</span><span style='color:#3daee9;'>\R</span><span style='color:#ff5500;'>)$</span>. Entonces 
 <span style='color:#644a9b;'>\ies</span>
 d_F(A,B)=<span style='color:#644a9b;'>\sqrt</span>{<span style='color:#644a9b;'>\sum</span>_{j=1}^m<span style='color:#644a9b;'>\sum</span>_{i=1}^n (A_{i,j}-B_{i,j})^2}
 <span style='color:#644a9b;'>\fes</span>
 <b>\end</b>{<b><span style='color:#0095ff;'>Cor</span></b>} 
 <b>\begin</b>{<b><span style='color:#0095ff;'>proof</span></b>}
 Por el Lema <b>\ref</b>{<b><span style='color:#0095ff;'>madjmtra</span></b>} <span style='color:#ff5500;'>$(A-B)^*=(A-B)'$</span>, luego
<span style='color:#644a9b;'>\ies</span>
 d_F(A,B)&amp;=&amp;<span style='color:#644a9b;'>\norm</span>{A-B}_F<span style='color:#644a9b;'>\\</span>
 &amp;=&amp;<span style='color:#644a9b;'>\sqrt</span>{tr((A-B)'(A-B))}<span style='color:#644a9b;'>\\</span>
 &amp;=&amp;<span style='color:#644a9b;'>\sqrt</span>{<span style='color:#644a9b;'>\sum</span>_{j=1}^m<span style='color:#644a9b;'>\sum</span>_{i=1}^n(A-B)'_{j,i}(A-B)_{i,j}}<span style='color:#644a9b;'>\\</span>
&amp;=&amp;<span style='color:#644a9b;'>\sqrt</span>{<span style='color:#644a9b;'>\sum</span>_{j=1}^m<span style='color:#644a9b;'>\sum</span>_{i=1}^n(A-B)_{i,j}(A-B)_{i,j}}<span style='color:#644a9b;'>\\</span>
&amp;=&amp;<span style='color:#644a9b;'>\sqrt</span>{<span style='color:#644a9b;'>\sum</span>_{j=1}^m<span style='color:#644a9b;'>\sum</span>_{i=1}^n(A_{i,j}-B_{i,j})(A_{i,j}-B_{i,j})}<span style='color:#644a9b;'>\\</span>
&amp;=&amp;<span style='color:#644a9b;'>\sqrt</span>{<span style='color:#644a9b;'>\sum</span>_{j=1}^m<span style='color:#644a9b;'>\sum</span>_{i=1}^n(A_{i,j}-B_{i,j})^2}<span style='color:#644a9b;'>\\</span>
 <span style='color:#644a9b;'>\fes</span> 

<b>\end</b>{<b><span style='color:#0095ff;'>proof</span></b>}<span style='color:#644a9b;'>\-</span>


<span style='color:#898887;'>%///////////////////////////////////////////////////////////////////////////////////////////////////////////////////////////////////////</span>
<span style='color:#898887;'>%///////////////////////////////////////////////////////////////////////////////////////////////////////////////////////////////////////</span>
<b>\chapter</b>[Algoritmos y datos]{<b>Algoritmos y datos</b>}<b>\label</b>{<b><span style='color:#0095ff;'>Apendice C</span></b>}
<span style='color:#898887;'>%///////////////////////////////////////////////////////////////////////////////////////////////////////////////////////////////////////</span>
<span style='color:#898887;'>%///////////////////////////////////////////////////////////////////////////////////////////////////////////////////////////////////////</span>

<span style='color:#644a9b;'>\lstset</span>{style=codigo}

<span style='color:#644a9b;'>\lstinputlisting</span>[label={Cod1.1},caption={Ajuste sistemas input-output en <span style='color:#ff5500;'>$</span><span style='color:#3daee9;'>\C</span><span style='color:#ff5500;'>$</span>},language=Python,firstline=1,lastline=85]{/home/armandorg/Documentos/Doctorado/Simulaciones/cod1.py}
 
  

<span style='color:#898887;'>%\begin{landscape}</span>
<b>\begin</b>{<b><span style='color:#0095ff;'>tiny</span></b>}
<b>\begin</b>{<b><span style='color:#0095ff;'>longtable</span></b>}{p{.7cm}rrrrrrrrr}
<span style='color:#644a9b;'>\caption</span>{<b>\label</b>{<b><span style='color:#0095ff;'>Tabla1</span></b>} Datos trimestrales de México, 1993-2018}<span style='color:#644a9b;'>\\</span> 
<span style='color:#644a9b;'>\multirow</span>{3}{1cm}{Año / Trimes- tre}<b>&amp;</b><span style='color:#644a9b;'>\multirow</span>{3}{1cm}{Consumo del Gobierno<span style='color:#ff5500;'>$^1$</span>}<b>&amp;</b><span style='color:#644a9b;'>\multirow</span>{3}{1.2cm}{Consumo Privado<span style='color:#ff5500;'>$^1$</span>}<b>&amp;</b><span style='color:#644a9b;'>\multirow</span>{3}{1.5cm}{Formación Bruta de Capital Fijo<span style='color:#ff5500;'>$^1$</span>}<b>&amp;</b>
<span style='color:#644a9b;'>\multirow</span>{3}{1cm}{Variación de Exis- tencias<span style='color:#ff5500;'>$^1$</span>}<b>&amp;</b><span style='color:#644a9b;'>\multirow</span>{3}{1.2cm}{Importación de Bienes Finales<span style='color:#ff5500;'>$^2$</span>}<b>&amp;</b><span style='color:#644a9b;'>\multirow</span>{3}{1.2cm}{Importación de Bienes Intermedios<span style='color:#ff5500;'>$^2$</span>}<b>&amp;</b>
<span style='color:#644a9b;'>\multirow</span>{3}{1.4cm}{Importación de Bienes de Capital<span style='color:#ff5500;'>$^2$</span>}<b>&amp;</b><span style='color:#644a9b;'>\multirow</span>{3}{1cm}{Exporta- ciones Totales<span style='color:#ff5500;'>$^2$</span>}<b>&amp;</b><span style='color:#644a9b;'>\multirow</span>{3}{.6cm}{Tipo de Cambio<span style='color:#ff5500;'>$^3$</span>}<span style='color:#644a9b;'>\\</span>
<span style='color:#644a9b;'>\\</span>
<span style='color:#644a9b;'>\\</span>[.1cm]<span style='color:#644a9b;'>\hline\hline</span>
<span style='color:#644a9b;'>\vspace</span>{-.1cm}<span style='color:#644a9b;'>\\</span>
1993/01<b>&amp;</b> 134 056.53<b>&amp;</b>1 017 774.94<b>&amp;</b> 324 289.55<b>&amp;</b> 46 372.24<b>&amp;</b> 1 798.68<b>&amp;</b> 10 914.75<b>&amp;</b> 2 671.22<b>&amp;</b> 11 769.29<b>&amp;</b>  3.11<span style='color:#644a9b;'>\\</span>
1993/02<b>&amp;</b> 139 849.02<b>&amp;</b>1 073 692.04<b>&amp;</b> 305 642.01<b>&amp;</b> 56 521.89<b>&amp;</b> 1 879.30<b>&amp;</b> 11 660.23<b>&amp;</b> 2 779.95<b>&amp;</b> 13 046.54<b>&amp;</b>  3.11<span style='color:#644a9b;'>\\</span>
1993/03<b>&amp;</b> 140 634.87<b>&amp;</b>1 079 579.95<b>&amp;</b> 323 636.66<b>&amp;</b> 19 135.45<b>&amp;</b> 1 938.79<b>&amp;</b> 11 726.94<b>&amp;</b> 2 660.52<b>&amp;</b> 12 818.20<b>&amp;</b>  3.12<span style='color:#644a9b;'>\\</span>
1993/04<b>&amp;</b> 154 256.09<b>&amp;</b>1 158 893.15<b>&amp;</b> 347 513.68<b>&amp;</b> 13 458.38<b>&amp;</b> 2 225.58<b>&amp;</b> 12 266.20<b>&amp;</b> 2 844.38<b>&amp;</b> 14 251.93<b>&amp;</b>  3.12<span style='color:#644a9b;'>\\</span>
1994/01<b>&amp;</b> 161 709.76<b>&amp;</b>1 125 906.78<b>&amp;</b> 375 637.40<b>&amp;</b> 80 398.47<b>&amp;</b> 2 067.48<b>&amp;</b> 12 964.39<b>&amp;</b> 3 041.23<b>&amp;</b> 13 775.92<b>&amp;</b>  3.17<span style='color:#644a9b;'>\\</span>
1994/02<b>&amp;</b> 171 024.52<b>&amp;</b>1 215 635.70<b>&amp;</b> 373 191.35<b>&amp;</b> 85 285.54<b>&amp;</b> 2 350.27<b>&amp;</b> 13 984.22<b>&amp;</b> 3 283.55<b>&amp;</b> 15 067.75<b>&amp;</b>  3.34<span style='color:#644a9b;'>\\</span>
1994/03<b>&amp;</b> 167 436.24<b>&amp;</b>1 234 138.85<b>&amp;</b> 394 947.53<b>&amp;</b> 36 990.27<b>&amp;</b> 2 284.03<b>&amp;</b> 14 258.33<b>&amp;</b> 3 316.14<b>&amp;</b> 15 064.15<b>&amp;</b>  3.39<span style='color:#644a9b;'>\\</span>
1994/04<b>&amp;</b> 174 406.13<b>&amp;</b>1 314 877.72<b>&amp;</b> 400 588.12<b>&amp;</b> 28 922.13<b>&amp;</b> 2 808.67<b>&amp;</b> 15 306.79<b>&amp;</b> 3 680.80<b>&amp;</b> 16 974.39<b>&amp;</b>  3.64<span style='color:#644a9b;'>\\</span>
1995/01<b>&amp;</b> 179 617.51<b>&amp;</b>1 284 669.88<b>&amp;</b> 343 814.98<b>&amp;</b> 116 175.01<b>&amp;</b> 1 469.61<b>&amp;</b> 14 348.91<b>&amp;</b> 2 371.48<b>&amp;</b> 18 786.74<b>&amp;</b>  6.03<span style='color:#644a9b;'>\\</span>
1995/02<b>&amp;</b> 195 254.98<b>&amp;</b>1 482 540.91<b>&amp;</b> 336 908.71<b>&amp;</b> 97 399.75<b>&amp;</b> 1 196.54<b>&amp;</b> 13 773.30<b>&amp;</b> 2 062.77<b>&amp;</b> 19 631.46<b>&amp;</b>  6.14<span style='color:#644a9b;'>\\</span>
1995/03<b>&amp;</b> 199 586.21<b>&amp;</b>1 590 023.26<b>&amp;</b> 383 658.05<b>&amp;</b> 87 100.20<b>&amp;</b> 1 216.29<b>&amp;</b> 14 598.61<b>&amp;</b> 2 057.71<b>&amp;</b> 20 087.17<b>&amp;</b>  6.21<span style='color:#644a9b;'>\\</span>
1995/04<b>&amp;</b> 216 167.34<b>&amp;</b>1 765 919.99<b>&amp;</b> 447 685.64<b>&amp;</b> 125 727.33<b>&amp;</b> 1 452.30<b>&amp;</b> 15 700.25<b>&amp;</b> 2 205.30<b>&amp;</b> 21 036.19<b>&amp;</b>  7.35<span style='color:#644a9b;'>\\</span>
1996/01<b>&amp;</b> 232 694.56<b>&amp;</b>1 854 077.06<b>&amp;</b> 506 711.90<b>&amp;</b> 131 221.40<b>&amp;</b> 1 455.89<b>&amp;</b> 16 140.09<b>&amp;</b> 2 340.07<b>&amp;</b> 21 870.38<b>&amp;</b>  7.52<span style='color:#644a9b;'>\\</span>
1996/02<b>&amp;</b> 247 727.30<b>&amp;</b>2 000 318.79<b>&amp;</b> 519 842.01<b>&amp;</b> 99 680.28<b>&amp;</b> 1 455.41<b>&amp;</b> 17 351.89<b>&amp;</b> 2 602.25<b>&amp;</b> 23 606.88<b>&amp;</b>  7.48<span style='color:#644a9b;'>\\</span>
1996/03<b>&amp;</b> 254 501.28<b>&amp;</b>2 123 101.37<b>&amp;</b> 600 914.71<b>&amp;</b> 85 555.16<b>&amp;</b> 1 623.86<b>&amp;</b> 18 531.84<b>&amp;</b> 2 679.46<b>&amp;</b> 24 247.08<b>&amp;</b>  7.56<span style='color:#644a9b;'>\\</span>
1996/04<b>&amp;</b> 279 446.10<b>&amp;</b>2 376 683.97<b>&amp;</b> 672 909.58<b>&amp;</b> 78 099.34<b>&amp;</b> 2 121.61<b>&amp;</b> 19 865.80<b>&amp;</b> 3 300.59<b>&amp;</b> 26 275.40<b>&amp;</b>  7.83<span style='color:#644a9b;'>\\</span>
1997/01<b>&amp;</b> 302 287.82<b>&amp;</b>2 355 533.98<b>&amp;</b> 678 673.66<b>&amp;</b> 159 187.94<b>&amp;</b> 1 518.04<b>&amp;</b> 18 982.09<b>&amp;</b> 3 028.92<b>&amp;</b> 25 098.39<b>&amp;</b>  7.86<span style='color:#644a9b;'>\\</span>
1997/02<b>&amp;</b> 322 560.04<b>&amp;</b>2 606 061.50<b>&amp;</b> 743 730.15<b>&amp;</b> 113 758.33<b>&amp;</b> 2 184.79<b>&amp;</b> 20 795.99<b>&amp;</b> 3 818.93<b>&amp;</b> 27 440.43<b>&amp;</b>  7.92<span style='color:#644a9b;'>\\</span>
1997/03<b>&amp;</b> 327 167.82<b>&amp;</b>2 675 387.74<b>&amp;</b> 836 181.30<b>&amp;</b> 79 086.66<b>&amp;</b> 2 289.54<b>&amp;</b> 22 426.69<b>&amp;</b> 3 768.92<b>&amp;</b> 28 176.38<b>&amp;</b>  7.81<span style='color:#644a9b;'>\\</span>
1997/04<b>&amp;</b> 357 456.29<b>&amp;</b>2 934 217.61<b>&amp;</b> 887 128.24<b>&amp;</b> 83 889.55<b>&amp;</b> 3 097.95<b>&amp;</b> 23 434.40<b>&amp;</b> 4 461.93<b>&amp;</b> 29 716.30<b>&amp;</b>  8.07<span style='color:#644a9b;'>\\</span>
1998/01<b>&amp;</b> 377 536.83<b>&amp;</b>2 921 015.89<b>&amp;</b> 948 018.82<b>&amp;</b> 176 814.83<b>&amp;</b> 2 703.31<b>&amp;</b> 23 008.60<b>&amp;</b> 4 162.25<b>&amp;</b> 28 106.17<b>&amp;</b>  8.43<span style='color:#644a9b;'>\\</span>
1998/02<b>&amp;</b> 401 250.40<b>&amp;</b>3 136 284.55<b>&amp;</b> 940 477.42<b>&amp;</b> 122 107.63<b>&amp;</b> 2 584.99<b>&amp;</b> 24 040.18<b>&amp;</b> 4 397.53<b>&amp;</b> 29 871.52<b>&amp;</b>  8.68<span style='color:#644a9b;'>\\</span>
1998/03<b>&amp;</b> 412 399.89<b>&amp;</b>3 261 141.12<b>&amp;</b>1 066 315.32<b>&amp;</b> 92 249.82<b>&amp;</b> 2 631.74<b>&amp;</b> 24 085.12<b>&amp;</b> 4 308.48<b>&amp;</b> 28 586.59<b>&amp;</b>  9.47<span style='color:#644a9b;'>\\</span>
1998/04<b>&amp;</b> 452 553.02<b>&amp;</b>3 509 423.24<b>&amp;</b>1 102 694.55<b>&amp;</b> 110 003.02<b>&amp;</b> 3 188.43<b>&amp;</b> 25 801.32<b>&amp;</b> 4 461.10<b>&amp;</b> 30 975.02<b>&amp;</b>  10.01<span style='color:#644a9b;'>\\</span>
1999/01<b>&amp;</b> 489 642.36<b>&amp;</b>3 568 342.77<b>&amp;</b>1 172 569.27<b>&amp;</b> 95 967.24<b>&amp;</b> 2 442.60<b>&amp;</b> 24 127.63<b>&amp;</b> 4 576.71<b>&amp;</b> 29 940.37<b>&amp;</b>  9.94<span style='color:#644a9b;'>\\</span>
1999/02<b>&amp;</b> 521 168.14<b>&amp;</b>3 801 665.72<b>&amp;</b>1 139 024.65<b>&amp;</b> 103 109.73<b>&amp;</b> 2 811.78<b>&amp;</b> 26 825.43<b>&amp;</b> 4 980.46<b>&amp;</b> 33 641.11<b>&amp;</b>  9.45<span style='color:#644a9b;'>\\</span>
1999/03<b>&amp;</b> 527 649.66<b>&amp;</b>3 918 757.54<b>&amp;</b>1 254 411.26<b>&amp;</b> 55 313.55<b>&amp;</b> 2 965.28<b>&amp;</b> 28 023.25<b>&amp;</b> 5 266.45<b>&amp;</b> 35 293.00<b>&amp;</b>  9.37<span style='color:#644a9b;'>\\</span>
1999/04<b>&amp;</b> 568 363.18<b>&amp;</b>4 153 842.28<b>&amp;</b>1 281 098.98<b>&amp;</b> 81 205.04<b>&amp;</b> 3 955.37<b>&amp;</b> 30 293.31<b>&amp;</b> 5 706.51<b>&amp;</b> 37 487.33<b>&amp;</b>  9.46<span style='color:#644a9b;'>\\</span>
2000/01<b>&amp;</b> 609 975.23<b>&amp;</b>4 201 895.45<b>&amp;</b>1 383 192.04<b>&amp;</b> 151 180.28<b>&amp;</b> 3 509.77<b>&amp;</b> 30 384.61<b>&amp;</b> 5 324.60<b>&amp;</b> 38 017.74<b>&amp;</b>  9.40<span style='color:#644a9b;'>\\</span>
2000/02<b>&amp;</b> 632 286.66<b>&amp;</b>4 485 433.73<b>&amp;</b>1 381 838.39<b>&amp;</b> 131 570.21<b>&amp;</b> 3 919.12<b>&amp;</b> 32 910.56<b>&amp;</b> 5 648.34<b>&amp;</b> 41 027.14<b>&amp;</b>  9.59<span style='color:#644a9b;'>\\</span>
2000/03<b>&amp;</b> 633 682.02<b>&amp;</b>4 628 401.55<b>&amp;</b>1 513 123.85<b>&amp;</b> 62 565.82<b>&amp;</b> 3 968.57<b>&amp;</b> 34 693.97<b>&amp;</b> 6 214.86<b>&amp;</b> 42 783.21<b>&amp;</b>  9.35<span style='color:#644a9b;'>\\</span>
2000/04<b>&amp;</b> 671 730.58<b>&amp;</b>4 874 553.68<b>&amp;</b>1 475 438.34<b>&amp;</b> 48 082.59<b>&amp;</b> 5 293.10<b>&amp;</b> 35 648.22<b>&amp;</b> 6 942.14<b>&amp;</b> 44 292.64<b>&amp;</b>  9.50<span style='color:#644a9b;'>\\</span>
2001/01<b>&amp;</b> 672 251.58<b>&amp;</b>4 672 918.23<b>&amp;</b>1 440 146.53<b>&amp;</b> 117 260.17<b>&amp;</b> 4 531.15<b>&amp;</b> 31 608.69<b>&amp;</b> 5 722.72<b>&amp;</b> 39 730.26<b>&amp;</b>  9.69<span style='color:#644a9b;'>\\</span>
2001/02<b>&amp;</b> 696 356.20<b>&amp;</b>4 849 357.45<b>&amp;</b>1 332 612.46<b>&amp;</b> 96 879.00<b>&amp;</b> 4 763.94<b>&amp;</b> 32 154.39<b>&amp;</b> 5 603.99<b>&amp;</b> 40 748.10<b>&amp;</b>  9.18<span style='color:#644a9b;'>\\</span>
2001/03<b>&amp;</b> 692 564.80<b>&amp;</b>4 872 250.16<b>&amp;</b>1 458 592.05<b>&amp;</b> 16 082.60<b>&amp;</b> 4 686.95<b>&amp;</b> 30 837.15<b>&amp;</b> 5 274.52<b>&amp;</b> 39 129.15<b>&amp;</b>  9.23<span style='color:#644a9b;'>\\</span>
2001/04<b>&amp;</b> 732 963.74<b>&amp;</b>5 088 672.66<b>&amp;</b>1 405 402.93<b>&amp;</b> 24 646.87<b>&amp;</b> 5 769.94<b>&amp;</b> 31 548.52<b>&amp;</b> 5 894.45<b>&amp;</b> 39 172.23<b>&amp;</b>  9.25<span style='color:#644a9b;'>\\</span>
2002/01<b>&amp;</b> 738 161.65<b>&amp;</b>4 840 452.83<b>&amp;</b>1 331 951.29<b>&amp;</b> 165 075.80<b>&amp;</b> 4 366.13<b>&amp;</b> 28 833.71<b>&amp;</b> 4 921.68<b>&amp;</b> 36 677.19<b>&amp;</b>  9.11<span style='color:#644a9b;'>\\</span>
2002/02<b>&amp;</b> 770 824.80<b>&amp;</b>5 225 346.00<b>&amp;</b>1 403 586.43<b>&amp;</b> 132 171.91<b>&amp;</b> 5 305.65<b>&amp;</b> 32 172.87<b>&amp;</b> 5 499.09<b>&amp;</b> 41 480.50<b>&amp;</b>  9.47<span style='color:#644a9b;'>\\</span>
2002/03<b>&amp;</b> 768 787.57<b>&amp;</b>5 250 586.43<b>&amp;</b>1 503 069.06<b>&amp;</b> 87 302.12<b>&amp;</b> 5 128.10<b>&amp;</b> 32 890.76<b>&amp;</b> 4 906.22<b>&amp;</b> 41 367.21<b>&amp;</b>  9.89<span style='color:#644a9b;'>\\</span>
2002/04<b>&amp;</b> 817 051.64<b>&amp;</b>5 458 095.01<b>&amp;</b>1 507 889.36<b>&amp;</b> 43 796.49<b>&amp;</b> 6 378.47<b>&amp;</b> 32 610.73<b>&amp;</b> 5 665.47<b>&amp;</b> 41 521.08<b>&amp;</b>  10.17<span style='color:#644a9b;'>\\</span>
2003/01<b>&amp;</b> 841 323.34<b>&amp;</b>5 256 478.10<b>&amp;</b>1 542 091.85<b>&amp;</b> 181 294.93<b>&amp;</b> 4 973.26<b>&amp;</b> 29 565.99<b>&amp;</b> 4 650.16<b>&amp;</b> 39 051.49<b>&amp;</b>  10.81<span style='color:#644a9b;'>\\</span>
2003/02<b>&amp;</b> 884 628.06<b>&amp;</b>5 483 515.38<b>&amp;</b>1 512 123.49<b>&amp;</b> 119 541.29<b>&amp;</b> 4 885.33<b>&amp;</b> 32 012.83<b>&amp;</b> 4 992.97<b>&amp;</b> 40 244.55<b>&amp;</b>  10.45<span style='color:#644a9b;'>\\</span>
2003/03<b>&amp;</b> 850 843.71<b>&amp;</b>5 534 132.37<b>&amp;</b>1 575 135.20<b>&amp;</b> 28 611.87<b>&amp;</b> 5 214.95<b>&amp;</b> 32 788.72<b>&amp;</b> 4 968.45<b>&amp;</b> 41 543.04<b>&amp;</b>  10.71<span style='color:#644a9b;'>\\</span>
2003/04<b>&amp;</b> 896 116.70<b>&amp;</b>5 830 030.95<b>&amp;</b>1 595 791.25<b>&amp;</b> 85 386.77<b>&amp;</b> 6 435.49<b>&amp;</b> 34 463.93<b>&amp;</b> 5 593.77<b>&amp;</b> 43 927.35<b>&amp;</b>  11.19<span style='color:#644a9b;'>\\</span>
2004/01<b>&amp;</b> 907 627.02<b>&amp;</b>5 640 640.91<b>&amp;</b>1 695 674.47<b>&amp;</b> 141 839.43<b>&amp;</b> 5 200.68<b>&amp;</b> 33 589.94<b>&amp;</b> 4 926.28<b>&amp;</b> 43 193.01<b>&amp;</b>  10.98<span style='color:#644a9b;'>\\</span>
2004/02<b>&amp;</b> 937 287.55<b>&amp;</b>5 955 078.92<b>&amp;</b>1 791 045.52<b>&amp;</b> 207 229.46<b>&amp;</b> 5 948.15<b>&amp;</b> 37 162.67<b>&amp;</b> 5 396.54<b>&amp;</b> 47 387.84<b>&amp;</b>  11.39<span style='color:#644a9b;'>\\</span>
2004/03<b>&amp;</b> 899 533.26<b>&amp;</b>6 086 046.68<b>&amp;</b>1 832 711.24<b>&amp;</b> 122 580.01<b>&amp;</b> 6 314.29<b>&amp;</b> 37 807.83<b>&amp;</b> 5 699.75<b>&amp;</b> 47 874.26<b>&amp;</b>  11.45<span style='color:#644a9b;'>\\</span>
2004/04<b>&amp;</b> 958 978.08<b>&amp;</b>6 445 208.85<b>&amp;</b>1 911 517.13<b>&amp;</b> 107 879.99<b>&amp;</b> 7 945.86<b>&amp;</b> 40 243.28<b>&amp;</b> 6 574.39<b>&amp;</b> 49 543.44<b>&amp;</b>  11.32<span style='color:#644a9b;'>\\</span>
2005/01<b>&amp;</b> 961 231.13<b>&amp;</b>6 176 929.94<b>&amp;</b>1 905 165.73<b>&amp;</b> 134 812.55<b>&amp;</b> 6 554.21<b>&amp;</b> 36 420.54<b>&amp;</b> 5 797.28<b>&amp;</b> 46 894.76<b>&amp;</b>  11.18<span style='color:#644a9b;'>\\</span>
2005/02<b>&amp;</b>1 008 967.19<b>&amp;</b>6 576 518.89<b>&amp;</b>1 950 469.28<b>&amp;</b> 96 646.41<b>&amp;</b> 7 289.19<b>&amp;</b> 41 090.79<b>&amp;</b> 6 274.72<b>&amp;</b> 53 977.04<b>&amp;</b>  10.97<span style='color:#644a9b;'>\\</span>
2005/03<b>&amp;</b> 992 525.04<b>&amp;</b>6 639 203.69<b>&amp;</b>1 967 492.53<b>&amp;</b> 36 298.27<b>&amp;</b> 7 686.72<b>&amp;</b> 41 578.05<b>&amp;</b> 6 606.59<b>&amp;</b> 54 183.35<b>&amp;</b>  10.71<span style='color:#644a9b;'>\\</span>
2005/04<b>&amp;</b>1 061 891.78<b>&amp;</b>6 857 958.41<b>&amp;</b>2 096 163.11<b>&amp;</b> 91 921.02<b>&amp;</b> 9 982.76<b>&amp;</b> 45 001.75<b>&amp;</b> 7 536.93<b>&amp;</b> 59 177.82<b>&amp;</b>  10.71<span style='color:#644a9b;'>\\</span>
2006/01<b>&amp;</b>1 078 524.49<b>&amp;</b>6 750 621.59<b>&amp;</b>2 112 334.84<b>&amp;</b> 146 743.88<b>&amp;</b> 8 237.27<b>&amp;</b> 42 953.44<b>&amp;</b> 6 928.61<b>&amp;</b> 58 852.39<b>&amp;</b>  10.60<span style='color:#644a9b;'>\\</span>
2006/02<b>&amp;</b>1 115 793.57<b>&amp;</b>7 141 213.96<b>&amp;</b>2 257 080.75<b>&amp;</b> 214 897.17<b>&amp;</b> 8 758.23<b>&amp;</b> 47 767.36<b>&amp;</b> 7 406.03<b>&amp;</b> 63 779.72<b>&amp;</b>  11.18<span style='color:#644a9b;'>\\</span>
2006/03<b>&amp;</b>1 071 233.04<b>&amp;</b>7 212 024.43<b>&amp;</b>2 361 997.90<b>&amp;</b> 104 521.39<b>&amp;</b> 9 534.88<b>&amp;</b> 48 568.90<b>&amp;</b> 7 658.94<b>&amp;</b> 63 139.91<b>&amp;</b>  10.95<span style='color:#644a9b;'>\\</span>
2006/04<b>&amp;</b>1 133 286.48<b>&amp;</b>7 499 115.15<b>&amp;</b>2 429 623.41<b>&amp;</b> 77 962.61<b>&amp;</b> 10 370.63<b>&amp;</b> 49 342.77<b>&amp;</b> 8 531.29<b>&amp;</b> 64 153.13<b>&amp;</b>  10.89<span style='color:#644a9b;'>\\</span>
2007/01<b>&amp;</b>1 140 500.96<b>&amp;</b>7 345 042.77<b>&amp;</b>2 364 026.75<b>&amp;</b> 157 876.20<b>&amp;</b> 9 142.87<b>&amp;</b> 46 051.96<b>&amp;</b> 7 527.92<b>&amp;</b> 60 269.04<b>&amp;</b>  11.02<span style='color:#644a9b;'>\\</span>
2007/02<b>&amp;</b>1 200 639.30<b>&amp;</b>7 691 347.19<b>&amp;</b>2 426 664.08<b>&amp;</b> 190 080.50<b>&amp;</b> 10 501.81<b>&amp;</b> 51 128.67<b>&amp;</b> 8 333.24<b>&amp;</b> 67 655.77<b>&amp;</b>  10.87<span style='color:#644a9b;'>\\</span>
2007/03<b>&amp;</b>1 180 499.31<b>&amp;</b>7 784 629.30<b>&amp;</b>2 551 401.94<b>&amp;</b> 75 329.63<b>&amp;</b> 10 924.89<b>&amp;</b> 53 358.50<b>&amp;</b> 8 380.44<b>&amp;</b> 70 269.40<b>&amp;</b>  10.96<span style='color:#644a9b;'>\\</span>
2007/04<b>&amp;</b>1 266 532.04<b>&amp;</b>8 086 075.64<b>&amp;</b>2 754 760.52<b>&amp;</b> 118 281.80<b>&amp;</b> 12 484.95<b>&amp;</b> 54 756.36<b>&amp;</b> 9 357.44<b>&amp;</b> 73 681.10<b>&amp;</b>  10.85<span style='color:#644a9b;'>\\</span>
2008/01<b>&amp;</b>1 244 310.11<b>&amp;</b>7 930 371.20<b>&amp;</b>2 628 697.80<b>&amp;</b> 93 548.08<b>&amp;</b> 11 259.63<b>&amp;</b> 52 234.59<b>&amp;</b> 8 237.74<b>&amp;</b> 70 084.13<b>&amp;</b>  10.81<span style='color:#644a9b;'>\\</span>
2008/02<b>&amp;</b>1 337 952.13<b>&amp;</b>8 410 688.29<b>&amp;</b>2 853 134.32<b>&amp;</b> 182 129.32<b>&amp;</b> 11 922.49<b>&amp;</b> 58 312.63<b>&amp;</b> 10 032.78<b>&amp;</b> 79 402.87<b>&amp;</b>  10.43<span style='color:#644a9b;'>\\</span>
2008/03<b>&amp;</b>1 309 872.13<b>&amp;</b>8 324 469.21<b>&amp;</b>2 878 769.51<b>&amp;</b> 56 736.99<b>&amp;</b> 13 266.88<b>&amp;</b> 61 159.24<b>&amp;</b> 10 468.06<b>&amp;</b> 78 466.80<b>&amp;</b>  10.32<span style='color:#644a9b;'>\\</span>
2008/04<b>&amp;</b>1 409 898.27<b>&amp;</b>8 409 489.45<b>&amp;</b>3 086 170.57<b>&amp;</b> 133 302.18<b>&amp;</b> 11 491.71<b>&amp;</b> 49 858.98<b>&amp;</b> 10 358.53<b>&amp;</b> 63 388.80<b>&amp;</b>  13.04<span style='color:#644a9b;'>\\</span>
2009/01<b>&amp;</b>1 393 431.91<b>&amp;</b>7 728 821.76<b>&amp;</b>2 625 853.32<b>&amp;</b> 81 439.88<b>&amp;</b> 7 020.70<b>&amp;</b> 37 574.52<b>&amp;</b> 7 361.65<b>&amp;</b> 49 664.85<b>&amp;</b>  14.38<span style='color:#644a9b;'>\\</span>
2009/02<b>&amp;</b>1 458 094.81<b>&amp;</b>7 793 042.76<b>&amp;</b>2 623 124.61<b>&amp;</b> 77 935.08<b>&amp;</b> 7 167.65<b>&amp;</b> 39 378.43<b>&amp;</b> 7 016.14<b>&amp;</b> 54 012.20<b>&amp;</b>  13.31<span style='color:#644a9b;'>\\</span>
2009/03<b>&amp;</b>1 432 551.80<b>&amp;</b>8 043 991.92<b>&amp;</b>2 690 814.40<b>&amp;</b> 60 035.39<b>&amp;</b> 8 854.48<b>&amp;</b> 44 809.55<b>&amp;</b> 7 643.03<b>&amp;</b> 58 520.88<b>&amp;</b>  13.27<span style='color:#644a9b;'>\\</span>
2009/04<b>&amp;</b>1 513 649.68<b>&amp;</b>8 460 424.93<b>&amp;</b>2 824 969.94<b>&amp;</b> 151 525.92<b>&amp;</b> 9 785.24<b>&amp;</b> 49 149.16<b>&amp;</b> 8 624.42<b>&amp;</b> 67 505.62<b>&amp;</b>  13.06<span style='color:#644a9b;'>\\</span>
2010/01<b>&amp;</b>1 496 786.96<b>&amp;</b>8 191 015.02<b>&amp;</b>2 766 614.53<b>&amp;</b> 207 833.29<b>&amp;</b> 9 104.78<b>&amp;</b> 50 315.82<b>&amp;</b> 6 804.15<b>&amp;</b> 66 596.62<b>&amp;</b>  12.76<span style='color:#644a9b;'>\\</span>
2010/02<b>&amp;</b>1 585 008.07<b>&amp;</b>8 618 735.08<b>&amp;</b>2 843 282.67<b>&amp;</b> 188 815.36<b>&amp;</b> 10 055.33<b>&amp;</b> 57 678.35<b>&amp;</b> 6 991.17<b>&amp;</b> 74 641.43<b>&amp;</b>  12.57<span style='color:#644a9b;'>\\</span>
2010/03<b>&amp;</b>1 560 216.70<b>&amp;</b>8 831 326.55<b>&amp;</b>2 859 818.49<b>&amp;</b> 75 139.68<b>&amp;</b> 10 269.77<b>&amp;</b> 59 874.74<b>&amp;</b> 7 696.45<b>&amp;</b> 75 589.82<b>&amp;</b>  12.80<span style='color:#644a9b;'>\\</span>
2010/04<b>&amp;</b>1 651 607.55<b>&amp;</b>9 298 298.71<b>&amp;</b>3 069 584.96<b>&amp;</b> 175 355.21<b>&amp;</b> 11 992.85<b>&amp;</b> 61 943.50<b>&amp;</b> 8 754.91<b>&amp;</b> 81 645.28<b>&amp;</b>  12.39<span style='color:#644a9b;'>\\</span>
2011/01<b>&amp;</b>1 644 255.49<b>&amp;</b>8 970 318.17<b>&amp;</b>2 909 996.66<b>&amp;</b> 219 464.37<b>&amp;</b> 11 583.08<b>&amp;</b> 60 603.55<b>&amp;</b> 7 706.05<b>&amp;</b> 81 801.20<b>&amp;</b>  12.06<span style='color:#644a9b;'>\\</span>
2011/02<b>&amp;</b>1 708 514.14<b>&amp;</b>9 331 501.54<b>&amp;</b>3 105 666.53<b>&amp;</b> 188 470.71<b>&amp;</b> 12 688.58<b>&amp;</b> 66 678.26<b>&amp;</b> 8 676.67<b>&amp;</b> 89 282.61<b>&amp;</b>  11.73<span style='color:#644a9b;'>\\</span>
2011/03<b>&amp;</b>1 718 061.39<b>&amp;</b>9 636 372.31<b>&amp;</b>3 305 363.61<b>&amp;</b> 55 615.19<b>&amp;</b> 13 851.03<b>&amp;</b> 69 022.69<b>&amp;</b> 9 094.40<b>&amp;</b> 88 087.09<b>&amp;</b>  12.31<span style='color:#644a9b;'>\\</span>
2011/04<b>&amp;</b>1 837 877.49<b>&amp;</b>10 078 909.43<b>&amp;</b>3 745 365.21<b>&amp;</b> 127 351.55<b>&amp;</b> 13 667.54<b>&amp;</b> 67 715.75<b>&amp;</b> 9 555.28<b>&amp;</b> 90 262.49<b>&amp;</b>  13.63<span style='color:#644a9b;'>\\</span>
2012/01<b>&amp;</b>1 839 767.80<b>&amp;</b>9 897 927.02<b>&amp;</b>3 450 147.09<b>&amp;</b> 229 466.37<b>&amp;</b> 12 819.15<b>&amp;</b> 66 131.69<b>&amp;</b> 8 954.86<b>&amp;</b> 89 608.97<b>&amp;</b>  13.00<span style='color:#644a9b;'>\\</span>
2012/02<b>&amp;</b>1 896 162.86<b>&amp;</b>10 045 719.25<b>&amp;</b>3 590 282.16<b>&amp;</b> 228 428.69<b>&amp;</b> 13 310.40<b>&amp;</b> 70 296.76<b>&amp;</b> 9 334.06<b>&amp;</b> 94 308.60<b>&amp;</b>  13.57<span style='color:#644a9b;'>\\</span>
2012/03<b>&amp;</b>1 859 125.80<b>&amp;</b>10 294 103.45<b>&amp;</b>3 663 796.28<b>&amp;</b> 43 563.98<b>&amp;</b> 13 223.02<b>&amp;</b> 69 661.27<b>&amp;</b> 9 531.30<b>&amp;</b> 91 234.18<b>&amp;</b>  13.17<span style='color:#644a9b;'>\\</span>
2012/04<b>&amp;</b>1 963 763.78<b>&amp;</b>10 714 140.29<b>&amp;</b>3 747 151.66<b>&amp;</b> 160 430.47<b>&amp;</b> 14 919.85<b>&amp;</b> 71 821.36<b>&amp;</b> 10 747.84<b>&amp;</b> 95 618.14<b>&amp;</b>  12.94<span style='color:#644a9b;'>\\</span>
2013/01<b>&amp;</b>1 912 523.20<b>&amp;</b>10 317 196.06<b>&amp;</b>3 364 545.79<b>&amp;</b> 257 660.45<b>&amp;</b> 13 647.34<b>&amp;</b> 66 619.77<b>&amp;</b> 9 079.70<b>&amp;</b> 88 228.08<b>&amp;</b>  12.65<span style='color:#644a9b;'>\\</span>
2013/02<b>&amp;</b>1 963 197.70<b>&amp;</b>10 713 017.10<b>&amp;</b>3 430 953.70<b>&amp;</b> 284 393.13<b>&amp;</b> 14 338.17<b>&amp;</b> 73 432.50<b>&amp;</b> 9 790.60<b>&amp;</b> 96 662.69<b>&amp;</b>  12.48<span style='color:#644a9b;'>\\</span>
2013/03<b>&amp;</b>1 968 384.17<b>&amp;</b>10 878 295.83<b>&amp;</b>3 415 749.49<b>&amp;</b> 113 480.67<b>&amp;</b> 14 500.31<b>&amp;</b> 73 171.22<b>&amp;</b> 9 608.29<b>&amp;</b> 96 307.24<b>&amp;</b>  12.91<span style='color:#644a9b;'>\\</span>
2013/04<b>&amp;</b>2 093 554.60<b>&amp;</b>11 368 494.77<b>&amp;</b>3 625 965.12<b>&amp;</b> 152 411.99<b>&amp;</b> 14 843.56<b>&amp;</b> 71 599.91<b>&amp;</b> 10 578.80<b>&amp;</b> 98 817.04<b>&amp;</b>  13.03<span style='color:#644a9b;'>\\</span>
2014/01<b>&amp;</b>2 062 083.99<b>&amp;</b>10 979 818.90<b>&amp;</b>3 408 460.76<b>&amp;</b> 280 428.76<b>&amp;</b> 13 264.96<b>&amp;</b> 69 590.03<b>&amp;</b> 9 209.38<b>&amp;</b> 90 759.10<b>&amp;</b>  13.23<span style='color:#644a9b;'>\\</span>
2014/02<b>&amp;</b>2 090 793.81<b>&amp;</b>11 375 851.46<b>&amp;</b>3 549 971.69<b>&amp;</b> 148 505.39<b>&amp;</b> 14 269.77<b>&amp;</b> 77 017.62<b>&amp;</b> 9 576.16<b>&amp;</b> 101 870.05<b>&amp;</b>  13.00<span style='color:#644a9b;'>\\</span>
2014/03<b>&amp;</b>2 127 278.63<b>&amp;</b>11 571 434.25<b>&amp;</b>3 675 050.93<b>&amp;</b> 106 510.63<b>&amp;</b> 14 898.86<b>&amp;</b> 77 879.31<b>&amp;</b> 10 061.94<b>&amp;</b> 101 120.72<b>&amp;</b>  13.12<span style='color:#644a9b;'>\\</span>
2014/04<b>&amp;</b>2 248 661.26<b>&amp;</b>12 111 339.17<b>&amp;</b>4 055 135.21<b>&amp;</b> 84 699.44<b>&amp;</b> 15 865.54<b>&amp;</b> 77 544.29<b>&amp;</b> 10 799.37<b>&amp;</b> 103 161.81<b>&amp;</b>  13.87<span style='color:#644a9b;'>\\</span>
2015/01<b>&amp;</b>2 257 960.33<b>&amp;</b>11 581 174.09<b>&amp;</b>3 810 144.69<b>&amp;</b> 320 411.69<b>&amp;</b> 12 673.56<b>&amp;</b> 70 307.65<b>&amp;</b> 9 623.47<b>&amp;</b> 90 460.71<b>&amp;</b>  14.95<span style='color:#644a9b;'>\\</span>
2015/02<b>&amp;</b>2 275 898.42<b>&amp;</b>11 840 307.89<b>&amp;</b>4 046 947.68<b>&amp;</b> 143 067.52<b>&amp;</b> 13 454.21<b>&amp;</b> 76 258.93<b>&amp;</b> 10 272.31<b>&amp;</b> 97 976.27<b>&amp;</b>  15.33<span style='color:#644a9b;'>\\</span>
2015/03<b>&amp;</b>2 248 161.07<b>&amp;</b>12 297 858.67<b>&amp;</b>4 369 901.87<b>&amp;</b> 27 864.08<b>&amp;</b> 15 162.78<b>&amp;</b> 76 782.40<b>&amp;</b> 10 617.32<b>&amp;</b> 95 890.31<b>&amp;</b>  16.43<span style='color:#644a9b;'>\\</span>
2015/04<b>&amp;</b>2 368 232.59<b>&amp;</b>12 932 143.56<b>&amp;</b>4 485 293.64<b>&amp;</b> 69 076.02<b>&amp;</b> 14 988.81<b>&amp;</b> 74 364.13<b>&amp;</b> 10 726.81<b>&amp;</b> 96 222.30<b>&amp;</b>  16.76<span style='color:#644a9b;'>\\</span>
2016/01<b>&amp;</b>2 349 138.77<b>&amp;</b>12 452 013.88<b>&amp;</b>4 283 785.14<b>&amp;</b> 336 504.33<b>&amp;</b> 11 721.62<b>&amp;</b> 68 521.82<b>&amp;</b> 8 889.94<b>&amp;</b> 85 146.83<b>&amp;</b>  18.06<span style='color:#644a9b;'>\\</span>
2016/02<b>&amp;</b>2 412 627.70<b>&amp;</b>12 824 465.62<b>&amp;</b>4 504 366.90<b>&amp;</b> 201 353.93<b>&amp;</b> 12 505.99<b>&amp;</b> 74 438.41<b>&amp;</b> 9 869.21<b>&amp;</b> 93 746.17<b>&amp;</b>  18.11<span style='color:#644a9b;'>\\</span>
2016/03<b>&amp;</b>2 393 451.28<b>&amp;</b>13 331 775.33<b>&amp;</b>4 665 180.02<b>&amp;</b> 55 162.05<b>&amp;</b> 13 727.96<b>&amp;</b> 75 965.00<b>&amp;</b> 10 462.04<b>&amp;</b> 94 918.85<b>&amp;</b>  18.75<span style='color:#644a9b;'>\\</span>
2016/04<b>&amp;</b>2 515 113.99<b>&amp;</b>14 153 946.45<b>&amp;</b>4 994 738.35<b>&amp;</b> 84 542.71<b>&amp;</b> 13 994.72<b>&amp;</b> 76 470.16<b>&amp;</b> 10 497.64<b>&amp;</b> 100 134.85<b>&amp;</b>  19.84<span style='color:#644a9b;'>\\</span>
2017/01<b>&amp;</b>2 526 625.83<b>&amp;</b>13 718 762.94<b>&amp;</b>4 805 083.64<b>&amp;</b> 356 311.36<b>&amp;</b> 13 055.59<b>&amp;</b> 75 171.01<b>&amp;</b> 9 253.03<b>&amp;</b> 94 708.75<b>&amp;</b>  20.33<span style='color:#644a9b;'>\\</span>
2017/02<b>&amp;</b>2 568 426.06<b>&amp;</b>14 063 805.48<b>&amp;</b>4 730 941.67<b>&amp;</b> 200 520.93<b>&amp;</b> 13 286.14<b>&amp;</b> 79 759.47<b>&amp;</b> 9 913.64<b>&amp;</b> 102 657.44<b>&amp;</b>  18.54<span style='color:#644a9b;'>\\</span>
2017/03<b>&amp;</b>2 522 967.79<b>&amp;</b>14 348 343.37<b>&amp;</b>4 810 109.96<b>&amp;</b> 71 669.85<b>&amp;</b> 14 722.61<b>&amp;</b> 82 519.46<b>&amp;</b> 10 658.47<b>&amp;</b> 101 851.48<b>&amp;</b>  17.82<span style='color:#644a9b;'>\\</span>
2017/04<b>&amp;</b>2 673 080.31<b>&amp;</b>15 097 438.55<b>&amp;</b>5 052 831.91<b>&amp;</b> 102 755.55<b>&amp;</b> 16 268.70<b>&amp;</b> 84 571.80<b>&amp;</b> 11 189.22<b>&amp;</b> 110 183.41<b>&amp;</b>  18.96<span style='color:#644a9b;'>\\</span>
2018/01<b>&amp;</b>2 687 714.89<b>&amp;</b>14 534 067.75<b>&amp;</b>4 949 016.43<b>&amp;</b> 383 926.90<b>&amp;</b> 14 523.59<b>&amp;</b> 82 060.11<b>&amp;</b> 10 434.86<b>&amp;</b> 105 242.02<b>&amp;</b>  18.74<span style='color:#644a9b;'>\\</span>
2018/02<b>&amp;</b>2 766 519.21<b>&amp;</b>15 104 267.30<b>&amp;</b>5 231 051.83<b>&amp;</b> 192 891.26<b>&amp;</b> 15 354.45<b>&amp;</b> 89 927.28<b>&amp;</b> 11 332.91<b>&amp;</b> 113 841.81<b>&amp;</b>  19.43<span style='color:#644a9b;'>\\</span>
2018/03<b>&amp;</b>2 667 651.25<b>&amp;</b>15 291 482.50<b>&amp;</b>5 274 734.95<b>&amp;</b> 99 234.55<b>&amp;</b> 16 462.57<b>&amp;</b> 91 157.19<b>&amp;</b> 12 116.38<b>&amp;</b> 114 062.56<b>&amp;</b>  18.96<span style='color:#644a9b;'>\\</span>
2018/04<b>&amp;</b>2 805 468.19<b>&amp;</b>15 909 520.09<b>&amp;</b>5 365 900.29<b>&amp;</b> 168 798.80<b>&amp;</b> 16 770.73<b>&amp;</b> 92 135.59<b>&amp;</b> 12 000.95<b>&amp;</b> 117 425.77<b>&amp;</b>  19.82<span style='color:#644a9b;'>\\\hline\hline</span>
<span style='color:#644a9b;'>\caption</span>*{<span style='color:#644a9b;'>\tiny</span>
<span style='color:#ff5500;'>$^1$</span> Millones de pesos a precios corrientes. Tomados del Banco de Información Económica del Instituto Nacional de Estadística y Geográfia: <span style='color:#644a9b;'>\url</span>{https://www.inegi.org.mx/sistemas/bie/}<span style='color:#644a9b;'>\\</span>
<span style='color:#ff5500;'>$^2$</span> Millones de dólares. Tomados del Banco de Información Económica del Instituto Nacional de Estadística y Geográfia: <span style='color:#644a9b;'>\url</span>{https://www.inegi.org.mx/sistemas/bie/}<span style='color:#644a9b;'>\\</span>
<span style='color:#ff5500;'>$^3$</span> Promedio trimestral pesos por dolar. Elaboración propia con datos de la Serie histórica diaria del tipo de cambio peso-dólar - (CF373) de Banco de México: 
<span style='color:#644a9b;'>\url</span>{http://www.anterior.banxico.org.mx/SieInternet/consultarDirectorioInternetAction.do?accion=consultarCuadro<b>&amp;</b>idCuadro=CF373<b>&amp;</b>sector=6<b>&amp;</b>locale=es}}<span style='color:#644a9b;'>\\\hline\hline</span>
<b>\end</b>{<b><span style='color:#0095ff;'>longtable</span></b>}
<b>\end</b>{<b><span style='color:#0095ff;'>tiny</span></b>}
<span style='color:#898887;'>%\end{landscape}</span>



<span style='color:#898887;'>%///////////////////////////////////////////////////////////////////////////////////////////////////////////////////</span>
<span style='color:#898887;'>%\\\\\\\\\\\\\\\\\\\\\\\\\\\\\\\\\\\\\\\\\\\\\\\\\\\\\\\\\\\\\\\\\\\\\\\\\\\\\\\\\\\\\\\\\\\\\\\\\\\\\\\\\\\\\\\\\\\</span>
<span style='color:#898887;'>%\cite{allem}</span>
<span style='color:#898887;'>%\cite{antmic}</span>
<span style='color:#898887;'>%\cite{ballesteros}</span>
<span style='color:#898887;'>%\cite{bay}</span>
<span style='color:#898887;'>%\cite{bcc}</span>
<span style='color:#898887;'>%\cite{cas}</span>
<span style='color:#898887;'>%\cite{chen}</span>
<span style='color:#898887;'>%\cite{des}</span>
<span style='color:#898887;'>%\cite{figueroa}</span>
<span style='color:#898887;'>%\cite{friedland}</span>
<span style='color:#898887;'>%\cite{hair}</span>
<span style='color:#898887;'>%\cite{har}</span>
<span style='color:#898887;'>%\cite{hinpri}</span>
<span style='color:#898887;'>%\cite{hoffman}</span>
<span style='color:#898887;'>%\cite{k1}</span>
<span style='color:#898887;'>%\cite{kai}</span>
<span style='color:#898887;'>%\cite{kalpaper}</span>
<span style='color:#898887;'>%\cite{kuna}</span>
<span style='color:#898887;'>%\cite{kunarao}</span>
<span style='color:#898887;'>%\cite{lay}</span>
<span style='color:#898887;'>%\cite{lazmor}</span>
<span style='color:#898887;'>%\cite{ljung}</span>
<span style='color:#898887;'>%\cite{mog}</span>
<span style='color:#898887;'>%\cite{myb}</span>
<span style='color:#898887;'>%\cite{row1}</span>
<span style='color:#898887;'>%\cite{row2}</span>
<span style='color:#898887;'>%\cite{sanliturk}</span>
<span style='color:#898887;'>%\cite{schu}</span>
<span style='color:#898887;'>%\cite{stoj}  </span>

<b><span style='color:#644a9b;'>\bibliographystyle</span></b>{<b><span style='color:#0095ff;'>jtbnew.bst</span></b>}
<b><span style='color:#644a9b;'>\bibliography</span></b>{<b><span style='color:#0095ff;'>trealizacion</span></b>}  

 
<span style='color:#644a9b;'>\printindex</span>
<b>\end</b>{<b><span style='color:#0095ff;'>document</span></b>}
</pre>
</body>
</html>
